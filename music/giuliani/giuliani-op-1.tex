\documentclass[a4paper]{article}

\begin{document}

\section*{Prefazione}

Lo studio della chitarra fù sempre la mia occupazione favorita, ed arrivarci alla perfezione lo scopo mio principale.

Anzioso di ritrovare il più giusto ed il più dritto sentiero, che conduce a questa meta, mi fù d'uopo aprire una strada non battuta, per avvicinarmi all' ideale, che fisso mi stava nella mente.

Vedendomi poi inoltrato a forza di zelo e di costanza, e non senza qualche successo, nacque in me il desiderio di rendere partecipi del frutto delle mie veglie quelli, che corrono l'istessa carriera, e di preservargli dagli sviamenti, mettendo in ordine le mie idee su tale assunto e somministrando loro una guida corta, sicura e nuova, quale, a mio sapere, fino adesso si desiderò ma invano.

Questi studj, che vengo a presentare al publico, sono il risultato delle lunghe e moltissime mie fatiche, confirmate % confermate?
dall' esperienza e dalla pratica; e sono persuaso, che gli amatori della chitarra, con un assiduo esercizio, in breve tempo saranno in grado di eseguire con espressione quanto e % è
stato composto in un genere più corretto per questo istrumento.

Gli esercizj seguenti sono adunque destinati per quelli, che, possedendo di già i primi elementi, desiderassero vieppiù perfezionarsi senza l'ajuto di un maestro.

Si divide quest' opera in quattro parti, cioè:

\end{document}

% Local Variables:
% TeX-engine: luatex
% End:
