\documentclass[a4paper]{book}

\linespread{1.5}
\setlength{\intextsep}{15pt}

\usepackage[hmarginratio={1:1},hmargin=1in,vmargin=1in]{geometry}
\usepackage{indentfirst}

\usepackage{titlesec}
\titleclass{\part}{top}
\titleformat{\part}[block]{\centering\normalfont\LARGE\bfseries}{}{0pt}{}[\markboth{}{}]
\titlespacing*{\part}{0pt}{50pt}{40pt}
\titleclass{\chapter}{straight}
\titleformat{\chapter}[block]{\centering\normalfont\LARGE\bfseries}{}{0pt}{}[\markboth{}{}]
\titlespacing*{\chapter}{0pt}{10pt}{10pt}
\titleclass{\section}{straight}
\titleformat{\section}[block]{\centering\normalfont\LARGE\bfseries}{}{0pt}{}
\titlespacing*{\section}{0pt}{0pt}{0pt}

\usepackage[labelformat=empty,textfont={Large,bf},justification=centering]{caption}
\usepackage{float}

\usepackage{lyluatex}

\usepackage[italian]{babel}

\usepackage{lipsum}

\title{%
  \Huge\textbf{METODO COMPLETO}\\[-1ex]
  \textsc{\large per}\\[-.5ex]
  \textbf{CHITARRA}\\[.5ex]
  \textit{\Large Composto espressamente\\
    per l'insegnamento di suo Figlio\\[.5ex]}
  \huge\textbf{Gustavo}}
\author{\Huge\bfseries FERDINANDO CARULLI}
\date{}

\begin{document}
\frontmatter

\maketitle
\thispagestyle{empty}
% \cleardoublepage

\part*{PREFAZIONE}

Ho composto, alcuni anni fa, un Metodo di Chitarra che il Pubblico accolse favorevolmente.

Trovando ora che la terza edizione di quest' opera è esaurita, approfitto dell' occasione onde farvi per la terza volta i cambiamenti e le aggiunte che l'esperienza mi additò necessarie per giovare alio studio della Chitarra.

Ebbi cura particolare in questa nuova Edizione del mio Metodo di presentare all' allievo gradatamente le difficoltà, e di mostrargliene l'esecuzione con altrettanti esempi.

Possa finalmente aver io raggiunto lo scopo che mi sono sempre proposto, e meritar l'approvazione degli amatori d'un istrumento, che non la cede a verun altro quando si sappia coll' arte cavarne degli effetti di cui è suscettibile.

\textit{NB\@.}  Quest' opera contiene tutto quanto è necessario per imparare a suonar bene la Chitarra; ma siccome gli esempi e gli esercizj ne sono succinti, ho composto un supplemento il quale contiene una quantità di pezzi che fanno seguito a tutto quanto si contiene in questo Metodo, più le Scale, Esercizj e Pezzi nei toni difficili.
\clearpage
\thispagestyle{empty}

\mainmatter

\renewcommand{\contentsname}{\Large\centering QUESTO METODO È DIVISO
  COME SEGUE.}
\tableofcontents
\thispagestyle{empty}
\clearpage
\thispagestyle{empty}

\part{PRIMA PARTE}

La Chitarra, così detta Francese, non ha che sei corde, di cui la prima, chiamata Cantino, è Mi; la seconda Si, la terza Sol, la quarta Re, la quint a La e la sesta Mi.

\begin{figure}[h]
  \centering
  \caption{ESEMPIO}
  \lilypondfile{strings.ly}
  \label{fig:strings}
\end{figure}

\chapter[Il modo di tener la Chitarra e di collocar le mani]{%
  MODO DI TENER LA CHITARRA\\
  \small E DI COLLOCAR LE MANI}

Si deve star seduto non troppo alto ne troppo basso; onde la Chitarra non salga troppo verso il petto o cada verso le ginocchia.

Si deve appoggiar I'istrumento sulla coscia sinistra; il manico sia più alto che la parte inferiore del corpo.  Le signore possono collocar su di un piccolo sgabello il piede sinistro.

La posizione del braccio sinistro non deve esser sempre la stessa; essa deve variare secondo che lo esige il movimento delle dita.

Il manico deve appoggiarsi sulle prime giunture del pollice e dell'indice della mano sinistra, lasciando liberi questi due diti.  Il pollice, che sta dietro il manico, non ha posizione fissa; ma a misura che le altre dita prendono posizioni più o meno difficili, esso deve trovarsi più infuori o più addentro del manico.

In alcuni Metodi gli Autori proibiscono agli allievi di valersi del poIlice della mano sinistra dal lato opposto alle altre dita sulla sesta corda e talvolta sulla quinta.
\clearpage

La Musica riesce tanto più piacevole quanto più è ricca d'armonia, e quattro diti non bastnndo per eseguire nel medesimo tempo un canto e dei bassi ragionati in diversi toni, bisogna adopefar necessariamente il pollice; cosi invito coloro, che vogliono suonare con maggior facilità, a valersene.

Il braccio destro debb'essere appoggiato sul lato che forma l'asse e la tavola armonica della Chitarra in linea retta del ponticello; la mano deve appoggiarsi leggiermente sul mignolo che deve posare quasi accanto al Cantino; e precisamente in mezzo della distanza dal ponticello all'apertura: questa mano non ha posizione fissa perché a misura che si vuol addolcire i suoni ed imitar l'Arpa, bisogna ravvicinarla all'apertura, e quando si vuol suonar forte bisogna accostarla al ponticello.

Si vedrà nella seconda parte il modo di pizzicar la difficoltà; ma nel prineipio, si pizzicherà la sesta, quinta e quarta corda col pollice della mano destra; la terza e seconda coll'indice, ed il cantino col medio.  Bisogna osservare però che negli arpeggi bisogna pizzicar talvolta la terza corda col pollice e talvolta la quinta e la quarta coll'indice ed il medio.

\begin{figure}[h]
  \centering
  \caption{ESEMPIO}
  \lilypondfile[indent=15pt]{right-hand-1.ly}
  \label{fig:right-hand-1}
\end{figure}

Quando il canto d'un pezzo di musica va discendendo, e che vi sono delle note basse d'accompagnamento, allora si è obbligato di pizzicar coll'indice fino alla quinta corda, e se invece il basso ascende bisogna che il pollice pizzichi fino alia seconda corda.

\begin{figure}[h]
  \centering
  \caption{ESEMPIO}
  \lilypondfile[indent=15pt]{right-hand-2.ly}
  \label{fig:right-hand-2}
\end{figure}

Il dito anulare non serve che nel pizzicato.

Bisogna far ben attenzione, pizzicando, che il pollice della mano destra, si trovi più in fuori degli altri diti, e che la mano non sia nè troppo alta ne troppo bassa indietro.
\clearpage

\chapter[Il modo di accordarla]{MODO D'ACCORDAR LA CHITARRA}

Il modo migliore d'accordar la Chitarra si e di accordarla all'orecchio a corde vuote, come si accorda il Violino.\ ecc..

Questa maniera non potendo essere descritta, indico la segneute per coloro che non potessero adoperare la prima.


Si accorda la quinta corda, che è La, col Diapason, istrumento d'acciajo cosi chiamato, oppure con un altro istrumento già stato accordato col Diapason; inseguito si pone un dito sul quinto tasto di questa stessa corda cho dà il Re, e si accorda la quarta all'unisono.  Si pone un dito al quarto tasto della terza corda che dà il Si, e si accorda la seconda all'unisono.  Si mette un dito sul quinto tasto del la seconda corda che dà il Mi, e si accorda il cantino all'unisono.  La sesta corda si accorda col cantino, ma due ottave più basso.

\chapter[La scala e gli esercizj per imparar a leggere le note alle
prima posizione]{%
  SCALA ALLA PRIMA POSIZIONE}

Le cifre indicano i diti della mano sinistra ed i tasti dove vanno collocati.

Le note a corda vuota sono indicate da un 0; l'indice da 1; il medio da 2; I'anulare da 3; ed il mignolo da 4.

\begin{figure}[h]
  \centering
  \lilypondfile{first-position.ly}
  \label{fig:first-position}
\end{figure}

\section*{ESERCIZI\\
  \normalsize\textsc{\mdseries per imparar bene a legare le note alle prima posizione}}

\begin{figure}[h]
  \centering
  \caption{\large SCALA}
  \lilypondfile{scale.ly}
  \label{fig:scale}
\end{figure}
\clearpage

\lilypondfile[indent=15pt]{scale-3-ex.ly}

\chapter[Scala con diesis e bemoli, ed un esercizio]{%
  SCALA\\
  \small CON DIESIS E BEMOLI}

\begin{figure}[H]
  \centering
  \lilypondfile{first-position-sharp.ly}
  \lilypondfile{first-position-flat.ly}
  \label{fig:first-position-chromatic}
\end{figure}

\section*{ESERCIZIO\\
  \normalsize\textsc{\mdseries per imparar bene a legar le note coi diesis ed i bemoli}}

\begin{figure}[H]
  \centering
  \lilypondfile{scale-chromatic-ex.ly}
\end{figure}
\clearpage

\chapter[Il piccolo e grande accordo]{DEGLI ACCORDI}

Vi è il piccolo ed il grande accordo.  Si chiama piccolo Accordo quando si e obbligato di prendere due o tre corde nel medesimo tasto col primo dito della mano sinistra e grande Accordo, quando se ne devono prendere cinque o sei.

{
  \setlength{\intextsep}{0pt}
  \begin{figure}[h]
    \centering
    \caption{ESEMPIO}
    \lilypondfile{barre.ly}
    \label{fig:barre}
  \end{figure}
}

\chapter[Il modi di pizzicar gli accordi]{%
  MODO DI PIZZICAR GLI ACCORDI\\[-1ex]
  \small COLLA MANO DESTRA}

Molti pizzicano gli accordi solamente col pollice, passandolo su tutte le corde.  Questa maniera non aggiunge veruna grazia alla mano e rende l'accordo troppo secco; cosi quando un accordo è di quattro note, bisogna pizzicarlo con quattro dita, ma assai celeremente, affinchè queste abbiano l'aria d'esser state pizzicate quasi insieme.

Quando l'accordo è di cinque note, si deve sdrucciolar il pollice sopra due corde e le altre corde cogli altri tre diti, e quando è di sei note, si deve sdrucciolar il pollice sopra tre corde, e le altre tre corde cogli altri tie diti.

\section*{ESEMPIO}

\textit{N.B\@.}  Il pollice della mano destia sarà indicato da un punto (.) l'indice da due punti (..) il medio da tre (...) e l'anulare da quattro (....)

\captionsetup{textfont={scriptsize,bf}}
{
  \setlength{\abovecaptionskip}{7pt}
  \begin{figure}[H]
    \centering
    \begin{minipage}{2.4in}
      \caption{ACCORDO di QUATTRO NOTE}
      \lilypondfile{pluck-4.ly}
    \end{minipage}
    \hfill
    \begin{minipage}{3in}
      \caption{ACCORDO di CINQUE NOTE}
      \lilypondfile{pluck-5.ly}
    \end{minipage}
  \end{figure}
  \begin{figure}[H]
    \centering
    \begin{minipage}{3.6in}
    \caption{ACCORDO di SEI NOTE}
    \lilypondfile{pluck-6.ly}
  \end{minipage}
  \end{figure}
}

\chapter[Gli arpeggi ed un esercizio]{DEGLI ARPEGGI}

Vi sono Arpeggi di tre, quattro, sei, otto, nove, dodici e sedici note, e si pizzicano con tre o quattro diti.  Se ne può fare una grandissima quantità, ma per esser breve, indichero i principali.

\begin{figure}[H]
  \centering
  \hspace*{15pt}
  \begin{minipage}{2.4in}
    \caption{ARPEGGI di TRE NOTE}
    \lilypondfile{arpeggio-3.ly}
  \end{minipage}
  \hfill
  \begin{minipage}{3in}
    \caption{ARPEGGI di QUATTRO NOTE}
    \lilypondfile{arpeggio-4.ly}
  \end{minipage}
  \hspace*{10pt}
\end{figure}

\begin{figure}[H]
  \centering
  \begin{minipage}{4.5in}
    \caption{ARPEGGI di SEI NOTE}
    \lilypondfile{arpeggio-6.ly}
  \end{minipage}
\end{figure}

\begin{figure}[H]
  \centering
  \begin{minipage}{3.2in}
    \caption{ARPEGGI di OTTO NOTE}
    \lilypondfile{arpeggio-8.ly}
  \end{minipage}
  \hfill
  \begin{minipage}{2.7in}
    \caption{ARPEGGI di NOVE NOTE}
    \lilypondfile{arpeggio-9.ly}
  \end{minipage}
\end{figure}

\begin{figure}[H]
  \centering
  \begin{minipage}{4.4in}
    \caption{ARPEGGI di DODICI NOTE}
    \lilypondfile{arpeggio-12.ly}
  \end{minipage}
\end{figure}

\begin{figure}[H]
  \centering
  \begin{minipage}{5.6in}
    \caption{ARPEGGI di SEDICI NOTE}
    \lilypondfile{arpeggio-16.ly}
  \end{minipage}
\end{figure}

\begin{figure}[H]
  \centering
  \begin{minipage}{6in}
    \caption{ARPEGGI a DOPPIE NOTE}
    \lilypondfile{arpeggio-double.ly}
  \end{minipage}
\end{figure}

\textit{N.B\@.}  Per addestrar le dita della mano dritta ed esercitarsi a far gli arpeggi con molta prestezza, si devono eseguir quelli che ho dati colla lezione seguente ad eccezione degli arpeggi a dodici od a sedici note.

Per suonar bene sulla Chitarra un pezzo di musica, bisogna, quando s'incontrano delle note di basso che non sono corde vuote, Iasciar sulla corda il dito fino a che un'altra nota obbliga di levarlo; quest'attenzione è necessaria per sostenere il suono di questa nota ed evitar quello che si produrrèbbe dalla vibrazione della corda vuota nel momento in cui il dito cesserebbe di premerla.

\begin{figure}[h]
  \centering
  \lilypondfile[indent=15pt]{arpeggio-ex.ly}
\end{figure}

\section*{\Large SEGUITO D'ARPEGGI DIFFICILI\\[-1.5ex]
  \small PER ESERCITARSI NELLA STESSA LEZIONE}

\begin{figure}[H]
  \centering
  \begin{minipage}{1.8in}
    \caption{ARPEGGIO di TRE NOTE}
    \lilypondfile{arpeggio-3-difficult.ly}
  \end{minipage}
  \hfill
  \begin{minipage}{1.9in}
    \caption{ARPEGGIO di QUATTRO NOTE}
    \lilypondfile{arpeggio-4-difficult.ly}
  \end{minipage}
  \hfill
  \begin{minipage}{2.1in}
    \caption{ARPEGGIO di SEI NOTE}
    \lilypondfile{arpeggio-6-difficult.ly}
  \end{minipage}
\end{figure}

\begin{figure}[H]
  \centering
  \begin{minipage}{1.8in}
    \caption{ARPEGGIO di OTTO NOTE}
    \lilypondfile{arpeggio-8-difficult.ly}
  \end{minipage}
  \hfill
  \begin{minipage}{1.9in}
    \caption{LO STESSO}
    \lilypondfile{arpeggio-8-difficult-2.ly}
  \end{minipage}
  \hfill
  \begin{minipage}{2.1in}
    \caption{LO STESSO}
    \lilypondfile{arpeggio-8-difficult-3.ly}
  \end{minipage}
\end{figure}

\begin{figure}[H]
  \centering
  \hfill
  \begin{minipage}{1.9in}
    \caption{ARPEGGIO di QUATTRO NOTE}
    \lilypondfile{arpeggio-4-difficult-2.ly}
  \end{minipage}
  \hfill
  \begin{minipage}{2.1in}
    \caption{ARPEGGIO di OTTO NOTE}
    \lilypondfile{arpeggio-8-difficult-4.ly}
  \end{minipage}
  \hfill{}
\end{figure}

Ciascun istrumento ha i suoi toni favoriti: si può suonare sulla Chitarra in tutti i toni; ma quelli che le convengono meglio sono La Maggiore e Minore, Re Maggiore e Minore, Mi Maggiore e Minore, Do, Sol, Fa.  Gli altri sono difficili; quindi ho dato le Scale, gli Accordi, gli Esercizj ed i pezzi seguenti nei toni più usitati e più facili pei principianti.

\chapter[Scale, accordi, esercizj e pezzi progressivi in varj toni i più necessarj]{%
  SCALE, ACCORDI, ESERCIZI,\\[.5ex]
  \small E PEZZI PROGRESSIVI IN DIFFERENTI TONI I PIÙ USITATI\\[-1em]
  ALLA PRIMA POSIZIONE}

\textit{N.B\@.}  Le cifre non indicano che i diti della mano sinistra.  Le note con una doppia coda vanno pizzicato col pollice della mano destra.

Nella Scala seguente bisogna che la mano sinistra si avanzi al secondo tasto per facilitare la digitazione.

Quando si iaiovano due note insieme che stanno ambedue sulla stessa corda, si fa la più alta al suo posto e la più bassa sulla corda che vien dopo.

\section*{ESEMPIO}

Re e Si sono ambidue sulla seconda corda; allora bisogna fàr il Re al suo posto ed il Si sulla terza corda al quarto tasto.

Sol e Mi sono ambidue sul cantino; bisogna fare il Sol al suo posto ed il Mi sulla seconda corda al quinto tasto.

\chapter[Seguito di pezzi progressivi per esercitarsi meglio]{%
  SEGUITO DI PEZZI PROGRESSIVI\\
  \small PER ESERCITARSI MEGLIO ALLA PRIMA POSIZIONE..}


\section*{FINE DELLA PRIMA PARTE}

\textit{\bfseries N.B\@.}  L'allievo passando alla Seconda Parte deve continuare ad esercitarsi con pezzi facili ch'egli troverà nelle opere 145. 120. 121. 122.\ ma specialmente nell'opera 114, opere tutte dell'Autore
\end{document}

% Local Variables:
% TeX-command-extra-options: "--shell-escape"
% End:
