\chapter{极限论}

\section{数列极限的概念与性质}

\begin{definition*}
  将一些数编好号后,按其编号从小到大排成一列,称为数列,记作\(a_1, a_2, \dots, a_n, \dots\)或\(\Seq{a_n}\).
\end{definition*}

\begin{definition*}
  从数列\(\Seq{a_n}\)中取出某些项后,按原来的顺序排成一个新的数列,称此数列为原数列\(\brce{a_n}\)的一个子列,记作\(\Seq{a_{n_k}}\),其中\(n_k \ge k,\ n_{k+1} > n_k\).
\end{definition*}

\begin{definition*}
  设\(\Seq{a_n}\)是一个数列,\(A\)是一个常数.对于任意的\(ε > 0\)都存在正整数\(N > 0\)使得当\(n > N\)时都有\(\abs*{a_n - A} < ε\)成立.这时,我们称\(A\)是数列\(\Seq{a_n}\)的极限,记作\(\displaystyle \lim_{n\to\infty} a_n = A\).

  \begin{remark}
    可以用形式化的符号来表示上述定义,即
    \[
      \lim_{n\to\infty} a_n = A \iff
      \paren[\big]{\forall ε > 0}
      \paren[\big]{\exists N > 0}
      \paren[\big]{\forall n > N}
      \paren[\big]{\abs[\big]{a_n - A} < ε}.
    \]
    它的否定形式就是
    \[
      \lim_{n\to\infty} a_n \ne A \iff
      \paren[\big]{\exists ε > 0}
      \paren[\big]{\forall N > 0}
      \paren[\big]{\exists n > N}
      \paren[\big]{\abs[\big]{a_n - A} \ge ε}.
    \]
  \end{remark}
\end{definition*}

\begin{example*}
  \(\displaystyle \lim_{n\to\infty} \paren[\Big]{1+\frac1n} = 1\).

  \begin{proof}
    取\(N = \ceil{\frac1ε}\)即可.
  \end{proof}
\end{example*}

\begin{example*}
  \(\displaystyle \lim_{n\to\infty} q^n = 0\ (\abs{q} < 1)\).

  \begin{proof}
    取\(N = \maxb*{1, \ceil*{\ln ε /\! \ln\abs q}}\)即可.
  \end{proof}
\end{example*}

\begin{example*}
  \(\displaystyle \lim_{n\to\infty} a^{1/n} = 1\ (a > 1)\).

  \begin{proof}
    取\(N = \ceil*{\ln a /\! \lnp{1+ε}}\)即可.
  \end{proof}

  \begin{proof}
    第二种方法就是把\(a^{1/n}\)分解成一个常数与一个无穷小量和的形式.因为\(a > 1\),所以\(a^{1/n} > 1\).令\(ε_n = a^{1/n} - 1\),就有
    \begin{align*}
      a^{1/n}
      &= 1 + ε_n
      && \reason{移项} \\
      a
      &= (1 + ε_n)^n
      && \reason{取\(n\)次幂} \\
      &= 1 + nε_n + \sum_{k=2}^n \binom{n}{k} ε_n^k
      && \reason{二项式定理} \\
      &> nε_n.
    \end{align*}
    因此,
    \begin{equation*}
      \paren[\big]{\forall ε > 0}
      \paren[\big]{\exists N = \ceil*{\frac aε} > 0}
      \paren[\big]{\forall n > N}
      \paren[\big]{\abs[\big]{a^{1/n} - 1} = ε_n < \frac an < ε}.
      \qedhere
    \end{equation*}
  \end{proof}
\end{example*}

\begin{example*}
  若\(\displaystyle \lim_{n\to\infty} a_n = A\),则\(\displaystyle \lim_{n\to\infty} \abs*{a_n} = \abs*{A}\).

  \begin{proof}
    因为\(\displaystyle \lim_{n\to\infty} a_n = A\),所以
    \begin{equation*}
      \paren[\big]{\forall ε > 0}
      \paren[\big]{\exists N > 0}
      \paren[\big]{\forall n > N}
      \paren[\big]{\abs[\big]{a_n - A} < ε}.
    \end{equation*}
    由反三角不等式
    \begin{equation*}
      \abs*{\abs{a_n} - \abs A} \le \abs*{a_n - A}
    \end{equation*}
    可知,取同样的\(N\)即可.
  \end{proof}
\end{example*}

\begin{example}
  \label{eg:limavg}
  若\(\displaystyle \lim_{n\to\infty} a_n = A\),则\(\displaystyle \lim_{n\to\infty} \frac{a_1 + a_2 + \dots + a_n}{n} = A\).

  \begin{proof}
    对于任意的\(ε > 0\),因为\(\displaystyle \lim_{n\to\infty} a_n = A\),所以存在正整数\(N_0\)使得当\(n > N_0\)时都有
    \begin{align*}
      \abs[\bigg]{\frac{a_1 + a_2 + \dots + a_n}{n} - A}
      &= \abs[\bigg]{\frac{a_1 + a_2 + \dots + a_n - nA}{n}} \\
      &= \abs[\bigg]{\frac{(a_1-A) + (a_2-A) + \dots + (a_n-A)}{n}} \\
      &\le \frac{\abs*{a_1-A} + \abs*{a_2-A} + \dots + \abs*{a_n-A}}{n} \\
      &= \frac{\abs*{a_1-A} + \dots + \abs*{a_{N_0}-A}}{n}
        + \frac{\abs*{a_{N_0+1}-A} + \dots + \abs*{a_n-A}}{n} \\
      &< \frac{N_0 \maxb{a_1-A, \dots, a_{N_0}-A}}{n} + \frac{(n-N_0)ε}{n} \\
      &< \frac{N_0 \maxb{a_1-A, \dots, a_{N_0}-A}}{n} + ε.
    \end{align*}
    此时,取\(N = \maxb[\Big]{N_0, \ceil[\Big]{\frac{N_0 \maxb{a_1-A, \dots, a_{N_0}-A}}{ε}}}\),当\(n > N\)时都有
    \begin{equation*}
      \abs*{\frac{a_1 + a_2 + \dots + a_n}{n} - A}
      < 2ε. \qedhere
    \end{equation*}
  \end{proof}
\end{example}

\begin{theorem*}
  若极限\(\displaystyle \lim_{n\to\infty} a_n\)存在,则其值唯一.

  \begin{proof}
    假设\(A_1\)和\(A_2\)都是数列\(\Seq{a_n}\)的极限且\(A_1 \ne A_2\).那么对于任意的\(ε > 0\),就可以找到正整数\(N_1\)和\(N_2\),当\(n > N_1\)时都有\(\abs*{a_n - A_1} < ε\),当\(n > N_2\)时都有\(\abs*{a_n - A_2} < ε\).这时,取\(N = \maxb{N_1, N_2}\),当\(n > N\)时都有
    \begin{gather*}
      2\abs[\bigg]{a_n - \frac{A_1+A_2}{2}} = \abs[\big]{2a_n - (A_1+A_2)} \le \abs[\big]{a_n - A_1} + \abs[\big]{a_n - A_2} < 2ε, \\
      \intertext{即}
      \abs[\bigg]{a_n - \frac{A_1+A_2}{2}} < ε.
    \end{gather*}
    这就说明\(\frac{A_1+A_2}{2}\)也是数列\(a_n\)的一个极限.那么对于任意的\(0 < ε \le \frac{\abs*{A_1-A_2}}{2}\),也能找到这样的正整数\(N\)使得当\(n > N\)时上述几个不等式都成立.但是这是不可能的,因为不论\(a_n\)多大,到\(A_1\)和\(A_2\)的距离,其中之一必然大于\(ε\).因此,\(A_1 = A_2\).
  \end{proof}
\end{theorem*}

\begin{theorem}
  \label{thm:cvgbnd}
  若数列\(\Seq{a_n}\)收敛,则其有界.

  \begin{proof}
    假设数列\(\Seq{a_n}\)收敛于\(A\).任取一个\(ε > 0\),则存在正整数\(N\)使得当\(n > N\)时都有\(\abs*{a_n - A} < ε\),即
    \begin{equation*}
      A - ε < a_n < A + ε
    \end{equation*}
    此时,\(\maxb*{\abs{A+ε}, \abs{A-ε}, \abs{a_1}, \dots, \abs{a_N}}\)就是数列\(\Seq{a_n}\)的一个界.
  \end{proof}
\end{theorem}

\begin{theorem*}[保号性]
  若\(\displaystyle \lim_{n\to\infty} a_n = A\)且\(A > 0\),则存在正整数\(N\)使得当\(n > N\)时都有\(a_n > 0\); 若\(a_n \ge 0\)且\(\displaystyle \lim_{n\to\infty} a_n\)存在,则\(\displaystyle \lim_{n\to\infty} a_n \ge 0\).

  \begin{proof}
    对前一个命题的证明:取\(ε = A\)时,能找到一个正整数\(N\)使得当\(n > N\)时都有\(\abs*{a_n - A} < A\),即\(0 < a_n < 2\,A\).对后一个命题的证明:假设\(\displaystyle \lim_{n\to\infty} a_n < 0\),那么根据前一个命题,就存在正整数\(N\)使得当\(n > N\)时都有\(a_n < 0\),这和\(a_n \ge 0\)是矛盾的,所以\(\displaystyle \lim_{n\to\infty} a_n \ge 0\).
  \end{proof}
\end{theorem*}

\begin{theorem}[数列极限的四则运算]
  \label{thm:seq4ops}
  设\(\displaystyle \lim_{n\to\infty} a_n = A\)和\(\displaystyle \lim_{n\to\infty} a_n = A\),则
  \begin{enumerate}
    \renewcommand{\labelenumi}{\enumparen{\arabic{enumi}}}
  \item \(\displaystyle \lim_{n\to\infty} \paren{a_n \pm b_n} = A \pm B\);
  \item \(\displaystyle \lim_{n\to\infty} a_n b_n = AB\);
  \item \(\displaystyle \lim_{n\to\infty} \tfrac{a_n}{b_n} = \tfrac AB\ (B \ne 0)\).
  \end{enumerate}

  \begin{proof}
    对加减法的证明:利用绝对值的三角不等式.对于任意的\(ε\)都能找到\(N_1\)和\(N_2\)使得当\(n > N_1\)时都有\(\abs*{a_n-A} < ε\)和当\(n > N_2\)时都有\(\abs*{b_n-B} < ε\).取\(N = \maxb{N_1, N_2}\),就有
    \begin{equation*}
      \abs*{\paren{a_n \pm b_n} - \paren{A \pm B}}
      = \abs*{a_n - A \pm b_n \mp B}
      = \abs*{\paren{a_n - A} \pm \paren{b_n - B}}
      \le \abs*{a_n - A} + \abs*{b_n - B}
      < 2ε.
    \end{equation*}
    对乘法的证明:由于定理~\ref{thm:cvgbnd},可以找到数列\(b_n\)的一个界\(M > 0\).对于任意的\(ε\)都能找到\(N_1\)和\(N_2\)使得当\(n > N_1\)时都有\(\abs*{a_n-A} < ε\)和当\(n > N_2\)时都有\(\abs*{b_n-B} < ε\).取\(N = \maxb{N_1, N_2}\),就有
    \begin{align*}
      \abs*{a_n b_n - AB}
      &= \abs*{a_n b_n - Ab_n + Ab_n - AB} \\
      &= \abs*{\paren{a_n - A}b_n + A\paren{b_n - B}} \\
      &\le \abs*{a_n-A}\abs*{b_n} + \abs*{A}\abs*{b_n-B} \\
      &\le M\abs*{a_n-A} + \abs*{A}\abs*{b_n-B} \\
      &< \paren*{M + \abs*{A}}ε.
    \end{align*}
    对除法的证明:只需证明\(\lim\limits_{n\to\infty} \frac1{b_n} = \frac1B\),再利用一次极限的乘法运算即可.对于任意的\(ε\)都能找到一个\(N_1\)使得当\(n > N_1\)时都有\(\abs*{b_n-B} < ε\).又由于\(B \ne 0\),可以找到一个正整数\(N_2\)使得\(\inf\limits_{\,\mathclap{n > N_2}}\Set*{\abs{b_n}} > 0\).取\(N = \maxb{N_1, N_2}\),就有
    \begin{equation*}
      \abs[\bigg]{\frac1{b_n} - \frac1B}
      = \abs[\bigg]{\frac{B - b_n}{b_n B}}
      \le \frac{\abs*{b_n - B}}{\abs*{B}
        \cdot \inf\limits_{\,\mathclap{n > N_2}}\Set*{\abs{b_n}}}
      < \frac{ε}{\abs*{B} \cdot \inf\limits_{\,\mathclap{n > N_2}}\Set*{\abs{b_n}}}.
      \qedhere
    \end{equation*}
  \end{proof}
\end{theorem}

\begin{definition*}
  对于任意的正数\(M\)都有正整数\(N\)使得当\(n > N\)时都有\(\abs*{a_n} > M\).这时,我们称数列\(\Seq{a_n}\)为无穷大量,记作\(\lim\limits_{n\to\infty} a_n = \infty\).
\end{definition*}

可以用形式化的符号来表示无穷大量的定义,即
\begin{alignat*}{2}
  \lim_{n\to\infty} a_n &= \infty &{}\iff{}
  &\paren[\big]{\forall M > 0}
  \paren[\big]{\exists N > 0}
  \paren[\big]{\forall n > N}
  \paren[\big]{\abs[\big]{a_n} > M}, \\
  \lim_{n\to\infty} a_n &= +\infty &{}\iff{}
  &\paren[\big]{\forall M > 0}
  \paren[\big]{\exists N > 0}
  \paren[\big]{\forall n > N}
  \paren[\big]{a_n > M}, \\
  \lim_{n\to\infty} a_n &= -\infty &{}\iff{}
  &\paren[\big]{\forall M > 0}
  \paren[\big]{\exists N > 0}
  \paren[\big]{\forall n > N}
  \paren[\big]{-a_n > M}.
\end{alignat*}
其中,\(+\infty\)叫作正无穷大量,\(-\infty\)叫作负无穷大量.

\begin{definition*}
  极限为\(0\)的数列叫作无穷小量.
\end{definition*}

\begin{example*}
  非零无穷小量的倒数是无穷大量.

  \begin{proof}
    设\(\lim\limits_{n\to\infty} a_n = 0\)且\(a_n \ne 0\).对于任意的\(M > 0\),取\(ε = 1/M > 0\),能找到一个正整数\(N\)使得当\(n > N\)时都有
    \begin{equation*}
      \abs*{a_n} < ε
      \implies
      \abs[\bigg]{\frac{1}{a_n}} = \frac{1}{\abs*{a_n}} > \frac1ε = M.
      \qedhere
    \end{equation*}
  \end{proof}
\end{example*}

\begin{theorem}
  \label{thm:sequnbndsubinf}
  无界数列存在为无穷大量的子列.

  \begin{proof}
    设数列\(\Seq{a_n}\)无界.证明分为两步,首先构造一个子列\(\Seq{a_{n_k}}\),然后说明这个子列是无穷大量.子列的构造过程是递推的.因为数列无界,所以能找到一个\(n_1\)使得\(\abs[\big]{a_{n_1}} > 1\).假设子列的前\(k\)项都已经构造好了.因为数列无界,所以能找到一个\(n_{k+1} > n_k\)使得\(\abs[\big]{a_{n_{k+1}}} > k+1\).如果不是这样,\(\maxb*{\abs[\big]{a_1}, \abs[\big]{a_2}, \dots, \abs[\big]{a_{n_k}}, k+1}\)就是原数列的一个界,矛盾.子列构造完毕.下面证明这个子列是无穷大量.对于任意的\(M > 0\)都能找到一个\(K = \ceil{M}\)使得当\(k > K\)时都有\(\abs[\big]{a_{n_k}} > k > K \ge M\).
  \end{proof}
\end{theorem}

\subpdfbookmark{思考}{B1.2.1.P}
\subsection*{思考}

\begin{enumerate}
\item 在数列极限的定义中,\(ε\)是不是一个无限小的正数?正整数\(N\)的选取是不是与\(ε\)有关?

  \ifshowsolp
  对\(ε\)来讲,就是一个任意的正实数,一旦给定,就确定下来了.题目中的无限小的正数,在标准实分析里是不存在的.正整数\(N\)的选取当然是和\(ε\)有关,有时候为了分析方便,还会给\(N\)标上下标\(N_ε\).
  \fi

\item 在数列极限的定义中,如果将\(n > N\)改成\(n \ge N\),是否有影响?如果将\(\forall ε > 0\)改成\(\forall ε \in \paren{0,1}\),是否有影响?

  \ifshowsolp
  都不影响.
  \fi
\end{enumerate}

\ifshowex
\currentpdfbookmark{练习}{B1.2.1.E}
\subsection*{练习}

\begin{enumerate}
\item 下列说法中,与\(\lim\limits_{n\to\infty} a_n = A\)不等价的是\uline{\makebox[10em]{}}.
  \begin{itemize}
    \renewcommand{\labelitemi}{\faCircleThin}
  \item \(
    \paren[\big]{\forall ε > 0}
    \paren[\big]{\exists N \in \N^+}
    \paren[\big]{\forall n > N}
    \paren[\big]{\abs[\big]{a_n - A} < \sqrt ε}
    \)
  \item \(
    \paren[\big]{\forall k \in \N}
    \paren[\big]{\exists N_k \in \Z^+}
    \paren[\big]{\forall n > N_k}
    \paren[\big]{\abs[\big]{a_n - A} < 1/2^k}
    \)
  \item \(
    \paren[\big]{\forall ε > 0}
    \paren[\big]{\exists N \in \N^+}
    \paren[\big]{\forall n > N}
    \paren[\big]{\abs[\big]{a_n - A} < 2ε}
    \)
    \ifshowsol
  \item[\faCircle]
    \else
  \item
    \fi
    \(
    \paren[\big]{\forall ε > 0}
    \paren[\big]{\exists N \in \N^+}
    \paren[\big]{\forall n > N}
    \paren[\big]{\abs[\big]{a_n - A} < ε/\!\sqrt n}
    \)
  \end{itemize}

  \ifshowsol
  一个反例是\(\lim\limits_{n\to\infty}\paren*{1 + 1/\!\sqrt n} = 1\),但是按照选项~D,它并不收敛.实际上,选项~D是数列收敛的充分不必要条件.
  \fi

\item 下列说法中,与“数列\(\Seq{a_n}\)不收敛于\(A\)”等价的是\uline{\makebox[10em]{}}.
  \begin{itemize}
    \renewcommand{\labelitemi}{\faCircleThin}
    \ifshowsol
  \item[\faCircle]
    \else
  \item
    \fi
    存在\(ε_0 > 0\)使得数列\(\Seq{a_n}\)中有无穷多项满足\(\abs*{a_n - A} \ge ε_0\)
  \item \(
    \paren[\big]{\exists ε_0 > 0}
    \paren[\big]{\exists N \in \Z^+}
    \paren[\big]{\forall n > N}
    \paren[\big]{\abs[\big]{a_n - A} \ge ε_0}
    \)
  \item \(
    \paren[\big]{\forall ε > 0}
    \paren[\big]{\exists N}
    \paren[\big]{\forall n > N}
    \paren[\big]{\abs[\big]{a_n - A} \ge ε_0}
    \)
  \item 数列\(\Seq{a_n}\)中,除有限项外,都满足\(\abs*{a_n - A} \ge ε_0\),其中\(ε_0\)是某个正数
  \end{itemize}

  \ifshowsol
  选项~B、C和~D都是题干的充分不必要条件.例如数列\(\Seq[\big]{A + \frac{1+(-1)^n}{2}}\)不收敛于\(A\),但是不满足选项~B、C、D.实际上,选项~B和~D是等价的,然后选项~C是选项~B的充分不必要条件.
  \fi

\item 下列说法中,正确的是\uline{\makebox[10em]{}}.
  \begin{itemize}
    \renewcommand{\labelitemi}{\faCircleThin}
  \item 数列\(\Seq{a_n}\)是否收敛与其前\(1000\)项有关
  \item 数列\(\Seq{a_n}\)是否收敛与其所有项均有关
  \item 数列\(\Seq{a_n}\)是否收敛仅与\(n\)充分大以后的某些项有关
    \ifshowsol
  \item[\faCircle]
    \else
  \item
    \fi
    数列\(\Seq{a_n}\)是否收敛仅与\(n\)充分大以后的所有项有关
  \end{itemize}

\item 下列数列中,极限为\(0\)的是\uline{\makebox[6em]{}}.
  \begin{itemize}
    \renewcommand{\labelitemi}{\faCircleThin}
  \item \(e^n/2^n\)
    \ifshowsol
  \item[\faCircle]
    \else
  \item
    \fi
    \(\paren{-1}^n/n\)
  \item \(n - 1/n\)
  \item \(n \sinp{1/n}\)
  \end{itemize}

\item 下列数列中,极限为\(1\)的是\uline{\makebox[6em]{}}.
  \begin{itemize}
    \renewcommand{\labelitemi}{\faCircleThin}
  \item \(n/a^n\ \paren{a > 1}\)
    \ifshowsol
  \item[\faCircle]
    \else
  \item
    \fi
    \(a^{1/n}\ \paren{a > 1}\)
  \item \(\paren[\big]{\sin n^2}\big/{n}\)
  \item \(\paren[\big]{n \sqrt{n+1}}\big/\brkt[\big]{\sqrt n \paren{2n-1}}\)
  \end{itemize}

\item 下列说法中,正确的是\uline{\makebox[10em]{}}.
  \begin{itemize}
    \renewcommand{\labelitemi}{\faCircleThin}
  \item 收敛数列的子列极限不一定相同
    \ifshowsol
  \item[\faCircle]
    \else
  \item
    \fi
    收敛数列的极限与其前有限项无关
  \item 数列不收敛则必无收敛子列
  \item 若收敛数列的极限大于零,则数列恒大于零
  \end{itemize}

\item 下列数列中,不是无穷大量的是\uline{\makebox[8em]{}}.
  \begin{itemize}
    \renewcommand{\labelitemi}{\faCircleThin}
  \item \(n/\!\ln n\)
  \item \(-n^2 + n\)
    \ifshowsol
  \item[\faCircle]
    \else
  \item
    \fi
    \(\brkt{n\paren{n^{7/3}+1}}/n^{15/4}\)
  \item \(\paren{-1}^n n^3 + n^2 - 10n\)
  \end{itemize}

\item 下列数列中,无界但不是无穷大量的是\uline{\makebox[6em]{}}.
  \begin{itemize}
    \renewcommand{\labelitemi}{\faCircleThin}
  \item \(n/\!\ln n\)
  \item \(\paren{-1}^n n^2 + n\)
    \ifshowsol
  \item[\faCircle]
    \else
  \item
    \fi
    \(n \sinp{n^n\!/2}\)
  \item \(e^n\!/n!\)
  \end{itemize}

\item 与命题“当\(n\to\infty\)时,\(a_n\to\infty\)”等价的是\uline{\makebox[10em]{}}.
  \begin{itemize}
    \renewcommand{\labelitemi}{\faCircleThin}
  \item \(
    \paren[\big]{\forall M > 0}
    \paren[\big]{\forall N > 0}
    \paren[\big]{\exists n > N}
    \paren[\big]{\abs[\big]{a_n} > M}
    \)
  \item \(
    \paren[\big]{\exists M > 0}
    \paren[\big]{\exists N > 0}
    \paren[\big]{\forall n > N}
    \paren[\big]{\abs[\big]{a_n} > M}
    \)
    \ifshowsol
  \item[\faCircle]
    \else
  \item
    \fi
    \(
    \paren[\big]{\forall M > 0}
    \paren[\big]{\exists N > 0}
    \paren[\big]{\forall n > N}
    \paren[\big]{\abs[\big]{a_n} > M}
    \)
  \item \(
    \paren[\big]{\exists N > 0}
    \paren[\big]{\forall M > 0}
    \paren[\big]{\forall n > N}
    \paren[\big]{\abs[\big]{a_n} > M}
    \)
  \end{itemize}

  \ifshowsol
  选项~A其实相当于是说数列\(\Seq{a_n}\)无界,选项~D是一个不可满足的命题.
  \fi

\item 下列说法中,正确的是\uline{\makebox[10em]{}}.
  \begin{itemize}
    \renewcommand{\labelitemi}{\faCircleThin}
  \item 一个数列如果不收敛,则它一定无界
  \item 若\(\lim\limits_{n\to\infty} a_n = A\)且\(A \ge 0\),则存在正整数\(N\)使得当\(n > N\)时都有\(a_n \ge 0\)
  \item 设有两个数列\(\Seq{a_n}\)和\(\Seq{b_n}\),那么\(\lim\limits_{n\to\infty} \paren{a_n - b_n} = \lim\limits_{n\to\infty} a_n - \lim\limits_{n\to\infty} b_n\)
    \ifshowsol
  \item[\faCircle]
    \else
  \item
    \fi
    存在发散数列\(\Seq{a_n}\)使得\(\Seq{\abs{a_n}}\)收敛
  \end{itemize}

  \ifshowsol
  对于选项~C,加上一个限制条件就成立了,就是这两个数列都收敛.设\(a_n = n + \frac1n, b_n = n\),那么\(\lim\limits_{n\to\infty} \paren{a_n - b_n} = \lim\limits_{n\to\infty} \frac1n = 0 \ne \infty - \infty = \lim\limits_{n\to\infty} a_n - \lim\limits_{n\to\infty} b_n\).
  \fi

\item 设\(a_k \ge 0,\ k = 1, 2, \dots, m\).求\(\lim\limits_{n\to\infty} \paren{a_1^n + a_2^n + \dots + a_m^n}^{1/n} =\)\uline{\makebox[6em]{}}.
  \begin{itemize}
    \renewcommand{\labelitemi}{\faCircleThin}
    \ifshowsol
  \item[\faCircle]
    \else
  \item
    \fi
    \(\max_{1 \le k \le m} \Set{a_k}\)
  \item \(\min_{1 \le k \le m} \Set{a_k}\)
  \item \(\paren{a_1 + a_2 + \dots + a_m}/{m}\) % report
  \item \(1\)
  \end{itemize}

  \ifshowsol
  \begin{proof}
    令\(A = \max_{1 \le k \le m} \Set{a_k}\).若\(A = 0\),则说明\(a_k = 0\)对于所有的\(1 \le k \le m\),所求极限自然也就等于零.若\(A > 0\),就有
    \begin{align*}
      \abs[\big]{\paren{a_1^n + a_2^n + \dots + a_m^n}^{1/n} - A}
      &= A \abs[\bigg]{\brkt[\Big]{\paren[\Big]{\frac{a_1}{A}}^n + \paren[\Big]{\frac{a_1}{A}}^n + \dots + \paren[\Big]{\frac{a_1}{A}}^n}^{\frac1n} - 1} \\
      &\le A \paren{m^{1/n} - 1}.
    \end{align*}
    对于任意的\(ε > 0\)取\(N = \ceil*{\ln m/\!\lnp{1+ε/A}}\)就可使得当\(n > N\)时都有
    \begin{align*}
      \frac{\ln m}{\lnp{1+ε/A}}
      &\le N < n
      && \iff \\
      \ln m
      &< n \lnp{1+ε/A}
      && \iff \\
      m
      &< \paren{1+ε/A}^n
      && \iff \\
      A \paren{m^{1/n} - 1}
      &< ε.
      &&\qedhere
    \end{align*}
  \end{proof}
  \fi
\end{enumerate}
\fi

\section{数列极限存在的充分条件}

\begin{theorem*}[数列极限的夹逼定理]
  若数列\(\Seq{a_n}, \Seq{b_n}, \Seq{c_n}\)满足:
  \begin{enumerate}[topsep=0ex,itemsep=0ex]
    \renewcommand{\labelenumi}{\enumparen{\arabic{enumi}}}
  \item 存在正整数\(N\)使得当\(n > N\)时都有\(a_n \le b_n \le c_n\),
  \item \(\lim\limits_{n\to\infty} a_n = \lim\limits_{n\to\infty} c_n = A\);
  \end{enumerate}
  则\(\lim\limits_{n\to\infty} b_n = A\).

  \begin{proof}
    对于任意的\(ε > 0\)都存在正整数\(N_1\)和\(N_2\)使得当\(n > N_1\)时都有\(\abs*{a_n - A} < ε\)和当\(n > N_2\)时都有\(\abs*{c_n - A} < ε\).取\(N = \maxb{N_1, N_2}\),当\(n > N\)时就有
    \begin{equation*}
      A - ε < a_n \le b_n \le c_n < A + ε
      \iff
      \abs*{b_n - A} < ε.
      \qedhere
    \end{equation*}
  \end{proof}
\end{theorem*}

\begin{example}
  \label{eg:factexp}
  求\(\displaystyle \lim_{n\to\infty} \frac{a^n}{n!}\).\rule[-2ex]{0ex}{0ex}

  \begin{remark}
    当\(a = 0\)时,极限显然是零.当\(a \ne 0\)时,我们构造数列\(a_n\)和\(c_n\)使得\(a_n \le a^n\!/n! \le c_n\)且\(\lim\limits_{n\to\infty} a_n = \lim\limits_{n\to\infty} c_n = 0\).令\(n_a = \ceil{\abs a},\ C_a = \frac{a^{n_a}}{n_a!}\),构造数列
    \begin{equation*}
      d_n =
      \begin{cases}
        C_a, & n \le n_a, \\
        C_a \cdot \paren[\bigg]{\dfrac{a}{n_a + 1}}^{n-n_a}, & n > n_a,
      \end{cases}
    \end{equation*}
    易知\(\lim\limits_{n\to\infty} d_n = 0\).当\(a > 0\)时,只要构造数列\(a_n = 0\)和\(c_n = d_n\)即可.当\(a < 0\)时,只有构造数列
    \begin{equation*}
      a_n =
      \begin{cases}
        0, & \text{\(n\)是偶数时}, \\
        d_n, & \text{\(n\)是奇数时},
      \end{cases}
      \quad\text{和}\quad
      c_n =
      \begin{cases}
        d_n, & \text{\(n\)是偶数时}, \\
        0, & \text{\(n\)是奇数时},
      \end{cases}
    \end{equation*}
    就有\(a_n \le a^n\!/n! \le c_n\)且\(\lim\limits_{n\to\infty} a_n = \lim\limits_{n\to\infty} c_n = 0\).综上所述,\(\displaystyle \lim_{n\to\infty} \frac{a^n}{n!} = 0\).\rule[-2ex]{0ex}{0ex}

    上面是为了直接套用夹逼定理的形式,实际上一种常见的方法是:欲证\(\lim\limits_{n\to\infty} a_n = A\),只需证明\(0 \le \abs*{a_n - A} \le b_n\)且\(\lim\limits_{n\to\infty} b_n = 0\)即可.
  \end{remark}
\end{example}

\begin{example*}
  求\(\displaystyle \lim_{n\to\infty} \sum_{k=1}^n \frac{k}{n^2+k}\).

  \begin{remark}
    构造数列
    \begin{equation*}
      a_n
      = \sum_{k=1}^n \frac{k}{n^2 + n}
      = \frac{1}{n^2 + n} \cdot \frac{n(n+1)}{2}
      = \frac12
      \quad\text{和}\quad
      c_n
      = \sum_{k=1}^n \frac{k}{n^2}
      = \frac{1}{n^2} \cdot \frac{n(n+1)}{2}
      = \frac{1 + 1/n}{2},
    \end{equation*}
    易知\(\displaystyle a_n \le \sum\limits_{k=1}^n \frac{k}{n^2+k} \le c_n\)且\(\displaystyle \lim_{n\to\infty} a_n = \lim_{n\to\infty} c_n = \frac12\).\rule[-3ex]{0ex}{0ex}
  \end{remark}
\end{example*}

\begin{theorem*}[单调有界收敛定理]
  若数列\(\Seq{a_n}\)单调增加且有上界,则\(\Seq{a_n}\)收敛且\(\lim\limits_{n\to\infty} a_n = \sup\Set{a_n}\); 若数列\(\Seq{a_n}\)单调减少且有下界,则\(\Seq{a_n}\)收敛且\(\lim\limits_{n\to\infty} a_n = \inf\Set{a_n}\).

  \begin{proof}
    只证单增的情况,单减的情况类似.因为数列有界,根据确界存在公理,数列有上确界\(M = \sup\Set{a_n}\).那么对于任意的\(ε > 0\)都存在一个\(N\)使得\(a_N > M - ε\).当\(n > N\)时,自然有
    \begin{gather*}
      M - ε < a_N \le a_n \le M < M + ε, \\
      \shortintertext{即}
      \abs*{a_n - M} < ε.
      \qedhere
    \end{gather*}
  \end{proof}
\end{theorem*}

\begin{theorem}
  \label{thm:seqe}
  证明\(\displaystyle \lim_{n\to\infty} \paren[\bigg]{1 + \frac1n}^n\)存在.

  \begin{proof}
    记\(a_n = \paren{1+1/n}^n\).根据单调有界收敛定理,只需证明对于所有的正整数\(n\)都有\(a_n \le a_{n+1}\)且\(a_n < 3\)即可.根据二项式定理,有
    \begin{align*}
      a_n
      &= \paren[\bigg]{1+\frac1n}^n
        = 1 + \binom{n}{1} \frac1n  + \binom{n}{2} \frac1{n^2} + \dots + \binom{n}{n} \frac1{n^n} \\
      &= 1 + 1 + \frac{n(n-1)}{2!\,n^2} + \dots + \frac{n(n-1)\dotsm1}{n!\,n^n} \\
      &= 2 + \frac1{2!} \cdot 1 \cdot \paren[\bigg]{1 - \frac1n} + \dots
        + \frac1{n!} \cdot 1 \cdot \paren[\bigg]{1 - \frac1n} \dotsm \paren[\bigg]{1 - \frac{n-1}n} \\
      &< 2 + \frac1{2!} \cdot 1 \cdot \paren[\bigg]{1 - \frac1{n+1}} + \dots
        + \frac1{(n+1)!} \cdot 1 \cdot \paren[\bigg]{1 - \frac1{n+1}} \dotsm \paren[\bigg]{1 - \frac{n}{n+1}} \\
      &= a_{n+1}.
    \end{align*}
    要证明\(a_n < 3\),只需证明\(1/2! + 1/3! + \dots + 1/n! < 1\)即可.有
    \begin{align*}
      \frac1{2!} + \frac1{3!} + \dots + \frac1{n!}
      &\le \frac{1}{1 \cdot 2} + \frac{1}{2 \cdot 3} + \dots + \frac{1}{\paren{n-1}n} \\
      &= \paren[\bigg]{1 - \frac12} + \paren[\bigg]{\frac12 - \frac13} + \dots + \paren[\bigg]{\frac1{n-1} - \frac1n} \\
      &= 1 - \frac1n < 1. \qedhere
    \end{align*}
  \end{proof}

  \begin{remark}
    实际上,这个极限就是自然常数\(e\).
  \end{remark}
\end{theorem}

\begin{example*}
  令\(\displaystyle a_n = 1 + \frac12 + \dots + \frac1n - \ln n\).证明数列\(\Seq{a_n}\)的极限存在.\rule{0ex}{3.5ex}

  \begin{proof}
    当\(x > 0\)时,有
    \begin{equation*}
      \frac{x}{1+x} < \lnp{1+x} < x.
    \end{equation*}
    因此有
    \begin{align*}
      a_{n+1} - a_n
      &= \frac{1}{1+n} - \lnp{1+n} + \ln n \\
      &= \frac{1/n}{1+1/n} - \lnp[\bigg]{1+\frac1n} < 0
    \end{align*}
    和
    \begin{align*}
      a_n
      &= 1 + \frac12 + \dots + \frac1n - \ln n \\
      &> \lnp{1+1} + \lnp[\bigg]{1+\frac12} + \dots + \lnp[\bigg]{1+\frac1n} - \ln n \\
      &= \lnp[\bigg]{2 \cdot \frac32 \dotsm \frac{n+1}{n}} - \ln n \\
      &= \lnp{n + 1} - \ln n > 0. \qedhere
    \end{align*}
  \end{proof}
\end{example*}

\begin{example*}
  证明极限\(\displaystyle \lim_{n\to\infty} \sum_{k=1}^n \frac1k\)不存在.

  \begin{proof}
    实际上,\(\displaystyle \lim_{n\to\infty} \sum_{k=1}^n \frac1k = \infty\).对于任意的\(M > 0\)都存在正整数\(N = \ceil[\big]{e^M-1}\)使得当\(n > N\)时都有
    \begin{equation*}
      \abs[\Bigg]{\sum_{k=1}^n \frac1k}
      = \sum_{k=1}^n \frac1k
      > \sum_{k=1}^n \lnp[\bigg]{1+\frac1k}
      = \lnp{n+1} > M.
      \qedhere
    \end{equation*}
  \end{proof}
\end{example*}

\begin{example*}
  令\(\displaystyle a_n = \sum_{k=1}^n \frac1{k^p}\).试讨论当\(p > 0\)时数列\(\Seq{a_n}\)的敛散性.

  \begin{remark}
    当\(0 < p \le 1\)时,有\(k^p \le k\).所以
    \begin{equation*}
      a_n = \sum_{k=1}^n \frac1{k^p} \ge \sum_{k=1}^n \frac1{k}
      > \sum_{k=1}^n \lnp[\bigg]{1+\frac1k} = \lnp{n+1}.
    \end{equation*}
    对于任意的\(M > 0\),取\(N = \ceil[\big]{e^M - 1}\),那么当\(n > N\)时,自然有\(a_n > \lnp{n+1} > M\).所以有
    \begin{equation*}
      \lim_{n\to\infty} a_n = +\infty
    \end{equation*}
    当\(0 < p \le 1\)时,由\(a_{n+1} - a_n = 1/(n+1)^p > 0\)可知数列单调增加,只需证明数列有上界即可.取\(\ell = \floor[\big]{\log_2 n} + 1\),有
    \begin{equation*}
      a_n
      = \sum_{k=1}^n \frac1{k^p}
      \le \sum_{k=1}^{2^\ell-1} \frac1{k^p}
      = \sum_{m=0}^{\ell-1} \sum_{j=2^m}^{2^{m+1}-1} \frac{1}{j^p}
      \le \sum_{m=0}^{\ell-1} \frac{2^m}{\paren{2^m}^p}
      = \sum_{m=0}^{\ell-1} \paren[\big]{2^{1-p}}^m
      = \frac{1-\paren{2^{1-p}}^\ell}{1-2^{1-p}}
      < \frac{1}{1-2^{1-p}}.
    \end{equation*}
    由单调有界收敛定理可知,数列\(\Seq{a_N}\)收敛.
  \end{remark}
\end{example*}

\subpdfbookmark{思考}{B1.2.2.P}
\subsection*{思考}

\begin{enumerate}
\item 若单调递增的数列\(\Seq{a_n}\)没有上界,这个数列是不是无穷大量?能否给出严格证明?

  \ifshowsolp
  是无穷大量,证明如下.

  \begin{proof}
    因为数列\(\Seq{a_n}\)没有上界,所以对任意的\(M > 0\)都存在一个正整数\(N\)使得\(a_N > M\).这时,对于所有的\(n > N\),因为数列\(\Seq{a_n}\)单调递增,所以
    \begin{equation*}
      a_n \ge a_N > M. \qedhere
    \end{equation*}
  \end{proof}
  \fi

\item 若数列\(\Seq{a_n}\)是正无穷大量,则\(\Seq{a_n}\)一定单调增加吗?

  \ifshowsolp
  未必.\(a_n = n + (-1)^n\),易证\(\lim\limits_{n\to\infty} a_n = +\infty\).此时,对于任意的正偶数\(k\)都有
  \begin{equation*}
    a_{k+1} = (k+1) + (-1)^{k+1} = k < k + 1 = k + (-1)^k = a_k.
  \end{equation*}
  \fi
\end{enumerate}

\ifshowex
\currentpdfbookmark{练习}{B1.2.2.E}
\subsection*{练习}

\begin{enumerate}
\item 下列数列中,收敛但极限不为\(1\)的是\uline{\makebox[6em]{}}.
  \begin{itemize}
    \renewcommand{\labelitemi}{\faCircleThin}
  \item \(\paren{2+1/n}^{1/n}\)
  \item \(n^{1/n}\)
    \ifshowsol
  \item[\faCircle]
    \else
  \item
    \fi
    \(\frac{1}{n^2+1} + \frac{2}{n^2+2} + \dots + \frac{n}{n^2+n}\)
  \item \(\paren{n!}^2\!/n^n\)
  \end{itemize}

  \ifshowsol
  对选项~B的证明:
  \begin{proof}
    对于任意的\(ε > 0\),取\(N = \ceil[\big]{2/ε^2 + 1}\),那么当\(n > N\)时都有
    \begin{gather*}
      n
      < \frac{n(n-1)}{2} ε^2
      < 1 + nε + \frac{n(n-1)}{2} ε^2 + \dots + ε^n
      = (1+ε)^n \\
      \shortintertext{即}
      n^{1/n} < 1 + ε.
      \qedhere
    \end{gather*}
  \end{proof}
  对选项~D的证明:
  \begin{proof}
    当\(n \ge 2\)时,有
    \begingroup
    \addtolength{\jot}{1ex}
    \begin{align*}
      \frac{(n!)^2}{n^n}
      &= \frac{n}{n} \cdot \frac{2(n-1)}{n} \cdot \frac{3(n-2)}{n}
      \dotsm \frac{2(n-1)}{n} \cdot \frac{n}{n} \\
      &=
        \begin{dcases}
          \brkt[\bigg]{2\paren[\bigg]{1-\frac1n} \cdot 3\paren[\bigg]{1-\frac2n}
            \dotsm \floor*{\frac{n}{2}} \paren[\bigg]{1-\frac{\floor{n/2}-1}{n}}}^2,
          & \text{\(n\)是偶数时,} \\
          \brkt[\bigg]{2\paren[\bigg]{1-\frac1n} \cdot 3\paren[\bigg]{1-\frac2n}
            \dotsm \floor*{\frac{n}{2}} \paren[\bigg]{1-\frac{\floor{n/2}-1}{n}}}^2
          \cdot \paren[\bigg]{\frac{\ceil{n/2}}{n}}^2,
          & \text{\(n\)是奇数时,}
        \end{dcases} \\
      &> \frac{\floor{n/2}!}{2^{\floor{n/2}}}.
    \end{align*}
    \endgroup
    易知\(\displaystyle \lim_{n\to\infty} \frac{(n!)^2}{n^n} = +\infty\).
  \end{proof}

  选项~A和~C可以直接套用夹逼定理.
  \fi

\item 以下说法中,错误的是\uline{\makebox[10em]{}}.
  \begin{itemize}
    \renewcommand{\labelitemi}{\faCircleThin}
    \addtolength{\itemsep}{1ex}
  \item 设\(x_1 > 0,\ y_1 > 0,\ x_{n+1} = \sqrt{x_n y_n}\,,\ y_{n+1} = (x_n+y_n)/2 \; (n \in \N^+)\).那么数列\(\Seq{x_n}\)与\(\Seq{y_n}\)收敛于同一个实数
    \ifshowsol
  \item[\faCircle]
    \else
  \item
    \fi
    若对所有的\(p \in \N^+\)都有\(\lim\limits_{n\to\infty} \abs*{a_{n+p}-a_n} = 0\),则数列\(\Seq{a_n}\)是柯西数列. % report
  \item 极限\(\displaystyle \lim_{n\to\infty} \sum_{k=1}^n \frac{\sin k}{2^k}\)存在
  \item 若\(\displaystyle \lim_{n\to\infty} \abs[\bigg]{\frac{a_{n+1}}{a_n}} = q < 1\),则\(\lim\limits_{n\to\infty} a_n = 0\)
  \end{itemize}

\item 下列说法中,错误的是\uline{\makebox[10em]{}}.
  \begin{itemize}
    \renewcommand{\labelitemi}{\faCircleThin}
    \addtolength{\itemsep}{.67ex}
  \item 若\(k\)是某一个正整数,则\(\lim\limits_{n\to\infty} x_n = a\)的充分必要条件是\(\lim\limits_{n\to\infty} x_{n+k} = a\)
    \ifshowsol
  \item[\faCircle]
    \else
  \item
    \fi
    若\(\displaystyle \lim_{n\to\infty} a_n = A,\ b_n = \frac{a_1 + a_2 + \dots + a_n}{n}\),则数列\(\Seq{b_n}\)不一定收敛.
  \item 若\(\lim\limits_{n\to\infty} a_n = a > 0,\ b_n = \sqrt[n]{a_1 a_2 \dotsm a_n}\)且所有的\(a_n \ne 0\),则\(\lim\limits_{n\to\infty} b_n = a\)
  \item 单调有界数列是柯西数列
  \end{itemize}

\item 下列说法中,正确的是\uline{\makebox[10em]{}}.
  \begin{itemize}
    \renewcommand{\labelitemi}{\faCircleThin}
    \ifshowsol
  \item[\faCircle]
    \else
  \item
    \fi
    单调递增数列要么收敛,要么是无穷大量
  \item 数列若不单调有界,则必不收敛
  \item 存在不收敛的柯西数列
  \item 收敛数列不一定有界
  \end{itemize}

\item 已知\(\displaystyle \lim_{n\to\infty} a_n = 1,\ \lim_{n\to\infty} b_n = 2\),求\(\displaystyle \lim_{n\to\infty} \frac{a_1 b_n + a_2 b_{n-1} + \dots + a_n b_1}{n}\).

  \ifshowsol
  可以模仿例~\ref{eg:limavg}的思路来证明这个极限的值是\(2\).
  \fi
\end{enumerate}
\fi

\section{Bolzano--Weierstrass定理与Cauchy收敛准则}

\begin{theorem}
  \label{thm:seqcvgsubseq}
  数列\(\Seq{a_n}\)收敛的充分必要条件是它的所有子列都收敛.

  \begin{remark}
    易证必要性;至于充分性,可以先证明所有子列都收敛于同一个数,然后通过反证法得到.
  \end{remark}
\end{theorem}

\begin{example*}
  设\(a_1 = 2,\ a_{n+1} = 2 + {1}/{a_n}\).试讨论数列\(\Seq{a_n}\)的敛散性. % https://mathworld.wolfram.com/ContinuedFraction.html

  \begin{remark}
    子列\(a_{2n}\)单调递减有下界,子列\(a_{2n-1}\)单调递增有上界.然后证明这两个子列的极限相等.这就说明原数列收敛.
  \end{remark}
\end{example*}

\begin{definition*}
  若闭区间\(\Seq{\brkt{a_n, b_n}}\)对于所有的\(n\)满足\(\brkt{a_{n+1}, b_{n+1}} ⊂ \brkt{a_n, a_n}\)且\(\lim\limits_{n\to\infty} (b_n - a_n) = 0\),则称\(\Seq{\brkt{a_n, b_n}}\)是一个区间套.
\end{definition*}

\begin{theorem*}[区间套定理]
  若\(\Seq{\brkt{a_n, b_n}}\)是一个区间套,则存在唯一的实数\(ξ\)使得\(ξ \in \brkt{a_n, b_n}\)对于所有的\(n\)都成立.

  \begin{remark}
    可以考虑使用单调有界收敛定理来证明此定理.
  \end{remark}
\end{theorem*}

\begin{example*}
  从区间套定理可以推出确界存在公理.

  % TODO: Complete the proof
  \begin{remark}
    可以通过二分法来构造区间套从而推出确界存在公理.这就说明确界存在公理、单调有界定理、区间套定理在逻辑上是等价的.
  \end{remark}
\end{example*}

\begin{theorem}[Bolzano--Weierstrass定理]
  \label{thm:bw}
  若数列\(\Seq{a_n}\)有界,则存在子列\(\Seq{a_{n_k}}\)收敛.

  % TODO: Complete the proof
  \begin{remark}
    还是可以通过二分法构造区间套来证明此定理的.
  \end{remark}
\end{theorem}

\begin{example}
  \label{eg:seqbndcmnidx}
  数列\(\Seq{a_n}\)和\(\Seq{b_n}\)都是有界的,可以找到一个共同的下标集\(\Set{n_k}\)使得子列\(\Seq{a_{n_k}}\)和\(\Seq{b_{n_k}}\)都收敛.

  \begin{proof}
    由上述定理可知,能找到一个下标集\(\Set{\bar n_k}\)使得子列\(\Seq{a_{\bar n_k}}\)收敛.再对有界子列\(\Seq{b_{\bar n_k}}\)使用一次这个定理,就能得到一个下标集\(\Set{n_k}\)使得子列\(\Seq{a_{n_k}}\)和\(\Seq{b_{n_k}}\)都收敛.
  \end{proof}
\end{example}

\begin{definition*}
  对于任意的\(ε > 0\)都存在正整数\(N\)使得当\(m > N,\ n > N\)时都有\(\abs*{a_n - a_m} < ε\).这时我们称数列\(\Seq{a_n}\)是一个柯西列(Cauchy sequence).
\end{definition*}

\begin{definition*}[柯西列的定价定义]
  \begin{math}
    \paren[\big]{\forall ε > 0}
    \paren[\big]{\exists N > 0}
    \paren[\big]{\forall n > N}
    \paren[\big]{\forall p > 0}
    \paren[\big]{\abs[\big]{a_{n+p} - a_n} < ε}.
  \end{math}
\end{definition*}

\begin{example*}
  证明数列\(\Seq{q^n}\)在\(\abs*{q} < 1\)条件下是柯西列.

  \begin{proof}
    对于任意的\(ε > 0\),取\(N = \maxb*{\ceil*{\ln ε /\!\ln\abs*q}, 1}\),那么当\(n > N\)时,对于所有的正整数\(p\)都有
    \begin{equation*}
      \abs*{q^{n+p} - q^n}
      = \abs*q^n \abs*{q^p - 1}
      < \abs*q^n
      < ε.
      \qedhere
    \end{equation*}
  \end{proof}
\end{example*}

\begin{example*}
  证明数列\(\Seq[\big]{\sum_{k=1}^n 1/k}\)不是柯西列.

  \begin{proof}
    前面我们已经证明了这个数列单调递增且为无穷大量.将这个数列记为\(\Seq{a_n}\).取\(ε = 1\),对于任意的正整数\(N\),先取\(n = N+1\),因为数列\(\Seq{a_n}\)单调递增且是无穷大量,所以一定能找到一个正整数\(p\)使得\(a_{n+p} > 2\,a_n \), 那么就有\(\abs[\big]{a_{n+p} - a_n} = a_{n+p} - a_n > a_n > 1\).
  \end{proof}
\end{example*}

% https://mathworld.wolfram.com/CauchyCriterion.html
% https://hsm.stackexchange.com/a/2861/13364
% https://en.wikipedia.org/wiki/Cauchy%27s_convergence_test
% https://personal.math.ubc.ca/~cass/courses/m220-00/cauchy.pdf
\begin{theorem}[柯西收敛准则]
  \label{thm:seqcvgcauchy}
  数列收敛的充分必要条件是它为柯西列.

  \begin{proof}
    易证必要性,用一次三角不等式即可.下面证明一下充分性.易从柯西列的定义推出柯西列一定是有界列.根据定理~\ref{thm:bw},有界列必有收敛子列, 将这个子列记为\(\Seq{a_{n_k}}\),其极限记为\(A\).那么对于任意的\(ε > 0\),都存在一个正整数\(K\)使得当\(k > K\)时都有\(\abs[\big]{a_{n_k} - A} < ε\),也存在一个正整数\(N_0\)使得当\(n > N_0,\ m > N_0\)时都有\(\abs*{a_n - a_m} < ε\).取\(N = 1 + \maxb*{K, N_0}\),当\(n > N\)时就有
    \begin{equation*}
      \abs[\Big]{a_n - A}
      \le \abs[\Big]{a_n - a_{n_N}} + \abs[\Big]{a_{n_N} - A}
      < 2ε.
      \qedhere
    \end{equation*}
  \end{proof}
\end{theorem}

\begin{example*}
  设\(b_n = \sum_{k=1}^n \abs*{a_{k+1} - a_k} \le C\).试证明数列\(\Seq{a_n}\)收敛.

  \begin{proof}
    数列\(\Seq{b_n}\)单调递增且有界,那么自然也就收敛,因而也是柯西列.所以对于任意的\(ε > 0\)都存在正整数\(N_0\)使得当\(n > N_0,\ p > 0\)时都有
    \begin{align*}
      ε
      &> \abs[\big]{b_{n+p} - b_{n}} \\
      &= \abs[\Bigg]{\,\sum_{k=1}^{n+p} \abs*{a_{k+1} - a_k} - \sum_{k=1}^n \abs*{a_{k+1} - a_k}\,} \\
      &= \abs[\Bigg]{{}\smashoperator[r]{\sum_{k=n+1}^{n+p}} \abs*{a_{k+1} - a_k}\,} = \smashoperator{\sum_{k=n+1}^{n+p}} \abs*{a_{k+1} - a_k} \\
      &\ge \abs[\Bigg]{{}\smashoperator[r]{\sum_{k=n+1}^{n+p}} \paren*{a_{k+1} - a_k}\,}
        = \abs[\big]{a_{n+p+1} - a_{n+1}}.
    \end{align*}
    取\(N = N_0 + 1\),当\(n > N,\ p > 0\)时就有\(\abs[\big]{a_{n+p} - a_n} < ε\).这就说明数列\(\Seq{a_n}\)是一个柯西列,自然也就收敛.
  \end{proof}
\end{example*}

\subpdfbookmark{思考}{B1.2.3.P}
\subsection*{思考}

\begin{enumerate}
\item 在闭区间套定理中,如果将闭区间改成开区间,结论是否成立?

  \ifshowsolp
  不成立,最终可能是一个空集.设\(a_n = 0,\ b_n = 1/n\),显然有\(\paren{a_{n+1}, b_{n+1}} \subset \paren{a_n, b_n}\)且\(\lim_{n\to\infty} \paren{b_n - a_n} = 0\).但是,
  \begin{equation*}
    \bigcap_{n=1}^\infty \paren{a_n, b_n} = \bigcap_{n=1}^\infty \paren[\Big]{0, \frac1n} = \emptyset.
  \end{equation*}
  \fi

\item 如何利用数列收敛与子列收敛的关系来证明一个数列非收敛?

  \ifshowsolp
  只要找到该数列的一个不收敛子列即可.
  \fi
\end{enumerate}

\ifshowex
\currentpdfbookmark{练习}{B1.2.3.E}
\subsection*{练习}

\begin{enumerate}
\item 若数列\(\Seq{a_n}\)满足条件\(\abs*{a_{n+1} - a_n} \le 1/2^n\),则\uline{\makebox[6em]{}}.
  \begin{itemize}
    \renewcommand{\labelitemi}{\faCircleThin}
  \item \(\lim_{n\to\infty} a_n = 0\)
  \item 数列\(\Seq{a_n}\)不一定收敛
    \ifshowsol
  \item[\faCircle]
    \else
  \item
    \fi
    \(\lim_{n\to\infty} a_n = A\)且\(\abs*{A - a_1} \le 1\)
  \item \(\lim_{n\to\infty} a_n = A\)且\(\abs*{A - a_1} > 1\)
  \end{itemize}

  \ifshowsol
  选项~A可以有反例\(a_n = 1\)来说明.选项~B可以由三角不等式推出这个数列是柯西列,进而收敛.选项~C可以先推出\(\abs*{a_n - a_1} \le 1 - 1/2^{n-1}\),然后用反证法说明\(\abs*{A - a_1} \ngtr 1\).
  \fi

\item 设\(\Seq*{\paren{a_n, b_n}}\)是一个开区间序列且满足:\enumparen{1}\(a_1 < a_2 < \dots < a_n < \dots < b_n < \dots < b_2 < b_1\) \enumparen{2}\(\lim_{n\to\infty} \paren{b_n - a_n} = 0\).那么就有\uline{\makebox[10em]{}}.
  \begin{itemize}
    \renewcommand{\labelitemi}{\faCircleThin}
    \ifshowsol
  \item[\faCircle]
    \else
  \item
    \fi
    存在唯一的实数\(ξ\)属于所有开区间\(\paren{a_n, b_n}\)且\(ξ = \lim_{n\to\infty} a_n\)
  \item 存在唯一的实数\(ξ\)属于所有开区间\(\paren{a_n, b_n}\)且\(ξ \ne \lim_{n\to\infty} a_n\)
  \item 至少存在两个不同的实数\(ξ\)和\(η\)属于所有开区间\(\paren{a_n, b_n}\)
  \item 不存在实数\(ξ\)属于所有开区间\(\paren{a_n, b_n}\)
  \end{itemize}

  \ifshowsol
  可以参考区间套定理的证明方式.
  \fi

\item 关于数列,下列四个结论中,正确的有\uline{\makebox[6em]{%
      \ifshowsol
      \enumparen{1}%
      \enumparen{2}
      \fi}}.
  \begin{enumerate}
    \renewcommand{\labelenumii}{\enumparen{\arabic{enumii}}}
  \item 单调的无界数列一定为无穷大量
  \item 无界数列存在无穷大量子列
  \item 数列收敛等价于有无穷多个子列收敛
  \item 发散数列存在无穷大量子列
  \end{enumerate}

\item 下面四个数列中,是柯西列的有\uline{\makebox[6em]{%
      \ifshowsol
      \enumparen{2}%
      \enumparen{3}
      \fi}}.
  \begin{enumerate}
    \renewcommand{\labelenumii}{\enumparen{\arabic{enumii}}}
  \item \(\Seq[\big]{1 + \frac12 + \dots + \frac1n}\)
  \item \(\Seq[\big]{1 + \frac1{2^2} + \dots + \frac1{n^2}}\)
  \item \(\Seq[\big]{\frac{n!}{n^n}}\)
  \item \(\Seq[\big]{\sin\frac{n^n}{4}}\)
  \end{enumerate}

\item 关于区间\(\paren{0,1}\)中所有的有理点排成的点列\(\Seq[\big]{\frac12, \frac13, \frac23, \frac14, \frac34, \frac15, \frac25, \frac35, \frac45, \dots}\),下列四个结论中,正确的有\uline{\makebox[6em]{%
      \ifshowsol
      \enumparen{1}
      \fi}}.
  \begin{enumerate}
    \renewcommand{\labelenumii}{\enumparen{\arabic{enumii}}}
  \item 对于任意的\(x \in [0,1]\)都存在该点列的一个子列收敛于\(x\)
  \item 不存在\(x \in [0,1]\)使得该点列的一个子列收敛于\(x\)
  \item 仅存在有限个\(x \in [0,1]\)使得该点列的一个子列收敛于\(x\)
  \item 至少存在有限个\(x \in [0,1]\)使得该点列的任何一个子列都不收敛于\(x\)
  \end{enumerate}

\item 下列说法,错误的是\uline{\makebox[6em]{}}.
  \begin{itemize}
    \renewcommand{\labelitemi}{\faCircleThin}
  \item 数列\(\Seq{a_n}\)单调,则\(\lim_{n\to\infty} a_n = A\)的充要条件是存在子列\(\Seq{a_{n_k}}\)满足\(\lim_{n\to\infty} a_{n_k} = A\)
    \ifshowsol
  \item[\faCircle]
    \else
  \item
    \fi
    若数列\(\Seq{a_n}\)不收敛,则必存在两个子列\(\Seq{a_{n_k}^{(1)}}\)和\(\Seq{a_{n_k}^{(2)}}\)分别收敛于两个不同的值
  \item 若数列\(\Seq{a_n}\)无界但非无穷大量,则必存在一个无穷大量子列和一个收敛子列
  \item 设\(S\)为非空有上界的实数集.若\(\sup S = A \notin S\),则存在单调递增数列\(\Seq{a_n} \subset S\)使得\(\lim_{n\to\infty} a_n = A\).
  \end{itemize}
\end{enumerate}
\fi

\section{函数极限的概念与性质}

\begin{definition*}
  \label{defn:limfunc}
  函数\(f\)在点\(x_0\)附近\footnote{在点\(x_0\)附近是指在这个点某个去心邻域上,也就是说在集合\(\paren{x_0 - δ, x_0 + δ} \setminus \Set{x_0}\)上,其中\(δ\)是某个正数.}有定义,\(A\)是某个常数.对于任意的\(ε > 0\),存在\(δ > 0\)使得当\(0 < \abs*{x-x_0} < δ\)时都有\(\abs*{\,f(x) - A} < ε\).这时,我们称\(A\)是函数\(f\)在\(x\)趋向于\(x_0\)时的极限,记作
  \begin{equation*}
    \setlength{\abovedisplayskip}{.5ex}
    \lim_{x \to x_0}\,f(x) = A.
  \end{equation*}
\end{definition*}

\begin{example*}
  证明\(\lim_{x\to1} x^2 = 1\).

  \begin{proof}
    对于任意的\(ε > 0\),取\(δ = \minb[\big]{\frac12, \frac23ε}\),当\(0 < \abs*{x - 1} < δ\)时就有
    \begin{equation*}
      \setlength{\abovedisplayskip}{.8ex}
      \abs[\big]{\,f(x) - 1}
      = \abs[\big]{x^2 - 1}
      = \abs[\big]{x-1} \abs[\big]{x+1}
      < δ \abs[\big]{x+1}
      < \frac32 δ
      \le ε.
      \rule[-2ex]{0ex}{0ex}
      \qedhere
    \end{equation*}
  \end{proof}
\end{example*}

\begin{example*}
  证明\(\lim_{x\to0} \cos x = 1\).

  \begin{proof}
    对于任意的\(ε > 0\),若使\(\abs[\big]{\cos x - 1} < ε\)成立,只需使\(1 - ε < \cos x\)成立即可,也就是\(1 - \cos x < ε\).考虑到
    \begin{equation*}
      1 - \cos x = 2 \sin^2 \frac{x}{2} \le \frac{x^2}{2},
    \end{equation*}
    这时只需取\(δ = \sqrt{2ε}\),就能使当\(0 < \abs{x} < δ\)时都有\(1 - \cos x < ε\).
  \end{proof}
\end{example*}

\begin{definition*}[单侧极限]
  对于任意的\(ε > 0\)都存在\(δ > 0\)使得当\(0 < x_0 - x < δ\)时都有\(\abs*{\,f(x) - A} < ε\).这时,我们称\(A\)是函数\(f\)在\(x\)从左侧趋向于\(x_0\)的极限,简称左极限,记作
  \begin{equation*}
    \lim_{x \to x_0^-} f(x) = A,
    \quad
    f_-(x_0) = A
    \txt{或}
    f(x_0-0) = A.
  \end{equation*}

  对于任意的\(ε > 0\)都存在\(δ > 0\)使得当\(0 < x - x_0 < δ\)时都有\(\abs*{\,f(x) - A} < ε\).这时,我们称\(A\)是函数\(f\)在\(x\)从右侧趋向于\(x_0\)的极限,简称右极限,记作
  \begin{equation*}
    \lim_{x \to x_0^+} f(x) = A,
    \quad
    f_+(x_0) = A
    \txt{或}
    f(x_0+0) = A.
  \end{equation*}
\end{definition*}

\begin{theorem}
  \label{thm:limfuncsided}
  函数极限存在的充分必要条件是它的左极限和右极限都存在且相等.

  \begin{proof}
    易证必要性.关于充分性,只需取\(δ = \minb{δ_1, δ_2}\)即可,其中\(δ_1\)和\(δ_2\)分别是左右极限所取的德尔塔.
  \end{proof}
\end{theorem}

\begin{example*}
  函数
  \begin{equation*}
    f(x) =
    \begin{cases}
      x+1, & x \ge 0, \\
      x-1, & x < 0
    \end{cases}
  \end{equation*}
  在\(x = 0\)处的极限不存在.
\end{example*}

\begin{example*}
  函数\(f(x) = \arctan\dfrac1x\)在\(x = 0\)处的极限不存在.\rule[-2ex]{0ex}{5.5ex}
\end{example*}

\begin{example*}
  函数\(f(x) = \dfrac{x}{\abs x + 1}\)在无穷处的极限不存在.\rule[-2ex]{0ex}{0ex}
\end{example*}

\begin{theorem*}[唯一性]
  函数的极限若存在则唯一.
\end{theorem*}

\begin{theorem*}[有界性]
  若函数\(f\)在\(x \to x_0\)时的极限存在,则其在\(x_0\)附近有界.
\end{theorem*}

\begin{theorem}[保号性]
  \label{thm:limfuncsgnprsv}
  若函数\(f\)在\(x \to x_0\)时收敛于某个非零数\(A\),则其在\(x_0\)附近拥有与\(A\)相同的正负号.
\end{theorem}

% https://gallica.bnf.fr/ark:/12148/bpt6k3116z/f688.item
% https://zbmath.org/46.0295.04
\begin{theorem}
  \label{thm:limequiv}
  函数\(f\)在\(x \to x_0\)时收敛于\(A\)的充分必要条件\footnote{这个条件又叫作极限的Heine定义.Sierpiński在1916年发现证明这两个定义等价,其实暗含了弱形式的选择公理.}是:对于任意的非\(x_0\)数列\(\Seq{x_n}\),当它收敛于\(x_0\)时,就一定有数列\(\Seq{\,f(x_n)}\)收敛于\(A\).

  \begin{proof}
    易证必要性.关于必要性,可以通过反证法证明.假设函数\(f\)在\(x \to x_0\)时不收敛于\(A\),则能构造一个数列\(\Seq{x_n}\)使得它收敛于\(x_0\)但是数列\(\Seq{\,f(x_n)}\)不收敛于\(A\).
  \end{proof}
  \begin{remark}
    实际上,对于单侧极限,也有类似的结论.只需把上面的非\(x_0\)数列\(\Seq{x_n}\)改成小于\(x_0\)的数列\(\Seq{x_n}\)(左侧极限)和大于\(x_0\)的数列\(\Seq{x_n}\)(右侧极限)即可.
  \end{remark}
\end{theorem}

\begin{example*}
  函数\(f\)在\(\brktparen{a, +\infty}\)上有定义,在\(\brkt{a, A}\)上有界,其中\(A\)是任意一个大于等于\(a\)的数,并且在趋向于正无穷时极限存在.证明函数\(f\)在\(\brktparen{a, +\infty}\)上有界.

  \begin{proof}
    记\(B\)为函数在趋向正无穷时的极限.那么对于任意的\(ε > 0\)都存在\(δ > 0\)使得当\(x > δ\)时都有\(\abs*{\,f(x) - B} < ε\),所以\(\abs[\big]{\,f(x)} < \abs[\big]{B} + ε\).因此,函数\(f\)在\(\paren{δ, +\infty}\)上有界.又因为函数\(f\)在\(\brkt{a, δ}\)上也有界,所以它在\(\brktparen{a, +\infty}\)上有界.
  \end{proof}
\end{example*}

\begin{example*}
  试判断极限\(\lim_{x\to0} \cos\frac1x\)存在与否.

  \begin{remark}
    不存在.下面给出两种证明.
  \end{remark}

  \begin{proof}
    对于任意的\(A \in \R\),都存在\(ε > 0\)使得当\(δ > 0\)时都能找到一个\(x\)满足\(0 < \abs{x} < δ\)且\(\abs[\big]{\cos\frac1x - A} \ge ε\).当\(A \notin \brkt{-1, 1}\)时,只需取\(ε = \abs{A} - 1\)即可.当\(A \in \paren{-1, 1}\)时,取\(ε = \frac{1-\abs{A}}{2}\)和\(x = \frac{1}{2\pi\paren*{\ceil{1/2\piδ}+1}}\)即可.当\(A \in \Set{-1, 1}\)时,取\(ε = 1/2\)和\(x = \frac{1}{\pi/2 + 2\pi\ceil*{1/2\piδ}}\)即可.
  \end{proof}

  \begin{proof}
    令\(x_n = \frac1{2n\pi},\ \bar x_n = \frac1{\pi/2 + 2n\pi}\).显然有非零数列\(\Seq{x_n}\)和数列\(\Seq{\bar x_n}\)都收敛于\(0\),但数列\(\Seq{\,f(x_n)}\)和\(\Seq{\,f(\bar x_n)}\)收敛于不同的值.
  \end{proof}
\end{example*}

\begin{example*}[Thomae函数]
  设\(x_0\)是一个实数.试证明函数
  \begin{equation*}
    f(x) =
    \begin{cases}
      \frac1q, & \text{\(x = \frac pq\),其中\(p \in \Z\)与\(q \in \N^+ \)互质,} \\
      1, & x = 0, \\
      0, & \otherwise
    \end{cases}
  \end{equation*}
  在\(x \to x_0\)时收敛于\(0\).

  \begin{remark}
    下面给出两种证明.
  \end{remark}

  \begin{proof}
    我们可以将此函数看成直尺上的刻度.可以先考察的此函数的一些性质,易知\(1\)和\(0\)分别是它的上下界,它是一个周期为\(1\)的函数.那么只需证明它在区间\(\paren{0,1}\)上的极限、在\(0\)处的右极限、在\(1\)处的左极限都是\(0\)即可.对于任意的\(ε > 0\),大于\(ε\)的刻度精度集合\(Q_ε^* = \Set[\big]{\,1/q \mid q \in \N^+ \tand 1/q \ge ε\,}\)显然是有限的.又因为由刻度精度\(1/p\)派生出来的刻度集合\(P_{1/q} = \Set[\big]{\,p/q \mid p \in \N^+ \tand p < q\,}\)也是有限的,那么刻度集合
    \begin{equation*}
      P_{Q_ε^*}
      = {}\smashoperator{\bigcup_{1/q \in Q_ε^*}} P_{1/q}
      = \Set[\big]{\,p/q \mid p,q \in \N^+,\ 1/q \ge ε,\  p < q\,}
    \end{equation*}
    也是有限的.当\(x_o \in \paren{0,1}\)时,令\(X = \paren[\big]{P_{Q_ε^*} \!\setminus \Set{x_0}} \cup \Set{0,1}\),取
    \begin{equation*}
      δ = \min\Set[\big]{\,\abs{x - x_0} \mid x \in X\,}
    \end{equation*}
    即可使当\(0 < \abs*{x - x_0} < δ\)时都有\(f(x) < ε\).同理,可证函数\(f\)在\(0\)处的右极限和在\(1\)处的左极限也是\(0\),从而它在\(\R\)上的所有极限都是\(0\).
  \end{proof}
  % TODO: Tidy up the proof
  \begin{proof}
    对于任意的收敛于\(x_0\)的非\(x_0\)数列\(\Seq{x_n}\),它一定能分割成有理数和无理数两个部分,这两个部分不可能都是有限的,也就是说这两个部分当中,至少有一个部分是无限的.若无理数的那个部分是无限的,将这个部分构成的子列记为\(\Seq{x_{n_k}}\).那么数列\(\Seq{\,f(x_{n_k})}\)自然收敛于\(0\).若有理数的那个部分是无限的,将这个部分构成的子列记为\(\Seq{\bar x_{n_k}}\).只要说明数列\(\Seq{\,f(\bar x_{n_k})}\)也收敛于\(0\),就能说明数列\(\Seq{\,f(x_n)}\)也收敛于\(0\),从而函数在\(x \to x_0\)时也收敛于\(0\).

    接下来,我们来证明数列\(\Seq{\,f(\bar x_{n_k})}\)确实收敛于\(0\).只需证明,对于任意的\(ε > 0\),只存在有限个项\(\bar x_{n_k}\)满足\(\,f(\bar x_{n_k}) \ge ε\)即可.因为取\(K = \max\Set[\big]{\,k \mid \,f(\bar x_{n_k}) \ge ε\,}\),就能使得当\(k > K\)时都有\(f(\bar x_{n_k}) < ε\).因为\(\bar x_{n_k}\)是有理数,必然能写成\(\bar x_{n_k} = p_k/q_k\)的形式.令
    \begin{equation*}
      P = \Set[\big]{\, p_k \mid \,f(p_k/q_k) \ge ε\,}, \quad
      Q = \Set[\big]{\, q_k \mid \,f(p_k/q_k) \ge ε\,}, \quad
      R = \Set[\big]{\, p_k/q_k \mid \,f(p_k/q_k) \ge ε\,}.
    \end{equation*}
    易证\(\card*Q \in \N\),从而\(\card*P \in \N\),所以\(\card*R \in \N\).若有无穷多项\(\bar x_{n_k}\)满足\(f(\bar x_{n_k}) \ge ε\),则对于任意的正整数\(K\)都存在一个\(k > K\)满足
    \begin{equation*}
      \abs[\big]{\bar x_{n_k} - x_0} \ge \min\Set[\big]{\,\abs{r - x_0} \mid r \in R \,} > 0.
    \end{equation*}
    这与子列\(\Seq[\big]{\bar x_{n_k}}\)收敛于\(x_0\)是矛盾的.因此只存在有限个项\(\bar x_{n_k}\)满足\(f(\bar x_{n_k}) \ge ε\).
  \end{proof}
\end{example*}

\subpdfbookmark{思考}{B1.2.4.P}
\subsection*{思考}

\begin{enumerate}
\item 用\(ε\)--\(N\)语言给出\(\lim_{x\to\infty} f(x) = A\)的定量描述.

  \ifshowsolp
  用形式语言的来定义,就是
  \begin{equation*}
    \paren[\big]{\forall ε > 0}
    \paren[\big]{\exists N > 0}
    \paren[\big]{\forall \abs{x} > N}
    \paren[\big]{\abs[\big]{\,f(x) - A} < ε}.
  \end{equation*}
  \fi

  % https://math.stackexchange.com/q/419909/147999
\item 思考函数极限与数列极限之间的关系,并尝试通过数列极限给出各种函数极限存在的柯西收敛原理.

  \ifshowsolp
  设函数\(f\)在点\(a\)附近有定义,当
  \begin{equation*}
    \paren[\big]{\forall ε > 0}
    \paren[\big]{\exists δ > 0}
    \paren[\big]{\forall \abs[\big]{x - a} < δ}
    \paren[\big]{\forall \abs[\big]{y - a} < δ}
    \paren[\big]{\abs[\big]{\,f(x) - f(y)} < ε}
  \end{equation*}
  时,我们称\(f\)在点\(a\)处柯西收敛.

  函数的柯西收敛原理无非是说:函数\(f\)在点\(a\)处极限存在的充分必要条件是它在点\(a\)处柯西收敛.

  \begin{proof}
    易证必要性.下面来证一下充分性.对于任意的\(ε > 0\)都存在\(δ > 0\)使得当\(\abs{x - a} < δ,\ \abs{y - a} < δ\)时都有\(\abs*{\,f(x) - f(y)} < ε\).这时,对于任何收敛于\(a\)的非\(a\)数列\(\Seq{x_n}\),自然存在一个正整数\(N\)使得:当\(n\)和\(m\)大于\(N\)时,都有\(\abs*{x_n - a} < δ\)和\(\abs*{x_m - a} < δ\).所以有\(\abs*{\,f(x_n) - f(x_m)} < ε\).这就说明数列\(\Seq{\,f(x_n)}\)是柯西列,从而收敛.与不同的数列\(\Seq{x_n}\)所对应的数列\(\Seq{\,f(x_n)}\)必然收敛于同一个数.因为假设收敛于不同的两个数,将不同的两个数列可以合并成一个新的数列\(\Seq{\bar x_n}\),从而推出\(\Seq{\,f(\bar x_n)}\)发散.这和前面的结论时矛盾的.因此,函数\(f\)在点\(a\)处的极限存在.
  \end{proof}
  \fi
\end{enumerate}

\ifshowex
\currentpdfbookmark{练习}{B1.2.4.E}
\subsection*{练习}

\begin{enumerate}
\item 若\(\lim_{x\to0} \,f(x) = 1\),则必定\uline{\makebox[6em]{}}.
  \begin{itemize}
    \renewcommand{\labelitemi}{\faCircleThin}
  \item \(f(0) = 1\)
  \item 函数\(f\)在原点没定义.
    \ifshowsol
  \item[\faCircle]
    \else
  \item
    \fi
    函数\(f\)在原点附近大于\(0\)
  \item 函数\(f\)在原点附近不等于\(1\)
  \end{itemize}

\item 函数\(f(x) = \abs x / x\)在原点处\uline{\makebox[6em]{}}.
  \begin{itemize}
    \renewcommand{\labelitemi}{\faCircleThin}
  \item 极限存在且为\(1\)
  \item 极限存在但不为\(1\)
    \ifshowsol
  \item[\faCircle]
    \else
  \item
    \fi
    极限不存在但在该点附近有界
  \item 极限不存在且在该点附近无界
  \end{itemize}

\item 函数
  \begin{equation*}
    f(x) =
    \begin{cases}
      \sin\frac1x, & x > 0, \\
      x \sin\frac1x, & x < 0
    \end{cases}
  \end{equation*}
  在原点处\uline{\makebox[6em]{}}.
  \begin{itemize}
    \renewcommand{\labelitemi}{\faCircleThin}
  \item 左右极限均存在且都为\(0\)
  \item 左右极限均不存在
    \ifshowsol
  \item[\faCircle]
    \else
  \item
    \fi
    左极限存在,但右极限不存在
  \item 左右极限都存在但不相同
  \end{itemize}

\item 函数
  \begin{equation*}
    f(x) =
    \begin{cases}
      2x, & x > 0, \\
      a \cos x + b \sin x, & x < 0
    \end{cases}
  \end{equation*}
  在原点处\uline{\makebox[6em]{}}.
  \begin{itemize}
    \renewcommand{\labelitemi}{\faCircleThin}
  \item 极限存在
  \item 极限不存在
  \item 当且仅当\(a = 0,\ b = 0\)时极限存在
    \ifshowsol
  \item[\faCircle]
    \else
  \item
    \fi
    当且仅当\(a = 0\)时极限存在
  \end{itemize}

\item 关于函数极限,下列结论中,正确的有\uline{\makebox[6em]{%
      \ifshowsol
      \enumparen{2}%
      \enumparen{4}
      \fi}}.
  \begin{enumerate}
    \renewcommand{\labelenumii}{\enumparen{\arabic{enumii}}}
  \item 若\(\lim_{x\to0} \,f(x^2) = A\),则\(\lim_{x\to0} \,f(x) = A\)
  \item 若\(\lim_{x\to0} \,f(x^3) = A\),则\(\lim_{x\to0} \,f(x) = A\)
  \item 若函数\(f\)是周期函数且\(\lim_{x\to0} \,f(x) = A\),则\(f(x) \equiv A\)
  \item 若函数\(f\)是周期函数且\(\lim_{x\to\infty} \,f(x) = A\),则\(f(x) \equiv A\)
  \end{enumerate}

\item 狄利克雷函数(定义~\ref{defn:dirichlet})\uline{\makebox[6em]{}}.
  \begin{itemize}
    \renewcommand{\labelitemi}{\faCircleThin}
  \item 在任意点处的极限都存在
    \ifshowsol
  \item[\faCircle]
    \else
  \item
    \fi
    在任意点处的极限都不存在
  \item 仅在有理点处的极限存在
  \item 仅在无理点处的极限存在
  \end{itemize}

\item 下列说法中,正确的是\uline{\makebox[10em]{}}.
  \begin{itemize}
    \renewcommand{\labelitemi}{\faCircleThin}
  \item
    \begin{math}
      \paren[\big]{\exists ε > 0}
      \paren[\big]{\forall δ > 0}
      \paren[\big]{\forall 0 < \abs{x} < δ}
      \paren[\big]{\abs[\big]{\,f(x) - A} < ε}
      \implies
      \lim_{x\to0} \,f(x) = A
    \end{math}
  \item
    \begin{math}
      \paren[\big]{\exists ε > 0}
      \paren[\big]{\exists δ > 0}
      \paren[\big]{\forall 0 < \abs{x} < δ}
      \paren[\big]{\abs[\big]{\,f(x) - A} < ε}
      \implies
      \lim_{x\to0} \,f(x) = A
    \end{math}
  \item
    \begin{math}
      \lim_{x\to0} \,f(x) = A
      \implies
      \paren[\big]{\forall ε > 0}
      \paren[\big]{\forall δ > 0}
      \paren[\big]{\forall 0 < \abs{x} < δ}
      \paren[\big]{\abs[\big]{\,f(x) - A} < ε}
    \end{math}
    \ifshowsol
  \item[\faCircle]
    \else
  \item
    \fi
    \begin{math}
      \paren[\big]{\forall ε > 0}
      \paren[\big]{\forall δ > 0}
      \paren[\big]{\forall 0 < \abs{x} < δ}
      \paren[\big]{\abs[\big]{\,f(x) - A} < ε}
      \implies
      \lim_{x\to0} \,f(x) = A
    \end{math}
  \end{itemize}
\end{enumerate}
\fi

\section{函数极限的运算}

% TODO: Expatiate on tending to infinity
\begin{theorem}[函数极限的四则运算]
  \label{thm:limfunc4ops}
  若\(\lim_{x \to x_0} \,f(x) = A,\ \lim_{x \to x_0} \,g(x) = B\),则
  \begin{enumerate}
    \renewcommand{\labelenumi}{\enumparen{\arabic{enumi}}}
  \item \(\displaystyle \lim_{x \to x_0} \paren*{\,f(x) \pm g(x)} = A \pm B\);
  \item \(\displaystyle \lim_{x \to x_0} \,f(x)\,g(x) = AB\);
  \item \(\displaystyle \lim_{x \to x_0} \tfrac{f(x)}{g(x)} = \tfrac AB\ (B \ne 0)\).
  \end{enumerate}

  \begin{proof}
    应用定理~\ref{thm:seq4ops}和定理~\ref{thm:limequiv}即可得证.
  \end{proof}
\end{theorem}

\begin{example*}
  求\(\!\lim\limits_{x\to+\infty} \dfrac{5a^x - 3b^x}{5a^x + 3b^x}\ (a > 0,\ b > 0)\).\rule{0ex}{4ex}

  \begin{remark}
    当\(a = b\)时,有\(\lim_{x\to+\infty} \frac{5a^x - 3b^x}{5a^x + 3b^x} = \lim_{x\to+\infty} \frac{5a^x - 3a^x}{5a^x + 3a^x} = \frac14\).当\(a > b\)时,有
    \begin{align*}
      \lim_{x\to+\infty} \frac{5a^x - 3b^x}{5a^x + 3b^x}
      &= \lim_{x\to+\infty} \frac{5 - 3(b^x\!/a^x)}{5 + 3(b^x\!/a^x)}
        = \lim_{x\to+\infty} \frac{5 - 3(b/a)^x}{5 + 3(b/a)^x}
      && \reason{变形} \\
      &= \frac{\lim_{x\to+\infty} \paren*{5 - 3(b/a)^x}}{\lim_{x\to+\infty} \paren*{5 + 3(b/a)^x}}
      && \reason{除法规则} \\
      &= \frac{\lim_{x\to+\infty} 5 - \lim_{x\to+\infty} 3(b/a)^x}{\lim_{x\to+\infty} 5 + \lim_{x\to+\infty} 3(b/a)^x}
      && \reason{加减法规则} \\
      &= \frac{5 - 3 \lim_{x\to+\infty} (b/a)^x}{5 + 3 \lim_{x\to+\infty} (b/a)^x}
      && \reason{乘法规则} \\
      &= 1.
    \end{align*}
    同理可知,当\(a < b\)时,有\(\lim_{x\to+\infty} \frac{5a^x - 3b^x}{5a^x + 3b^x} = -1\).所以,
    \begin{equation*}
      \lim_{x\to+\infty} \frac{5a^x - 3b^x}{5a^x + 3b^x} =
      \begin{cases}
        \dfrac14, & a = b, \\
        1, & a > b, \\
        -1, & a < b.
      \end{cases}
    \end{equation*}
  \end{remark}
\end{example*}

\begin{example*}
  求\(\displaystyle \lim_{x\to2} \frac{x^2 + x - 6}{\sqrt{x\mathstrut} - \sqrt{4-x\mathstrut}}\).

  \begin{remark}
    分母有理化后,得到
    \begin{equation*}
      \lim_{x\to2} \frac{x^2 + x - 6}{\sqrt{x\mathstrut} - \sqrt{4-x\mathstrut}}
      = \lim_{x\to2} \frac{%
        \paren[\big]{x^2 + x - 6}
        \paren[\big]{\sqrt{x\mathstrut} + \sqrt{4-x\mathstrut}}}{%
        x - 4 + x}
      = \lim_{x\to2} \frac{%
        \paren[\big]{x-2}
        \paren[\big]{x+3}
        \paren[\big]{\sqrt{x\mathstrut} + \sqrt{4-x\mathstrut}}}{%
        2(x-2)}
      = 5\sqrt2\,.
    \end{equation*}
  \end{remark}
\end{example*}

\begin{theorem}[复合函数的极限]
  \label{thm:limfunccomp}
  若函数\(f\)在点\(B\)处的极限是\(A\),函数\(g\)在点\(x_0\)处的极限是\(B\)且在点\(x_0\)附近不等于\(B\),则复合函数\(f \circ g\)在点\(x_0\)的极限是\(A\).

  \begin{proof}
    应用定理~\ref{thm:limequiv}即可得证.
  \end{proof}
\end{theorem}

\begin{example*}
  令\(f(u) = \abs*{\sgn x}\,,\ g(x) = x \Fn D(x)\).复合函数\(f \circ g\)在原点处的极限存在吗?

  \begin{remark}
    不存在.若“生吞活剥”地套用上述定理,就可能得出极限存在且为\(1\)的结论.实际上,
    \begin{equation*}
      (\,f \circ g)(x)
      = \abs[\Big]{\sgn\paren[\big]{x \Fn D(x)}} =
      \begin{cases}
        \Fn D(x), & x \ne 0, \\
        0, & x = 0.
      \end{cases}
    \end{equation*}
    函数的极限在某点处的极限和它在这点的取值无关,狄利克雷函数在实轴上处处极限不存在,因此该复合函数在原点自然也不存在.误用上述定理的原因是:函数\(g\)在原点附近总是能取到\(0\),由于无理数是稠密的.
  \end{remark}
\end{example*}

\begin{theorem}
  \label{thm:limfuncpowexp}
  在同一个极限过程中,若函数\(f\)的极限是正数\(A\)且函数\(g\!\)的极限是\(B\),则函数\(\,f^g\!\)的极限是\(A^B\).

  \begin{proof}
    运用两次定理~\ref{thm:limfunccomp}和一次定理~\ref{thm:limfunc4ops},有
    \begin{align*}
      \lim_{x \to x_0\!} \,f(x)^{g(x)}
      &= {}\smashoperator{\lim_{x \to x_0\!}} \expb[\Big]{\ln \,f(x)^{g(x)}}
        = {}\smashoperator{\lim_{x \to x_0\!}} \expb[\Big]{g(x) \ln \,f(x)}
        = \expb[\Big]{\lim_{x \to x_0\!} g(x) \ln \,f(x)} \\
      &= \expb[\Big]{\lim_{x \to x_0\!} g(x) \smashoperator[r]{\lim_{x \to x_0\!}} \ln \,f(x)}
        = \expb[\Big]{B \ln {}\smashoperator{\lim_{x \to x_0\!}} \,f(x)} \\
      &= \expb[\big]{B \ln A} = \expb[\big]{\ln A^B} = A^B.
    \end{align*}
    注意,在使用定理~\ref{thm:limfunccomp}的时候,用到了指数函数和对数函数都是连续函数(见\hyperref[ch:cont]{下章})的事实.
  \end{proof}
\end{theorem}

\begin{theorem}[函数极限的夹逼定理]
  \label{thm:funcsqueeze}
  若函数\(f,\ g,\ h\)满足:
  \begin{enumerate}[topsep=0ex,itemsep=0ex]
    \renewcommand{\labelenumi}{\enumparen{\arabic{enumi}}}
  \item 在点\(x_0\)附近有\(f \le g \le h\),
  \item \(\lim\limits_{x \to x_0} \,f(x) = \lim\limits_{x \to x_0} h(x) = A\),
  \end{enumerate}
  则\(\lim\limits_{x \to x_0} g(x) = A\).

  \begin{proof}
    应用定理~\ref{thm:limequiv}即可得证.
  \end{proof}
\end{theorem}

\begin{example*}
  求\(\lim\limits_{x\to0} x \floor[\bigg]{\dfrac1x}\).

  \begin{remark}
    实际上,当\(x > -1\)时,有
    \begin{equation*}
      \Fn H(-x) + \frac{\floor{1/x}}{\floor{1/x}+1} \Fn H(x)
      \le
      x \floor[\bigg]{\dfrac1x}
      \le
      \frac{\floor{1/x}}{\floor{1/x}+1} \Fn H(-x) + \Fn H(x),
    \end{equation*}
    其中\(\Fn H\)为单位阶跃函数(定义~\ref{defn:heaviside}).所以,\(\lim\limits_{x\to0} x \floor[\bigg]{\dfrac1x} = 1\).
  \end{remark}
\end{example*}

\begin{theorem*}
  \(\lim\limits_{x\to0} \dfrac{\sin x}{x} = 1\).

  \begin{proof}
    根据三角函数在几何上的意义,当\(0 < x < \pi/2\)时,有
    \begin{equation*}
      \sin x < x < \tan x
      \iff
      1 < \frac{x}{\sin x} < \sec x
      \iff
      \cos x < \frac{\sin x}{x} < 1.
    \end{equation*}
    使用一次夹逼定理,得
    \begin{equation*}
      \lim_{x\to0^+\negthickspace} \frac{\sin x}{x} = 1,
    \end{equation*}
    又因为这个函数是偶函数,所以\(\lim\limits_{x\to0} \dfrac{\sin x}{x} = 1\).
  \end{proof}
\end{theorem*}

\begin{example*}
  常见的极限结论:
  \begin{equation*}
    \lim_{x\to0} \frac{\tan x}{x} = 1,
    \quad
    \lim_{x\to0} \frac{1 - \cos x}{x^2} = \frac12,
    \quad
    \lim_{x\to0} \frac{\arcsin x}{x} = 1,
    \quad
    \lim_{x\to0} \frac{\arctan x}{x} = 1.
  \end{equation*}
\end{example*}

\begin{example*}
  求\(\lim\limits_{x\to0} \dfrac{\sin ax}{\sin bx}\ (b \ne 0)\).

  \begin{remark}
    当\(a \ne 0\)时,有
    \begin{equation*}
      \lim_{x\to0} \frac{\sin ax}{\sin bx}
      = \lim_{x\to0} \frac{\sin ax}{ax} \frac{ax}{bx} \frac{bx}{\sin bx}
      = \frac ab.
    \end{equation*}
    当\(a = 0\)时,有\(\lim\limits_{x\to0} \dfrac{\sin ax}{\sin bx} = 0 = \dfrac ab\).所以,无论\(a\)的取值,都有\(\lim\limits_{x\to0} \dfrac{\sin ax}{\sin bx} = \dfrac ab\).
  \end{remark}
\end{example*}

\begin{example*}
  求\(\lim\limits_{x\to1} \dfrac{\sinp{1-x}}{\sqrt x - 1}\).\rule{0ex}{3.5ex}

  \begin{equation*}
    \lim_{x\to1} \frac{\sinp{1-x}}{\sqrt x - 1}
    = \lim_{x\to1} \frac{-\paren{\sqrt x + 1} \sinp{x-1}}{x - 1}
    = -2.
  \end{equation*}
\end{example*}

\begin{example*}
  求\(\;\smashoperator[l]{\lim\limits_{x\to\pi/2}} \dfrac{\cos x}{\pi/2 - x}\).\rule{0ex}{3.5ex}

  \begin{equation*}
    \lim_{x\to\pi/2} \frac{\cos x}{\pi/2 - x}
    = - \smashoperator{\lim_{x\to\pi/2}} \frac{\cosp{x-\pi/2+\pi/2}}{x - \pi/2}
    = - \smashoperator{\lim_{x\to\pi/2}} \frac{-\sinp{x-\pi/2}}{x-\pi/2}
    = 1.
  \end{equation*}
\end{example*}

\begin{theorem*}
  \(\!\lim\limits_{x\to\infty} \paren[\bigg]{1 + \dfrac1x}^x = e\).\rule{0ex}{3.5ex}

  \begin{proof}
    应用定理~\ref{thm:seqe}和定理~\ref{thm:funcsqueeze}即可得证.
  \end{proof}
\end{theorem*}

\begin{example*}
  求\(\lim\limits_{x\to0} \dfrac{\ln(1+x)}{x} = 1\).\rule{0ex}{3.5ex}

  \begin{remark}
    令\(t = 1/x\),有
    \begin{equation*}
      \lim_{x\to0} \frac{\ln(1+x)}{x}
      = \lim_{x\to0} \ln(1+x)^{1/x}
      = \lim_{t\to\infty} \lnp[\bigg]{1 + \frac1t}^t
      = \ln e
      = 1.
    \end{equation*}
  \end{remark}
\end{example*}

\begin{example*}
  求\(\lim\limits_{x\to0} \dfrac{e^x-1}{x}\).

  \begin{remark}
    令\(t = e^x - 1\),有
    \begin{equation*}
      \lim_{x\to0} \frac{e^x-1}{x}
      = \lim_{t\to0} \frac{t}{\ln(1+t)}
      = 1.
    \end{equation*}
  \end{remark}
\end{example*}

\begin{example*}
  求\(\lim\limits_{x\to0} \dfrac{a^x-1}{x}\ (a > 0)\).

  \begin{remark}
    当\(a \ne 1\)时,有
    \begin{equation*}
      \lim_{x\to0} \frac{a^x-1}{x}
      = \lim_{x\to0} \frac{e^{x \ln a}-1}{x \ln a} \ln a
      = \ln a.
    \end{equation*}
    当\(a = 1\)时,有
    \begin{equation*}
      \lim_{x\to0} \frac{a^x-1}{x}
      = \lim_{x\to0} \frac{0}{x}
      = 0 = \ln a.
    \end{equation*}
    所以,\(\lim\limits_{x\to0} \dfrac{a^x-1}{x} = \ln a\).
  \end{remark}
\end{example*}

\begin{example*}
  求\(\lim\limits_{x\to\infty} \paren[\bigg]{\dfrac{x+5}{x+2}}^{\mathrlap{x+3}}\).

  \begin{remark}
    稍作变形,有
    \begin{align*}
      \lim_{x\to\infty} \paren[\bigg]{\frac{x+5}{x+2}}^{x+3}\negthickspace
      &= \lim_{x\to\infty} \paren[\bigg]{1 + \frac{3}{x+2}}^{x+2} \paren[\bigg]{%
        1 + \frac{3}{x+2}} \\
      &= \lim_{x\to\infty} \paren[\bigg]{1 + \frac{1}{(x+2)/3}}^{(x+2)/3\cdot3} \paren[\bigg]{%
        1 + \frac{3}{x+2}} \\
      &= e^3.
    \end{align*}
  \end{remark}
\end{example*}

\begin{example*}
  求\(\lim\limits_{x\to0} (\cos x)^{1/\!\sin^2 x}\).

  \begin{remark}
    稍作变形,有
    \begin{align*}
      \lim_{x\to0} (\cos x)^{1/\!\sin^2 x}
      &= \lim_{x\to0} \expb[\Bigg]{\frac{\ln \cos x}{\sin^2 x}} \\
      &= \lim_{x\to0} \expb[\Bigg]{
        \frac{\lnp{1 - 2 \sin^2 \frac x2}}{
        \paren{2 \sin\frac x2 \cos\frac x2}^2}} \\
      &= \lim_{x\to0} \expb[\Bigg]{
        \frac{\lnp{1 - 2 \sin^2 \frac x2}}{-2 \sin^2 \frac x2}
        \cdot \frac{1}{-2 \cos^2 \frac x2}} \\
      &= e^{-1/2} = \frac1{\!\sqrt e\,}.
    \end{align*}
  \end{remark}
\end{example*}

\subpdfbookmark{思考}{B1.2.5.P}
\subsection*{思考}

复合函数求极限的条件是什么?若不满足该条件,会出现什么问题?

\ifshowsolp
\pskip
要在自变量趋向的点附近存在一个去心邻域,使得内部函数在此邻域上不等于所趋向的极限值.若不满足,则可能所求的极限不存在,或者等于其他值.
\fi

\ifshowex
\currentpdfbookmark{练习}{B1.2.5.E}
\subsection*{练习}

\begin{enumerate}
\item 函数
  \begin{equation*}
    f(x) = \frac{2+e^{1/x}}{1+e^{4/x}} + \frac{\sin x}{\abs x}
  \end{equation*}
  在原点处\uline{\makebox[6em]{}}.
  \begin{itemize}
    \renewcommand{\labelitemi}{\faCircleThin}
    \ifshowsol
  \item[\faCircle]
    \else
  \item
    \fi
    极限存在
  \item 左极限存在,右极限不存在
  \item 左极限不存在,右极限存在
  \item 左右极限都存在但不相等
  \end{itemize}

  \ifshowsol
  实际上,有
  \begin{gather*}
    \lim_{\,x\to0^+} \,f(x)
    = {}\smashoperator[l]{\lim_{\,x\to0^+}} \paren[\bigg]{\frac{2/e^{1/x} + 1}{1/e^{1/x} + e^{3/x}} + \frac{\sin x}{x}}
    = 1
    \siand
    \lim_{\,x\to0^-} \,f(x)
    = {}\smashoperator[l]{\lim_{\,x\to0^-}} \paren[\bigg]{\frac{2+e^{1/x}}{1+e^{4/x}} - \frac{\sin x}{x}}
    = 1.
  \end{gather*}
  所以函数\(f\)在原点处的极限是\(1\).
  \fi

\item 若\(\lim\limits_{x \to x_0} \,f(x)\)存在且\(\lim\limits_{x \to x_0} g(x)\)不存在,则\uline{\makebox[10em]{}}.
  \begin{itemize}[itemsep=1ex]
    \renewcommand{\labelitemi}{\faCircleThin}
  \item \(\lim\limits_{x \to x_0} \,f(x)\,g(x)\)和\(\lim\limits_{x \to x_0} \dfrac{g(x)}{f(x)}\)一定都不存在
  \item \(\lim\limits_{x \to x_0} \,f(x)\,g(x)\)和\(\lim\limits_{x \to x_0} \dfrac{g(x)}{f(x)}\)一定都存在
  \item 在\(\lim\limits_{x \to x_0} \,f(x)\,g(x)\)和\(\lim\limits_{x \to x_0} \dfrac{g(x)}{f(x)}\)中恰有一个存在
    \ifshowsol
  \item[\faCircle]
    \else
  \item
    \fi
    \(\lim\limits_{x \to x_0} \paren*{\,f(x)+g(x)}\)和\(\lim\limits_{x \to x_0} \paren*{\,f(x)-g(x)}\)一定都不存在
  \end{itemize}

  \ifshowsol
  令\(f(x) = x\)和\(g(x) = 1/x\),可以证伪选项~A和~B.在此基础上,令\(g(x) = 1/x^2\),可以证伪选项~C.实际上,在题干的条件下,\(\lim\limits_{x \to x_0} \paren*{\,f(x)+g(x)}\)、\(\lim\limits_{x \to x_0} \paren*{\,f(x)-g(x)}\)和\(\lim\limits_{x \to x_0} \dfrac{g(x)}{f(x)}\)一定都不存在.当\(\lim\limits_{x \to x_0} \,f(x) \ne 0\)时,\(\lim\limits_{x \to x_0} \,f(x)\,g(x)\)一定不存在.当\(\lim\limits_{x \to x_0} \,f(x) = 0\)时,\(\lim\limits_{x \to x_0} \,f(x)\,g(x)\)可能存在也可能不存在.
  \fi

\item 若\(\lim_{x\to\infty} \paren[\bigg]{\dfrac{x^2+1}{x+1} - ax -b} = 0\),则\(a\)和\(b\)的值分别为\uline{\makebox[3em]{%
      \ifshowsol
      \(-1\)
      \fi}}和\uline{\makebox[3em]{%
      \ifshowsol
      \(1\)
      \fi}}.

\item 求\(\lim\limits_{n\to\infty} \sin^2 \pi\sqrt{n^2+1}\).

  \ifshowsol
  稍作变形,有
  \begin{align*}
    \lim_{n \to \infty} \sin^2 \pi\sqrt{n^2+1}
    &= \lim_{n \to \infty} \sin^2 \brce[\big]{\paren[\big]{\sqrt{n^2+1} - \sqrt{n^2} + \sqrt{n^2}} \pi} \\
    &= \lim_{n \to \infty}
      \bigl\lbrace
      \sin\brkt[\big]{\paren[\big]{\sqrt{n^2+1} - \sqrt{n^2}}\pi}
      \cos\pi\sqrt{n^2} \\
    &\hphantom{= \lim_{n \to \infty} \lbrace} +
      \cos\brkt[\big]{\paren[\big]{\sqrt{n^2+1} - \sqrt{n^2}}\pi}
      \sin\pi\sqrt{n^2}
      \bigr\rbrace^2 \\
    &= \lim_{n \to \infty}
      \sin^2\brkt[\big]{\paren[\big]{\sqrt{n^2+1} - \sqrt{n^2}}\pi}
      \cos^2\pi\sqrt{n^2} \\
    &= \lim_{n \to \infty} \sin^2\brkt[\big]{\paren[\big]{\sqrt{n^2+1} - \sqrt{n^2}}\pi} \\
    &= 0.
  \end{align*}
  \fi

\item 若\(\!\lim\limits_{x\to-\infty} \paren[\big]{\sqrt{x^2 - x + 1} - ax - b} = 0\),则\(a\)和\(b\)的值分别为\uline{\makebox[3em]{%
      \ifshowsol
      \(-1\)
      \fi}}和\uline{\makebox[3em]{%
      \ifshowsol
      \(1/2\)
      \fi}}.

\item 求\(\lim\limits_{x\to0} \paren{2 \sin x + \cos x}^{1/x}\).

  \ifshowsol
  稍作变形,有
  \begin{align*}
    \lim_{x\to0} \paren{2 \sin x + \cos x}^{1/x}
    &= \lim_{x\to0} \expb[\bigg]{
      \frac{\lnp{1 + 2 \sin x + \cos x - 1}}{2 \sin x + \cos x - 1}
      \cdot
      \frac{2 \sin x + \cos x - 1}{x}} \\
    &= e^2.
  \end{align*}
  \fi

\item 若函数
  \begin{equation*}
    f(x) =
    \begin{dcases}
      \frac{\sin x}{x}, & x \ne 0, \\
      0, & x = 0
    \end{dcases}
    \txt{且}
    g(t) = t \sin\frac1t,
    \qquad
  \end{equation*}
  则\(\lim\limits_{t\to0} \,f\,\paren*{g(t)}\)\uline{\makebox[6em]{}}.
  \begin{itemize}
    \renewcommand{\labelitemi}{\faCircleThin}
  \item 等于\(1\)
  \item 等于\(0\)
  \item 等于\(-1\)
    \ifshowsol
  \item[\faCircle]
    \else
  \item
    \fi
    不存在
  \end{itemize}

  \ifshowsol
  若令\(t_n = 1/n\pi\),则\(\lim\limits_{n\to\infty} \,f\,\paren*{g(t_n)} = 0\).若令\(t_n = 1/(n\pi+1)\),则\(\lim\limits_{n\to\infty} \,f\,\paren*{g(t_n)} = 1\).根据定理~\ref{thm:limequiv},所以\(\lim\limits_{t\to0} \,f\,\paren*{g(t)}\)不存在.
  \fi

\item 求\(\!\lim\limits_{\,x\to0^+\!} \paren{\cos\sqrt x}^{\pi/x}\).

  \ifshowsol
  略作变形,有
  \begin{equation*}
    \begin{split}
      \lim_{\,x\to0^+\!} \paren{\cos\sqrt x}^{\pi/x}
      &= \lim_{\,x\to0^+\!} \expb[\bigg]{\frac{\pi \ln\cos\sqrt x}{x}}
      = \lim_{\,x\to0^+\!} \expb[\bigg]{
        \frac{\pi \lnp{1 + \cos\sqrt x - 1}}{\cos\sqrt x - 1}
        \cdot
        \frac{\cos\sqrt x - 1}{x}} \\
      &= e^{-\pi/2}.
    \end{split}
  \end{equation*}j
  \fi

\item 求\(\lim\limits_{x\to1} \dfrac{\sinp{x-1}}{\sqrt x - 1}\).
  \ifshowsol
  \begin{equation*}
    \lim_{x\to1} \frac{\sinp{x-1}}{\sqrt x - 1}
    = \lim_{x\to1} \frac{\sinp{x-1}}{x-1} \cdot \frac{x-1}{\sqrt x - 1}
    = 2\,.
  \end{equation*}
  \fi
\end{enumerate}
\fi

\section{无穷小量及其(阶的)比较}

\begin{definition*}
  若\(\!\lim\limits_{\,x \to x_0\!} \,f(x) = 0\),则称函数\(f\)在\(x \to x_0\)时是一个无穷小量,记作\(f(x) = \littleo(1)\ (x \to x_0)\).
\end{definition*}

\begin{definition*}
  对于任意的\(M > 0\)都存在\(δ > 0\)使得当\(0 < \abs*{x - x_0} < δ\)时都有\(\abs*{\,f(x)} > M\).这时,我们称函数\(f\)在\(x \to x_0\)时是一个无穷大量,记作\(\!\lim\limits_{\,x \to x_0\!} \,f(x) = \infty\).
\end{definition*}

\begin{theorem*}
  无穷大量的倒数是无穷小量.
\end{theorem*}

\begin{theorem*}
  非零无穷小量的倒数是无穷大量.
\end{theorem*}

\begin{theorem*}
  无穷小量与有界变量的乘积还是无穷小量.
\end{theorem*}

\begin{theorem*}
  函数\(f\)极限是\(A\),当且仅当它可以写成\(A\)与一个无穷小量之和的形式.
\end{theorem*}

% https://portal.tpu.ru/SHARED/k/KONVAL/Textbooks/Tab1/Konev-Limits_of_Sequences_and_Functions_Textbook.pdf
\begin{definition*}
  在同一个极限过程中,函数\(\,f\mkern2mu\)和\(\mkern1mu g\)都是无穷小量.
  \begin{enumerate}[topsep=3pt,itemsep=0ex]
    \renewcommand{\labelenumi}{\enumparen{\arabic{enumi}}}
  \item 若\(f/g\)收敛于某个非零常数,则称\(\,f\mkern2mu\)和\(\mkern2mu g\)是\emph{同阶无穷小量};特别地,若此常数是\(1\),则称\(\,f\,\)和\(g\)是\emph{等价无穷小量},记作\(f \sim g\);
  \item 若\(\,f/g\)也是无穷小量,则称\(\,f\mkern1mu\)是\(\mkern1mu g\)的\emph{高阶无穷小量}\footnote{关于小o和大O关系,最常见的定义,除了还要求\(g\)是一个正函数之外,其实并不关心\(\mkern2mu f\mkern2mu\)和\(\mkern2mu g\)是否收敛或者为无穷大量,只要它们的商满足条件即可.若\(\mkern2mu f\mkern1mu\)是\(g\)的高阶无穷小量,则一定满足\(\,f = \littleo(\abs{g})\);反之不然\,.这里的小o和大O记号,其实表达的是一个函数的集合,更严谨的用法应该是\(\,f \in \littleop g\)和\(\,f \in \bigOp g\).},称\(\mkern2mu g\)是\(\,f\mkern2mu\)的\emph{低阶无穷小量},记作\(f= \littleo(g)\);特别地,若\(\,f/g^n\)收敛于某个非零常数,则称\(\,f\mkern2mu\)是\(\mkern2mu g\)的\(n\)阶无穷小量.
  \end{enumerate}

  \begin{remark}
    我们简称函数\(x-x_0\)在点\(x_0\)处和函数\(1/x\)在无穷处的\(k\)阶无穷小量为\(k\)阶无穷小量.也就是说,函数\(f\)在点\(x_0\)处是\(k\)阶无穷小量,当且仅当\(0 < \abs[\Big]{\lim\limits_{\,x \to x_0\!} \frac{f(x)}{(x-x_0)^k}} < +\infty\);函数\(f\)在无穷处是\(k\)阶无穷小量,当且仅当\(0 < \abs[\Big]{\lim\limits_{x\to\infty} x^k\,f(x)} < +\infty\).
  \end{remark}
\end{definition*}

\begin{definition*}
  在同一个极限过程中,函数\(\,f\mkern2mu\)和\(\mkern2mu g\)都是无穷小量.若\(\,f/g\)有界,则称\(\,f\mkern2mu\)和\(\mkern2mu g\)有大O关系\footnote{同上.},记作\(\,f = \bigO(g)\).
\end{definition*}

\begin{example*}
  证明\(\sqrt[\leftroot{-2}\uproot{2}k]{1+x} - 1 \sim \dfrac xk\ (x \to 0)\).

  \begin{proof}
    利用公式\(a^n-b^n = (a-b)(a^{n-1} + a^{n-2}b + \dots + b^{n-1})\),稍作变形,有
    \begin{align*}
      \lim_{x\to0} \frac{\sqrt[\leftroot{-2}\uproot{2}k]{1+x} - 1}{x/k}
      &= \lim_{x\to0} \frac{k(1+x-1)}{
        x \sum_{j=0}^{k-1} (1+x)^{(k-1-j)/k}}
      && \reason{分子分母同乘以\(\Sigma\)} \\
      &= k\big/\!\lim_{x\to0} \sum_{j=0}^{k-1} (1+x)^{(k-1-j)/k}
      && \reason{除法法则} \\
      &= k\bigg/\!\sum_{j=0}^{k-1} \lim_{x\to0} (1+x)^{(k-1-j)/k}
      && \reason{加法法则} \\
      &= k/k = 1.
      && \reason{复合法则}
         \qedhere
    \end{align*}
  \end{proof}
\end{example*}

\begin{theorem*}
  若\(f \sim α,\ g \sim β \),则\(\lim \,f/g = \lim α/β\).

  \begin{proof}
    \begin{equation*}
      \lim \frac{\,f}{g}
      = \lim \frac{\,f}{α} \cdot \frac{α}{β} \cdot \frac{β}{g}
      = \lim \frac{α}{β}
      \qedhere
    \end{equation*}
  \end{proof}
\end{theorem*}

\begin{example*}
  求\(\lim\limits_{x\to0} \dfrac{\tan x - \sin x}{x^3}\).\rule[-2ex]{0ex}{0ex}

  \begin{remark}
    稍作变形,有
    \begin{equation*}
      \lim_{x\to0} \frac{\tan x - \sin x}{x^3}
      = \lim_{x\to0} \frac{1}{\cos x} \cdot \frac{1 - \cos x}{x^2} \cdot \frac{\sin x}{x}
      = \frac12.
    \end{equation*}
  \end{remark}
\end{example*}

\begin{example*}
  求\(\lim\limits_{x\to0} \dfrac{e^{1-\cos x}-1}{x^2}\).\rule[-2ex]{0ex}{3.5ex}

  \begin{remark}
    使用等价无穷小量替换和复合法则,有
    \begin{equation*}
      \lim_{x\to0} \frac{e^{1-\cos x}-1}{x^2}
      = \lim_{x\to0} \frac{e^{1-\cos x}-1}{2(1-\cos x)}
      = \frac12.
    \end{equation*}
  \end{remark}
\end{example*}

\begin{example*}
  求\(\lim\limits_{x\to0} \dfrac{\ln(1+ax^m)}{1-\cos(1-\cos x)}\ (a \ne 0, m > 0)\).\rule[-2ex]{0ex}{3.5ex}

  \begin{remark}
    使用等价无穷小量替换和复合法则,有
    \begin{equation*}
      \lim_{x\to0} \frac{\ln(1+ax^m)}{1-\cos(1-\cos x)}
      = \lim_{x\to0} \frac{ax^m}{(1-\cos x)^2\!/2}
      = \lim_{x\to0} \frac{ax^m}{x^4\!/8}
      = \lim_{x\to0} 8\,ax^{m-4} =
      \begin{cases}
        0, & m > 4, \\
        8\,a, & m = 4, \\
        \infty, & 0 < m < 4.
      \end{cases}
    \end{equation*}
  \end{remark}
\end{example*}

\subpdfbookmark{思考}{B1.2.6.P}
\subsection*{思考}

\begin{enumerate}
\item 两个无穷小量之和是不是无穷小量?两个无穷大量之和是不是无穷大量?

  \ifshowsolp
  是.不是.前者可由极限的加法法则得出.后者可以构造反例,函数\(f(x) = x,\ g(x) = -x\).这里的\(\,f\mkern1mu\)和\(\mkern1mu g\)在无穷处都是无穷大量,但是\(f+g\)是常函数.
  \fi

\item 任意两个无穷小量是否都可以比阶?试举例说明.

  \ifshowsolp
  不一定.例如,函数\(x \sin\frac1x\)和\(x\)在原点处都是无穷小量,但是它们的商发散.
  \fi
\end{enumerate}

\ifshowex
\currentpdfbookmark{练习}{B1.2.6.E}
\subsection*{练习}

\begin{enumerate}
\item 下列说法中,正确的是\uline{\makebox[6em]{}}.
  \begin{itemize}
    \renewcommand{\labelitemi}{\faCircleThin}
    \ifshowsol
  \item[\faCircle]
    \else
  \item
    \fi
    无穷小量与无穷大量之和为无穷大量
  \item 无穷小量与无穷大量之积为无穷大量
  \item 无穷小量与无穷大量之差为无穷小量
  \item 无穷小量与无穷大量之积为无穷小量
  \end{itemize}

  \ifshowsol
  令\(f(x) = \sin x,\ g(x) = \frac1x\),则在原点处\(\,f\mkern2mu\)和\(\mkern1mu g\)分别是无穷小量和无穷大量.但是\(\,f \cdot g\)即不是无穷小量,也不是无穷大量.所以选项~B和~D都错了.设\(\,f\mkern2mu\)和\(\mkern2mu g\)在点\(a\)处分别是无穷小量和无穷大量.对于任意的\(M > 0\)都存在一个去心邻域使得当\(x\)在此邻域上时都有
  \begin{gather*}
    \abs*{\,f(x)} < M
    \txt{且}
    2\,M < \abs*{g(x)} \\
    \shortintertext{即}
    \abs*{\,f(x) \pm g(x)}
    \ge \abs[\Big]{\abs*{g(x)} - \abs*{\,f(x)}}
    \ge \abs*{g(x)} - \abs*{\,f(x)}
    > M.
  \end{gather*}
  所以选项~A对了而选项~C错了.
  \fi

\item 当\(x \to 0\)时,下列函数中不是无穷小量的是\uline{\makebox[6em]{}}.
  \begin{itemize}
    \renewcommand{\labelitemi}{\faCircleThin}
  \item \(\sin(\tan x^2)\)
  \item \(x \cos\frac1x\)
    \ifshowsol
  \item[\faCircle]
    \else
  \item
    \fi
    \(\sin\paren{\cos x}\)
  \item \(\lnp{\sin x + 1}\)
  \end{itemize}

\item 当\(x \to 0^+\)时,下列函数中不是无穷大量的是\uline{\makebox[6em]{}}.
  \begin{itemize}
    \renewcommand{\labelitemi}{\faCircleThin}
  \item \(x - \ln x\)
  \item \(x + \ln x\)
    \ifshowsol
  \item[\faCircle]
    \else
  \item
    \fi
    \(x \ln x\)
  \item \(\dfrac{\ln x}{x}\)\rule{0ex}{3.5ex}
  \end{itemize}

  \ifshowsol
  对于选项~C,令\(t = -\ln x\),实际上有
  \begin{equation*}
    \lim_{\,x \to 0^+\!} \!x \ln x
    = - \smashoperator{\lim_{\,t \to +\infty\!}} te^{-t}
    = 0.
  \end{equation*}
  \fi

\item 当\(x \to 0^+\)时,下列无穷小量按照其阶由低到高排列正确的是\uline{\makebox[10em]{}}.
  \begin{itemize}
    \renewcommand{\labelitemi}{\faCircleThin}
  \item \(\sin x^2,\ \sin(\tan x),\ e^{x^3}-1,\ \lnp{1+\sqrt x}\)
  \item \(\lnp{1+\sqrt x},\ \sin x^2,\ \sin(\tan x),\ e^{x^3}-1\)
  \item \(\sin(\tan x),\ \lnp{1+\sqrt x},\ \sin x^2,\ e^{x^3}-1\)
    \ifshowsol
  \item[\faCircle]
    \else
  \item
    \fi
    \(\lnp{1+\sqrt x},\ \sin(\tan x),\ \sin x^2,\ e^{x^3}-1\)
  \end{itemize}

  \ifshowsol
  实际上,有
  \begin{equation*}
    \lnp{1+\sqrt x} \sim \sqrt x, \quad
    \sin(\tan x) \sim x, \quad
    \sin x^2 \sim x^2, \quad
    e^{x^3}-1 \sim x^3.
  \end{equation*}
  \fi

\item 当\(n \to \infty\)时,下列无穷大量按照其阶由低到高排列正确的是\uline{\makebox[6em]{}}.
  \begin{itemize}
    \renewcommand{\labelitemi}{\faCircleThin}
    \ifshowsol
  \item[\faCircle]
    \else
  \item
    \fi
    \(\sqrt n,\ n^2,\ e^n,\ n!,\ n^n\)
  \item \(n^2,\ e^n,\ n!,\ \sqrt n,\ n^n\)
  \item \(\sqrt n,\ n^2,\ n!,\ e^n,\ n^n\)
  \item \(\sqrt n,\ n^2,\ n!,\ n^n,\ e^n\)
  \end{itemize}

  \ifshowsol
  关于\(e^n = \littleop{n!}\),参见例~\ref{eg:factexp}.下面证明一下,对于给定的数\(a > 1\)和正整数\(k\),都有\(n^k = \littleop{a^n}\).

  \begin{proof}
    令\(α = a - 1\),则
    \begin{gather*}
      a^n = (1+α)^n = \sum_{j=0}^n \binom nj α^j > \binom{n}{k+1} α^{k+1}
      = α^{k+1} n^{k+1} + \bigOp{n^k}, \\
      \shortintertext{所以有}
      0 < \frac{n^k}{a^n} < \frac{n^k}{α^{k+1} n^{k+1} + \bigOp{n^k}}
      = \frac{1}{α^{k+1} n + \bigOp{1}}.
      \qedhere
    \end{gather*}
  \end{proof}
  \fi

\item 当\(x \to 0^+\)时,与\(\sqrt x\)等价的无穷小量是\uline{\makebox[6em]{}}.
  \begin{itemize}
    \renewcommand{\labelitemi}{\faCircleThin}
  \item \(1 - e^{\sqrt x}\)
    \ifshowsol
  \item[\faCircle]
    \else
  \item
    \fi
    \(\ln \dfrac{1+x}{1-\sqrt x}\)\rule[-2ex]{0ex}{5ex}
  \item \(\sqrt{1 + \sqrt x} - 1\)
  \item \(1 - \cos\sqrt x\)
  \end{itemize}

  \ifshowsol
  实际上,有
  \begin{equation*}
    1 - e^{\sqrt x} \sim -\sqrt x, \quad
    \ln \dfrac{1+x}{1-\sqrt x} \sim \sqrt x, \quad
    \sqrt{1 + \sqrt x} - 1 \sim \frac{\sqrt x}{2}, \quad
    1 - \cos\sqrt x \sim \frac x2.
  \end{equation*}
  \fi

\item 当\(x \to 0\)时,有\(\paren{1-ax^2}^{1/4} - 1 \sim x \sin x\).求\(a\)的值.

  \ifshowsol
  因为\(x \sin x \sim x^2\)和\(\paren{1-ax^2}^{1/4} - 1 \sim -ax^2/4\),所以有\(a = 4\).
  \fi
\end{enumerate}
\fi
