\documentclass{article}

\usepackage{amsmath, amsthm}
\usepackage{CJKutf8, cancel}
\usepackage[a4paper, hmargin=1.25in, vmargin=1in]{geometry}

\setlength{\parskip}{12pt}

\newcommand*{\intertexti}[1]{\intertext{\indent#1}}
\newcommand*{\reason}[1]{\langle #1 \rangle}
\newcommand*{\abs}[1]{\left\lvert #1 \right\rvert}
\newcommand*{\ex}[2]{\textbf{例#1:}#2}
\newcommand*{\ds}[1]{\( \displaystyle #1 \)}
\newcommand*{\exds}[2]{\ex{#1}\ds{#2}}
\newcommand*{\paren}[1]{\ensuremath{\left(#1\right)}}
\newcommand*{\brkt}[1]{\ensuremath{\left[#1\right]}}

\begin{document}
\begin{CJK*}{UTF8}{gkai}
\subsection*{\S 2. 换元积分法}
\subsubsection*{一. 第一换元法} \ 

\vspace{-3em}
设$ \int f(u) \, dx = F(u) + C $且$\varphi(x) \in C^1$,
则$ \int f(\varphi(x)) \varphi'(x) \, dx = F(\varphi(x)) +C $.

\textbf{证明:}$ (\text{左边})' = f(\varphi(x)) \varphi'(x) = (\text{右边})' $. \qed

在实际使用中, 换元法体现了莱布尼茨微分记号的优越性. 将
\begin{align*}
	\varphi'(x) \, dx
		& = d\varphi(x) \\
	\intertexti{代入原式, 得}
	\int f(\varphi(x)) \varphi'(x) \, dx
		& = \int f(\varphi(x)) \, d\varphi(x)
	\intertexti{用$ u $代替$ \varphi(x) $}
		& = \int f(u) \, du \\
		& = F(u) + C \\
		& = F(\varphi(x)) + C.
\end{align*}

\exds{1}{ \int x \sin(x^2) \, dx. }

将$x \, dx = \frac{1}{2} \, dx$代入得
\begin{align*}
	\int x \sin(x^2) \, dx
		&= \frac{1}{2} \int \sin(x^2) \, d(x^2) \\
	\intertexti{用$u$代替$x^2$}
		&= \frac{1}{2} \int \sin u \, du \\
		&= -\frac{1}{2} \cos(u) + C \\
		&= -\frac{1}{2} \cos(x^2) + C.
\end{align*}

\exds{2}{ \int \cot x \, dx. }
\begin{align*}
	\text{原式}
		&= \int \frac{\cos x}{\sin x} \, dx
			&& \reason{ \cot x = \frac{\cos x}{\sin x} }\\
		&= \int \frac{1}{\sin x} \, d(\sin x)
			&& \reason{ d\sin x = \cos x \, dx } \\
		&= \int \frac{du}{u}
			&& \reason{ u = \sin x } \\
		&= \ln \abs{u} + C
			&& \reason{ d\ln \abs{u} = \frac{1}{u} \, du } \\
		&= \ln \abs{\sin x} + C.
			&& \reason{ u = \sin x }
\end{align*}

\exds{3}{ \int \frac{dx}{a^2 + x^2} \qquad (a \ne 0). }
\begin{align*}
	\text{原式} &= \frac{1}{a^2} \int \frac{dx}{1+\paren{\frac{x}{a}}^2}
		&& \reason{ \text{因为$a \ne 0$, 提取因子$\frac{1}{a^2}$} } \\
	&= \frac{1}{a^2} \int \frac{d(au)}{1+u^2}
		&& \reason{ u = \frac{x}{a} } \\
	&= \frac{1}{a} \int \frac{du}{1+u^2}
		&& \reason{ d(au) = a \, du } \\
	&= \frac{1}{a}\arctan u + C
		&& \reason{ d\arctan u = \frac{1}{1+u^2} \, du } \\
	&= \frac{1}{a}\arctan \frac{x}{a} + C.
		&& \reason{ u = \frac{x}{a} }
\end{align*}

\exds{4}{ \int \frac{dx}{\sqrt{a^2 - x^2}} \qquad (a > 0). }
\begin{align*}
	\text{原式} &= \frac{1}{a} \int \frac{dx}{\sqrt{1-\paren{\frac{x}{a}}^2}}
		&& \reason{ \text{因为$a > 0$, 提取因子$\frac{1}{a}$} } \\
	&= \frac{1}{a} \int \frac{d(au)}{\sqrt{1-u^2}}
		&& \reason{ u = \frac{x}{a} } \\
	&= \int \frac{du}{\sqrt{1-u^2}}
		&& \reason{ d(au) = a \, du } \\
	&= \arcsin u + C
		&& \reason{ d\arcsin u = \frac{1}{\sqrt{1-u^2}} \, du } \\
	&= \arcsin \frac{x}{a} + C.
		&& \reason{ u = \frac{x}{a} }
\end{align*}

\subsection*{二. 第二换元法} \ 

\vspace{-3em}
设$ f(x) $为连续函数, $ x = \varphi(t) $连续可导且有反函数, 则
\[ \int f(x) \, dx = \int f(\varphi(t)) \varphi'(t) \, dt. \]

若右边的原函数可求得, 记\ds{ G(t) = \int f(\varphi(t)) \varphi'(t) \, dt }, 则
\[ \int f(x) \, dx = G(\varphi^{-1}(x)) + C. \]

\exds{1}{ \int \sqrt{a^2 - x^2} \, dx. }
\begin{align*}
	\intertexti{假设$ a > 0 $, 则$ -a \le x \le a$}
	\text{原式} &= \int a(\sqrt{1-\sin^2 t}) \, d(a\sin t)
		&& \reason{x = a \sin t, \  t \in [-\frac{\pi}{2}, \frac{\pi}{2}] } \\
	&= a^2 \int \cos^2 t \, dt
		&& \reason{ \cos t = \sqrt{1-\sin^2 t},\ d(a\sin t) = a\cos t \, dt } \\
	&= a^2 \int \frac{1+\cos 2t}{2} \, dt
		&& \reason{ \cos^2 t = \frac{1+\cos 2t}{2} } \\
	&= \frac{a^2}{2} t + \frac{\sin 2t}{4} + C
		&& \reason{ d\frac{t}{2} = \frac{1}{2} \, dt,\ d\frac{\sin 2t}{4} = \frac{\cos 2t}{2} \, dt } \\
	&= \frac{a^2}{2} \arcsin \frac{x}{a} + \frac{ \sin(2\arcsin\frac{x}{a})}{2} + C.
		&& \reason{ t = \arcsin\frac{x}{a} }
\end{align*}

\exds{2}{ \int \frac{dx}{\sqrt{x^2 - a^2}}. }

假设$a>0$, 则$x>a$或$x<-a$
\begin{align*}
	\text{原式} &= \int \frac{dx}{a \sqrt{\sec^2 t - 1}}
		&& \reason{ x = a\sec t, \  t \in (0, \frac{\pi}{2}) } \\
	&= \int \frac{ \bcancel{a} \sec t \,\cancel{\tan t} }{\bcancel{a} \,\cancel{\tan t}} \, dt
		&& \reason{ d(a\sec t) = a \sec t \tan t \, dt } \\
	&= \int \frac{dt}{\cos t}
		&& \reason{ \sec t = \frac{1}{\cos t} } \\
	&= \int \frac{\cos t \, dt}{\cos^2 t}
		&& \reason{ \text{分子分母同乘以$\cos t$} } \\
	&= \int \frac{d\sin t}{1 - \sin^2 t}
		&& \reason{ d\sin t = \cos t \, dt, \  \cos^2 t = 1 - \sin^2 t} \\
	&= \int \frac{du}{1-u^2}
		&& \reason{ u = \sin t } \\
	&= \frac{1}{2}\paren{\int \frac{du}{1-u} + \int \frac{du}{1+u} }
		&& \reason{ \frac{1}{1-u^2} = \frac{1}{2}\paren{\frac{1}{1-u} + \frac{1}{1+u}} } \\
	&= \frac{1}{2}(\ln \abs{1+u} - \ln \abs{1-u} + C)
		&& \reason{ d\ln \abs{1+u} = \frac{1}{1+u},\ d\ln \abs{1-u} = -\frac{1}{1-u} } \\
	&= \frac{1}{2} \ln \abs{\frac{1+\sin t}{1 - \sin t}} + C
		&& \reason{ \text{对数的性质},\ u = \sin t } \\
	&= \frac{1}{2} \ln \abs{ \frac{1 + \sqrt{1-\paren{\frac{a}{x}}^2}}
								{1 - \sqrt{1-\paren{\frac{a}{x}}^2}} } + C
		&& \reason{ \sin^2 t = 1 - \paren{\frac{a}{x}}^2} \\
	&= \frac{1}{2} \ln \abs{ \frac{\paren{1 + \sqrt{1-\paren{\frac{a}{x}}^2}}^2}
								{\paren{\frac{a}{x}}^2} } +C
		&& \reason{\text{分母有理化}} \\
	&= \ln \paren{x + \sqrt{x^2-a^2}} + C.
		&& \reason{\text{对数的性质}}
\end{align*}

\exds{3}{ \int \frac{dx}{x^2 \sqrt{x^2+1}}. }
\begin{align*}
	\text{原式} &= \int \frac{d\tan t}{\tan^2 t \sqrt{\tan^2 t + 1}}
		&& \reason{\text{$x = \tan t$,
						因为$x \ne 0$,
						所以$t \in (-\frac{\pi}{2},\frac{\pi}{2})\setminus\{0\}$}} \\
	&= \int \frac{\sec^{\cancel{2}} t \, dt}{\tan^2 t \,\cancel{\sec t}}
		&& \reason{ \sec^2 t - \tan^2 t = 1 } \\
	&= \int \frac{\cos t \, dt}{\sin^2 t}
		&& \reason{ \sec t = \frac{1}{\cos t},\  \tan t = \frac{\sin t}{\cos t} } \\
	&= \int \frac{d\sin t}{\sin^2 t}
		&& \reason{ d\sin t = \cos t \, dt } \\
	&= -\frac{1}{\sin t} + C
		&& \reason{ \int u^p \, du = \frac{u^{p+1}}{p+1} + C } \\
	&= -\frac{\sqrt{1+x^2}}{x} + C.
		&& \frac{1}{\sin t} = \frac{\sqrt{1+x^2}}{x}
\end{align*}

\end{CJK*}
\end{document}