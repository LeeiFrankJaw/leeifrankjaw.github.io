\chapter{导数应用}

% https://ncatlab.org/nlab/show/mean+value+theorem
\section{微分中值定理}

\begin{definition*}
  \label{defn:extrema}
  存在点\(x_0\)的一个\(\delta\)邻域\(V_\delta(x_0)\)使得当\(x \in V_\delta(x_0)\)时都有\(f(x) \le f(x_0)\).这时,我们把点\(x_0\)称为函数\(f\)的一个极大值点,把\(f(x_0)\)称为函数\(f\)的一个极大值.极小值点和极小值也有类似的定义.

  极大值和极小值统称为极值,极大值点和极小值点统称为极值点.
\end{definition*}


\begin{definition*}
  \label{defn:stationary}
  若函数\(f\)在点\(x_0\)处的导数为零,则称点\(x_0\)为函数\(f\)的一个驻点.
\end{definition*}

% https://digitalcommons.ursinus.edu/triumphs_calculus/11/
\begin{theorem}[费马驻点定理]
  \label{thm:fermat}
  函数在极值点处若可导则导数为零(可导的极值点必为驻点).
\end{theorem}

% https://doi.org/10.4169/000298910x476086
% https://doi.org/10.2307/4145046
% https://en.wikipedia.org/wiki/Darboux's_theorem_(analysis)
\begin{corollary}[达布定理]
  \label{cor:darboux}
  导函数满足介值性质.
\end{corollary}

\begin{corollary}[罗尔定理]
  \label{cor:rolle}
  若函数\(f\)在闭区间\(\brkt{a,b}\)上连续,在开区间\(\paren{a,b}\)上可导且\(\,f(a) = f(b)\),则存在~\(x_0 \in \paren{a,b}\)满足\(\,f'(x_0) = 0\).
\end{corollary}

\begin{theorem*}[广义罗尔定理]
  \label{thm:genrolle}
  若函数\(f\)在广义开区间\(\paren{a,b}\)上可导且极限\(\!\lim\limits_{x\to a^+\!} \,f(x) = \lim\limits_{x\to b^-\!} \,f(x)\)存在,则存在\(x_0 \in \paren{a,b}\)满足\(\,f'(x_0) = 0\).
\end{theorem*}

\begin{example*}
  证明方程\(x^5+x-1 = 0\)只有一个实根.

  \begin{proof}
    先证明存在实根.令\(\,f(x) = x^5+x-1\).因为\(\,f(0)\,f(1) = -1 < 0\),所以由定理~\ref{thm:bolzano}可知函数在区间\(\paren{a,b}\)上存在零点.再证至多存在一个实根.假设有两个实根\(x_1\)和\(x_2\),即\(\,f(x_1) = f(x_2) = 0\).由推论~\ref{cor:rolle}可知,存在\(x_0 \in \paren{x_1, x_2}\)满足\(\,f'(x_0) = 0\).这是不可能的,因为对于任意的\(x \in \R\)都有~\(\,f'(x) = 5x^4 + 1 > 0\).
  \end{proof}
\end{example*}

\begin{theorem}
  \label{thm:polymaxnzero}
  \(n\)次多项式函数至多能有\(n\)个零点.

  \begin{proof}
    令\(n\)次多项式\(P_n(x) = \!\sum\limits_{k=0}^n c_k x^k\).假设\(P_n\)有\(n+1\)个零点,由推论~\ref{cor:rolle}可知,\(P_n'\)至少有\(n\)个零点,\(P_n^{(2)}\)至少有\(n-1\)个零点,依此类推,\(P_n^{(n)}\)至少有一个零点,但是又有\(P_n^{(n)}(x) = n!\,c_n \ne 0\).
  \end{proof}
\end{theorem}

% https://en.wikipedia.org/wiki/Rodrigues%27_formula
% https://en.wikipedia.org/wiki/Legendre_polynomials
\begin{example*}[勒让德多项式]
  证明多项式
  \begin{equation*}
    P_n(x) = \frac1{2^nn!} \ddn xn \paren{x^2-1}^n
  \end{equation*}
  在区间\(\brkt{-1,1}\)上恰有\(n\)个零点.

  \begin{proof}
    多项式\(\paren{x^2-1}^n\)显然恰有\(2\)个零点,分别为\(-1\)和\(1\).由推论~\ref{cor:rolle}可知,多项式
    \begin{equation*}
      \ddx \paren{x^2-1}^n = n(x^2-1)^{n-1} \paren{2x}
    \end{equation*}
    在区间\(\paren{-1,1}\)上存在一个零点.又因为\(-1\)和\(1\)还是它的两个零点,所以它至少有\(3\)个零点.同理,对于任意的\(0 \le k < n\),都有
    \begin{equation*}
      \ddn xk (x^2-1)^n = \frac{n!}{(n-k)!}(x^2-1)^{n-k} (2x)^k + \sum_j c_j (x^2-1)^{a_j} x^{b_j},
    \end{equation*}
    其中\(a_j > n-k\).因此,\(-1\)和\(1\)都是它们的零点.又由推论~\ref{cor:rolle}可知,多项式
    \begin{equation*}
      \ddn xk (x^2-1)^n
    \end{equation*}
    在区间\(\paren{-1,1}\)上至少有\(k\)个零点,再加上\(-1\)和\(1\),一共至少有\(k+2\)个零点.所以多项式
    \begin{equation*}
      \ddn x{n-1} (x^2-1)^n
    \end{equation*}
    至少有\(n+1\)个零点.再使用一次推论~\ref{cor:rolle},就能得到多项式
    \begin{equation*}
      \ddn xn (x^2-1)^n
    \end{equation*}
    至少有\(n\)个零点.又因为定理~\ref{thm:polymaxnzero},所以它恰有\(n\)个零点.因此,勒让德多项式\(P_n\)也恰有\(n\)个零点.
  \end{proof}
\end{example*}

\begin{example*}
  设函数\(f\)在闭区间\(\brkt{a,b}\)上二阶可导且\(\,f(a) = f(b) = 0\).请证明
  \begin{enumerate}
    \renewcommand{\labelenumi}{\enumparen{\arabic{enumi}}}
  \item 存在\(x_0 \in \paren{a,b}\)满足\(\,f(x_0) + \,f'(x_0) = 0\);
  \item 若\(\,f'(a)\,f'(b) > 0\),则存在\(x_1 \in \paren{a,b}\)满足\(\,f\dpr(x_1) = 0\);
  \item 若\(\,f'(a)\,f'(b) > 0\),则存在\(x_2 \in \paren{a,b}\)满足\(\,f\dpr(x_2) + 2\,f'(x_2) + \,f(x_2) = 0\).
  \end{enumerate}

  \begin{proof}
    构造函数\(\Fn F(x) = e^x\,f(x)\).自然有\(\Fn F(a) = \Fn F(b) = 0\).根据推论~\ref{cor:rolle},存在\(x_0 \in \paren{a,b}\)使得
    \begin{equation*}
      F'(x_0) = e^{x_0}\paren[\big]{\,f(x_0) + f'(x_0)} = 0.
    \end{equation*}
    因为\(e^{x_0}\)自然不会等于零,所以有\(\,f(x_0) + \,f'(x_0) = 0\).命题\enumparen{1}得证.

    不失一般地,假设\(\,f'(a)\)和\(\,f'(b)\)都是正数.因为\(\,f(a) = \,f(b) = 0 \),所以根据推论~\ref{cor:rolle}可知,存在~\(x_3 \in \paren{a,b}\)满足\(\,f'(x_3) = 0\).又因为函数\(\,f'\!\)是导函数,所以由推论~\ref{cor:darboux}可知,函数\(\,f'\!\)满足介值性质.从而存在\(x_4 \in \paren{a, x_3}\)和\(x_5 \in \paren{x_3, b}\)满足
    \begin{equation*}
      0 < f'(x_4) = f'(x_5) < \minb[\big]{\,f'(a), \,f'(b)}.
    \end{equation*}
    再使用一次推论~\ref{cor:rolle}即可证明命题\enumparen{2}.

    对函数\(\Fn F\)使用证明命题\enumparen{2}同样的方法,即可证明命题\enumparen{3}.
  \end{proof}
\end{example*}

\begin{theorem}[拉格朗日中值定理]
  \label{thm:lmvt}
  设\(a < b\).若函数\(f\)在闭区间\(\brkt{a,b}\)上连续且在开区间\(\paren{a,b}\)上可导,则存在\(x_0 \in \paren{a,b}\)满足
  \begin{equation*}
    f'(x_0) = \frac{\,f(b) - \,f(a)}{b-a}.
  \end{equation*}
\end{theorem}

\begin{corollary}
  \label{cor:lmvtconst}
  函数为常函数的充分必要条件是它的导函数为恒零函数.
\end{corollary}

\begin{corollary}
  \label{cor:lmvtmono}
  函数在开区间上单调递增的充分条件是它的导函数在此区间上恒为正;函数在开区间上单调递减的充分条件是它的导函数在此区间上恒为负.
\end{corollary}

\begin{corollary*}
  设\(a < b\),函数\(f\)在区间\(\brktparen{a,b}\)上连续,在区间\(\paren{a,b}\)上可导.若极限\(\!\lim\limits_{x\to a^+} \,f'(x) = A\)存在,则函数\(f\)在点\(a\)处的右导数存在且\(\,f_+'(a) = A\).
\end{corollary*}

\begin{corollary*}
  若函数\(f\)在闭区间\(\brkt{a,b}\)上可导,则导函数\(\,f'\!\)在此区间上不存在第一类间断点.
\end{corollary*}

\begin{example*}
  证明不等式
  \begin{equation*}
    \frac{x}{1+x} < \lnp{1+x} < x
  \end{equation*}
  对于任意的\(x > 0\)都成立.

  \begin{proof}
    只考虑\(x > 0\)的情况.构造函数\(\Fn F(x) = x - \lnp{1+x}\).显然\(\Fn F(\Fn0) = \Fn0\)且
    \begin{equation*}
      \Fn F'(x) = 1 - \frac{1}{1+x} = \frac{x}{1+x} > 0.
    \end{equation*}
    由定理~\ref{thm:lmvt}可知,对于任意的\(x > 0\)都存在\(x_0 \in \paren{0,x}\)满足
    \begin{equation*}
      \Fn F(x) = \Fn F(x) - \Fn F(\Fn0) = x \Fn F'(x_0) > 0,
    \end{equation*}
    即\(\lnp{1+x} < x\).类似地,构造函数\(\Fn G(x) = \lnp{1+x} - \frac{x}{1+x}\).显然\(\Fn G(\Fn0) = \Fn0\)且
    \begin{equation*}
      \Fn G'(x) = \frac{x}{(1+x)^2} > 0.
    \end{equation*}
    由定理~\ref{thm:lmvt}可知,对于任意的\(x > 0\)都存在\(x_0 \in \paren{0,x}\)满足
    \begin{equation*}
      \Fn G(x) = \Fn G(x) - \Fn G(\Fn0) = x \Fn G'(x_0) > 0,
    \end{equation*}
    即\(\frac{x}{1+x} < \lnp{1+x}\).
  \end{proof}
  % TODO: Add proof from lecture
\end{example*}

\begin{example*}
  设函数\(f\)在区间\(\parenbrkt{-\infty,0}\)上可导且实数\(A > 0\).请证明:若\(\negthickspace\lim\limits_{x\to-\infty} f'(x) = A\),则
  \begin{math}
    \negthickspace\lim\limits_{x\to-\infty} f(x) = -\infty.
  \end{math}

  \begin{proof}
    由极限定义可知,存在\(M < 0\)使得当\(x < M\)时都有\(\,f'(x) > A/2\).又由定理~\ref{thm:lmvt}可知,对于任意的\(x < M\)都存在\(x_0 \in \paren{x, M}\)满足
    \begin{equation*}
      f(x) = \,f(M) + (x - M)\,f'(x_0) < \,f(M) + (x - M) \frac A2 = \,f(M) - \frac{MA}{2} + \frac A2 x.
    \end{equation*}
    当\(x \to -\infty\)时,上述不等式右侧是负无穷大量,那么左侧自然也是负无穷大量.
  \end{proof}
\end{example*}

\begin{theorem}[柯西中值定理]
  \label{thm:cmvt}
  若函数\(\,f\mkern2mu\)和\(\mkern1mu g\)在闭区间\(\brkt{a,b}\)上连续,在开区间\(\paren{a,b}\)上可导,则存在\(t_0 \in \paren{a,b}\)满足
  \begin{equation*}
    \paren[\big]{\,f(b)-\,f(a)}\,g'(t_0) = \paren[\big]{g(b)-g(a)}\,f'(t_0).
  \end{equation*}
  特别地,当\(\,f'\!\)和\(g'\!\)在开区间\(\paren{a,b}\)上不同时为零且点\(\paren{\,f(a), \,g(a)}\)和点\(\paren{\,f(b), \,g(b)}\)不为同一个点时,则存在\(t_0 \in \paren{a,b}\)满足广义公式
  \begin{equation*}
    \frac{\,f(b)-\,f(a)}{\mkern2mu g(b)-\mkern1mu g(a)} = \frac{\mkern1mu f'(t_0)}{g'(t_0)}.
  \end{equation*}
\end{theorem}

\begin{example*}
  请证明:若函数\(f\)在闭区间\(\brkt{a,b}\)上可导,则存在\(x_0 \in \paren{a,b}\)满足
  \begin{equation*}
    \frac{1}{e^a - e^b}
    \begin{vsmallmatrix}
      e^a & e^b \\
      \,f(a) & \,f(b) \rule{0ex}{2ex}
    \end{vsmallmatrix}
    = \,f(x_0) - \,f'(x_0).
  \end{equation*}

  \begin{proof}
    令\(\Fn F(x) = e^{-x}\,f(x)\)和\(\Fn G(x) = e^{-x}\).那么就有\(\Fn F'(x) = e^{-x}\paren{\,f'(x) - \,f(x)}\)和\(\Fn G'(x) = -e^{-x}\).根据定理~\ref{thm:cmvt},存在\(x_0 \in \paren{a,b}\)满足
    \begin{equation*}
      \frac{1}{e^a - e^b}
      \begin{vsmallmatrix}
        e^a & e^b \\
        \,f(a) & \,f(b) \rule{0ex}{2ex}
      \end{vsmallmatrix}
      = \frac{e^a e^b}{e^a e^b}
      \cdot \frac{e^{-b}\,f(b) - \,e^{-a}\,f(a)}{\mkern.5mu e^{-b}-\,e^{-a}}
      = \frac{\Fn F(\Fn b) - \,\Fn F(a)}{\Fn G(\Fn b) - \,\Fn G(a)}
      = \frac{\mkern1mu\Fn F'(x_0)}{\Fn G'\mkern-2mu(x_0)}
      = \,f(x_0) - \,f'(x_0).
      \qedhere
    \end{equation*}
  \end{proof}
\end{example*}

\begin{example*}
  设函数\(f\)在闭区间\(\brkt{a,b}\)上可导且\(a > 0\).请证明:在开区间\(\paren{a,b}\)上存在\(c\)和\(d\)满足
  \begin{equation*}
    f'(c) = \frac{a+b}{2d} \,f'(d).
  \end{equation*}

  \begin{proof}
    令\(g(x) = x^2\).显然\(g'(x) = 2x > 0\)对于任意的\(x \in \paren{a,b}\).由定理~\ref{thm:cmvt}可知,存在\(d \in \paren{a,b}\)满足
    \begin{equation}
      \frac{f(b)-f(a)}{g(b) - g(a)}
      = \frac{\,f(b)-f(a)}{b^2 - a^2}
      = \frac{f(b)-f(a)}{(b-a)(a+b)}
      = \frac{\,f'(d)}{2d}.
      \label{eq:1}
    \end{equation}
    又定理~\ref{thm:lmvt}可知,存在\(c \in \paren{a,b}\)满足
    \begin{equation}
      \frac{\,f(b)-f(a)}{b-a} = \,f'(c)
      \label{eq:2}
    \end{equation}
    把式~\eqref{eq:2}代入式~\eqref{eq:1}即可得证.
  \end{proof}
\end{example*}

\subpdfbookmark{思考}{B1.5.1.P}
\subsection*{思考}

\begin{enumerate}
\item 有人说:因为当函数\(f\)在区间\(\brkt{a,b}\)上可导时,其导函数\(f'\)在\(\brkt{a,b}\)上具有介值性质,所以\(f'\)在\(\brkt{a,b}\)上连续.你是否认可此人的说法?

  \ifshowsolp
  不认可.练习~\ref{B1.4.1.E}题~\ref{B1.4.1.E4}中的函数就是一个反例.
  \fi

\item 考察拉格朗日中值定理中的\(\,f(x) - \,f(x_0) = f'(\xi)(x - x_0)\).如果\(x_0\)固定,那么当\(x\)变化时\(\xi\)和\(x\)有什么关系?点\(\xi\)能否表示为\(x\)的函数?

  \ifshowsolp
  点\(\xi\)一定坐落在\(x_0\)与\(x\)之间.可以,但这个函数并不唯一,只能说明至少存在一个这样的函数.举个极端点的例子,对于恒等函数\(\,f(x) = x\)来说,任意的\(\minb{x_0, x} \le \xi \le \maxb{x_0, x}\)都能是某个这样函数的函数值.
  \fi

\item 拉格朗日中值定理和柯西中值定理的几何意义是什么?在证明这两个定理的过程中我们是如何构造辅助函数的?能否构造其他辅助函数?

  \ifshowsolp
  % TODO: Complete the answer
  可以构造其他辅助函数.
  \fi
\end{enumerate}

\ifshowex
\currentpdfbookmark{练习}{B1.5.1.E}
\subsection*{练习}

\begin{enumerate}
\item 设函数\(\,f\mkern2mu\)和\(\mkern1mu g\)可导.下列说法中,正确的是\uline{\hspace{10em}}.
  \begin{itemize}
    \renewcommand{\labelitemi}{\faCircleThin}
  \item 若函数\(f\)在闭区间\(\brkt{a,b}\)上单调递增,则\(\,f'\!\)在此区间上恒大于零
    \ifshowsol
  \item[\faCircle]
    \else
  \item
    \fi
    若导函数\(\,f'\!\)在闭区间\(\brkt{a,b}\)上恒为零,则函数\(f\)在此区间上恒为常数
  \item 若导函数\(\,f'\!\)和\(g'\!\)在闭区间\(\brkt{a,b}\)上恒相等,则函数\(\,f\mkern2mu\)和\(\mkern2mu g\)在此区间上也都恒相等
  \item 若存在\(c\)使得\(\,f'(c) = 0\),则存在\(\paren{a,b}\)使得\(c \in \paren{a,b}\)且\(\,f(a) = f(b)\)
  \end{itemize}

  \ifshowsol
  函数\(x^3\)是选项A和D的反例.选项B就是推论~\ref{cor:lmvtconst}.选项C中的两个函数可能相差一个常数.
  \fi

\item 设函数\(f\)可导.关于拉格朗日中值定理,下列形式中,错误的是\uline{\hspace{10em}}.
  \begin{itemize}
    \renewcommand{\labelitemi}{\faCircleThin}
  \item \(f(b) - f(a) = \,f'(c) \paren{b-a},\ c \in \paren{a,b}\)
  \item \(f(x) = f(x_0) + f'(x_0 + \theta\fdx) \fdx,\ \theta \in \paren{0,1}\)
  \item \(f(b) - f(a) = \,f'\paren[\big]{a + \theta(b-a)} \paren{b-a},\ \theta \in \paren{0,1}\)
    \ifshowsol
  \item[\faCircle]
    \else
  \item
    \fi
    \(f'(a + \theta b) = \frac{\,f(b)-\,f(a)}{b-a},\ \theta \in \paren{0,1}\)
  \end{itemize}

\item 下列等式中,正确的是\uline{\hspace{10em}}.
  \begin{itemize}
    \renewcommand{\labelitemi}{\faCircleThin}
  \item \(2\arctan x + \arcsin\frac{2x}{1+x^2} = \pi,\ \abs{x} \ge 1\)
  \item \(\arctan x = \arcsin\frac{x}{\sqrt{1+x^2}} + \frac\pi2,\ -\infty < x < +\infty\)
    \ifshowsol
  \item[\faCircle]
    \else
  \item
    \fi
    \(\arcsin x + \arccos x = \frac\pi2,\ \abs{x} \le 1\)
  \item \(\arcsin x = \arctan\frac{x}{1-x^2} - \frac\pi2,\ \abs{x} < 1\)
  \end{itemize}

\item 关于方程\(x^n + px + q = 0\)的实根个数,下列结论中,正确的有\ifshowsol
  \uline{\makebox[6em]{\enumparen{1}\enumparen{2}\enumparen{4}}}.
  \else
  \uline{\hspace{6em}}.
  \fi
  \begin{enumerate}
    \renewcommand{\labelenumii}{\enumparen{\arabic{enumii}}}
  \item 当\(n\)为偶数时,方程至多有\(2\)个不同实根
  \item 当\(n\)为奇数时,方程至多有\(3\)个不同实根
  \item 当\(n\)为偶数时,方程至少有一个实根
  \item 当\(n\)为奇数时,方程至少有一个实根
  \end{enumerate}

\item 若函数\(f\)在\(\R\)上可导,则\uline{\hspace{10em}}.
  \begin{itemize}
    \renewcommand{\labelitemi}{\faCircleThin}
    \ifshowsol
  \item[\faCircle]
    \else
  \item
    \fi
    当\(\!\lim\limits_{x\to+\infty}\,f'(x) = +\infty\)时,必有\(\!\lim\limits_{x\to+\infty}\,f(x) = +\infty\)
  \item 当\(\!\lim\limits_{x\to+\infty}\,f(x) = +\infty\)时,必有\(\!\lim\limits_{x\to+\infty}\,f'(x) = +\infty\)
  \item 当\(\!\lim\limits_{x\to-\infty}\,f'(x) = -\infty\)时,必有\(\!\lim\limits_{x\to-\infty}\,f(x) = -\infty\)
  \item 当\(\!\lim\limits_{x\to-\infty}\,f(x) = -\infty\)时,必有\(\!\lim\limits_{x\to-\infty}\,f'(x) = -\infty\)
  \end{itemize}

\item 设函数\(f\)在区间\(\brktparen{a,+\infty}\)上可导且\(c\)为常数.下列结论中,正确的有\ifshowsol
  \uline{\makebox[6em]{\enumparen{4}}}.
  \else
  \uline{\hspace{6em}}.
  \fi
  \begin{enumerate}
    \renewcommand{\labelenumii}{\enumparen{\arabic{enumii}}}
  \item 若\(\!\lim\limits_{x\to+\infty}\,f(x) = c\),则\(\!\lim\limits_{x\to+\infty}\,f'(x) = 0\)
  \item 若\(\!\lim\limits_{x\to+\infty}\,f'(x) = 0\),则\(\!\lim\limits_{x\to+\infty}\,f(x) = c\)
  \item 若\(\!\lim\limits_{x\to+\infty}\,f(x) = +\infty\),则\(\!\lim\limits_{x\to+\infty}\,f'(x) = +\infty\)
  \item 若\(\!\lim\limits_{x\to+\infty}\,f'(x) = +\infty\),则\(\!\lim\limits_{x\to+\infty}\,f(x) = +\infty\)
  \end{enumerate}

  \ifshowsol
  函数\(\frac{\sin x^2}{x}\)是命题\enumparen{1}的反例,对数函数\(\ln x\)是命题\enumparen{2}的反例,恒等函数\(x\)是命题\enumparen{3}的反例.
  \fi

\item 设函数\(f\)在闭区间\(\brkt{a,b}\)上二阶可导且\(\,f(a) = f(b) = 0,\ f\dpr(x) \ne 0\ \forall x \in \brkt{a,b}\).下列说法中,错误的是\uline{\hspace{8em}}.
  \begin{itemize}
    \renewcommand{\labelitemi}{\faCircleThin}
  \item 函数\(f\)在开区间\(\paren{a,b}\)上不等于零
    \ifshowsol
  \item[\faCircle]
    \else
  \item
    \fi
    导函数\(f'\!\)在开区间\(\paren{a,b}\)上不等于零
  \item \(\,f_+'(a)\)与\(\,f_-'(b)\)异号
  \item 函数\(f\)在开区间\(\paren{a,b}\)上只有一个点\(c\)使得\(\,f'(c) = 0\)
  \end{itemize}

\item 设\(y\)是\(x\)由方程\(\arctan x = x \arctan' y\)确定的一个函数.求\(\lim\limits_{x\to0} \dfrac{y^2}{x^2}\).

  \ifshowsol
  显然有\(\arctan' y = \frac{1}{1+y^2}\),所以\(y^2 = \frac{x}{\arctan x} - 1\).因此,
  \begin{equation*}
    % \setlength{\abovedisplayskip}{.8ex}
    \lim_{x\to0} \frac{y^2}{x^2}
    = \lim_{x\to0} \frac{x-\arctan x}{x^2 \arctan x}
    = \lim_{x\to0} \frac{x-\arctan x}{x^3}
    = \lim_{x\to0} \frac{1-1/(1+x^2)}{3x^2}
    = \lim_{x\to0} \frac{x^2}{3x^2(1+x^2)}
    = \frac13.
  \end{equation*}
  \fi
\end{enumerate}
\fi

\section{洛必达法则}

% https://doi.org/10.1007/BF02086277
% https://eudml.org/doc/156872
% https://gallica.bnf.fr/ark:/12148/bpt6k205444w/f171.item
% https://archive.org/details/infinimentpetits1716lhos00uoft/page/145/mode/1up
% https://archive.org/details/vorlesungenbera01stolgoog/page/n180/mode/1up
% https://www.jstor.org/stable/2322330
% https://www.jstor.org/stable/3609821
% https://mathworld.wolfram.com/LHospitalsRule.html
% Baby Rudin (109)
% https://doi.org/10.1016/B978-0-08-013473-4.50010-4 (221)
\begin{theorem*}
  \label{thm:lhospital}
  设函数\(\,f\mkern2mu\)和\(\mkern1mu g\)满足:
  \begin{enumerate}
    \renewcommand{\labelenumi}{\enumparen{\arabic{enumi}}}
  \item 在广义点\(x_0\)附近可导且导数非零;
  \item \(\!\lim\limits_{x\to x_0\!} f(x) = \!\lim\limits_{x\to x_0\!} g(x) = 0\)或者\(\!\lim\limits_{x\to x_0\!} g(x) = +\infty\).
  \end{enumerate}
  若
  \begin{equation*}
    \lim_{x\to x_0} \frac{\,f'(x)}{g'(x)} = K,
  \end{equation*}
  则
  \begin{equation*}
    \lim_{x\to x_0} \frac{\,f(x)}{g(x)}
    = K.
  \end{equation*}
\end{theorem*}

\begin{example*}
  求\(\lim\limits_{x\to0} \dfrac{x-x\cos x}{x - \sin x}\).
  \begin{equation*}
    \lim_{x\to0} \frac{x-x\cos x}{x - \sin x}
    = \lim_{x\to0} \frac{1 - \cos x + x \sin x}{1 - \cos x}
    = 3.
  \end{equation*}
\end{example*}

\begin{example*}
  求\(\lim\limits_{x\to0} \dfrac{e^x\!-e^{-x}\!-2x}{x-\sin x}\).
  \begin{gather*}
    \lim_{x\to0} \frac{e^x\!-e^{-x}\!-2x}{x-\sin x}
    = \lim_{x\to0} \frac{e^x\!+e^{-x}\!-2}{1-\cos x}
    = \lim_{x\to0} \frac{(e^x\!-1)^2}{e^x\,x^2\!/2}
    = \lim_{x\to0} \frac{2x^2}{e^xx^2}
    = 2  \\
    \shortintertext{或者}
    2 \lim_{x\to0} \frac{\sinh x-x}{x-\sin x}
    = 2 \lim_{x\to0} \frac{\cosh x-1}{1-\cos x}
    = 2 \lim_{x\to0} \frac{\sinh x}{\sin x}
    = 2 \lim_{x\to0} \frac{\cosh x}{\cos x}
    = 2.
  \end{gather*}
\end{example*}

\begin{example}
  \label{eg:lhospital00ce}
  求\(\lim\limits_{x\to0} \dfrac{x^2 \sin\frac1x}{\sin x}\).

  \begin{remark}
    此例洛必达法则不适用.直接求极限,有
    \begin{equation*}
      \lim_{x\to0} \frac{x^2 \sin\frac1x}{\sin x}
      = \lim_{x\to0} \frac{x}{\sin x} \cdot x \sin\frac1x
      = 0.
    \end{equation*}
  \end{remark}
\end{example}

\begin{example*}
  求\(\!\lim\limits_{x\to+\infty\!} \!\dfrac{\pi/2 - \arctan x}{1/x}\).
  \begin{equation*}
    \lim_{x\to+\infty\!} \!\frac{\pi/2 - \arctan x}{1/x}
    = \lim_{x\to+\infty} \frac{x^2}{1+x^2}
    = 1.
  \end{equation*}
\end{example*}

\begin{example*}
  求\(\!\lim\limits_{x\to0^+\!} \dfrac{\ln x}{1/x}\).
  \begin{equation*}
    \lim_{x\to0^+\!} \frac{\ln x}{1/x}
    = \lim_{x\to0^+\!} \frac{1/x}{-1/x^2}
    = 0.
  \end{equation*}
\end{example*}

\begin{example*}
  设\(a > 1\).求\(\!\lim\limits_{x\to+\infty\!} \dfrac{\,x^2}{a^x}\).
  \begin{equation*}
    \lim_{x\to+\infty\!} \frac{\,x^2}{a^x}
    = \lim_{x\to+\infty\!} \frac{2x}{a^x \ln a}
    = \lim_{x\to+\infty\!} \frac{2}{a^x \operatorname{ln^2} a}
    = 0.
  \end{equation*}
\end{example*}

\begin{example*}
  设\(a > 0\).求\(\!\lim\limits_{x\to+\infty\!} \dfrac{\ln x}{\,x^a}\).
  \begin{equation*}
    \lim_{x\to+\infty\!} \frac{\ln x}{\,x^a}
    = \lim_{x\to+\infty\!} \frac{x^{-1}}{ax^{a-1}}
    = 0.
  \end{equation*}
\end{example*}

\begin{example}
  \label{eg:lhospitalinfinfce}
  求\(\!\lim\limits_{x\to+\infty\!} \!\dfrac{x+\sin x}{x}\).

  \begin{remark}
    此例洛必达法则不适用.直接求极限,有
    \begin{equation*}
      \lim_{x\to+\infty\!} \!\frac{x+\sin x}{x} = \lim_{x\to+\infty\!} \paren*{1 + \frac{\sin x}{x}} = 1.
    \end{equation*}
  \end{remark}
\end{example}

\begin{example*}
  求\(\!\lim\limits_{x\to0^+} \!x \ln x\).
  \begin{equation*}
    \lim_{x\to0^+} \!x \ln x
    = \lim_{x\to0^+\!} \frac{\ln x}{1/x}
    = 0.
  \end{equation*}
\end{example*}

\begin{example*}
  求\(\lim\limits_{x\to1}{}\paren[\bigg]{\dfrac{x}{x-1} - \dfrac1{\ln x}}\).
  \begin{equation*}
    \lim_{x\to1}{}\paren[\bigg]{\dfrac{x}{x-1} - \dfrac1{\ln x}}
    = \lim_{x\to1} \frac{x \ln x - x + 1}{(x-1) \ln x}
    = \lim_{x\to1} \frac{x \ln x - x + 1}{(x-1)^2}
    = \lim_{x\to1} \frac{\ln x + 1 - 1}{2(x-1)}
    = \frac12.
  \end{equation*}
\end{example*}

\begin{example*}
  求\(\!\lim\limits_{x\to0^+\!} x^x\).
  \begin{equation*}
    \lim_{x\to0^+\!} x^x
    = \exp \lim_{x\to0^+\!} x \ln x
    = e^0 = 1.
  \end{equation*}
\end{example*}

\begin{example*}
  求\(\lim\limits_{x\to0} {}(\cos x)^{1/\!\sin^2 x}\).
  \begin{equation*}
    \lim_{x\to0} {}(\cos x)^{1/\!\sin^2 x}
    = \exp \lim_{x\to0} \frac{\ln \cos x}{\sin^2 x}
    = \exp \lim_{t\to1^-\!} \frac{\ln t}{1-t^2}
    = \frac1{\!\sqrt e\,}.
  \end{equation*}
\end{example*}

\begin{example*}
  求\(\!\lim\limits_{x\to+\infty\!} x^{1/x}\).
  \begin{equation*}
    \lim_{x\to+\infty\!} x^{1/x}
    = \exp \!\lim_{x\to+\infty\!} \!\frac{\ln x}{x}
    = e^0 = 1.
  \end{equation*}
\end{example*}

\begin{example*}
  设函数\(\,f(x) = e^x + ax^2 + bx + c\)与函数\(g(x) = x - \sin x\)在\(x \to 0\)时是等价无穷小量.求\(a,b,c\).

  \begin{remark}
    因为当\(x \to 0\)时有\(\,f(x) \to 1 + c,\ g(x) \to 0,\ f(x)/g(x) \to 1\),所以\(1 + c = 0 \implies c = -1\).又因为
    \begin{equation*}
      \lim_{x\to0} \frac{\,f'(x)}{g'(x)}
      = \lim_{x\to0} \frac{e^x + 2ax + b}{1-\cos x}
      = K,
    \end{equation*}
    当\(1 + b \ne 0\)时有\(K = \pm\infty\),所以\(b = -1\).同理可得\(2a = -1 \implies a = -1/2\).
  \end{remark}
\end{example*}

\subpdfbookmark{思考}{B1.5.2.P}
\subsection*{思考}

\begin{enumerate}
\item 对于\(\frac00\)或\(\frac\infty\infty\)型不定式,如果极限\(\lim\frac{\,f'}{g'}\)既不存在也不趋向于无穷大,那么\(\lim\frac{\,f}{g}\)是否一定不存在?

  \ifshowsolp
  未必一定不存在.例~\ref{eg:lhospital00ce}和~\ref{eg:lhospitalinfinfce}分别是\(\frac00\)和\(\frac\infty\infty\)型不定式的此类反例.
  \fi

\item 能否利用\(\lim\limits_{x\to0} \frac{-\sin x}{1} = 0\)来求\(\lim\limits_{x\to0} \frac{\cos x}{x}\)的值?

  \ifshowsolp
  不能,因为\(\lim\limits_{x\to0} \frac{\cos x}{x}\)是\(\frac10\)型的,既不是\(\frac00\)型的也不是\(\frac\infty\infty\)型的.
  \fi

\item 能否利用\(\lim\limits_{x\to0} \frac{\sin x}{x} = \lim\limits_{x\to0} \frac{\cos x}{1} = 1\)来求重要极限\(\lim\limits_{x\to0} \frac{\sin x}{x}\)的值?

  \ifshowsolp
  作为求值技巧,可以.作为证明,不可以;因为存在循环论证,即求三角函数的导函数时用到了此重要极限.
  \fi
\end{enumerate}

\ifshowex
\currentpdfbookmark{练习}{B1.5.2.E}
\subsection*{练习}

\begin{enumerate}
\item 求\(\lim\limits_{x\to0} \dfrac{e^x\!-1}{\sin x}\).
  \ifshowsol
  \begin{equation*}
    \lim_{x\to0} \frac{e^x\!-1}{\sin x}
    = \lim_{x\to0} \frac{e^x}{\cos x}
    = 1.
  \end{equation*}
  \fi

\item 求\(\!\lim\limits_{x\to0^+\!} x^{\sin x}\).
  \ifshowsol
  \begin{equation*}
    \lim_{x\to0^+\!} x^{\sin x}
    = \exp \lim_{x\to0^+\!} \sin x \ln x
    = e^0 = 1.
  \end{equation*}
  \fi

\item 求\(\lim\limits_{x\to0} {}\paren[\bigg]{\dfrac1x - \dfrac1{e^x\!-1}}\).
  \ifshowsol
  \begin{equation*}
    \lim_{x\to0} {}\paren[\bigg]{\frac1x - \frac1{e^x\!-1}}
    = \lim_{x\to0} \frac{e^x\!-1-x}{x \paren{e^x\!-1}}
    = \lim_{x\to0} \frac{e^x\!-1}{e^x\!-1 + xe^x}
    = \lim_{x\to0} \frac{e^x}{2e^x + xe^x}
    = \frac12.
  \end{equation*}
  \fi

\item 求\(\lim\limits_{x\to0} \dfrac{(1+x)^{1/x}\!-e}{x}\).
  \ifshowsol
  \begin{equation*}
    \begin{split}
      \lim_{x\to0} \frac{(1+x)^{1/x}\!-e}{x}
      &= \lim_{x\to0} \frac{(1+x)^{1/x}}{1} \cdot \frac{x/(1+x) - \ln(1+x)}{x^2} \\
      &= e \lim_{x\to0} \frac{1/(1+x)^2 - 1/(1+x)}{2x} \\
      &= e \lim_{x\to0} \frac{-2/(1+x)^3 + 1/(1+x)^2}{2} \\
      &= - \frac e2.
    \end{split}
  \end{equation*}
  \fi

\item 设函数\(\,f(x) = x - \sin ax\)与函数\(g(x) = x^2 \lnp{1-bx}\)在\(x \to 0\)时是等价无穷小量.求\(a,b\).

  \ifshowsol
  首先,当\(b = 0\)时,函数\(g\)退化成恒零函数,从而不能和函数\(f\)是等价无穷小量.因此,\(b \ne 0\).又因为
  \begin{equation*}
    \lim_{x\to0} \frac{\,f'(x)}{g'(x)}
    = \lim_{x\to0} \frac{1-a\cos ax}{2x\ln(1-bx) - bx^2\!/(1-bx)}
    = K,
  \end{equation*}
  当\(1-a \ne 0\)时有\(K = \pm\infty\),所以\(a = 1\).因为\(\lim\limits_{x\to0}\,f\dpr(x) = \lim\limits_{x\to0} \sin x = 0\)和
  \begin{equation*}
    \lim_{x\to0} g\dpr(x)
    = \lim_{x\to0} {}\brce[\bigg]{2\ln(1-bx) - \frac{4bx}{1-bx} - \frac{b^2x^2}{(1-bx)^2}}
    = 0,
  \end{equation*}
  所以还需再用一次洛必达法则.有
  \begin{equation*}
    \lim_{x\to0} \frac{\,f\trpr(x)}{g\trpr(x)}
    = -\frac{1}{6b} = 1
    \implies
    b = -\frac16.
  \end{equation*}
  \fi

\item 求\(\lim\limits_{x\to0} \dfrac{\sin\sin x - \sin x}{x^3}\).
  \ifshowsol
  \begin{equation*}
    \begin{split}
      \lim_{x\to0} \frac{\sin\sin x - \sin x}{x^3}
      &= \lim_{x\to0} \frac{\cos x \cos\sin x - \cos x}{3x^2} \\
      &= - \lim_{x\to0} \frac{(1 - \cos\sin x) \cos x}{3\sin^2 x} \\
      &= - \lim_{t\to0} \frac{\paren{1 - \cos t} \sqrt{1-t^2}}{3t^2} \\
      &= - \frac16.
    \end{split}
  \end{equation*}
  \fi
  
\item 求\(\lim\limits_{x\to0} {}\paren[\bigg]{\dfrac{\ln(1+x)}{x}}^{\frac1{e^x-1}}\).
  \ifshowsol
  \begin{equation*}
    \begin{split}
      \lim_{x\to0} \paren[\bigg]{\frac{\ln(1+x)}{x}}^{\frac1{e^x-1}}
      &= \exp \lim_{x\to0} \frac{\ln\frac{\ln(1+x)}{x}}{e^x-1} \\
      &= \exp \lim_{x\to0} \frac{x/\ln(1+x)}{e^x} \cdot \frac{x/(1+x)-\ln(1+x)}{x^2} \\
      &= \exp \lim_{x\to0} \frac{1/(1+x)^2 - 1/(1+x)}{2x} \\
      &= \exp \lim_{x\to0} \frac{-2/(1+x)^3 + 1/(1+x)^2}{2} \\
      &= \frac1{\!\sqrt e\,}.
    \end{split}
  \end{equation*}
  \fi
  
\item 设\(\lim\limits_{x\to0} {}\brce[\bigg]{\dfrac1x - \paren[\bigg]{\dfrac1x - a} e^x} = 1\).求\(a\).

  \ifshowsol
  因为
  \begin{equation*}
    \lim_{x\to0} {}\brce[\bigg]{\frac1x - \paren[\bigg]{\frac1x - a} e^x}
    = \lim_{x\to0} \frac{1-(1-ax)e^x}{x}
    =  \lim_{x\to0} \frac{e^x(a+ax-1)}{1}
    = a-1 = 1,
  \end{equation*}
  所以\(a = 2\).
  \fi
\end{enumerate}
\fi

\section{函数的单调性与极值}

\subpdfbookmark{思考}{B1.5.3.P}
\subsection*{思考}

\ifshowex
\currentpdfbookmark{练习}{B1.5.3.E}
\subsection*{练习}
\fi

\section{函数的凸性与拐点}

\subpdfbookmark{思考}{B1.5.4.P}
\subsection*{思考}

\ifshowex
\currentpdfbookmark{练习}{B1.5.4.E}
\subsection*{练习}
\fi

\section{泰勒公式}

\subpdfbookmark{思考}{B1.5.5.P}
\subsection*{思考}

\ifshowex
\currentpdfbookmark{练习}{B1.5.5.E}
\subsection*{练习}
\fi

% Local Variables:
% TeX-engine: luatex
% TeX-master: "微积分B"
% End:
