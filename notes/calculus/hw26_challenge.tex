\documentclass{article}

\usepackage{amsmath, amssymb}
\usepackage[a4paper, hmargin=1.25in, vmargin=1in]{geometry}
\usepackage[colorlinks=true, urlcolor=blue]{hyperref}
\usepackage{graphicx, cancel}

\newcommand*{\parasp}{\setlength{\parskip}{10pt}}
\newcommand*{\paren}[1]{\left( #1 \right)}
\newcommand*{\Prb}[1]{\section*{Problem #1}}
\newcommand{\Q}[1]{\textbf{Question:} #1}
\newcommand*{\A}[1]{\textbf{Answer:} #1}
\newcommand{\Prblm}[3]{\Prb{#1} \Q{#2} \\[6pt] \A{#3}}
\newcommand*{\cm}{\,\mathrm{cm}}
\newcommand*{\R}{\mathbb{R}}
\newcommand*{\Rp}{(0,+\infty)}
\newcommand*{\Rm}{(-\infty,0)}
\newcommand*{\deduce}{\mathrel{\Downarrow}}
\newcommand*{\abs}[1]{\left\lvert #1 \right\rvert}
\newcommand*{\reason}[1]{\langle \, \text{#1} \, \rangle}

\DeclareMathOperator{\arccosh}{arccosh}

\everymath{\displaystyle}

\begin{document}
    \Prblm{1}{$ \frac{d}{dx} \int_{t=0}^{\arcsin x} \ln \left| \sin t + \cos t
                \right| \, dt $.}
    {$ \frac{1}{\sqrt{1-x^2}}\ln \left| x + \sqrt{1-x^2} \right| $.}
    
    \begin{align*}
    \frac{d}{dx} \int_{t=0}^{\arcsin x} \ln \left| \sin t + \cos t \right| \, dt
        &= \frac{d}{dx} \int_{t=0}^{u} \ln \left| \sin t + \cos t
            \right| \, dt
            && \reason{$ u = \arcsin x $} \\
        &= \ln |\sin u + \cos u| \cdot \frac{du}{dx}
            && \reason{$\textstyle \frac{d}{dx} \int_{t=a}^{u(x)} f(t) \,dt =
                        f(u(x)) \frac{du(x)}{dx} $} \\
        &= \frac{\ln |x + \cos(\arcsin x)|}{\sqrt{1-x^2}}
            && \reason{$ u = \arcsin x $ and evaluate $ \tfrac{du}{dx} $} \\
        &= \frac{\ln |x + \sqrt{1-x^2}|}{\sqrt{1-x^2}}
            && \reason{$ \cos(\arcsin x) = \sqrt{1-x^2} $}
    \end{align*}
    
    \Prblm{2}{$ \frac{d}{dx} \int_{t=\sin x}^{\tan x} e^{-t^2} \, dt $.}
    {$ e^{-\tan^2 x} \cdot \sec^2 x - e^{-\sin^2 x} \cdot \cos x $.}
    
    \begin{align*}
    \frac{d}{dx} \int_{t=\sin x}^{\tan x} e^{-t^2} \, dt
        &= \frac{d}{dx} \paren{\int_{t=0}^{\tan x} e^{-t^2} \, dt -
            \int_{t=0}^{\sin x} e^{-t^2} \, dt}
            && \reason{additivity} \\
        &= \frac{d}{dx} \int_{t=0}^{\tan x} e^{-t^2} \, dt
            - \frac{d}{dx} \int_{t=0}^{\sin x} e^{-t^2} \, dt
            && \reason{linearity} \\
        &= e^{-\tan^2 x} \cdot \sec^2 x - e^{-\sin^2 x} \cdot \cos x.
            && \reason{$\textstyle \frac{d}{dx} \int_{t=a}^{u(x)} f(t) \,dt =
                f(u(x)) \frac{du(x)}{dx} $} \\
    \end{align*}
    
    \Prblm{3}
    {Which of the following is the leading order term in the Taylor series about
    $ x=0 $ of
        \[ f(x) = \int_{t=0}^x\ln(\cosh t) \, dt \]
    \textbf{Hint:} yes, there's more than one way to do this problem\dots\ Try
    using the F.T.I.C. to compute the derivatives.}
    {$ \frac{x^3}{3!} = \frac{x^3}6 $.}
    
    \parasp
    The constant term of the Taylor series must be zero since $\textstyle f(0) =
    \int_{t=0}^0 x\ln(\cosh t) \, dt $. So if the first derivative at $ x=0 $ is
    nonzero, then $ f'(0)\cdot x $ must be the leading order term. And
        \[ \frac{d}{dx} f(x) = \frac{d}{dx} \int_{t=0}^x\ln(\cosh t) \, dt
                             = \ln(\cosh x)\,, \]
    so $ f'(0) = \ln(\cosh 0) = \ln 1 = 0 $. Oops, we have to take the second
    derivative at $ x=0 $,
        \[ \frac{d^2}{dx^2} f(x) = \frac{d}{dx} \ln(\cosh x)
                                 = \frac{\sinh x}{\cosh x}\,, \]
    so $ f''(0) = \sinh 0 / \cosh 0 = 0 $. Oops again, let's take the third derivative,
        \[ \frac{d^3}{dx^3} f(x) = \frac{d}{dx} \frac{\sinh x}{\cosh x}
                                 = \frac{1}{\cosh^2 x}\,, \]
    so $ f'''(0) = 1 $. Thus, the leading order term is
        \[ \frac{x^3}{3!} = \frac{x^3}6. \]
    
    \Prblm{4}
    {We usually use Riemann sums to approximate integrals, but we can go the
    other way, too, using an antiderivative to approximate a sum. Using only
    your head (no paper, no calculator), tell me which of the following is the
    best estimate for
        \[ \sum_{n=0}^{100} n^3. \]}
    {$ 2.5\times 10^7 $.}
    
    \begin{align*}
        \sum_{n=0}^{100} n^3
            &\approx \int_0^{100} x^3 \, dx \\
            &= 100^4/4 \\
            &= 2.5\times 10^7
    \end{align*}
    
    \begin{figure}[h]
    \centering
    \includegraphics[width=0.7\linewidth]{hw26_challenge_fig1}
    \label{fig:hw26_challenge_fig1}
    \end{figure}

\end{document}