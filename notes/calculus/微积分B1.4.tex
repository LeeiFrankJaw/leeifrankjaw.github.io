\chapter{导数与微分}

\section{导数与微分的概念}

% https://math.libretexts.org/@go/page/116560#Notations_for_the_Derivative
% https://www.jstor.org/stable/1967725
\begin{definition*}
  \label{defn:deriv}
  设函数\(y = f(x)\)在点\(x_0\)处及其附近有定义.若极限
  \begin{equation*}
    \lim_{\fwdf x \to 0} \frac{f(x_0+\fwdf x) - f(x_0)}{\fwdf x}
  \end{equation*}
  存在,则称函数\(f\)在点\(x_0\)处可导,称此极限为函数\(f\)在此处的导数,记作
  \begin{equation*}
    f'(x_0),
    \quad
    \left.\dd[f]x\right\vert_{\mathrlap{x_0}},
    \quad
    \dd[f]x(x_0),
    \quad
    \dot y(x_0)
    \txt{或}
    (\Fn Df)(x_0).
  \end{equation*}
  我们把所有可导点与其导数的对应关系叫作导函数,也简称成导数,记作
  \begin{equation*}
    f', \quad \dd[f]x, \quad \dot y \txt{或} \!\Fn Df.
  \end{equation*}
\end{definition*}

\begin{example*}
  试讨论函数\(f(x) = x^2 \Fn D(x)\)在\(\R\)上的可导性.

  \begin{remark}
    此函数除了在原点处可导之外,在其他处皆不可导.对于任意的\(x_0 \ne 0\),分别由无理数和有理数的方式趋向于\(x_0\),易知导数定义中的极限必有一种方式使其为无穷大量.只有当\(x = x_0\)时,无论由什么方式趋向于\(x_0\),导数定义中极限都收敛且等于零.
  \end{remark}
\end{example*}

\begin{example*}
  常函数的导数都是零.
\end{example*}

\begin{example*}
  求函数\(f(x) = x^n\)的导函数.

  \begin{remark}
    令\(\fwdf\,f(x) = f(x+\fwdf x) - f(x) = \paren{x+\fwdf x}^n - x^n = \fwdf x \sum_{k=1}^n \paren{x+\fwdf x}^{n-k}\,x^{k-1}\),所以有
    \begin{equation*}
      f'(x)
      = \lim_{\fwdf x \to 0} \frac{\fwdf\,f(x)}{\fwdf x}
      = \lim_{\fwdf x \to 0} \sum_{k=1}^n \paren{x+\fwdf x}^{n-k}\,x^{k-1}
      = nx^{n-1}.
    \end{equation*}
  \end{remark}
\end{example*}

\begin{example*}
  求指数函数\(f(x) = e^x\)的导函数.
  \begin{equation*}
    f'(x)
    = \lim_{\fwdf x \to 0} \frac{e^{x+\fwdf x} - e^x}{\fwdf x}
    = \lim_{\fwdf x \to 0} e^x \cdot \frac{e^{\fwdf x} - 1}{\fwdf x}
    = e^x \lim_{\fwdf x \to 0} \frac{e^{\fwdf x} - 1}{\fwdf x}
    = e^x.
    \rule[-3.5ex]{0ex}{0ex}
  \end{equation*}
\end{example*}

\begin{example*}
  求函数\(f(x) = a^x\)的导函数,其中\(a > 0\)且\(a \ne 1\).
  \begin{equation*}
    f'(x)
    = \lim_{\fwdf x \to 0} \frac{a^{x+\fwdf x} - a^x}{\fwdf x}
    = \lim_{\fwdf x \to 0} a^x \cdot \frac{a^{\fwdf x} - 1}{\fwdf x}
    = a^x \lim_{\fwdf x \to 0} \frac{a^{\fwdf x} - 1}{\fwdf x}
    = a^x \ln a.
    \rule[-3.5ex]{0ex}{0ex}
  \end{equation*}
\end{example*}

\begin{example*}
  求函数\(f(x) = \ln x\)的导函数.
  \begin{equation*}
    f'(x)
    = \lim_{\fwdf x \to 0} \frac{\lnp{x+\fwdf x}- \lnp x}{\fwdf x}
    = \lim_{\fwdf x \to 0} \frac1x \cdot \frac{\lnp{1+\fwdf x/x}}{\fwdf x/x}
    = \frac1x \lim_{\fwdf x \to 0} \frac{\lnp{1+\fwdf x/x}}{\fwdf x/x}
    = \frac1x.
  \end{equation*}
\end{example*}

\begin{example*}
  求正弦函数的导函数.
  \begin{align*}
    \sin'x
    &= \lim_{\fwdf x \to 0} \frac{\sinp{x+\fwdf x}- \sin x}{\fwdf x} \\
    &= \lim_{\fwdf x \to 0} \frac{2 \cosp{x+\fwdf x/2} \sinp{\fwdf x/2}}{\fwdf x} \\
    &= \lim_{\fwdf x \to 0} \cosp{x+\fwdf x/2} \cdot \frac{\sinp{\fwdf x/2}}{\fwdf x/2} \\
    &= \cos x.
  \end{align*}
\end{example*}

\begin{example*}
  若函数\(f\)在点\(a\)处的导数大于零,则存在\(\delta > 0\)使得:当\(a < x < a+\delta\)时都有\(f(x) > f(a)\);当\(a-\delta < x < a\)时都有\(f(x) < f(a)\).

  \begin{proof}
    令\(\fwdf x = x - a\).由定理~\ref{thm:limfuncsgnprsv}可知,存在\(\delta > 0\)使得当\(0 < \abs{\fwdf x} < \delta\)时都有\(\fwdf\,f(a)/\!\fwdf x > 0\).所以当\(\fwdf x > 0\)时,有
    \begin{equation*}
      \fwdf\,f(a) = f(a+\fwdf x) - f(a) > 0
      \iff
      f(a+\fwdf x) = f(x) > f(a).
    \end{equation*}
    同理,当\(\fwdf x < 0\)时,有\(f(x) < f(a)\).
  \end{proof}
\end{example*}

% https://www.mathcounterexamples.net/a-function-whose-derivative-at-0-is-one-but-which-is-not-increasing-near-0/
% https://math.stackexchange.com/q/2768994/147999
\begin{example*}
  函数\(f\)在点\(a\)处的导数大于零,不能保证一定存在\(a\)的一个邻域使得函数\(f\)在此邻域上单调递增.

  \begin{remark}
    这里提供两个的反例
    \begin{equation*}
      f(x) =
      \begin{cases}
        x + 2 x^2 \sin\frac1x, & x \ne 0, \\
        0, & x = 0,
      \end{cases}
      \qquad
      g(x) =
      \begin{cases}
        x, & x \in \Q, \\
        \sin x, & x \notin \Q.
      \end{cases}
    \end{equation*}
    根据导数的定义,有\(f'(0) = g'(0) = 1\),但显然它们在原点附近都不单调递增.
  \end{remark}
\end{example*}

\begin{definition*}
  设函数\(f\)在点\(x_0\)处及其右侧附近有定义.若极限
  \begin{equation*}
    \lim_{\fwdf x \to 0^+} \frac{\fwdf\,f(x_0)}{\fwdf x}
  \end{equation*}
  存在,则称函数\(f\)在点\(x_0\)处右可导,称此单侧极限为函数\(f\)在此处的右导数,记作\(f'_+(x_0)\).左导数也有类似的定义和记号.
\end{definition*}

\begin{theorem}
  \label{thm:derivsided}
  函数在某处可导的充分必要条件是它在其处的左右导数皆存在且相等.

  \begin{proof}
    应用定理~\ref{thm:limfuncsided}即可得证.
  \end{proof}
\end{theorem}

\begin{example*}
  试讨论函数
  \begin{equation*}
    f(x) =
    \begin{cases}
      \sqrt x, & x \in \brkt{0,1}, \\
      x, & x > 1
    \end{cases}
  \end{equation*}
  在点\(1\)处的可导性.

  \begin{remark}
    显然有\(f'_+(1) = 1\),剩下就看左导数是否存在且相等即可.有
    \begin{equation*}
      f'_-(1)
      = \lim_{\fwdf x \to 0^-} \frac{\sqrt{1+\fwdf x} - 1}{\fwdf x}
      = \lim_{\fwdf x \to 0^-} \frac{\fwdf x}{\raisebox{-.1ex}{$\fwdf x$}\,\paren[\big]{\raisebox{-.1ex}{$\sqrt{1+\fwdf x} + 1$}}}
      = \frac12.
    \end{equation*}
    所以它在点\(1\)处不可导.
  \end{remark}
\end{example*}

\begin{theorem}
  \label{thm:deriv2cont}
  可导蕴含连续.

  \begin{proof}
    使用反证法.假设函数\(f\)在点\(x_0\)处不连续,那么存在\(\varepsilon_0 > 0\)使得:对于任意的\(\delta > 0\)都存在\(\fwdf x_\delta\)满足\(0 < \abs*{\fwdf x_\delta} < \delta\)且\(\abs*{\fwdf\,f_\delta(x_0)} \ge \varepsilon_0\).这时,令\(1/\delta_0 = 1 + \abs*{\,f'(x_0)}/\varepsilon_0\),则存在\(\fwdf x\)满足\(0 < \abs*{\fwdf x} < \delta_0\)且
    \begin{equation*}
      \abs*{\frac{\fwdf\,f(x_0)}{\fwdf x} - f'(x_0)}
      \ge \abs*{\frac{\fwdf\,f(x_0)}{\fwdf x}} - \abs*{\,f'(x_0)}
      \ge \frac{\varepsilon_0}{\abs*{\fwdf x}} - \abs*{\,f'(x_0)}
      > \varepsilon_0.
    \end{equation*}
    显然,对于任意的\(\delta > 0\),也能找到对应的\(\fwdf x\)使得上述不等式成立.所以函数\(f\)在点\(x_0\)处不可导,自相矛盾.
  \end{proof}
  \begin{proof}
    根据定理~\ref{thm:limfunc4ops}和~\ref{thm:contequivdefn},有
    \begin{equation*}
      \lim_{\fwdf x \to 0} \fwdf\,f(x_0)
      = \lim_{\fwdf x \to 0} \frac{\fwdf\,f(x_0)}{\fwdf x} \cdot \fwdf x
      = f'(x_0) \cdot 0 = 0. \rule[-2.5ex]{0ex}{0ex}
      \qedhere
    \end{equation*}
  \end{proof}
\end{theorem}

\begin{example*}
  常数\(a\)和\(b\)取什么值时,可以使得函数
  \begin{equation*}
    f(x) =
    \begin{cases}
      x^2, & x \ge 0, \\
      ax+b, & x < 0
    \end{cases}
  \end{equation*}
  在原点处可导?

  \begin{remark}
    显然有\(f'_-(0) = a\)和\(f'_+(0) = 0\).根据定理~\ref{thm:derivsided},有\(a = 0\).又因为\(f_-(0) = b,\ f_+(0) = f(0) = 0\),根据定理~\ref{thm:deriv2cont},有\(b = 0\).
  \end{remark}
\end{example*}

\begin{remark}[导数的几何意义]
  \label{rem:derivgeom}
  \leavevmode
  \begin{enumerate}
  \item 切线方程:曲线\(y = f(x)\)在点\(\paren{x_0, \,y_0}\)处的切线方程为
    \begin{equation*}
      y = x\,f'(x_0) + y_0 - x_0\,f'(x_0).
    \end{equation*}
  \item 法线方程:曲线\(y = f(x)\)在点\(\paren{x_0, \,y_0}\)处的法线方程为
    \begin{equation*}
      y = - \frac{x}{f'(x_0)} + y_0 + \frac{x_0}{f'(x_0)}.
    \end{equation*}
  \end{enumerate}
\end{remark}

\begin{example*}
  求曲线\(y = \ln x\)过原点的切线方程.

  \begin{remark}
    假设此切线是函数\(\ln x\)在点\(x_0\)处的切线,则
    \begin{equation*}
      y_0 - x_0 \ln'x_0 = \ln x_0 - 1 = 0
      \iff
      x_0 = e.
    \end{equation*}
    所以此切线的方程是\(y = x/e\).
  \end{remark}
\end{example*}

\begin{example*}
  设曲线\(y = f(x)\)与曲线\(y = \ln x\)在\(x = 1\)处相切.求\(f(1)\)和\(f'(1)\).

  \begin{remark}
    自然有\(f(1) = \ln 1 = 0\)和\(f'(1) = \ln' 1 = 1\).
  \end{remark}
\end{example*}

% https://en.wikipedia.org/wiki/Differential_of_a_function
% https://math.stackexchange.com/q/23902/147999
\begin{definition*}
  \label{defn:diff}
  设函数\(f\)在点\(x_0\)处及其附近有定义.若存在一个线性函数\(A \fwdf x\)使得当\(\fwdf x \to 0\)时有
  \begin{equation*}
    \fwdf\,f(x_0) = A \fwdf x + \littleop{\fwdf x},
  \end{equation*}
  则称函数\(f\)在点\(x_0\)处可微,称\(A \fwdf x\)为函数\(f\)在点\(x_0\)处的微分,记作\(\diff f\,\smash{\Big\vert_{\mathrlap{x_0}}\,} = A \fwdf x\).
\end{definition*}

\begin{example*}
  请用定义来求函数\(f(x) = x^3\)的微分.

  \begin{remark}
    因为
    \begin{equation*}
      \fwdf\,f(x)
      = (x+\fwdf x)^3 - x^3
      = 3x^2 \fwdf x + 3x \fwdf x^2 + \fwdf x^3
      = 3x^2 \fwdf x + \littleop{\fwdf x},
    \end{equation*}
    所以函数\(f\)的微分是\(3x^2 \fwdf x\).
  \end{remark}
\end{example*}

\begin{theorem}
  \label{thm:diffequivderiv}
  可微的充分必要条件是可导.

  \begin{proof}
    易证必要性.下面来证明一下充分性.因为
    \begin{equation*}
      \lim_{\fwdf x \to 0} \frac{\fwdf\,f(x_0) - \,f'(x_0) \fwdf x}{\fwdf x} = 0,
    \end{equation*}
    所以\(\fwdf\,f(x_0) - \,f'(x_0) \fwdf x = \littleop{\fwdf x} \implies \fwdf\,f(x_0) = \,f'(x_0) \fwdf x + \littleop{\fwdf x}\).
  \end{proof}
\end{theorem}

\begin{remark}
  显然地,恒等函数\(\,f(x) = x\)的微分是\(1 \cdot \fdx\),即\(\dx = \fdx\).所以我们今后一般用\(\dy = y' \dx\)来表示函数\(y = f(x)\)微分.
\end{remark}

% https://doi.org/10.1016/B978-0-08-013473-4.50008-6 (167)
% TODO: Interpret geometrically the differential

\subpdfbookmark{思考}{B1.4.1.P}
\subsection*{思考}

\begin{enumerate}
\item 连续与可导都是函数在一点的性质,请说明它们之间的相互关系.

  \ifshowsolp
  可导一定连续,连续未必可导(定理~\ref{thm:deriv2cont}).
  \fi

\item 导数与左右导数是什么关系?请举出左右导数都存在但不相等的例子.

  \ifshowsolp
  见定理~\ref{thm:derivsided}.一个简单的例子就是绝对值函数\(\abs x\).
  \fi

\item 请说明函数\(f\)在点\(x_0\)处可导与曲线\(y = f(x)\)在点\(\paren[\big]{x_0, \,f(x_0)}\)处的切线存在性之间的关系.

  \ifshowsolp
  可导则切线一定存在,切线存在未必可导.例如函数\(\!\sqrt[\leftroot{-2}\uproot{2}3] x\)在原点处的切线存在但不可导.
  \fi
\end{enumerate}

\ifshowex
\currentpdfbookmark{练习}{B1.4.1.E}
\subsection*{练习}
\label{B1.4.1.E}

\begin{enumerate}
\item 下列关于导数的说法中,错误的是\uline{\hspace{10em}}.
  \begin{itemize}
    \renewcommand{\labelitemi}{\faCircleThin}
  \item 可导函数一定连续
    \ifshowsol
  \item[\faCircle]
    \else
  \item
    \fi
    连续函数一定可导
  \item 可导偶函数的导函数为奇函数
  \item 可导奇函数的导函数为偶函数
  \end{itemize}

  \ifshowsol
  下面来说明一下后两个选项.设函数\(f\)为偶函数,则
  \begin{equation*}
    \begin{split}
      f'(-x)
      &= \lim_{\fdx \to 0} \frac{\fwdf\,f(-x)}{\fdx}
      = \lim_{\fdx \to 0} \frac{\,f(\fdx-x) - \,f(-x)}{\fdx} \\
      &= \lim_{\fdx \to 0} \frac{\,f(x-\fdx) - \,f(x)}{\fdx}
      = \lim_{h \to 0} \frac{\,f(x+h) - \,f(x)}{-h} \\
      &= - \,f'(x).
    \end{split}
  \end{equation*}
  设函数\(f\)为奇函数,则
  \begin{equation*}
    \begin{split}
      f'(-x)
      &= \lim_{\fdx \to 0} \frac{\fwdf\,f(-x)}{\fdx}
      = \lim_{\fdx \to 0} \frac{\,f(\fdx-x) - \,f(-x)}{\fdx} \\
      &= \lim_{\fdx \to 0} \frac{\,f(x) - \,f(x-\fdx)}{\fdx}
      = \lim_{h \to 0} \frac{\,f(x) - \,f(x+h)}{-h} \\
      &= \,f'(x).
    \end{split}
  \end{equation*}
  \fi

\item 函数
  \begin{equation*}
    f(x) =
    \begin{cases}
      \frac23 x^3, & x \le 1, \\
      x^2, & x > 1
    \end{cases}
  \end{equation*}
  在点\(1\)处的\uline{\hspace{10em}}.
  \begin{itemize}
    \renewcommand{\labelitemi}{\faCircleThin}
  \item 左右导数都存在
    \ifshowsol
  \item[\faCircle]
    \else
  \item
    \fi
    左导数存在,右导数不存在
  \item 左导数不存在,右导数存在
  \item 左右导数都不存在
  \end{itemize}

  \ifshowsol
  函数\(f\)显然在此处左连续,所以有\(f'_-(1) = \paren[\big]{\frac23 x^3}'\big\vert_1 = 2x^2\big\vert_1 = 2\).又因为\(f(1) = 2/3 \ne 1 = f_-(1)\),函数\(f\)显然不是右连续的,因此也不是右可导的.
  \fi

\item 函数\(\,f(x) = \abs{\sin x}\)在原点处\uline{\hspace{8em}}.
  \begin{itemize}
    \renewcommand{\labelitemi}{\faCircleThin}
  \item 既不可导,又不连续
    \ifshowsol
  \item[\faCircle]
    \else
  \item
    \fi
    连续,但不可导
  \item 可导,但不连续
  \item 连续且可导
  \end{itemize}

  \ifshowsol
  因为\(f_-(0) = -\sin_-0 = 0 = \sin_+0 = f_+(0)\)且\(f(0) = 0\),所以它在原点处连续.又因为~\(f'_-(0) = -\sin'_-0 = -1 \ne 1 = \sin'_+0 = f'_+(0)\),所以它在原点处不可导.
  \fi

\item \label{B1.4.1.E4}函数
  \begin{equation*}
    f(x) =
    \begin{cases}
      x^2 \sin\frac1x, & x \ne 0, \\
      0, & x = 0
    \end{cases}
  \end{equation*}
  在原点处\uline{\hspace{8em}}.
  \begin{itemize}
    \renewcommand{\labelitemi}{\faCircleThin}
  \item 既不可导,也不连续
  \item 连续,但不可导
  \item 可导,但不连续
    \ifshowsol
  \item[\faCircle]
    \else
  \item
    \fi
    连续且可导
  \end{itemize}

  \ifshowsol
  根据定理~\ref{thm:deriv2cont},可导则连续.又有
  \begin{equation*}
    f'(0)
    = \lim_{\fdx\to0} \frac{\fwdf\,f(0)}{\fdx}
    = \lim_{\fdx\to0} \frac{\,f(\fdx)}{\fdx}
    = \lim_{\fdx\to0} \frac{\fdx^2 \sin\frac1\fdx}{\fdx}
    = \lim_{\fdx\to0} \fdx \sin\frac1\fdx
    = 0.
  \end{equation*}
  \fi

\item 函数
  \begin{equation*}
    f(x) =
    \begin{cases}
      x, & x < 0, \\
      \lnp{1+x}, & x \ge 0
    \end{cases}
  \end{equation*}
  在原点处\uline{\hspace{8em}}.
  \begin{itemize}
    \renewcommand{\labelitemi}{\faCircleThin}
  \item 既不可导,也不连续
  \item 连续,但不可导
  \item 可导,但不连续
    \ifshowsol
  \item[\faCircle]
    \else
  \item
    \fi
    连续且可导
  \end{itemize}

\item 若函数\(g\)是有界函数,函数
  \begin{equation*}
    f(x) =
    \begin{dcases}
      \frac{1 - \cos x}{\sqrt x}, & x > 0, \\
      x^2 g(x), & x \le 0
    \end{dcases}
  \end{equation*}
  在原点处\uline{\hspace{8em}}.
  \begin{itemize}
    \renewcommand{\labelitemi}{\faCircleThin}
  \item 极限不存在
  \item 极限存在,但不连续
  \item 连续,但不可导
    \ifshowsol
  \item[\faCircle]
    \else
  \item
    \fi
    可导
  \end{itemize}

  \ifshowsol
  因为\(\,f(0) = 0\),所以\(\,f'(0) = \!\lim\limits_{\fdx\to0} f(\fdx)/\!\fdx\).又因为
  \begin{gather*}
    \lim_{\fdx\to0^-}\negthickspace\frac{\,f(\fdx)}{\fdx}
    = \negthickspace\lim_{\fdx\to0^-}\negthickspace\frac{\fdx^2 g(\fdx)}{\fdx}
    = \negthickspace\lim_{\fdx\to0^-}\negthickspace\fdx\,g(\fdx)
    = 0 \\
    \siand
    \lim_{\fdx\to0^+}\negthickspace\frac{\,f(\fdx)}{\fdx}
    = \negthickspace\lim_{\fdx\to0^+}\negthickspace\frac{1-\cosp\fdx}{\fdx^{1.5}}
    = \negthickspace\lim_{\fdx\to0^+}\negthickspace\frac{\!\sqrt{\scriptstyle\fdx}\,}{2}
    = 0,
  \end{gather*}
  所以\(\,f'(0) = 0\).
  \fi

\item 设\(\Fn F(x) = \,f(x) \paren[\big]{1+\abs{\sin x}}\).若函数\(f\)可导,则\(\,f(0) = 0\)是函数\(\Fn F\)在原点处可导的\uline{\hfill}.
  \begin{itemize}
    \renewcommand{\labelitemi}{\faCircleThin}
    \ifshowsol
  \item[\faCircle]
    \else
  \item
    \fi
    充分必要条件
  \item 充分不必要条件
  \item 必要不充分条件
  \item 既不充分又不必要条件
  \end{itemize}

  \ifshowsol
  \begin{proof}
    注意到\(\Fn F(0) = \,f(0)\).有
    \begin{equation}
      \begin{split}
        \Fn F'(0)
        &= \lim_{\fdx\to0} \frac{\fwdf F(0)}{\fdx}
        = \lim_{\fdx\to0} \frac{\,f(\fdx)\paren[\big]{1+\abs{\sin\fdx}} - f(0)}{\fdx} \\
        &= \lim_{\fdx\to0} \paren[\bigg]{
          \frac{\,f(\fdx) - f(0)}{\fdx}
          + f(\fdx) \frac{\abs{\sin\fdx}}{\fdx}}.
      \end{split}
      \label{eq:B1.4.1.E7}
    \end{equation}
    假设\(f(0) = 0\).因为函数\(f\)可导,所以连续,即\(\lim_{\fdx\to0} \,f(\fdx) = f(0) = 0\).式~\eqref{eq:B1.4.1.E7}就变成\(F'(0) = f'(0)\).充分性得证.假设\(f(0) \ne 0\).那么式~\eqref{eq:B1.4.1.E7}中的\(\fdx\)从左右两个方向趋向于零,就会得到不相等的左右极限,即函数\(F\)在原点处不可导.必要性得证.
  \end{proof}
  \fi

\item 若函数\(f\)在原点处连续且\(\lim\limits_{h\to0} \frac{\,f(h^2)}{h^2} = 1\),则\uline{\hspace{8em}}.
  \begin{itemize}
    \renewcommand{\labelitemi}{\faCircleThin}
  \item \(f(0) = 0\)且\(\,f'_-(0)\)存在
  \item \(f(0) = 1\)且\(\,f'_-(0)\)存在
    \ifshowsol
  \item[\faCircle]
    \else
  \item
    \fi
    \(f(0) = 0\)且\(\,f'_+(0)\)存在
  \item \(f(0) = 1\)且\(\,f'_+(0)\)存在
  \end{itemize}

  \ifshowsol
  注意到
  \begin{equation}
    \lim_{h\to0} \frac{\,f(h^2)}{h^2} = \!\lim_{\,x\to0^+\!} \!\frac{\,f(x)}{x} = 1.
    \label{eq:B1.4.1.E8}
  \end{equation}
  因为函数连续,所以\(\!\lim\limits_{x\to0^+} f(x) = \,f(0)\).如果\(\,f(0) \ne 0\),那么式~\eqref{eq:B1.4.1.E8}中的极限发散.自相矛盾,所以~\(\,f(0) = 0\).从而有
  \begin{equation*}
    f'_+(0)
    = \!\lim_{\,x\to0^+\!} \!\frac{\,f(x)}{x} = 1.
  \end{equation*}
  \fi

\item 若函数\(f\)在点\(x_0\)处可导,则
  \begin{math}
    \lim\limits_{h\to0} \frac{\,f(x_0-h) - \,f(x_0)}{h} =
  \end{math}
  \ifshowsol
  {\setlength{\ULdepth}{.9ex}%
    \uline{\makebox[6em]{\(-\,f'(x_0)\)}}}.

  根据定理~\ref{thm:limfunccomp},有
  \begin{equation*}
    \lim_{h\to0} \frac{\,f(x_0-h) - \,f(x_0)}{h}
    = - \lim_{h\to0} \frac{\,f(x_0-h) - \,f(x_0)}{-h}
    = - \lim_{\fdx\to0} \frac{\,f(x_0+\fdx) - \,f(x_0)}{\fdx}
    = - \,f'(x_0).
  \end{equation*}
  \else
  \uline{\hspace{6em}}.
  \fi

\item 若函数\(f\)在点\(a\)处及其附近有定义,则\(f\)在点\(a\)处可导的充分必要条件是\uline{\hfill}.
  \begin{itemize}
    \renewcommand{\labelitemi}{\faCircleThin}
  \item \(\lim\limits_{h\to0} h\paren[\big]{\,f(a+1/h) - \,f(a)}\)存在
  \item \(\lim\limits_{h\to0} \frac{\,f(a+2h) - \,f(a+h)}{h}\)存在
    \ifshowsol
  \item[\faCircle]
    \else
  \item
    \fi
    \(\lim\limits_{h\to0} \frac{\,f(a) - \,f(a-h)}{h}\)存在
  \item \(\lim\limits_{h\to0} \frac{\,f(a+h) - \,f(a-h)}{h}\)存在
  \end{itemize}

  \ifshowsol
  令\(\,f(x) = \abs x,\ a = 0\).这就是选项~B和~D的共同反例.实际上,选项~B和~D只是必要不充分条件.选项~A即不充分也不必要.函数\(\arctan\abs x\)说明了不充分性,函数\(x^2\)说明了不必要性.关于选项~C,有
  \begin{equation*}
    \lim_{h\to0} \frac{\,f(a) - \,f(a-h)}{h}
    = \lim_{h\to0} \frac{\,f(a-h) - \,f(a)}{-h}
    = \lim_{\fdx\to0} \frac{\,f(a+\fdx) - \,f(a)}{\fdx}
    = \,f'(a).
  \end{equation*}
  \fi

\item 周期函数\(f\)在\(\R\)上可导,周期为\(3\).若\(\lim\limits_{x\to0} \frac{\,f(1-x)-\,f(1)}{2x} = -1\),则曲线\(y = f(x)\)在点\(\paren[\big]{4, \,f(4)}\)处的切线斜率为
  \ifshowsol
  \uline{\makebox[4em]{\(2\)}}.

  实际上,有
  \begin{equation*}
    f'(1)
    = \!\lim_{\fdx\to0} \!\frac{\,f(1+\fdx) - \,f(1)}{\fdx}
    = \lim_{x\to0} \frac{\,f(1-x) - \,f(1)}{-x}
    = -2 \lim_{x\to0} \frac{\,f(1-x)-\,f(1)}{2x}
    = 2.
  \end{equation*}
  从而\(\,f'(4) = \,f'(1) = 2\).
  \else
  \uline{\hspace{4em}}.
  \fi

\item 求曲线\(y = 1/x\)在点\(\paren{2,1/2}\)处的切线方程.

  \ifshowsol
  因为\(\,f'(2) = -1/4\),自然有
  \begin{equation*}
    y = \frac12 - \frac14 \paren{x-2}.
  \end{equation*}
  \fi

\item 求曲线\(y = 1/x\)在点\(\paren{2,1/2}\)处的法线方程.

  \ifshowsol
  因为\(\,f'(2) = -1/4\),自然有
  \begin{equation*}
    y = \frac12 + 4(x-2).
  \end{equation*}
  \fi

\item 设函数\(f\)在点\(x_0\)处可导,\(\fdy = \,f(x_0+h) - \,f(x_0)\),\(\dy\)为\(\,f\)在\(x_0\)处的微分.当\(h \to 0\)时,必有\uline{\hspace*{6em}}.
  \begin{itemize}
    \renewcommand{\labelitemi}{\faCircleThin}
  \item \(\dy\)是\(h\)的同阶无穷小量
  \item \(\fdy - \dy\)是\(h\)的同阶无穷小量
  \item \(\dy\)是\(h\)的高阶无穷小量
    \ifshowsol
  \item[\faCircle]
    \else
  \item
    \fi
    \(\fdy - \dy\)是\(h\)的高阶无穷小量
  \end{itemize}

  \ifshowsol
  若\(\,f'(x_0) \ne 0\),则\(\dy\)是\(h\)的高阶无穷小量;若\(\,f'(x_0) = 0\),则\(\dy\)是\(h\)的同阶无穷小量.因此,选项~A和~C都不对.
  \fi

\item 设函数\(f\)在点\(x_0\)处可导,当自变量由\(x_0\)增至\(x_0+\fdy\)时记函数\(f\)的增量为\(\fdy\),\(\dy\)为\(f\)在\(x_0\)处的微分.求\(\!\lim\limits_{\fdx\to0} \!\frac{\fdy - \dy}{\fdx}\).

  \ifshowsol
  显然等于零.
  \fi
\end{enumerate}
\fi

\section{导数与微分的运算}

\begin{theorem*}
  \label{thm:deriv4ops}
  若函数\(\,f\mkern2mu\)和\(\mkern1mu g\)均可导,则有:
  \begin{enumerate}
    \renewcommand{\labelenumi}{\enumparen{\arabic{enumi}}}
  \item \(\paren{\,f \pm \,g}' = \,f' \pm \,g'\);
  \item \(\paren{\,fg}' = \,f'g + \,fg'\);
  \item \(\paren{\,f/g}' = \frac{\,f'g - \,fg'\!}{g^2} = \frac{\,f'\!}{g} - \frac{\,fg'\!}{g^2}\),当定义域在\(\Set{x \mid g(x) \ne 0}\)上时.
  \end{enumerate}
  % TODO: Complete the proof
\end{theorem*}

\begin{example*}
  求正切函数\(\tan x\)的导函数.
  \begin{equation*}
    \paren{\tan x}'
    = \paren*{\frac{\sin x}{\cos x}}'
    = \frac{\sin' x \cos x - \sin x \cos' x}{\cos^2 x}
    = \frac{\cos^2 x + \sin^2 x}{\cos^2 x}
    = \frac{1}{\cos^2 x}
    = \sec^2 x.
  \end{equation*}
\end{example*}

\begin{example*}
  求余切函数\(\cot x\)的导函数.
  \begin{equation*}
    \paren{\cot x}'
    = \paren*{\frac{\cos x}{\sin x}}'
    = \frac{\cos' x \sin x - \cos x \sin' x}{\sin^2 x}
    = \frac{-\sin^2 x - \cos^2 x}{\sin^2 x}
    = -\frac{1}{\sin^2 x}
    = -\csc^2 x.
  \end{equation*}
\end{example*}

\begin{example*}
  求正割函数\(\sec x\)的导函数.
  \begin{equation*}
    \paren{\sec x}'
    = \paren[\bigg]{\frac{1}{\cos x}}'
    = -\frac{\cos' x}{\cos^2 x}
    = \frac{\sin x}{\cos^2 x}
    = \sec x \tan x.
  \end{equation*}
\end{example*}

\begin{example*}
  求余割函数\(\csc x\)的导函数.
  \begin{equation*}
    \paren{\csc x}'
    = \paren*{\frac{1}{\sin x}}'
    = -\frac{\sin' x}{\sin^2 x}
    = -\frac{\cos x}{\sin^2 x}
    = -\csc x \cot x.
  \end{equation*}
\end{example*}

% https://qr.ae/p2rr3G
% https://en.wikipedia.org/wiki/Vieta%27s_formulas
\begin{example*}
  求函数
  \begin{equation*}
    f(x) = \prod_{k=0}^n \paren{x + k}
  \end{equation*}
  在原点处的导数.

  \begin{remark}
    展开后,只需关注最低次项即可.有
    \begin{gather*}
      f(x) = \sum_{k=1}^{n+1} c_k\,x^k,
      \shortintertext{从而}
      f'(x) = c_1 + \sum_{k=2}^{n+1} c_k\,kx^{k-1}.
    \end{gather*}
    所以有\(\,f'(0) = c_1 = n!\).
  \end{remark}
\end{example*}

\begin{theorem}[复合函数的导数]
  \label{thm:derivcomp}
  若函数\(\,f\mkern2mu\)和\(\mkern1mu g\)都可导,则\(\paren*{\,f\mkern1mu \circ g}' = \paren*{\,f'\mkern-4mu \circ g}\,g'\).

  \begin{proof}
    根据定理~\ref{thm:diffequivderiv},只需证明\(\diff\paren*{\,f\mkern1mu \circ g} =  \paren{\,f'\mkern-4mu \circ g}\,g' \dx\)即可.令\(u = g(x)\),有
    \begin{align}
      \fwdf\,f(u)
      &= f'(u) \fdu + \littleop{\fdu},
        \label{eq:derivcomp1} \\
      \fdu
      &= g'(x) \fdx + \littleop{\fdx}.
        \label{eq:derivcomp2}
    \end{align}
    把式~\eqref{eq:derivcomp2}代入式~\eqref{eq:derivcomp1}就得到
    \begin{equation*}
      \begin{split}
        \fwdf\paren{\,f\mkern1mu \circ g}(x)
        % &= \paren{\,f\mkern1mu \circ g}(x+\fdx) - \paren{\,f\mkern1mu \circ g}(x) \\
        % &= f\mkern2mu\paren*{\mkern1mu g(x) + \fdu} - \,f\mkern2mu\paren*{\mkern1mu g(x)} \\
        &= \fwdf\,f(u) \\
        &= f'(u) \fdu + \littleop{1} \fdu \\
        &= f'\paren*{\mkern1mu g(x)}\,g'(x) \fdx + f'\paren*{\mkern1mu g(x)} \littleop{\fdx} \\
        &\qquad + \littleop{1}\,g'(x) \fdx + \littleop{1} \littleop{\fdx} \\
        % &= f'\paren*{\mkern1mu g(x)}\,g'(x) \fdx + \littleop{\fdx} \\
        &= \paren*{(\,f'\mkern-4mu \circ g)\,g'}(x) \fdx + \littleop{\fdx}.
        \qedhere
      \end{split}
    \end{equation*}
  \end{proof}
  \begin{proof}
    令\(u = g(x)\).当函数\(g\)在点\(x\)处存在一个去心邻域使得它在此邻域上都不等于\(g(x)\)时,自然有
    \begin{equation*}
      \lim_{\fdx\to0} \frac{\Fn\Delta\paren{\,f\mkern1mu \circ g}(x)}{\fdx}
      = \lim_{\fdx\to0} \frac{\Fn\Delta\paren{\,f\mkern1mu \circ g}(x)}{\fwdf g(x)}
      \cdot \frac{\fwdf g(x)}{\fdx}
      = \lim_{\fdu\to0}\frac{\fwdf\,f(u)}{\fdu} \lim_{\fdx\to0} \frac{\fwdf g(x)}{\fdx}
      = \paren*{(\,f'\mkern-4mu \circ g)\,g'}(x).
    \end{equation*}
    当函数\(g\)在点\(x\)处不存在一个去心邻域使得它在此邻域上都不等于\(g(x)\)时,能推出\(g'(x) = 0\).再由定理~\ref{thm:limequiv}和~\ref{thm:seqcvgsubseq}可知,\(\paren{\,f\mkern1mu \circ g}'(x) = 0 = f'\paren*{\mkern1mu g(x)}\,g'(x) = \paren*{(\,f'\mkern-4mu \circ g)\,g'}(x)\).
  \end{proof}
\end{theorem}

\begin{example*}
  求幂函数\(f(x) = x^a\)的导函数,其中\(a\)是任意非零实数.
  \begin{equation*}
    f'(x)
    = \paren{x^a}'
    = \paren{\exp\ln x^a}'
    = \paren[\big]{\expb{a \ln x}}'
    = \expb{a \ln x} \cdot \frac{a}{x}
    = x^a \cdot \frac ax
    = ax^{a-1}.
  \end{equation*}
\end{example*}

\begin{example*}
  求双曲正弦函数的导函数.
  \begin{equation*}
    \sinh' x
    = \paren[\bigg]{\frac{e^x - e^{-x}}{2}}'
    = \frac{e^x + e^{-x}}{2}
    = \cosh x.
  \end{equation*}
\end{example*}

\begin{example*}
  求双曲余弦函数的导函数.
  \begin{equation*}
    \cosh' x
    = \paren[\bigg]{\frac{e^x+e^{-x}}{2}}'
    = \frac{e^x - e^{-x}}{2}
    = \sinh x.
  \end{equation*}
\end{example*}

\begin{example*}
  用链式法则求反双曲正弦函数的导函数.
  \begin{equation*}
    \arcsinh' x
    = \paren[\Big]{\ln\paren[\big]{x + \sqrt{\smash{x^2} + 1}}}'
    = \frac{1}{x + \sqrt{\smash{x^2} + 1}} \paren[\Bigg]{1 + \frac{2x}{2\sqrt{x^2 + 1}}}
    = \frac{1}{\negthickspace\sqrt{x^2 + 1}}.
  \end{equation*}
\end{example*}

% https://doi.org/10.1016/B978-0-08-013473-4.50008-6 (170)
\begin{theorem*}[一阶微分形式的不变性]
  一阶微分的形式具有不变性.

  \begin{proof}
    应用定理~\ref{thm:diffequivderiv}和~\ref{thm:derivcomp}即可得证.
  \end{proof}
\end{theorem*}

\begin{theorem}
  \label{thm:derivinv}
  若函数\(f\)在点\(x_0\)处可导且\(f'(x_0) \ne 0\),则其反函数\(\,f^{-1}\)在点\(f(x_0)\)处可导且
  \begin{equation*}
    \dd[f^{-1}]y \Bigg\vert_{\,f(x_0)} \mkern-11mu = \frac1{\,f'(x_0)}.
  \end{equation*}

  \begin{proof}
    令\(x = f^{-1}(y)\).根据定义~\ref{defn:funcinv}和定理~\ref{thm:limfunccomp}和~\ref{thm:deriv2cont},有
    \begin{equation*}
      \begin{split}
        \dd[f^{-1}]y \Bigg\vert_{\,f(x_0)} \mkern-11mu
        &= \lim_{\,y\to y_0\!} \frac{\,f^{-1}(y) - \,f^{-1}(y_0)}{y - y_0}
        = \lim_{\,y\to y_0\!} \frac{\,f^{-1}(y) - \,f^{-1}(y_0)}{\,f\paren[\big]{\,f^{-1}(y)} - \,f\paren[\big]{\,f^{-1}(y_0)}} \\
        &= \lim_{\,x\to x_0\!} \frac{x - x_0}{\,f(x) - \,f(x_0)}
        = \frac{1}{\,f'(x_0)}.
        \qedhere
      \end{split}
    \end{equation*}
  \end{proof}
\end{theorem}

\begin{example*}
  求反正弦函数的导函数.
  \begin{equation*}
    \arcsin'x
    = \frac1{\sin'\arcsin x}
    = \frac1{\cos\arcsin x}
    = \frac1{\negthickspace\sqrt{1 - \sin^2\arcsin x}}
    = \frac1{\negthickspace\sqrt{1 - x^2}}.
  \end{equation*}
\end{example*}

\begin{example*}
  求反正切函数的导函数.
  \begin{equation*}
    \arctan'x
    = \frac1{\tan'\arctan x}
    = \frac1{\sec^2\arctan x}
    = \frac1{1 + \tan^2\arctan x}
    = \frac1{1 + x^2}.
  \end{equation*}
\end{example*}

\begin{example*}
  求反余弦函数的导函数.
  \begin{equation*}
    \arccos'x
    = \frac1{\cos'\arccos x}
    = - \frac1{\sin\arccos x}
    = - \frac1{\negthickspace\sqrt{1 - \cos^2\arccos x}}
    = - \frac1{\negthickspace\sqrt{1 - x^2}}.
  \end{equation*}
\end{example*}

\subpdfbookmark{思考}{B1.4.2.P}
\subsection*{思考}

\begin{enumerate}
\item 函数\(f\)与其反函数\(f^{-1}\)的导数在每一点处的乘积等于\(1\).这个结论对吗?

  \ifshowsolp
  不对.函数\(f(x) = x^3\!\)与其反函数\(f^{-1}(x) = \sqrt[\leftroot{-2}\uproot{2}3]x\)在点\(2\)处的导数分别是\(12\)和\(\sqrt[\leftroot{-2}\uproot{2}3]{2}\big/{6}\),显然它们的乘积\(2\sqrt[\leftroot{-2}\uproot{2}3]2 > 1\).
  \fi

\item 在复合函数的求导法则\(\paren{\,f\mkern1mu \circ g}' = \paren{\,f'\mkern-4mu \circ g}\,g'\)中,\(\paren{\,f\mkern1mu \circ g}'\)与\(f'\mkern-4mu \circ g\)的含义有什么不同?若令\(u = g(x)\),能得到\(\,f'(u) = f'(u)\,u'\)吗?

  \ifshowsolp
  前者是先复合再求导,后者是先求导再复合.不能.
  \fi
\end{enumerate}

\ifshowex
\currentpdfbookmark{练习}{B1.4.2.E}
\subsection*{练习}

\begin{enumerate}
\item 设曲线\(y = x^3 + ax\)与曲线\(y = bx^2 + c\)在点\(\paren{-1,0}\)处相切.求常数\(a,\ b,\ c\).

  \ifshowsol
  解方程组
  \begin{equation*}
    \left\{
      \begin{aligned}
        -1 - a &= 0 \\
        b + c &= 0 \\
        3 + a &= -2b
      \end{aligned}
    \right.
  \end{equation*}
  得到\(a = b = -1,\ c = 1\).
  \fi

\item 求函数\(\ln\ln x\)的导函数.
  \ifshowsol
  \begin{equation*}
    (\ln\ln x)'
    = \frac{\ln' x}{\ln x}
    = \frac{1}{x \ln x}.
  \end{equation*}
  \fi

\item 求函数\(2\cos3x\)的导函数.
  \ifshowsol
  \begin{equation*}
    (2\cos3x)' = 2(\cos3x)' = 2(-\sin3x)(3x)' = -6\sin3x.
  \end{equation*}
  \fi

\item 设函数\(f\)可导.求函数\(f\mkern1mu \circ f\mkern1mu \circ f\)的导函数.
  \ifshowsol
  \begin{equation*}
    (\,f\mkern1mu \circ f\mkern1mu \circ f)'
    = (\,f'\mkern-4mu \circ f\mkern1mu \circ f) (\,f\mkern1mu \circ f)'
    = (\,f'\mkern-4mu \circ f\mkern1mu \circ f) (\,f'\mkern-4mu \circ f) \,f'.
  \end{equation*}
  \fi

\item 求函数\(\dfrac{\arcsin x}{x}\)的导函数.
  \ifshowsol
  \begin{equation*}
    \paren[\bigg]{\frac{\arcsin x}{x}}'
    = \frac{x\big/\!\sqrt{1-\smash{x^2}} - \arcsin x}{x^2}
    = \frac{1}{x \sqrt{1-\smash{x^2}}} - \frac{\arcsin x}{x^2}
  \end{equation*}
  \fi

\item 求函数\(1/\!\ln x\)的导函数.
  \ifshowsol
  \begin{equation*}
    \paren[\bigg]{\frac1{\ln x}}'
    = - \frac{\ln'x}{\operatorname{ln^2} x}
    = - \frac{1}{x \operatorname{ln^2} x}.
  \end{equation*}
  \fi

\item 求函数\(\ln\sin x\)的微分.
  \ifshowsol
  \begin{equation*}
    \diff(\ln\sin x)
    = \frac1{\sin x} \diff(\sin x)
    = \frac{\cos x}{\sin x} \dx
    = \cot x \dx
  \end{equation*}
  \fi

\item 求函数\(\operatorname{ln^3} x^2\)的微分.
  \ifshowsol
  \begin{equation*}
    \diff\paren{\operatorname{ln^3} x^2}
    = 3\operatorname{ln^2} x^2 \diff\paren{\ln x^2}
    = 3\operatorname{ln^2} x^2 \cdot \frac1{x^2} \diff\paren{x^2}
    = 3\operatorname{ln^2} x^2 \cdot \frac1{x^2} \cdot 2x \dx
    = \frac{6 \operatorname{ln^2} x^2}{x} \dx.
  \end{equation*}
  \fi

\item 设函数\(\,f(x) = -2x^2\)且\(\,f'(a) = 4\).求常数\(a\).

  \ifshowsol
  因为\(f'(x) = -4x\),所以\(f'(a) = -4\,a = 4\),即\(a = -1\).
  \fi

\item 函数\(f\)满足\(y = f\,\paren[\Big]{\frac{x+1}{x-1}} \)和\(\,f'(x) = \arctan\sqrt x\).求\(\dydx\bigg\vert_{2}\).
  \ifshowsol
  \begin{equation*}
    \dydx\Bigg\vert_{2}
    = f'(3) \cdot \paren[\bigg]{\frac{x+1}{x-1}}'\Bigg\vert_{2}
    = \arctan\sqrt3 \cdot \paren[\bigg]{-\frac{2}{(x-1)^2}} \Bigg\vert_{2}
    = - \frac23 \pi.
  \end{equation*}
  \fi

\item 设函数\(\,f(x) = \prod\limits_{k=1}^n \paren{e^{kx}-k}\),其中\(n\)是正整数.求\(\,f'(0)\).

  下面给出两种方法.
  \begin{remark}[方法一]
    因为\(\,f(0) = 0\),所以有
    \begin{equation*}
      \begin{split}
        f'(0)
        &= \lim_{\fdx \to 0} \frac{\,f(\fdx)}{\fdx}
        = \lim_{\fdx \to 0} \frac{e^x-1}{\fdx} \prod_{k=2}^n \paren{e^{k\fdx}-k} \\
        &= \prod_{k=2}^n \paren{1-k}
        = (-1)^{n-1} \prod_{k=1}^{n-1} \paren{k} \\
        &= (-1)^{n-1} (n-1)!\,.
      \end{split}
    \end{equation*}
  \end{remark}
  \begin{remark}[方法二]
    这是对乘积形式的式子常见的处理技巧.因为
    \begin{equation*}
      \begin{split}
        f'(x)
        &= f(x) \,\ddx \ln\,f(x) \\
        &= f(x) \,\ddx \ln\prod_{k=1}^n \paren{e^{kx}-k} \\
        &= f(x) \,\ddx \sum_{k=1}^n \lnp{e^{kx}-k} \\
        &= f(x) \sum_{k=1}^n \ddx \lnp{e^{kx}-k} \\
        &= f(x) \sum_{k=1}^n \frac{ke^{kx}}{e^{kx}-k} \\
        &= \sum_{k=1}^n ke^{kx} \,\smashoperator{\prod_{\substack{1 \le j \le n \\ j \ne k}}}
        \,\paren{e^{jx}-j}.
      \end{split}
    \end{equation*}
    所以有
    \begin{equation*}
      f'(0)
      = \sum_{k=1}^n \prod_{\substack{1 \le j \le n \\ j \ne k}} \!\paren{1-j}
      = \,\smashoperator{\prod_{\substack{1 \le j \le n \\ j \ne 1}}} \,\paren{1-j}
      = (-1)^{n-1} (n-1)!\,.
    \end{equation*}
  \end{remark}
\end{enumerate}
\fi

\section{几种特殊函数的求导法、高阶导数}

\begin{theorem*}
  设有方程\(f(x,y) = 0\).把\(x\)看成自变量,把\(y\)看成因变量,则由此方程确定的函数\(y\)在点\(\paren{x_0, \,y_0}\)处关于\(x\)的导数满足
  \begin{equation*}
    \diff f(x,y) = \pdpdx\,f(x,y) \dx + \pdpdy\,f(x,y) \dy.
  \end{equation*}
\end{theorem*}

\begin{example*}
  设\(y\)是\(x\)由方程\(x = y + \varepsilon \sin y\)确定的一个函数,其中\(0 < \varepsilon < 1\).求\(y'\).

  \begin{remark}
    因为
    \begin{equation*}
      \dx
      = \diff(y + \varepsilon \sin y)
      = \dy + \varepsilon \cos y \dy
      = (1 + \varepsilon \cos y) \dy,
    \end{equation*}
    所以
    \begin{equation*}
      \dydx = \frac1{1 + \varepsilon \cos y}.
    \end{equation*}
  \end{remark}
\end{example*}

\begin{example*}
  设\(y\)是\(x\)由\(e^{xy} + \tan xy = y\)确定的一个函数.求\(y'(0)\).

  \begin{remark}
    当\(x = 0\)时,有
    \begin{math}
      e^0 + \tan 0 = 1 = y.
    \end{math}
    又因为
    \begin{gather*}
      \begin{split}
        \dy
        &= \diff(e^{xy} + \tan xy) \\
        &= e^{xy} \paren{y \dx + x \dy} + \paren{y \dx + x \dy} \sec^2xy \\
        &= \paren{y \dx + x \dy} \paren{e^{xy} + \sec^2xy},
      \end{split}
      \shortintertext{所以}
      \dydx = \paren[\bigg]{y + x \dydx} \paren{e^{xy} + \sec^2xy}, \\
      \shortintertext{从而}
      \dydx\Bigg\vert_{\substack{x = 0 \\ y = 1}} \mkern-11mu
      = (1 + 0)(e^0 + \sec^2 0)
      = 2.
    \end{gather*}
  \end{remark}
\end{example*}

% https://mathcurve.com/courbes2d.gb/foliumdedescartes/foliumdedescartes.shtml
% https://mathworld.wolfram.com/FoliumofDescartes.html
% https://en.wikipedia.org/wiki/Folium_of_Descartes
\begin{example*}
  求曲线\(x^3 + y^3 = 9xy\)在点\(\paren{2,4}\)处的切线方程和法线方程.

  \begin{remark}
    因为
    \begin{equation*}
      \diff(x^3+y^3)
      = 3x^2 \dx + 3y^2 \dy
      = 9(y \dx + x \dy)
      = \diff(9xy),
    \end{equation*}
    所以当\(x = 2,\ y = 4\)时就有
    \begin{gather*}
      12\dx + 48\dy
      = 9(4\dx + 2\dy)
      = 36\dx + 18\dy
      \shortintertext{即}
      \dydx\Bigg\vert_{\substack{x = 2 \\ y = 4}} \mkern-10mu
      = \frac{36-12}{48-18}
      = \frac{4}{5}.
    \end{gather*}
    因此,这条曲线在此处的切线方程和法线方程分别是
    \begin{equation*}
      y =  \frac45 x + \frac{12}{5}
      \txt{和}
      y = - \frac54 x + \frac{13}{2}.
    \end{equation*}
  \end{remark}
\end{example*}

\begin{theorem*}[参数方程的求导法]
  设参数方程
  \begin{equation*}
    x = x(t), \quad
    y = y(t).
  \end{equation*}
  若函数\(\fdx\)在点\(t_0\)处连续且在此处附近异号,则
  \begin{equation*}
    \dydx\Bigg\vert_{t=t_0} \mkern-15mu
    = \lim_{t\to t_0} \frac{\fdy(t)}{\fdx(t)}.
  \end{equation*}
  特别地,若函数\(x\)在点\(t_0\)处可导且\(x'(t_0) \ne 0\),则\(y\)在点\(t = t_0\)处关于\(x\)的导数是
  \begin{equation*}
    \dydx\Bigg\vert_{t=t_0} \mkern-15mu
    = \frac{\dy/\!\dt}{\dx/\!\dt}\Bigg\vert_{t=t_0} \mkern-22mu.
  \end{equation*}
  % TODO: Add proof
\end{theorem*}

\begin{example*}
  求曲线\(r = 1 + \cos\theta\)在\(\theta = \pi/4\)处的切线方程.

  \begin{remark}
    因为\(x = r \cos\theta = \cos\theta + \cos^2\theta\)和\(y = \sin\theta + \cos\theta \sin\theta\),所以有
    \begin{equation*}
      \dydx
      = \frac{\dy/\!\dtheta}{\dx/\!\dtheta}
      = \frac{\cos\theta + \cos2\theta}{-\sin\theta - \sin2\theta}
      = - \frac{\cos\theta + \cos2\theta}{\sin\theta + \sin2\theta}.
    \end{equation*}
    因此,当\(\theta = \pi/4\)时有
    \begin{equation*}
      \dydx\Bigg\vert_{\theta=\frac\pi4} \mkern-11mu
      = - \frac{1/{\scriptstyle\sqrt2}}{1/{\scriptstyle\sqrt2} + 1}
      = - \frac{1}{1 + {\scriptstyle\sqrt2}}
      = 1 - \sqrt2.
    \end{equation*}
  \end{remark}
\end{example*}

\begin{example*}
  设参数方程\(x = 2t + \abs t,\ y = 5t^2 + 3t\abs t\).求\(y\)在点\(t = 0\)处关于\(x\)的导数.
  \begin{equation*}
    \dydx\Bigg\vert_{t=0} \mkern-11mu
    = \frac{\dy/\!\dt}{\dx/\!\dt} \Bigg\vert_{t=0} \mkern-11mu
    = \lim_{t\to0} \frac{5t^2 + 3t\abs t}{2t + \abs t}
    = 0.
  \end{equation*}
\end{example*}

\begin{theorem*}[对数求导法]
  函数\(f\)在非零点处满足
  \begin{equation*}
    f'(x) = f(x) \,\ddx \ln\,f(x).
  \end{equation*}
\end{theorem*}

\begin{example*}
  求函数\(\,f(x) = x^{x^2}\)的导函数.
  \begin{equation*}
    f'(x)
    = f(x) \,\ddx \ln\,f(x)
    = x^{x^2} \paren{2x\ln x + x}.
  \end{equation*}
\end{example*}

\begin{example*}
  求函数
  \begin{equation*}
    f(x) = \frac{\prod\limits_{k=1}^n g_k(x)}{\prod\limits_{k=1}^n h_k(x)}
  \end{equation*}
  的导函数.
  \begin{equation*}
    f'(x)
    = f(x) \,\ddx \ln\,f(x)
    = \paren[\Bigg]{\prod_{k=1}^n \frac{\ g_k(x)}{h_k(x)}}
    \paren[\Bigg]{\sum_{k=1}^n \paren[\bigg]{\frac{g_k'(x)}{g_k(x)} - \frac{h_k'(x)}{h_k(x)}}}.
  \end{equation*}
\end{example*}

\begin{example*}
  求函数\(f(x) = \sqrt{\frac{(x-1)(x-2)}{(x-3)(x-4)}}\)的导函数.
  \begin{remark}
    当\(x \in \paren{-\infty,1} \cup \paren{2,3} \cup \paren{4,+\infty}\)时,有
    \begin{equation*}
      f'(x)
      = f(x) \,\ddx \ln\,f(x)
      = \frac12
      \sqrt{\frac{(x-1)(x-2)}{(x-3)(x-4)}}
      \paren*{\frac1{x-1} + \frac1{x-2} - \frac1{x-3} - \frac1{x-4}}.
    \end{equation*}
  \end{remark}
\end{example*}

\begin{definition*}[高阶导数]
  \label{defn:derivhio}
  令\(\,f^{(0)} = \dd[^0f]{x^0} = f\).当\(n > 1\)时,令
  \begin{equation*}
    f^{(n)} = \paren[\big]{\,f^{(n-1)}}'
    \txt{和}
    \dd[^nf]{x^n} = \ddx \dd[^{n-1}f]{x^{n-1}\negthickspace}.
  \end{equation*}
  我们把\(\,f^{(n)}\)称为\(n\)阶导数,把大于一阶的导数称为高阶导数.
\end{definition*}

\begin{theorem*}[莱布尼茨公式]
  \begin{equation*}
    (\,fg)^{(n)} = \sum_{k=0}^n \binom nk \,f^{(n-k)} g^{(k)}.
  \end{equation*}
\end{theorem*}

\begin{example*}
  求函数\(\,f(x) = a^x\)的\(n\)阶导函数.
  \begin{equation*}
    f^{(n)}(x) = a^x \ln^n a.
  \end{equation*}
\end{example*}

\begin{example*}
  求函数\(\,f(x) = \frac1{1+x}\)的\(n\)阶导函数.
  \begin{equation*}
    f^{(n)}(x) = (-1)^n \frac{n!}{(1+x)^{n+1}}.
  \end{equation*}
\end{example*}

\begin{example*}
  求函数\(\,f(x) = \lnp{1+x}\)的\(n\)阶导函数.
  \begin{equation*}
    f^{(n)}(x) = (-1)^{n-1} \frac{(n-1)!}{(1+x)^n}.
  \end{equation*}
\end{example*}

\begin{example*}
  求正弦函数的\(n\)阶导函数.
  \begin{equation*}
    \sin^{(n)}x = \sinp{x + {n\pi}/{2}}.
  \end{equation*}
\end{example*}

\begin{example*}
  求余弦函数的\(n\)阶导函数.
  \begin{equation*}
    \cos^{(n)}x = \cosp{x + {n\pi}/{2}}.
  \end{equation*}
\end{example*}

\begin{example*}
  求函数\(y = \frac{x}{1+x^2}\)的\(n\)阶导函数.

  \begin{remark}
    因为\(x = (1+x^2)\,y\),所以有
    \begin{gather*}
      \begin{split}
        \dd[^n]{x^n} (1+x^2)\,y
        &= \sum_{k=0}^n \binom nk (1+x^2)^{(n-k)} y^{(k)} \\
        &= n(n-1)y^{(n-2)} + 2nxy^{(n-1)} + (1+x^2)\,y^{(n)} \\
        &=
        \begin{cases}
          1, & n = 1, \\
          0, & n > 1,
        \end{cases}
      \end{split}
      \shortintertext{即}
      y^{(n)} =
      \begin{cases}
        \frac{1 - 2xy}{1+x^2}, & n = 1, \\
        - \frac1{1+x^2}\paren[\big]{n(n-1)y^{(n-2)} + 2nxy^{(n-1)}}, & n > 1.
      \end{cases}
    \end{gather*}
  \end{remark}
\end{example*}

\begin{example*}
  求函数\(y = \frac1{(1+x)(2+3x)}\)的\(n\)阶导函数.

  \begin{remark}
    因为
    \begin{gather*}
      y = \frac1{(1+x)(2+3x)} = \frac{3}{2+3x} - \frac1{1+x}, \\
      \shortintertext{所以}
      y^{(n)} = (-1)^n \frac{3^{n+1}n!}{(2+3x)^{n+1}} - (-1)^n \frac{n!}{(1+x)^{n+1}}.
    \end{gather*}
  \end{remark}
\end{example*}

\begin{example*}
  求函数\(y = x^2 \sin x\)的\(n\)阶导函数.
  \begin{equation*}
    \begin{split}
      \dd[^n]{x^n} x^2 \sin x
      &= \sum_{k=0}^n \binom nk (x^2)^{(n-k)} \sin^{(k)}x \\
      &= n(n-1) \sinp[\Big]{x+(n-2)\frac{\pi}{2}}
      + 2nx \sinp[\Big]{x+(n-1)\frac{\pi}{2}}
      + x^2 \sinp[\Big]{x+\frac{n\pi}{2}}.
    \end{split}
  \end{equation*}
\end{example*}

\subpdfbookmark{思考}{B1.4.3.P}
\subsection*{思考}

\begin{enumerate}
  % https://doi.org/10.1016/B978-0-08-013473-4.50008-6 (180)
\item 把函数\(y = \,f(x)\)的二阶微分定义成\(\diff^2y = \,f^{(2)}(x) \dx^2\).一阶微分具有形式不变性,二阶微分是否也有形式不变性?

  \ifshowsolp
  没有.令\(y = \,f(u),\ u = g(x)\),有
  \begin{gather*}
    \diff_u^2 y
    = f^{(2)}(u) \du^2
    = f^{(2)}\paren[\big]{g(x)} \paren[\big]{g'(x)}{}^2 \dx^2, \\
    \diff_x^2 y
    = \paren[\Big]{\,f^{(2)}\paren[\big]{g(x)} \paren[\big]{g'(x)}^2 + f'\paren[\big]{g(x)}\,g^{(2)}(x)} \dx^2.
  \end{gather*}
  \fi

\item 由参数方程\(x = \varphi(t),\ y = \psi(t)\)确定的函数\(y = y(x)\)的求导公式\(\dy/\!\dx = \psi'(t)/\varphi'(t)\)是如何确定的?如何求二阶导数\(\diff^2y/\!\dx^2\)和三阶导数\(\diff^3y/\!\dx^3\)?

  \ifshowsolp
  确定的方式多种多样,最简单的方式就是把\(t\)看成\(x\)的函数,然后根据定理~\ref{thm:derivcomp}和~\ref{thm:derivinv}就能得到.使用同样的方法,能得到
  \begin{gather*}
    \dd[^2y]{x^2}
    = \ddt \frac{\psi'}{\varphi'} \cdot \dd[t]x
    = \frac{\psi^{(2)}}{\paren{\varphi'}^2}
    - \frac{\psi'\varphi^{(2)}}{\paren{\varphi'}^3}, \\
    \siand
    \begin{split}
      \ddn[y]x3
      &= \ddt \paren[\Bigg]{\frac{\psi^{(2)}}{\paren{\varphi'}^2}
        - \frac{\psi'\varphi^{(2)}}{\paren{\varphi'}^3}} \cdot \dd[t]x \\
      &= \frac{\psi^{(3)}}{(\varphi')^3}
      - \frac{\psi^{(2)}\varphi^{(2)}
        + \psi'\varphi^{(3)}
        + 2\psi^{(2)}\varphi^{(2)}}{(\varphi')^4}
      + \frac{3\psi'(\varphi^{(2)})^2}{(\varphi')^5}.
    \end{split}
  \end{gather*}
  \fi
\end{enumerate}

\ifshowex
\currentpdfbookmark{练习}{B1.4.3.E}
\subsection*{练习}

\begin{enumerate}
\item 设\(x\)的函数\(y\)由方程\(e^{x+y} = xy + 1\)确定.求\(\dy\).
  \ifshowsol
  \begin{equation*}
    e^{x+y} \dx + e^{x+y} \dy
    = y \dx + x \dy
    \implies
    \dy = \frac{y-e^{x+y}}{e^{x+y}-x} \dx.
  \end{equation*}
  \fi

\item 设函数\(f\)二阶可导且\(y = \,f(x^2)\).求\(y^{(2)}\).
  \ifshowsol
  \begin{equation*}
    y^{(2)}
    = \paren[\big]{\,f'(x^2) \cdot 2x}'
    = 2\,f'(x^2) + 4x^2\,f^{(2)}(x^2).
  \end{equation*}
  \fi

\item 设参数方程\(x = a \cos t,\ y = b \sin t\).求\(\ddn[y]x2\).
  \ifshowsol
  \begin{gather*}
    \dydx = - \frac{b \cos t}{a \sin t}, \\[1ex]
    \ddn[y]x2
    = - \ddt \frac{b \cos t}{a \sin t} \cdot \dd[t]x
    = - \frac{b}{a^2 \sin t} - \frac{b \cos^2 t}{a^2 \sin^3 t}
    = - \frac{b}{a^2} \csc^3 t.
  \end{gather*}
  \fi

\item 设参数方程\(x = \sin t,\ y = t \sin t + \cos t\).求\(\ddn[y]x2 \bigg\vert_{t=\frac\pi4}\mkern-24mu\).
  \ifshowsol
  \begin{equation*}
    \dydx = \frac{\sin t + t \cos t - \sin t}{\cos t} = t,
    \quad
    \ddn[y]x2 = \ddt t \cdot \dd[t]x = \sec t,
    \quad
    \ddn[y]x2 \Bigg\vert_{t=\frac\pi4} \mkern-11mu = \sqrt2.
  \end{equation*}
  \fi

\item 设\(x\)的函数\(y\)由方程\(y - 2x = (x-y) \lnp{x-y}\)确定.求\(\ddn[y]x2\).
  \ifshowsol
  \begin{gather*}
    -2\dx + \dy
    = \paren[\big]{\ln(x-y) + 1} \dx - \paren[\big]{\ln(x-y) + 1} \dy, \\[1ex]
    \dydx = \frac{\ln(x-y)+3}{\ln(x-y)+2} = 1 + \frac{1}{\ln(x-y)+2}, \\[1ex]
    \ddn[y]x2
    = - \frac{1-y'}{(x-y)\paren[\big]{\ln(x-y)+2}{}^2}
    = \frac{1}{(x-y)\paren[\big]{\ln(x-y)+2}{}^3}.
  \end{gather*}
  \fi

\item 设\(x\)的函数\(y\)由方程\(xy + e^y = x + 1\)确定.求\(\dy\Big\vert_{x=0}\)和\(\ddn[y]x2\bigg\vert_{x=0}\mkern-22mu\).
  \ifshowsol
  \begingroup
  \addtolength{\jot}{1ex}
  \begin{gather*}
    y \dx + x \dy + e^y \dy = \dx, \\
    \dydx = \frac{1-y}{x+e^y}, \\
    \dy\Big\vert_{x=0} = \dy\Big\vert_{\substack{x=0 \\ y =0}} = \dx, \\
    \ddn[y]x2 = \frac{-y'}{x+e^y} - \frac{(1-y)(1+e^yy')}{(x+e^y)^2}, \\
    \ddn[y]x2\Bigg\vert_{x=0}
    = \ddn[y]x2\Bigg\vert_{\substack{x=0 \\ y =0}}
    = -1 - 2 = -3.
  \end{gather*}
  \endgroup
  \fi

\item 求函数\(\,f(x) = x \sin 2x\)的\(5\)阶导函数.
  \ifshowsol
  \begin{equation*}
    f^{(5)}(x)
    = 5 \,\ddn x4 \sin 2x + x \,\ddn x5 \sin 2x
    = 16 (5\sin 2x + 2x \cos 2x).
  \end{equation*}
  \fi

\item 求函数\(\,f(x) = x^{\sin x}\)的导函数.
  \ifshowsol
  \begin{equation*}
    f'(x)
    = f(x) \,\ddx \ln\,f(x)
    = x^{\sin x} \,\ddx \sin x \ln x
    = x^{\sin x} \paren[\Big]{\cos x \ln x + \frac{\sin x}{x}}.
  \end{equation*}
  \fi

\item 求函数\(\,f(x) = x^x\)的导函数.
  \ifshowsol
  \begin{equation*}
    f'(x)
    = f(x) \,\ddx \ln\,f(x)
    = x^x \paren{\ln x + 1}.
  \end{equation*}
  \fi

\item 设\(x\)的函数\(y\)由方程\(x^2 + x^2y^2 + y^2 = 3\)确定.求\(\dy/\!\dx\).
  \ifshowsol
  \begin{gather*}
    2x \dx + 2xy^2 \dx + 2x^2y \dy + 2y \dy = 0, \\
    \dydx = - \frac{2x+2xy^2}{2x^2y+2y} = - \frac{x(1+y^2)}{(x^2+1)y}.
  \end{gather*}
  \fi
\end{enumerate}
\fi
