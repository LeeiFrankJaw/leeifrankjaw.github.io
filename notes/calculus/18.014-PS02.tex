\documentclass[a4paper]{article}

\title{Solutions to Problem Set~2}
\author{L. F. \textsc{Jaw}}

\usepackage[T1]{fontenc}
\usepackage{textcomp}
\usepackage{mathtools,amssymb,amsthm}
\usepackage[hmargin=1in,vmargin=1in]{geometry}
\usepackage{graphicx,xcolor}
\usepackage[pdfusetitle]{hyperref}
\hypersetup{%
  colorlinks=true,
  urlcolor=[rgb]{0,0.2,0.6},
  linkcolor={.},
  bookmarksdepth=2}
\usepackage{bookmark}
\usepackage{float}

\frenchspacing

\newcommand*{\parasp}{\setlength{\parskip}{10pt plus 2pt minus 3pt}}
\newcommand*{\noparasp}{\setlength{\parskip}{0pt plus 1pt}}
\newcommand*{\setparasp}[1]{\setlength{\parskip}{#1}}
\newcommand*{\pskip}{\vskip 10pt plus 2pt minus 3pt}
% \newcommand\LEFTRIGHT[3]{\left#1 #3 \right#2}
\newcommand\SetSymbol[1][]{%
  \nonscript\:#1\vert
  \allowbreak
  \nonscript\:
  \mathopen{}}
% \newcommand*{\paren}[1]{\LEFTRIGHT(){#1}}
\DeclarePairedDelimiterX{\paren}[1]{\lparen}{\rparen}{%
  \renewcommand{\mid}{\SetSymbol[\delimsize]}#1}
% \newcommand*{\brkt}[1]{\LEFTRIGHT[]{#1}}
\DeclarePairedDelimiterX{\brkt}[1]{\lbrack}{\rbrack}{%
  \renewcommand{\mid}{\SetSymbol[\delimsize]}#1}
\DeclarePairedDelimiterX{\brce}[1]{\lbrace}{\rbrace}{%
  \renewcommand{\mid}{\SetSymbol[\delimsize]}#1}
\newcommand*{\unit}[1]{\,\mathrm{#1}}
\newcommand*{\DeclareUnit}[1]{\expandafter\def\csname#1\endcsname{\unit{#1}}}
\DeclareUnit{cm}
% \renewcommand*{\m}{\unit{m}}
\DeclareUnit{m}
\DeclareUnit{kg}
\DeclareUnit{s}
\newcommand*{\R}{\mathbb{R}}
\newcommand*{\Z}{\mathbb{Z}}
\newcommand*{\N}{\mathbb{N}}
\newcommand*{\Q}{\mathbb{Q}}
% \newcommand*{\Rp}{(0,+\infty)}
% \newcommand*{\Rm}{(-\infty,0)}
\newcommand*{\deduce}{\mathrel{\Downarrow}}
% \newcommand*{\abs}[1]{\left\lvert #1 \right\rvert}
\DeclarePairedDelimiter{\abs}{\lvert}{\rvert}
% \newcommand*{\ceil}[1]{\left\lceil#1\right\rceil}
\DeclarePairedDelimiter{\ceil}{\lceil}{\rceil}
\DeclarePairedDelimiter{\floor}{\lfloor}{\rfloor}
\newcommand*{\textop}[1]{\mathbin{\text{#1}}}
\newcommand*{\tand}{\textop{and}}
\newcommand*{\tor}{\textop{or}}
\newcommand*{\txt}[2][\quad]{#1 \text{#2} #1}
\newcommand*{\qand}{\txt{and}}
\newcommand*{\iand}{\intertext{and}}
\newcommand*{\DeclareText}[1]{\expandafter\def\csname#1\endcsname{\text{#1}}}
\DeclareText{otherwise}
\newcommand*{\tfor}{\text{for }}
\newcommand*{\qfor}{\txt{for}}

\newcommand*{\enumparen}[1]{(\makebox[0.6em][c]{#1})}
\renewcommand{\labelenumii}{\enumparen{\theenumii}}
\newcommand*{\upstar}{\textsuperscript{\normalfont\textasteriskcentered}}%
\makeatletter
\newcommand*{\bonus}{\@itemlabel\upstar}%
\def\contitem{%
  \def\H@item{%
    \@inmatherr\item
    \@noitemargtrue
    \@ifnextchar[\@item{\@item[\@itemlabel]}}}
\makeatother

\DeclareMathOperator{\arccosh}{arccosh}
\DeclareMathOperator{\arcsinh}{arcsinh}
\DeclareMathOperator{\arctanh}{arctanh}
\DeclareMathOperator{\arccoth}{arccoth}
\DeclareMathOperator{\sech}{sech}
\DeclareMathOperator{\arcsech}{arcsech}
\DeclareMathOperator{\sgn}{sgn}
\DeclareMathOperator{\var}{var}
\DeclareMathOperator{\Ber}{Bernoulli}
\DeclareMathOperator{\Cov}{Cov}
\DeclareMathOperator{\E}{E}
\def\argmax{\qopname\relax m{arg\,max}}
\DeclarePairedDelimiterXPP{\Eb}[1]{\E}{\lbrack}{\rbrack}{}{%
  \renewcommand{\mid}{\SetSymbol[\delimsize]}#1}
\DeclarePairedDelimiterXPP{\varp}[1]{\var}{\lparen}{\rparen}{}{%
  \renewcommand{\mid}{\SetSymbol[\delimsize]}#1}
\DeclarePairedDelimiterXPP{\Covp}[1]{\Cov}{\lparen}{\rparen}{}{%
  \renewcommand{\mid}{\SetSymbol[\delimsize]}#1}
\DeclarePairedDelimiterXPP{\expp}[1]{\exp}{\lbrace}{\rbrace}{}{#1}
\newcommand*{\Fn}[1]{\mathop{\relax #1}\nolimits}
\newcommand*{\fn}[1]{\mathop{\relax\kern0pt #1}\nolimits}
\newcommand*{\gammaf}{\Fn{\Gamma}}
\renewcommand*{\Pr}{\Fn{P}}
\DeclarePairedDelimiterXPP{\Prp}[1]{\Pr}{\lparen}{\rparen}{}{%
  \renewcommand{\mid}{\SetSymbol[\delimsize]}#1}
\newcommand*{\pnorm}{\Fn{\Phi}}
\DeclarePairedDelimiterXPP{\pnormp}[1]{\pnorm}{\lparen}{\rparen}{}{#1}
\newcommand*{\dnorm}{\fn{\varphi}}
\DeclarePairedDelimiterXPP{\dnormp}[1]{\dnorm}{\lparen}{\rparen}{}{#1}
\newcommand*{\qnorm}{\Fn{\Phi}^{-1}}
%\newcommand*{\diff}{\mathop{}\!d}
\newcommand*{\diff}{\mathop{}\!\mathit{d}}
%\newcommand*{\diff}{\mathop{}\!\mathrm{d}}
\newcommand*{\dx}{\diff x}
\newcommand*{\dy}{\diff y}
\newcommand*{\dz}{\diff z}
\newcommand*{\ds}{\diff s}
\newcommand*{\dt}{\diff t}
\newcommand*{\du}{\diff u}
\newcommand*{\dv}{\diff v}
\newcommand*{\dtheta}{\diff \theta}
\newcommand*{\dd}[2][]{\frac{\diff#1}{\diff#2}}
\newcommand*{\ddx}{\frac{\diff}{\dx}}
\newcommand*{\ddt}{\frac{\diff}{\dt}}
\newcommand*{\ddy}{\dd y}
\newcommand*{\ddtheta}{\frac{\diff}{\dtheta}}
\newcommand*{\ddz}{\dd z}
\newcommand*{\fwdf}{\mathop{}\!\Delta}
\newcommand*{\dydx}{\frac\dy\dx}
\newcommand*{\pdpd}[2][]{\frac{\partial#1}{\partial#2}}
\newcommand*{\pdpdx}{\frac\partial{\partial x}}
\newcommand*{\pdpdy}{\frac\partial{\partial y}}
\newcommand*{\pdpdz}{\frac\partial{\partial z}}
\newcommand*{\pdpdu}{\frac\partial{\partial u}}
\newcommand*{\pdpdv}{\frac\partial{\partial v}}
\newcommand*{\pdpdt}{\frac\partial{\partial t}}
\newcommand*{\pdzpdx}{\frac{\partial z}{\partial x}}
\newcommand*{\pdzpdy}{\frac{\partial z}{\partial y}}
\newcommand*{\pdzpdt}{\frac{\partial z}{\partial t}}
\newcommand*{\pdxpdt}{\frac{\partial x}{\partial t}}
\newcommand*{\pdypdt}{\frac{\partial y}{\partial t}}

% \usepackage[lite,subscriptcorrection,nofontinfo]{mtpro2}
\usepackage{fontspec}

\setmainfont{Palatino Linotype}[Ligatures=TeX,Numbers=OldStyle]
\setmonofont{Source Code Pro}
% \usepackage[integrals]{wasysym}
\usepackage{fontawesome}

\usepackage[math-style=TeX]{unicode-math}
\setmathfont{TeX Gyre Pagella Math}

\usepackage{microtype}

\contitem
\let\reason\text
\let\vect\symbf

\AtBeginDocument{%
  % \renewcommand{\perp}{\mathrel{\bot}}
  \let\leq\leqslant
  \let\le\leq
  \let\geq\geqslant
  \let\ge\geq}


\begin{document}
\maketitle

This is my own solution to Problem Set~2 of
\href{https://ocw.mit.edu/courses/mathematics/18-014-calculus-with-theory-fall-2010/assignments/}{18.014}.
The first four problems are from Apostol's \textit{Calculus} (1: 57, 60, 70).

\begin{enumerate}
\item
  \begin{enumerate}
  \item Let \(f(x) = \sum_{k=0}^n c_k x^k\) be either a zero polynomial or a
    polynomial of degree at most \(n\).  If \(f(x) = 0\) for \(n+1\)
    distinct real values of \(x\), then \(f\) is a zero polynomial.

    \begin{proof}
      Let \(x_k\) denote the zeros for \(f\).  The statement clearly holds for
      \(n=0\).  If \(f\) is of degree \(0\), then \(f(x_0) = c_0 = 0\), which
      contradicts the assumption \(c_0 \ne 0\).  Thus, \(f\) must be a zero
      polynomial.  Suppose the statement holds for some \(n \ge 0\).  We are
      going to show that the statement also holds for \(n+1\).  Assume that
      \(f\) is either a zero polynomial or a polynomial of degree at most
      \(n+1\) and have \(n+2\) distinct zeros.  The polynomial \(f\) can be
      factored into \((x-x_0) \, h(x)\), where \(h\) is either a zero
      polynomial or a polynomial of degree at most \(n\).  Our induction basis
      implies the degree of \(f\) cannot be \(0\).  If the degree of \(f\) is
      at least one, then this is so by the result we proved in Recitation~5.
      If \(f\) is a zero polynomial, then \(f(x) = 0 = (x-x_0) \cdot 0\).  In
      both cases, the stated decomposition holds.  Other than \(x_0\), \(f\)
      still has \(n+1\) distinct zeros.  By Theorem~I.11, \(h\) must have
      \(n+1\) distinct zeros.  By our induction hypothesis, \(h\) is a zero
      polynomial.  This means \(f\) is also a zero polynomial.
    \end{proof}

    An alternative view, which is less wordy, to look at the problem is as
    follows.

    \begin{proof}
      Suppose the degree of the polynomial \(f\) is well-defined.  Then let
      \(m\) denote the degree of \(f\).  By the result from Recitation~5, we
      have
      \begin{align*}
        f(x) &= h_0(x) \\
             &= (x-x_0) \, h_1(x) \\
             &= (x-x_0)(x-x_1) \, h_2(x) \\
             &= \brce[\bigg]{\prod_{k=0}^{m-1} (x-x_k)} \, h_m(x),
      \end{align*}
      where each \(h_i\) is a polynomial of degree \(m-i\).  Since
      \(0 \le m \le n\) and \(f\) has \(n+1\) distinct zeros, there is at
      least one \(x_m\) different from \(x_0, x_1, \dotsc, x_{m-1}\) such
      that
      \[
        f(x_m) = \brce[\bigg]{\prod_{k=0}^{m-1} (x_m-x_k)} \, h_m(x_m) = 0.
      \]
      The product in the braces is not zero, then by Theorem~I.11
      \(h_m(x_m) = 0\).  This is impossible since \(h_m\) is of degree
      \(0\).  Thus, \(f\) must be a zero polynomial.
    \end{proof}

  \item Let \(f(x) = \sum_{k=0}^n c_k x^k\) and
    \(g(x) = \sum_{k=0}^m b_k x^k\) be polynomials of degree \(n\) and \(m\)
    respectively and \(m \ge n\).  Prove that if \(g(x) = f(x)\) for \(m+1\)
    distinct real values of \(x\), then \(m = n\), \(b_k = c_k\) for each
    \(k\), and \(g(x) = f(x)\) for all real \(x\).

    \begin{proof}
      Let \(h(x) = g(x) - f(x)\).  Then it is easy to see that \(h\) is
      either a zero polynomial or a polynomial of degree at most \(m\).  The
      polynomial \(h\) has \(m+1\) distinct zeros since \(g(x) = f(x)\) for
      \(m+1\) distinct real values of \(x\).  By the result from the
      previous part, \(h\) is thus a zero polynomial.  We have
      \begin{align*}
        h(x) &= g(x) - f(x) \\
             &= \sum_{k=0}^m b_k x^k - \sum_{k=0}^n c_k x^k \\
             &= \sum_{k=0}^n b_k x^k + \sum_{\mathclap{k=n+1}}^m b_k x^k - \sum_{k=0}^n c_k x^k \\
             &= \sum_{\mathclap{k=n+1}}^m b_k x^k + \sum_{k=0}^n \paren[\big]{b_k x^k - c_k x^k} \\
             &= \sum_{\mathclap{k=n+1}}^m b_k x^k + \sum_{k=0}^n \paren{b_k - c_k} x^k \\
             &= 0.
      \end{align*}
      If \(m > n\), then the above won't hold since \(b_m \ne 0\).  Thus,
      \(m = n\) and \(b_k - c_k = 0\) for each \(k\).
    \end{proof}
  \end{enumerate}

\item Let \(A = \brce{1, 2, 3, 4, 5}\), and let \(\mathcal{M}\) denote the
  class of all subsets of \(A\).  (There are \(32\) altogether, counting
  \(A\) itself and the empty set \(\emptyset\).)  For each set \(S\) in
  \(\mathcal{M}\), let \(n(S)\) denote the number of distinct elements in
  \(S\).  If \(S = \brce{1, 2, 3, 4}\) and \(T = \brce{3, 4, 5}\), compute
  \(n(S \cup T)\), \(n(S \cap T)\), \(n(S - T)\), and \(n(T - S)\).  Prove
  that the set function \(n\) satisfies the first three axioms for area.

  We have \(S \cup T = \brce{1, 2, 3, 4, 5}\), \(S \cap T = \brce{3, 4}\),
  \(S - T = \brce{1, 2}\), and \(T - S = \brce{5}\).  Thus,
  \(n(S \cup T) = 5\), \(n(S \cap T) = 2\), \(n(S - T) = 2\) and
  \(n(T - S) = 1\).

  \begin{proof}
    The empty set \(\emptyset\) has the least number of distinct elements,
    namely \(0\).  Any nonempty set \(S\) has at least one element and hence
    \(n(S) > 0\).  Thus, \(n(S) \ge 0\) for each set \(S \in \mathcal{M}\).
    This proves the nonnegative property.

    Now suppose \(S\) and \(T\) are both in \(\mathcal{M}\).  For any
    \(x \in S \cup T\), either \(x \in S\) or \(x \in T\).  If \(x \in S\),
    then \(x \in A\) since \(S \subseteq A\).  Similarly, if \(x \in T\),
    then \(x \in A\) since \(T \subseteq A\).  Thus, \(x \in A\).  This
    means \(S \cup T\) is a subset of \(A\) and thus is in \(\mathcal{M}\).
    In the same fashion, we can show that \(S \cap T\) is also in
    \(\mathcal{M}\).  It is easy to see that the sets \(S \cap T\),
    \(S - T\), \(T - S\) are disjoint from each other and
    \begin{gather*}
      S = (S-T) \cup (S \cap T), \\
      T = (T-S) \cup (S \cap T), \\
      S \cup T = (S-T) \cup (S \cap T) \cup (T-S).
    \end{gather*}
    Thus, we have
    \begin{gather}
      \label{eq:1}
      n(S) = n(S-T) + n(S \cap T), \\
      \label{eq:2}
      n(T) = n(T-S) + n(S \cap T), \\
      \label{eq:3}
      n(S \cup T) = n(S - T) + n(S \cap T) + n(T - S).
    \end{gather}
    Substitute~\eqref{eq:1} and~\eqref{eq:2} into~\eqref{eq:3} and we obtain
    \[
      n(S \cup T) = n(S) + n(T) - n(S \cap T).
    \]
    This proves the additive property.

    With additional condition \(S \subseteq T\), we have \(S \cup T = T\),
    \(S \cap T = S\), and \(S - T = \emptyset\).  Substitute into
    \eqref{eq:3} and we get
    \begin{gather*}
      n(T) = n(\emptyset) + n(S) + n(T - S), \text{ or} \\
      n(T - S) = n(T) - n(S).
    \end{gather*}
    This proves the difference property.
  \end{proof}

\item
  \begin{enumerate}
  \item Compute \(\int_0^9 [\sqrt t] \dt\).

    We can find a partition \(\brce{0, 1, 4, 9}\) for the step function
    \([\sqrt t]\) and then
    \begin{displaymath}
      \int_0^9 [\sqrt t] \dt = 0 \times 1 + 1 \times (4-1) + 2 \times (9-4) = 13.
    \end{displaymath}

  \item If \(n\) is a positive integer, prove that
    \(\int_0^{n^2} [\sqrt t] \dt = n(n-1)(4n+1)/6\).

    \begin{proof}
      We can find a partition \(\brce{0, 1, 4, \dotsc, n^2}\) for the step
      function \([\sqrt t]\) and then
      \begin{align*}
        \int_0^{n^2} [\sqrt t] \dt
          &= \sum_{k=1}^n (k-1)\brce[\big]{k^2 - (k-1)^2} \\
          &= \sum_{k=1}^n (k-1)(2k-1) \\
          &= \sum_{k=1}^n (2k^2 - 3k + 1) \\
          &= 2 \sum_{k=1}^n k^2 - 3 \sum_{k=1}^n k + \sum_{k=1}^n 1 \\
          &= \frac{n(n+1)(2n+1)}{3} - \frac{3n(n+1)}{2} + n \\
          &= \frac{n(n-1)(4n+1)}{6}. \qedhere
      \end{align*}
  \end{proof}
  \end{enumerate}

\item If, instead of defining integrals of step functions by using formula
  (1.3), we used the definition
  \[
    \int_a^b s(x) \dx = \sum_{k=1}^n s_k^3 \cdot (x_k - x_{k-1}),
  \]
  a new and different theory of integration would result.  Which of the
  following properties would remain valid in this new theory?
  \begin{enumerate}
    \everymath{\displaystyle}
  \item \(\int_a^b s + \int_b^c s = \int_a^c s\).

    This is valid.

    \begin{proof}
      Let \(P_1 = \brce{x_0, x_1, \dotsc, x_n}\) be a partition for \(s\) on
      \([a, b]\) and \(P_2 = \brce{x_0', x_1', \dotsc, x_m'}\) be a partition
      for \(s\) on \([b, c]\).  Then \(s(x) = s_k\) on every open interval
      \((x_{k-1}, x_k)\) and \(s(x) = s_j'\) on every open interval
      \((x_{j-1}', x_j')\), where \(0 \le k \le n\), \(0 \le j \le m\), and
      \(s_k\) and \(s_j'\) are all constants.  Notice that \(x_n = x_0' = b\).
      We now can form a new partition
      \(P = \brce{x_0, x_1, \dotsc, x_n, x_1', \dotsc, x_m'}\).  We set
      \(x_k = x_{k-n}'\) and \(s_k = s_{k-n}'\) when \(k > n\).  Then
      \begin{align*}
        \int_a^b s + \int_b^c s
          &= \sum_{k=1}^n s_k^3 (x_k - x_{k-1}) + \sum_{j=1}^m s_j'^3 (x_j' - x_{j-1}') \\
          &= \sum_{k=1}^n s_k^3 (x_k - x_{k-1}) + \sum_{\mathclap{k=n+1}}^{\mathclap{n+m}} s_k^3 (x_k - x_{k-1}) \\
          &= \sum_{k=1}^{\mathclap{n+m}} s_k^3 (x_k - x_{k-1}) \\
          &= \int_a^c s. \qedhere
      \end{align*}
    \end{proof}

  \item \(\int_a^b c \cdot s = c \int_a^b s\).

    This is invalid.  Let \(P = \brce{x_0, x_1, \dotsc, x_n}\) be a partition
    for \(s\) on \([a, b]\) and \(s(x) = s_k\) on every open interval
    \((x_k, x_{k-1})\).  Then
    \begin{displaymath}
      \int_a^b c \cdot s
        = \sum_{k=1}^n (cs_k)^3 (x_k - x_{k-1})
        = \sum_{k=1}^n c^3 s_k^3 (x_k - x_{k-1})
        \ne c \sum_{k=1}^n s_k^3 (x_k - x_{k-1})
        = c \int_a^b s.
      \end{displaymath}
      However, we do have
      \[
        \int_a^b c \cdot s = c^3 \int_a^b s.
      \]
  \end{enumerate}

\item Prove, using properties of the integral, that for \(a, b > 0\)
  \[
    \int_1^a \frac{\dx}{x} + \int_1^b \frac{\dx}{x} = \int_1^{ab} \frac{\dx}{x}.
  \]
  Define a function \(f(w) = \int_1^w \frac1x \dx\), for
  \(w \in \mathbf{R}^+\).  Rewrite the equation above in terms of the
  function \(f\).  Give an example of a function that has the same
  property as the one displayed here by \(f\).

  \begin{proof}
    We have
    \begin{align*}
      \int_1^b \frac{\dx}{x}
        &= \frac1a \int_a^{ab} \frac{\dx}{x/a} && \reason{by Theorem~1.19,} \\
        &= \int_a^{ab} \frac{\dx}{x} && \reason{by homogeneity and cancellation,} \\
        &= \int_a^1 \frac{\dx}{x} + \int_1^{ab} \frac{\dx}x && \reason{by Theorem~1.17,} \\
        &= -\int_1^a \frac{\dx}{x} + \int_1^{ab} \frac{\dx}x && \reason{by our convention.}
    \end{align*}
    Move the first term of the RHS to the LHS and we obtain the equation.
  \end{proof}

  We can rewrite the equation in terms of the function \(f\) as
  \[
    f(a) + f(b) = f(ab).
  \]
  The logarithm functions observe the stated property
  \[
    \log a + \log b = \log ab.
  \]

\item Suppose we define
  \(\int_a^b s(x) \dx = \sum s_k \paren{x_{k-1} - x_k}^2\) for a step
  function \(s(x)\) with partition \(P = \brce{x_0, x_1, \dotsc, x_n}\).
  Is this integral well-defined?  That is, will the value of the integral
  be independent of the choice of partition?  (If well-defined, prove it.
  If not well-defined, provide a counterexample.)

  Clearly, this integral is not well-defined since the factor
  \(\paren{x_{k-1} - x_k}^2\) does not observe telescoping property
  under refinement.  A counterexample will be as follows.  Let us define
  \(s(x) = 1\) on \([0, 2]\).  Then both \(P_1 = \brce{0, 2}\) and
  \(P_2 = \brce{0, 1, 2}\) are partitions for \(s\), with \(P_2\) being
  a refinement of \(P_1\).  However, we have
  \begin{displaymath}
    \int_0^2 s = \paren{2-0}^2 = 4 \ne 2 = \paren{1-0}^2 + \paren{2-1}^2 = \int_0^2 s.
  \end{displaymath}

\item[\bonus] Define the function (where \(n\) is in the positive integers)
  \[
    f(x) =
    \begin{cases}
      x, & x = \frac1{n^2}, \\[1ex]
      0, & x \ne \frac1{n^2}.
    \end{cases}
  \]
  Prove that \(f\) is integrable on \([0, 1]\) and that
  \(\int_0^1 f(x) \dx = 0\).

  \begin{proof}
    For any \(\epsilon > 0\), by the Archimedean property of real numbers
    we can always find a positive integer \(n_0\) such that
    \(1/n_0^2 < \epsilon\). Then we can define two step functions \(s\)
    and \(t\) on \([0, 1]\) as follows.  Let
    \begin{displaymath}
      s(x) =
      \begin{cases}
        x, & x = \frac{1}{n^2} \tand x > \frac{1}{n_0^2}, \\[1ex]
        0, & \otherwise,
      \end{cases}
      \qand
      t(x) =
      \begin{cases}
        x, & x = \frac{1}{n^2} \tand x > \frac{1}{n_0^2}, \\[1ex]
        \frac{1}{n_0^2}, & 0 \le x \le \frac{1}{n_0^2}, \\[1ex]
        0, & \otherwise.
      \end{cases}
    \end{displaymath}
    It is easy to verify that
    \begin{displaymath}
      \int_0^1 s = 0
      \qand
      \int_0^1 t = \frac{1}{n_0^4}.
    \end{displaymath}
    Therefore, we have
    \begin{gather*}
      \int_0^1 t - \int_0^1 s = \frac{1}{n_0^4} \le \frac{1}{n_0^2} < \epsilon \\
      \iand
      \int_0^1 s \le 0 \le \int_0^1 t.
    \end{gather*}
    By Theorem~2 from Course Note~C, the function \(f\) is integrable on
    \([0, 1]\) and \(\int_0^1 f(x) \dx = 0\).
  \end{proof}
\end{enumerate}
\end{document}
