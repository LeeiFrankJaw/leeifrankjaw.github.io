\chapter{连续函数}
\label{ch:cont}

\section{连续函数的概念与性质}

\begin{definition}[连续]
  \label{defn:cont}
  函数\(\,f\mkern2mu\)在点\(x_0\)处连续,点\(x_0\)是函数\(\,f\mkern2mu\)的连续点,当且仅当:\(\!\lim\limits_{\,x\to x_0\!} \,f(x) = \,f(x_0)\).
\end{definition}

\begin{example*}
  函数\(\,f(x) = x \Fn D(x)\).

  \begin{remark}
    易知此函数在\(x \ne 0\)处的极限都不存在,因而都不连续;它在原点处的极限存在且等于函数值,因而在此处连续.
  \end{remark}
\end{example*}

\begin{theorem}
  \label{thm:contequivdefn}
  函数\(\,f\mkern2mu\)在点\(x_0\)处连续的充分必要条件是:\(\lim\limits_{\fdx \to 0} \fwdf\,f(x_0) = 0\),其中\(\fdx = x - x_0,\ \fwdf\,f(x_0) = {}\,f(x_0+\fdx) - f(x_0)\).

  \begin{proof}
    先证必要性.根据定理~\ref{thm:limfunccomp}和~\ref{thm:limfunc4ops},有
    \begin{equation*}
      \lim_{\fdx \to 0} \paren*{\,f(x_0 + \fdx) - f(x_0)}
      = \lim_{x\to x_0} \paren*{\,f(x) - f(x_0)}
      = 0.
    \end{equation*}
    再证充分性.根据定理~\ref{thm:limfunccomp}和~\ref{thm:limfunc4ops},有
    \begin{equation*}
      \begin{split}
        \lim_{x\to x_0} f(x)
        &= \lim_{x\to x_0} \paren*{\,f(x_0 + x - x_0) - f(x_0) + f(x_0)} \\
        &= \lim_{\fdx \to 0} \paren*{\,f(x_0 + \fdx) - f(x_0) + f(x_0)} \\
        &= \lim_{\fdx \to 0} \paren*{\fwdf\,f(x_0) + f(x_0)} = \,f(x_0).
          \qedhere
      \end{split}
    \end{equation*}
  \end{proof}
\end{theorem}

\begin{definition*}
  若\(\lim\limits_{x\to x_0^-}\,f(x) = \,f(x_0)\),则称函数\(\,f\mkern2mu\)在点\(x_0\)处左连续;若\(\lim\limits_{x\to x_0^+} \,f(x) = \,f(x_0)\),则称函数\(\,f\mkern2mu\)在点\(x_0\)处右连续.
\end{definition*}

\begin{theorem}
  \label{thm:contsided}
  函数\(\,f\mkern2mu\)在点\(x_0\)处连续的充分必要条件是它在此处左连续且右连续.

  \begin{proof}
    使用上述定义和定理~\ref{thm:limfuncsided}即可得证.
  \end{proof}
\end{theorem}

\begin{theorem}
  \label{thm:contequiv}
  函数\(\,f\mkern2mu\)在点\(x_0\)处连续的充分必要条件是:对于任意的数列\(\Seq{x_n}\),当它收敛于\(x_0\)时,就一定有数列\(\Seq{\,f(x_n)}\)收敛于\(\,f(x_0)\).

  % TODO: Complete the proof
  \begin{proof}
    使用定义~\ref{defn:cont}、定理~\ref{thm:limequiv}和定理~\ref{thm:seqcvgsubseq}即可得证.
  \end{proof}
\end{theorem}

\begin{remark}
  我们用\(\,f \in C(a, b)\)来表示“函数\(\,f\mkern2mu\)在开区间\((a,b)\)上的每一点都连续”.类似地,在闭区间或者半开半闭区间上,当它在除去端点的每一点都连续,在闭的左端点右连续,在闭的右端点左连续时,也有类似的记号.此时,我们简称函数\(\,f\mkern2mu\)在此区间上连续.
\end{remark}

\begin{example*}
  证明函数\(\,f(x) = x^2\)在\(\R\)上连续.

  \begin{proof}
    对于任意的\(ε > 0\)都存在\(0 < δ \le \sqrt{x_0^2+ε} - \abs{\,x_0\mkern1mu}\)使得当\(0 < \abs*{\,x-x_0\mkern1mu} < δ\)时都有
    \begin{equation*}
      \abs*{\,x^2 - x_0^2\,}
      = \abs*{\,x+x_0\mkern1mu} \abs*{\,x-x_0\mkern1mu}
      \le (\abs{\,x\,} + \abs{\,x_0\mkern1mu})δ
      < (2\abs{\,x_0\mkern1mu} + δ)δ
      = 2\abs{\,x_0\mkern1mu}\,δ + δ^2
      \le ε.
      \qedhere
    \end{equation*}
    \begin{remark}
      也可以取\(δ = \minb{c, ε/(2\abs{\,x_0\mkern1mu} + c)}\),其中\(c\)是某个正常数.
    \end{remark}
  \end{proof}
\end{example*}

\begin{theorem}
  \label{thm:contsin}
  证明正弦函数在\(\R\)上连续.

  \begin{proof}
    对于任意的\(ε > 0\)都存在正数\(δ = ε\)使得当\(0 < \abs*{\,x-x_0\mkern1mu} < δ\)时都有
    \begin{align*}
      \abs*{\,\sin x - \sin x_0\,}
      &= \abs*{\,2 \sin\frac{x-x_0}{2} \cos\frac{x+x_0}{2}\,}
        = 2 \abs*{\,\sin\frac{x-x_0}{2}\,} \, \abs*{\,\cos\frac{x+x_0}{2}\,} \\
      &\le 2 \abs*{\,\sin\frac{x-x_0}{2}\,}
        < 2 \abs*{\,\frac{x-x_0}{2}\,}
        = \abs*{\,x-x_0\mkern1mu}
        < δ = ε.
        % \rule[-2ex]{0ex}{0ex}
        \qedhere
    \end{align*}
  \end{proof}
\end{theorem}

% https://en.wikipedia.org/wiki/Classification_of_discontinuities
% Baby Rudin, 3rd ed. (94)
\begin{definition*}
  函数定义域上不是连续点的点叫作间断点.左右极限都存在的间断点叫作\kenten{第一类间断点},否则叫作\kenten{第二类间断点}.左右极限不相等的第一类间断点叫作\kenten{可去间断点},否则叫作\kenten{跳跃间断点}.
\end{definition*}

\begin{theorem}[Bolzano定理]
  \label{thm:bolzano}
  若函数\(\,f\mkern2mu\)在闭区间\(\brkt{a,b}\)上连续且\(\,f(a)\ f(b) < 0\),则存在\(ξ \in \paren{a,b}\)使得\({}\,f(ξ) = 0\).

  % TODO: Complete the proof
  \begin{proof}
    根据二分法来构造区间套,再使用定理~\ref{thm:cvgsgnprsv}和~\ref{thm:contequiv},最后再论证一下\(ξ\)不会是区间的端点,即可得证.
  \end{proof}
\end{theorem}

\begin{example*}
  试讨论方程\(2^x + x = 0\)在开区间\(\paren{-1,0}\)上解的个数。

  \begin{remark}
    令\(\,f(x) = 2^x + x\),则该方程在此区间上解的个数就等于函数\(\,f\mkern2mu\)在此区间上零点的个数.因为\({}\,f(-1) = -1/2\)且\(\,f(0) = 1\),所以\(\,f\mkern2mu\)在此区间上至少有一个零点.又因为\(\,f\mkern2mu\)严格单调递增,所以它在此区间上只有一个零点,即原方程在此区间上只有一个解.
  \end{remark}
\end{example*}

% https://mathworld.wolfram.com/FixedPointTheorem.html
% https://en.wikipedia.org/wiki/Brouwer_fixed-point_theorem
\begin{corollary}[不动点定理]
  \label{cor:fixedpoint}
  若函数\(\,f\mkern2mu\)在闭区间\(\brkt{a,b}\)上连续且有\(\,f(a) > a\)和\(\,f(b) < b\),则存在点\(ξ\)使得\({}\,f(ξ) = ξ\).

  \begin{proof}
    构造函数\(\Fn F(x) = \,f(x) - x\).因为函数\(\Fn F(x)\)在\(\brkt{a,b}\)上也连续且有\(\Fn F(a) > 0\)和\(\Fn F(b) < 0\),所以存在\(ξ\)使得\(\Fn F(ξ) = 0\),即\(\,f(ξ) = ξ\).
  \end{proof}
\end{corollary}

% https://mathworld.wolfram.com/IntermediateValueTheorem.html
% https://en.wikipedia.org/wiki/Intermediate_value_theorem
\begin{corollary}[介值定理]
  \label{cor:ivt}
  若函数\(\,f\mkern2mu\)在\(\brkt{a,b}\)上连续,则函数\(\,f\mkern2mu\)在\(\paren{a,b}\)上可以取到所有\(\,f(a)\)和\(\,f(b)\)之间(不含)的数\footnote{英文叫作intermediate value theorem,简称为IVT.}.

  \begin{proof}
    当\(\,f(a) = \,f(b)\)时,此定理空虚地为真.当\(\,f(a) \ne \,f(b)\)时,对于任意的\(μ \in \paren[\big]{\,f(a), \,f(b)}\),构造函数\(\Fn F(x) = \,f(x) - μ\),那么根据定理~\ref{thm:bolzano},存在\(ξ \in \paren{a,b}\)使得\(\Fn F(ξ) = 0\),即\(\,f(ξ) = μ\).由于\(μ\)的任意性,此定理得证.
  \end{proof}
\end{corollary}

\begin{corollary*}
  若函数\(\,f\mkern2mu\)在\(\R\)上连续且有界,则函数\(\,f\mkern2mu\)能取到所有上下确界之间的数(不含).

  \begin{proof}
    当\(\inf\,f = \sup\,f\)时,此定理空虚地为真.当\(\inf\,f < \sup\,f\)时,对于任意的\(μ \in \paren{\inf\,f, \sup\,f\,}\),根据确界的定义,存在\(\,f(x_1) \in \paren{\inf\,f,\,μ}\)和\(\,f(x_2) \in \paren{μ,\,\sup\,f}\);又根据推论~\ref{cor:ivt},存在位于\(x_1\)和\(x_2\)之间(不含)的数\(ξ\)使得\(\,f(ξ) = μ\).由于\(μ\)的任意性,此定理得证.
  \end{proof}
\end{corollary*}

% TODO: Add the example of area of line sliding through a triangle

\begin{theorem}[连续函数的四则运算]
  \label{thm:cont4ops}
  若函数\(\,f\mkern2mu\)和\(\mkern1mu g\)在点\(x_0\)处都连续,则函数\(\,f \pm g,\ fg,\ f\mkern2mu/g\ (g(x_0) \ne 0)\)也都在此处连续.

  \begin{proof}
    使用定义~\ref{defn:func4ops}和定理~\ref{thm:limfunc4ops}即可得证.
  \end{proof}
\end{theorem}

\begin{theorem}[复合函数的连续性]
  \label{thm:contcomp}
  若函数\(\mkern1mu g\)在点\(x_0\)处连续且函数\(\,f\mkern2mu\)在点\(u_0 = g(x_0)\)处连续,则复合函数\(\,f \circ g\)在点\(x_0\)处连续.

  \begin{proof}
    因为函数\(\,f\mkern2mu\)在点\(u_0\!\)处连续,所以对于任意的\(ε > 0\)都存在\(δ_1 > 0\)使得当\(\abs*{\ u - u_0\,} < δ_1\)时都有\(\abs*{\mkern6mu f(u)-f(u_0)\,} < ε\).又因为函数\(\mkern1mu g\)在点\(x_0\)处连续,所以存在\(δ > 0\)使得当\(\abs*{\ x - x_0\,} < δ\)时都有\(\abs*{\ g(x) - g(x_0)\,} = \abs*{\ g(x) - u_0\,} < δ_1\),从而\(\abs*{\mkern6mu f\,\paren[\big]{g(x)} - f(u_0)\,} = \abs*{\mkern6mu f\,\paren[\big]{g(x)} - f\,\paren[\big]{g(x_0)}\,} < ε\).
  \end{proof}

  \begin{proof}
    使用定理~\ref{thm:contequiv}即可得证.
  \end{proof}
\end{theorem}

\begin{theorem*}
  若函数\(\,f\mkern2mu\)和\(\mkern1mu g\)在点\(x_0\)处连续且\(\,f(x_0) > 0\),则函数\(\,f^g\!\)在此处也连续.

  \begin{proof}
    仿照定理~\ref{thm:limfuncpowexp}的证明,使用两次定理~\ref{thm:contcomp}和一次定理~\ref{thm:cont4ops}即可得证.
  \end{proof}

  \begin{proof}
    使用两次定理~\ref{thm:contequiv}和一次定理~\ref{thm:limfuncpowexp}即可得证.
  \end{proof}
\end{theorem*}

\begin{theorem*}
  若函数\(\,f\mkern2mu\)在某个闭(开)区间上严格单调连续,则其反函数\(\,f^{-1}\!\)也在某个闭(开)区间上严格同向单调连续.

  % TODO: Complete the proof
  \begin{proof}
    对于闭区间的情况,使用定义~\ref{defn:funcinv}和~\ref{defn:funcmono}、推论~\ref{cor:ivt}即可得证.对于开区间的情况,还要再用上公理~\ref{ax:lubglb}、定义~\ref{defn:bnd}和~\ref{defn:supinf}.
  \end{proof}
\end{theorem*}

\begin{theorem*}
  初等函数在定义域内(除去孤立点)连续.
\end{theorem*}

\begin{example*}
  讨论函数\(\,f(x) = \sqrt{x(x-1)} + \sqrt{x\mathstrut}\)在其自然定义域上的连续性.

  \begin{remark}
    它的自然定义域是\(\Set{0} \cup \brktparen{1, +∞}\).显然地,它在\(\paren{1, +∞}\)上的每一点处都连续,在点\(1\)处右连续,在原点处(孤立点)不连续.
  \end{remark}
\end{example*}

\subpdfbookmark{思考}{B1.3.1.P}
\subsection*{思考}

\begin{enumerate}
\item 函数在一点连续与单侧连续之间有什么关系?

  \ifshowsolp
    参见定理~\ref{thm:contsided}.
  \fi

\item 函数在一点连续能否推出函数在此点附近也连续?

  \ifshowsolp
    不能.
  \fi

\item 基本初等函数在其定义域内是连续的,这个结论是如何得到的?

  \ifshowsolp
    其实这要把基本初等函数分成两类来看:一类是超越函数(transcedental functions)、另一类是代数函数(algebraic functions).代数函数基本上可以由实数公理推出来.超越函数中的指数函数可以定义成
    \begin{equation*}
      \exp x \coloneq \lim_{n\to\infty} \paren*{1 + \frac xn}^n,
    \end{equation*}
    然后通过讨论这个数列的单调性和有界性论证它总是收敛的,并且研究出它的很多性质满足我们中学学过的指数运算法则.由这些结论,我们可以证明指数函数是连续的.再由此定义对数函数为指数函数的反函数,得到它的一系列性质.这样,对于超越的幂函数,例如\(x^π\),就可以变形成\(\expb{π \ln x}\),再由定理~\ref{thm:contcomp}和定理~\ref{thm:cont4ops}得出其连续性.

    对于超越函数中的三角函数,正弦函数的连续性已经由定理~\ref{thm:contsin}给出,余弦函数的连续性可由\(\cos x = \sinp{π/2-x}\)再加上定理~\ref{thm:contcomp}得出.剩下的三角函数,对正弦函数和余弦函数使用定理~\ref{thm:cont4ops}即可得到它们的连续性.
  \fi
\end{enumerate}

\ifshowex
\currentpdfbookmark{练习}{B1.3.1.E}
\subsection*{练习}

\begin{enumerate}
\item 原点是函数
  \begin{equation*}
    f(x) =
    \begin{dcases}
      \frac{e^{1/x}-1}{e^{1/x}+1}, & x \ne 0, \\
      1, & x = 0
    \end{dcases}
  \end{equation*}
  的\uline{\makebox[6em]{}}.
  \begin{itemize}
    \renewcommand{\labelitemi}{\faCircleThin}
  \item 可去间断点
    \ifshowsol
    \item[\faCircle]
    \else
    \item
    \fi
    跳跃间断点
  \item 第二类间断点
  \item 连续点
  \end{itemize}

  \ifshowsol
    函数\(\,f\mkern2mu\)在原点处的左右极限分别是\(-1\)和\(1\).
  \fi

\item 令
  \begin{equation*}
    f(x) = \frac1{e^{{x}/\paren{x-1}} - 1}
  \end{equation*}
  则\uline{\makebox[10em]{}}.
  \begin{itemize}
    \renewcommand{\labelitemi}{\faCircleThin}
  \item 原点和点\(1\)都是第一类间断点
  \item 原点和点\(1\)都是第二类间断点
  \item 原点是第一类间断点,点\(1\)是第二类间断点
    \ifshowsol
    \item[\faCircle]
    \else
    \item
    \fi
    原点是第二类间断点,点\(1\)是第一类间断点
  \end{itemize}

  \ifshowsol
    函数\(\,f\mkern2mu\)在原点处的左右极限分别是\(+\infty\)和\(-\infty\),在点\(1\)处的左右极限分别是\(-1\)和\(0\).
  \fi

\item 求函数
  \begin{equation*}
    f(x) = \frac{(e^x+x) \tan x}{x (e^{1/x}-e)}
  \end{equation*}
  在\(\brkt{-π,π}\)上的第一类间断点.

  \ifshowsol
    函数\(\,f\mkern2mu\)在此区间上有\(4\)个间断点,分别是\(-π/2,\ 0,\ 1,\ π/2\).它在这\(4\)个间断点处的左右极限分别是\(-\infty\)和\(+\infty\)、\(-e^{-1}\)和\(0\)、\(+\infty\)和\(-\infty\)、\(-\infty\)和\(+\infty\).所以原点是函数\(\,f\mkern2mu\)的跳跃间断点.
  \fi

\item 求函数
  \begin{equation*}
    f(x) = (1+x)^{x/\!\tanp{x-π/4}}
  \end{equation*}
  在\(\paren{0,2π}\)上的第二类间断点.

  \ifshowsol
    函数\(\,f\mkern2mu\)在此区间上有\(4\)个间断点,分别是\(π/4,\ 3π/4,\ 5π/4,\ 7π/4\).它在这\(4\)个间断点处的左右极限分别是\(0\)和\(+\infty\)、\(1\)和\(1\)、\(0\)和\(+\infty\)、\(1\)和\(1\).所以点\(π/4\)和\(5π/4\)是函数\(\,f\mkern2mu\)的无穷间断点.
  \fi

\item 原点是函数\(\,f(x) = \cos^2(1/x)\)的\uline{\makebox[6em]{}}.
  \begin{itemize}
    \renewcommand{\labelitemi}{\faCircleThin}
  \item 可去间断点
  \item 跳跃间断点
  \item 无穷间断点
    \ifshowsol
    \item[\faCircle]
    \else
    \item
    \fi
    振荡间断点
  \end{itemize}

\item 令
  \begin{equation*}
    f(x) = \lim_{n\to\infty} \frac{x^{2n+1}+1}{x^{2n+1}-x^{n+1}+x},
  \end{equation*}
  则下列说法正确的是\uline{\makebox[6em]{}}.
  \begin{itemize}
    \renewcommand{\labelitemi}{\faCircleThin}
  \item 原点是可去间断点
    \ifshowsol
    \item[\faCircle]
    \else
    \item
    \fi
    点\(1\)是可去间断点
  \item 点\(-1\)是可去间断点
  \item 点\(1\)和点\(-1\)均为跳跃间断点
  \end{itemize}

  \ifshowsol
    实际上,有
    \begin{equation*}
      f(x) =
      \begin{dcases}
        1, & \abs{\,x\,} > 1, \\
        0, & x = -1, \\
        2, & x = 1, \\
        1/x, & \abs{\,x\,} < 1 \tand x \ne 0.
      \end{dcases}
    \end{equation*}
    所以点\(-1, 0, 1\)分别是函数\(\,f\mkern2mu\)的跳跃间断点、第二类间断点、可去间断点.
  \fi

\item 原点是函数\uline{\makebox[6em]{}}的可去间断点.
  \begin{itemize}
    \renewcommand{\labelitemi}{\faCircleThin}
  \item
    \begin{math}
      f(x) =
      \begin{dcases}
        x + 1/x, & x \ne 0, \\
        1, & x = 0
      \end{dcases}
    \end{math}
  \item
    \begin{math}
      f(x) =
      \begin{dcases}
        (1+x^2)/x^2, & x \ne 0, \\
        1, & x = 0
      \end{dcases}
    \end{math}
    \ifshowsol
    \item[\faCircle]
    \else
    \item
    \fi
    \(\,f(x) = \floor{\cos x}\)
  \item \(\,f(x) = \sgnp x\)
  \end{itemize}

  \ifshowsol
    原点分别是选项~A、B中函数的第二类间断点、选项~C中函数的可去间断点、选项~D中的跳跃间断点.
  \fi

\item 下列说法中,错误的是\uline{\makebox[10em]{}}.
  \begin{itemize}
    \renewcommand{\labelitemi}{\faCircleThin}
  \item 如果函数\(\,f\mkern2mu\)在点\(a\)处连续,那么函数\(\abs*{\mkern6mu f\ }\)也在点\(a\)处连续
    \ifshowsol
    \item[\faCircle]
    \else
    \item
    \fi
    如果函数\(\abs*{\mkern6mu f\ }\)在点\(a\)处连续,那么函数\(\,f\mkern2mu\)也在点\(a\)处连续
  \item 如果函数\(\,f\mkern2mu\)在点\(a\)处连续,那么函数\(\,f^3\!\)也在点\(a\)处连续
  \item 如果函数\(\,f^3\!\)在点\(a\)处连续,那么函数\(\,f\mkern2mu\)也在点\(a\)处连续
  \end{itemize}

  \ifshowsol
    由于反三角不等式,选项~A成立;由于定理~\ref{thm:contcomp},选项~C和~D成立.选项~B的一个反例是函数\(2\Fn H - 1\),其中\(\Fn H\)是单位阶跃函数(定义~\ref{defn:heaviside}).
  \fi

\item 函数\uline{\makebox[10em]{}}在原点处连续.
  \begin{itemize}
    \renewcommand{\labelitemi}{\faCircleThin}
    \ifshowsol
    \item[\faCircle]
    \else
    \item
    \fi
    \begin{math}
      f(x) =
      \begin{dcases}
        e^{-1/x^2}, & x \ne 0, \\
        0, & x = 0
      \end{dcases}
    \end{math}
  \item \(\,f(x) = \floor x\)
  \item \(\,f(x) = \sgnp{\sin x}\)
  \item
    \begin{math}
      f(x) =
      \begin{dcases}
        \paren{\sin x}/\abs{\,x\,}, & x \ne 0, \\
        1, & x = 0
      \end{dcases}
    \end{math}
    % \rule{0ex}{5ex}
  \end{itemize}

  \ifshowsol
    原点是选项~B、C、D中函数的跳跃间断点.选项~A中的函数在原点处的左右极限都是\(0\)且等于此处的函数值,因此连续.
  \fi

\item 若函数\(\,f\mkern2mu\)和\(\mkern2mu g\)在点\(x_0\)处均不连续,则\uline{\makebox[10em]{}}.
  \begin{itemize}
    \renewcommand{\labelitemi}{\faCircleThin}
  \item 函数\(\,f+g\)在点\(x_0\)处不连续
  \item 函数\(\,f\mkern2mu/g\)在点\(x_0\)处不连续
  \item 函数\(\,fg\)在点\(x_0\)处不连续
    \ifshowsol
    \item[\faCircle]
    \else
    \item
    \fi
    函数\(\,f+g,\ f\mkern2mu/g,\ fg\)都可能在点\(x_0\)处连续
  \end{itemize}

  \ifshowsol
    令\(\,f = \Fn H,\, g = 1-\Fn H\),则\(\,f + g = 1,\ fg = 0\),因此选项~A和~C不正确.令
    \begin{equation*}
      f(x) = g(x) = \frac{1}{x \Fn H(x) + \Fn H(-x)},
    \end{equation*}
    则\(\,f\mkern2mu/g = 1\),因此选项~B不正确.

    或者令\(\,f = 2\Fn H - 1,\, g = 1 - 2\Fn H\),则\(\,f + g = 0,\ f\mkern2mu/g = fg = -1\),因此选项~A、B和C都不正确.
  \fi
\end{enumerate}
\fi

\section{闭区间上连续函数的性质}

\begin{theorem}
  \label{thm:contclsintvbnd}
  闭区间上的连续函数是有界函数.

  \begin{proof}
    运用反证法.假设函数\(\,f\mkern2mu\)在闭区间上无界.那么构造有界数列\(\Seq{x_n}\)使得数列\(\Seq{\,f(x_n)}\)是无穷大量.由定理~\ref{thm:bw}可知,存在收敛于闭区间上某个常数\(L\)的子列\(\Seq{x_{n_k}}\).又因为函数\(\,f\mkern2mu\)是连续的,所以数列\(\Seq{\,f(x_{n_k})}\)收敛于\(\,f(L)\),矛盾.
  \end{proof}

  \begin{remark}
    上述证明中的\(L\)有可能是区间的端点,但正因为函数在闭区间的左端点右连续、右端点左连续,所以最后能保证数列\(\Seq{\,f(x_{n_k})}\)收敛于\(L\).把闭区间换成开区间,就无法保证此结果了.
  \end{remark}
\end{theorem}

% \pskip
\begin{theorem}[最值定理]
  \label{thm:evt}
  闭区间上的连续函数能取到最大值和最小值\footnote{英文叫作extreme value theorem,简称EVT.}.

  \begin{proof}
    使用定理~\ref{thm:contclsintvbnd}和公理~\ref{ax:lubglb}可以得到上确界\(M = \sup\,f\).令\(ε_0 = \abs[\big]{\,M - \maxb{\,a, b\,}\,}\),其中\(a,b\)分别为闭区间的端点.若\(ε_0 = 0\),则说明区间的端点之一就是最大值.否则,令\(ε_n = ε_0/2^n\).根据定义~\ref{defn:supinf},则能找到有界数列\(\Seq{x_n}\)使得\(M \ge f(x_n) > M - ε_n\)对于所有正整数\(n\)都成立.易知数列\(\Seq{\,f(x_n)}\)收敛于\(M\).使用定理~\ref{thm:bw},找到收敛于\(L\)的子列\(\Seq{x_{n_k}}\).由于定理~\ref{thm:cvgsgnprsv},有\(a \le L \le b\).又因为函数\(\,f\mkern2mu\)连续,使用定理~\ref{thm:contequiv}即可得到\(\,f(L) = M\).类似地,可以证明函数\(\,f\mkern2mu\)能取到最小值.
  \end{proof}
\end{theorem}

% TODO: Add the example of area under a line sliding up through a curve

% https://proofwiki.org/wiki/Continuous_Image_of_Closed_Interval_is_Closed_Interval
% https://en.wikipedia.org/wiki/Intermediate_value_theorem
\begin{theorem*}
  闭区间上的连续函数,其像集也是闭区间.

  \begin{proof}
    使用定理~\ref{thm:evt}和推论~\ref{cor:ivt}即可得证.
  \end{proof}
\end{theorem*}

\subpdfbookmark{思考}{B1.3.2.P}
\subsection*{思考}

\begin{enumerate}
\item 在最值定理中,若将闭区间改成开区间,则结论不再成立.请举例说明.

  \ifshowsolp
    其实对于定义在开区间\(\paren{0,1}\)上的恒等函数\(\,f(x) = x\)这样的简单函数,既取不到最小值,又取不到最大值.更不用说像定义在开区间上的\(\paren{-π/2,π/2}\)上的正切函数\(\,f(x) = \tan x\)和定义在开区间\(\paren{-∞,+∞}\)上的反正切函数\(\,f(x) = \arctan x\)了.
  \fi

\item 函数\(\,f(x) = x^{2m} + x^{2m-1} + \dots + x - 1\)和\(g(x) = x^{2m} + x^{2m-1} + \dots + x + 1\)在\(\R\)上是否存在零点?

  \ifshowsolp
    此处假定\(m \ge 1\).函数\(\,f\mkern2mu\)在\(\R\)上显然是有零点的;因为\(\,f(0) = -1\)且\(\,f(1) = 2m-1 > 0\),由定理~\ref{thm:bolzano}可知它在\(\paren{-1,1}\)上至少有一个零点.函数\(\,f\mkern2mu\)在\(\R\)上不存在零点,原因如下.有
    \begin{align*}
      g(x)
      &=
        \begin{dcases}
          1 + \sum_{k = 1}^{\smash{2m}} x^k, & x \ne 1, \\
          2m+1, & x = 1,
        \end{dcases} \\
      &=
        \begin{dcases}
          \frac{1-x^{2m+1}\negthickspace}{1-x}, & x \ne 1, \\
          2m+1, & x = 1.
        \end{dcases}
    \end{align*}
    可见函数\(\mkern1mu g\)在\(\R\)上都是正数,因此无零点.
  \fi
\end{enumerate}

\ifshowex
\currentpdfbookmark{练习}{B1.3.2.E}
\subsection*{练习}

\begin{enumerate}
\item 下列说法中,正确的是\uline{\makebox[10em]{}}.
  \begin{itemize}
    \renewcommand{\labelitemi}{\faCircleThin}
  \item 若函数\(\,f \in C\brkt{a,b}\)且\(\,f(a)\ f(b) < 0\),则存在唯一的\(ξ \in \paren{a,b}\)满足\(\,f(ξ) = 0\)
    \ifshowsol
    \item[\faCircle]
    \else
    \item
    \fi
    若函数\(\,f \in C\brkt{a,b}\),则存在\(ξ \in \brkt{a,b}\)满足\(\,f(ξ) = \paren[\big]{\,f(a) + f(b)}/{2}\)
  \item 若函数\(\,f \in C\paren{a,b}\)且\(\,f(a)\ f(b) < 0\),则存在唯一的\(ξ \in \paren{a,b}\)满足\(\,f(ξ) = 0\)
  \item 若\(\,f \in C\paren{a,b}\),则存在\(ξ \in \paren{a,b}\)满足\(\,f(ξ) = \paren[\big]{\,f(a) + f(b)}/{2}\)
  \end{itemize}

  \ifshowsol
    令\(a = 0,\ b = 3π\),余弦函数就是选项~A和~C的反例.令\(a = -1,\ b = 1\),则函数\(x^2\)显然不满足选项~D.对于选项~B,当\(\,f(a) \ne \,f(b)\)时,使用推论~\ref{cor:ivt}即可;当\(\,f(a) = \,f(b)\)时,取\(ξ = a\)或者\(ξ = b\)即可.
  \fi

\item 下列说法中,正确的是\uline{\makebox[10em]{}}.
  \begin{itemize}
    \renewcommand{\labelitemi}{\faCircleThin}
  \item 若函数\(\,f \in C\paren{a,b}\),则函数\(\,f\mkern2mu\)在\(\paren{a,b}\)上一定有最大值和最小值
  \item 若函数\(\,f \in C\paren{a,b}\),则函数\(\,f\mkern2mu\)在\(\paren{a,b}\)上一定没有最大值和最小值
  \item 若函数\(\,f \in C\brkt{a,b}\),则函数\(\,f\mkern2mu\)在\(\paren{a,b}\)上一定有最大值和最小值
    \ifshowsol
    \item[\faCircle]
    \else
    \item
    \fi
    若函数\(\,f \in C\brkt{a,b}\),则函数\(\,f\mkern2mu\)在\(\brkt{a,b}\)上一定有最大值和最小值
  \end{itemize}

  \ifshowsol
    令\(a = 0,\ b = 2π\),则函数\(x\)在区间\(\paren{a,b}\)上没有最大值和最小值,但函数\(\sin x\)在区间\(\paren{a,b}\)上有最大值和最小值.因此,选项~A、B、C都不正确.选项~D就是定理~\ref{thm:evt}.
  \fi

\item 设\(a_1,\ a_2,\ a_3\)均为正数且\(λ_1 < λ_2 < λ_3\),则方程
  \begin{equation*}
    \frac{a_1}{x-λ_1} + \frac{a_2}{x-λ_2} + \frac{a_3}{x-λ_3} = 0
  \end{equation*}
  \uline{\makebox[10em]{}}.
  \begin{itemize}
    \renewcommand{\labelitemi}{\faCircleThin}
  \item 在\(\paren{λ_1,λ_2}\)和\(\paren{λ_2,λ_3}\)内均无实根
  \item 在\(\paren{λ_1,λ_2}\)内有一个实根,在\(\paren{λ_2,λ_3}\)内无实根
  \item 在\(\paren{λ_1,λ_2}\)内无实根,在\(\paren{λ_2,λ_3}\)内又一个实根
    \ifshowsol
    \item[\faCircle]
    \else
    \item
    \fi
    在\(\paren{λ_1,λ_2}\)和\(\paren{λ_2,λ_3}\)内各有一个实根
  \end{itemize}

  \ifshowsol
    使用定理~\ref{thm:bolzano}和单调性即可得知.
  \fi

\item 下列函数中,在区间\(\paren{0,1}\)上必有零点的是\uline{\makebox[10em]{}}.
  \begin{itemize}
    \renewcommand{\labelitemi}{\faCircleThin}
  \item \(\,f \in C\paren{0,1}\)且\(\,f(0)\ f(1) < 0\)
  \item \(\,f \in C\paren{0,1}\)且\(\,f(1/2)\ f(1) < 0\)
  \item \(\,f \in C\paren{0,1}\)且\(\,f(0)\ f(1/2) < 0\)
    \ifshowsol
    \item[\faCircle]
    \else
    \item
    \fi
    \(\,f \in C\paren{0,1}\)且\(\,f(1/4)\ f(1/2) < 0\)
  \end{itemize}

  \ifshowsol
    选项~A的一个反例是函数\(\,f(x) = 1 - 2 \fn H(-x)\).选项~B、C的一个共同反例是函数\(g(x) = \paren[\big]{1 - 2 \fn H(x-1)}\ f(x)\).
  \fi

\item 下列函数中,在其定义域上有最大值和最小值的是\uline{\makebox[10em]{}}.
  \begin{itemize}
    \renewcommand{\labelitemi}{\faCircleThin}
  \item
    \begin{math}
      f(x) =
      \begin{dcases}
        \ln\abs{\,x\,}, & x \ne 0, \\
        0, & x = 0
      \end{dcases}
    \end{math}
    \ifshowsol
    \item[\faCircle]
    \else
    \item
    \fi
    \begin{math}
      f(x) = \ln(1 + \abs{\,x\,}),\ x \in \brkt{-1,1}
    \end{math}
  \item
    \begin{math}
      f(x) = \ln\abs{\,x\,}, x \in \brkt{-1,1} \setminus \Set{0}
    \end{math}
  \item
    \begin{math}
      f(x) =
      \begin{dcases}
        \ln\abs{\,x\,}, & 0 < \abs{\,x\,} < 1, \\
        0, & x = 0
      \end{dcases}
    \end{math}
  \end{itemize}

  \ifshowsol
    选项~A既没最大值也没最小值,选项~C和~D有最大值但没最小值.
  \fi

\item 下列函数中,在其定义域内有最大值的是\uline{\makebox[10em]{}}.
  \begin{itemize}
    \renewcommand{\labelitemi}{\faCircleThin}
    \ifshowsol
    \item[\faCircle]
    \else
    \item
    \fi
    \begin{math}
      f(x) = x/e^x,\ x \in \paren{0,+∞}
    \end{math}
  \item
    \begin{math}
      f(x) = 1/e^x,\ x \in \paren{0,+∞}
    \end{math}
  \item
    \begin{math}
      f(x) = 1/e^x,\ x \in \paren{0,1}
    \end{math}
  \item
    \begin{math}
      f(x) = 1/e^x,\ x \in \parenbrkt{0,1}
    \end{math}
  \end{itemize}

  \ifshowsol
    选项~B和~C既没最大值也没最小值,选项~D有最小值但没有最大值.下面来证明一下选项~A.

    \begin{proof}
      取一个正数\(x_0\),自然有\(\,f(x_0) > 0\).因为函数\(\,f\mkern2mu\)在正无穷处的极限是零,所以存在\(δ > 0\)使得当\(x > δ\)时都有\(\,f(x) < \,f(x_0)\).显然最大值不可能在区间\(\paren{δ, +∞}\)上.根据定理~\ref{thm:cont4ops}可知,函数\(\,f\mkern2mu\)在\(\R\)上是连续的,所以在\(\brkt{0,δ}\)上也是连续的.再使用定理~\ref{thm:evt}可知函数在此区间上有最大值.又因为~\(\,f(0) = 0\),所以原点不可能是函数的最大值.因此,最大值只能在\(\paren{0,δ}\)上了,自然也在\(\paren{0,+∞}\)上.
    \end{proof}
  \fi

\item 下列函数中,在\(\brktparen{0,+∞}\)上有界的是\uline{\makebox[6em]{}}.
  \begin{itemize}
    \renewcommand{\labelitemi}{\faCircleThin}
  \item \(\,f(x) = x^2 - 1\)
    \ifshowsol
    \item[\faCircle]
    \else
    \item
    \fi
    \(\,f(x) = \paren[\big]{\ln(1+x)}/\paren{1+x}\)
  \item \(\,f(x) = x^2 \sin\paren[\big]{1/\paren{1+x}}\)
  \item
    \begin{math}
      f(x) =
      \begin{dcases}
        e^{1/x}, & x > 0, \\
        0, & x = 0
      \end{dcases}
    \end{math}
  \end{itemize}

  \ifshowsol
    选项~B参照前一题.关于选项~C,当\(x \to +∞\)时,有
    \begin{align*}
      f(x)
      &= x^2 \sin\frac1{1+x} \\
      &= x⋅\frac{\sinp[\big]{1/(1+x)}}{1/(1+x)} - \frac{\sinp[\big]{1/(1+x)}}{1/(1+x)} + \sin\frac1{1+x} \\
      &\ge \frac12x - 2.
    \end{align*}
    可见选项~C中的函数\(\,f\mkern2mu\)有下界但无上界(下界是零).选项~D中的函数也是有下界但无上界(下界也是零).
  \fi

\item 下列函数中,在\(\R\)上有界的是\uline{\makebox[10em]{}}.
  \begin{itemize}
    \renewcommand{\labelitemi}{\faCircleThin}
  \item \(\,f \in C\paren{-∞,+∞}\)且\(\!\lim\limits_{x\to+∞\!}\,f(x) = 1\)
  \item \(\,f \in C\paren{-∞,+∞}\)且\(\!\lim\limits_{x\to-∞\!}\,f(x) = 1\)
    \ifshowsol
    \item[\faCircle]
    \else
    \item
    \fi
    \(\,f \in C\paren{-∞,+∞}\)且\(\lim\limits_{x\to∞}\,f(x) = 1\)
  \item \(\,f \in C\paren{-∞,+∞}\)且\(\abs*{\mkern6mu f(x)\,} \le \abs{\,x\,} / 2\)对于所有的\(x \in \R\)都成立
  \end{itemize}

  \ifshowsol
    函数\(1 + e^{-x}\)是选项~A的反例,函数\(1+e^x\)是选项~B的反例,函数\(x/4\)是选项~D的反例.下面来证明一下选项~C.

    \begin{proof}
      因为函数在无穷处的极限是\(1\),所以对于某个正数\(ε\)存在\(δ > 0\)使得当\(\abs{\,x\,} > δ\)时都有\({}\abs*{\mkern6mu f(x)-1\,} < ε\).因此函数在\(\paren{-∞,-δ} \cup \paren{δ,+∞}\)上有界.又因为函数在\(\brkt{-δ,δ}\)上连续,所以根据定理~\ref{thm:contclsintvbnd}它在此区间上也有界.综上所述,它在\(\R\)上有界.
    \end{proof}
  \fi

\item 下列形结论中,错误的是\uline{\makebox[10em]{}}.
  \begin{itemize}
    \renewcommand{\labelitemi}{\faCircleThin}
  \item 若函数\(\,f \in C\brkt{a,b}\)且\(\,f(a) < a,\ f(b) > b\),则存在\(ξ \in \paren{a,b}\)满足\(\,f(ξ)=ξ\)
    \ifshowsol
    \item[\faCircle]
    \else
    \item
    \fi
    若函数\(\,f \in C\parenbrkt{-∞,a}\)且\(\,f(a) = 0, \lim\limits_{x\to-∞\!}\,f(x) = 0\),则函数\(\,f\mkern2mu\)在区间\(\parenbrkt{-∞,a}\)上有正的最大值
  \item 方程\(x^5 - 3x - 1 = 0\)至少有一个根介于\(1\)和\(2\)之间
  \item 方程\(x = a + b \sin x\ (a > 0,\ b > 0)\)至少有一个不大于\(a+b\)的正根
  \end{itemize}

  \ifshowsol
    选项~A可以参照定理~\ref{cor:fixedpoint}的证明.选项~C和~D都可以由定理~\ref{thm:bolzano}得到.关于选项~D,令
    \begin{equation*}
      f(x) = x - a - b \sin x,
    \end{equation*}
    则有\(\,f(0) = -a < 0,\ f(a+b) = b\paren[\big]{1 - \sinp{a+b}}\).若\(\sinp{a+b} = 1\),则\(\,f(a+b) = 0\).若\(\sinp{a+b} \ne 1\),则\(\,f(a+b) > 0\).根据定理~\ref{thm:bolzano},函数\(\,f\mkern2mu\)在区间\(\paren{0,\ a+b}\)上有一个零点.选项~B的一个反例是\(\,f(x) = (x-a)e^x\),此函数在区间\(\parenbrkt{-\infty,a}\)上有负的最小值和为零的最大值.
  \fi
\end{enumerate}
\fi

\section{函数的一致连续性}

\begin{example*}
  函数\(\,f(x) = 1/x\)在区间\(\paren{0,+\infty}\)上处处连续.

  \begin{proof}
    令\(x_0 \in \paren{0,+\infty}\),则对于任意的正数\(ε\)都存在\(δ = \minb{\,{x_0}/{2}, \paren{{x_0^2}/{2}}ε\,}\)使得当\(\abs*{\,x - x_0\,} < δ\)时都有
    \begin{equation*}
      \abs*{\,\frac1x - \frac1{x_0}\,}
      = \abs*{\,\frac{x_0 - x}{x x_0}\,}
      < \frac{2}{x_0^2} \abs{\,x_0 - x\,}
      < \frac{2}{x_0^2} δ
      \le ε.
      \qedhere
    \end{equation*}
  \end{proof}

  \begin{proof}
    使用定理~\ref{thm:cont4ops}即可得证.
  \end{proof}
\end{example*}

\begin{example*}
  参见定理~\ref{thm:contsin}的证明过程.
\end{example*}

\begin{definition}[一致连续]
  \label{defn:unicont}
  设函数\(\,f\mkern2mu\)在区间\(I\)上有定义.对于任意的\(ε > 0\)都存在\(δ > 0\)使得:对于\(I\)上的任意两点\(x\)和\(y\),当\(\abs*{\,x - y\,} < δ\)时都有\(\abs*{\mkern6mu f(x) - f(y)\,} < ε\).此时,我们称函数\(\,f\mkern2mu\)在\(I\)上\kenten{一致连续}.
\end{definition}

% https://math.stackexchange.com/a/728364/147999
\begin{example*}
  函数\(\sqrt{x\mathstrut}\)在区间\(\brktparen{a,+\infty}\)上一致连续,其中\(a > 0\).

  \begin{proof}
    对于任意的正数\(ε\)都存在\(δ = 2ε \sqrt{a\mathstrut}\)使得:对于\(\brktparen{a,+\infty}\)上的任意两点\(x\)和\(y\),当\(\abs*{\,x - y\,} < δ\)时都有
    \begin{equation*}
      \abs[\Big]{\,\sqrt{x\mathstrut}\, - \sqrt{\smash{y}\mathstrut}\ }
      = \abs*{\,\frac{x - y}{\sqrt{x}\, + \sqrt{\smash[b]{y}}\,}\,}
      \le \frac{\abs*{\,x - y\,}}{2 \sqrt a}
      < \frac{δ}{2 \sqrt a}
      = ε.
      \qedhere
    \end{equation*}
  \end{proof}
\end{example*}

% https://math.stackexchange.com/q/569928/147999
\begin{example}
  \label{eg:unicontsqrt}
  函数\(\,f(x) = \sqrt x\)在区间\(\brktparen{0,+\infty}\)上也是一致连续的.

  % \begin{remark}
  %   下面给出两种证明方法.
  % \end{remark}
  \begin{proof}
    对于任意的\(ε > 0\),令\(δ = ε^2\),则对于区间\(\brktparen{0,+\infty}\)上任意两点\(x\)和\(y\),当\(\abs*{\,x - y\,} < δ\)时都有
    \begin{gather*}
      \abs[\Big]{\,\sqrt{x\mathstrut}\, - \sqrt{\smash{y}\mathstrut}\ }^2
      \le \abs[\Big]{\,\sqrt{x\mathstrut}\, - \sqrt{\smash{y}\mathstrut}\ } \, \paren[\Big]{\abs[\Big]{\,\sqrt{x\mathstrut}\ } + \abs[\Big]{\,\sqrt{\smash{y}\mathstrut}\ }}
      = \abs[\Big]{\,\sqrt{x\mathstrut}\, - \sqrt{\smash{y}\mathstrut}\ } \, \abs[\Big]{\,\sqrt{x\mathstrut}\, + \sqrt{\smash{y}\mathstrut}\ }
      = \abs*{\,x - y\,}
      < δ = ε^2,
      \shortintertext{即}
      \abs[\Big]{\,\sqrt{x\mathstrut}\, - \sqrt{\smash{y}\mathstrut}\ }
      < ε.
      \qedhere
    \end{gather*}
  \end{proof}

  % https://math.stackexchange.com/a/5098463/147999
  \begin{proof}
    对于任意的\(ε > 0\),令\(δ = ε^2\).定义区间\(I_a = \brktparen{a,a+δ}\),我们称\(I_a\)为滑动窗口.对于在滑动窗口上的任意两点\(x\)和\(y\)都有
    \begin{gather*}
      \sqrt{a\mathstrut} \le \sqrt{x\mathstrut}\, < \sqrt{a+δ\mathstrut}
      \ \qand
      \sqrt{a\mathstrut} \le \sqrt{\smash{y}\mathstrut}\, < \sqrt{a+δ\mathstrut}, \\
      \shortintertext{从而}
      - \paren[\big]{\sqrt{a+δ\mathstrut} - \sqrt{a\mathstrut}}
      < \sqrt{x\mathstrut}\, - \sqrt{\smash{y}\mathstrut}\,
      < \sqrt{a+δ\mathstrut} - \sqrt{a\mathstrut},
      \shortintertext{即}
      \abs[\Big]{\,\sqrt{x\mathstrut}\, - \sqrt{\smash{y}\mathstrut}\ }
      < \sqrt{a+δ\mathstrut}\, - \sqrt{a\mathstrut}
      = \frac{δ}{\sqrt{a+δ}\, + \sqrt{a\vphantom{δ}}}
      \le \frac{δ}{\sqrtδ}
      = ε.
    \end{gather*}
    注意到上面的关系对于所有的\(a \ge 0\)的成立.又由于说“区间\(\brktparen{0,+\infty}\)上的任意两点\(x\)和\(y\)当\({}\abs*{\,x - y\,} < δ\)时满足某个条件”\kenten{等价于}说“所有滑动窗口\(\brktparen{a,a+δ}\)上的任意两点\(x\)和\(y\)满足某个条件”,因而得证.
  \end{proof}
\end{example}

\begin{theorem*}[一致连续函数的线性组合]
  \label{thm:unicontlincomb}
  若函数\(\,f\mkern2mu\)和\(\mkern1mu g\)在\(I\)上一致连续,则它们的线性组合也在\(I\)上一致连续.

  \begin{proof}
    取\(δ = \minb{δ_1, δ_2}\),再利用三角不等式即可得证.其中\(δ_1\)和\(δ_2\)是函数\(\,f\mkern2mu\)和\(\mkern2mu g\)对于同一个\(ε\)各自取的德尔塔.
  \end{proof}
\end{theorem*}

\begin{theorem*}[复合函数的一致连续性]
  \label{thm:unicontcomp}
  若函数\(\,f\mkern2mu\)在区间\(I_{\text{外}}\!\)上一致连续、函数\(\mkern1mu g\)在区间\(I_{\text{内}}\!\)上一致连续、\(\ran g \subset I_{\text{外}}\),则函数\(\,f \circ g\)在区间\(I_{\text{内}}\!\)上一致连续.

  \begin{proof}
    因为函数\(\,f\mkern2mu\)在\(I_{\text{外}}\!\)上一致连续,所以对于任意的\(ε > 0\)都存在\(δ_{\text{外}}\! > 0\)使得:对于\(I_{\text{外}}\!\)上的任意两点\(u\)和\(v\),当\(\abs*{\,u - v\,} < δ_{\text{外}}\!\)时都有\(\abs*{\mkern6mu f(u) - f(v)\,} < ε\).又因为函数\(\mkern1mu g\)在\(I_{\text{内}}\!\)上一致连续,所以存在\(δ_{\text{内}}\! > 0\)使得:对于\(I_{\text{内}}\!\)上的任意两点\(x\)和\(y\),当\(\abs*{\,x - y\,} < δ_{\text{内}}\!\)都有\(\abs*{\mkern4mu g(x) - g(y)\,} < δ_{\text{外}}\),从而有~\(\abs*{\mkern6mu f\,\paren*{g(x)} - f\,\paren*{g(y)}\,} < ε\).
  \end{proof}
\end{theorem*}

\begin{theorem}
  \label{thm:unicont2cont}
  一致连续蕴含连续.

  \begin{proof}
    使用定义~\ref{defn:unicont}和~\ref{defn:cont}即可得证.
  \end{proof}
\end{theorem}

\begin{theorem}
  \label{thm:unicontbndintvbnd}
  一致连续函数在有界区间上有界.

  \begin{proof}
    使用反证法.假设一致连续函数在有界区间上无界.那么可以构造有界数列\(\Seq{x_n}\)使得数列\({}\Seq{\,f(x_n)}\)为无穷大量.由定理~\ref{thm:bw}可知,存在收敛于某个常数\(L\)的子列\(\Seq{x_{n_k}}\).从而对于所有的\(δ > 0\)都能在此有界区间上找到两点\(x_{n_a}\)和\(x_{n_b}\)使得\(\abs[\big]{\,x_{n_a} - x_{n_b}\,} < δ\)且\(\abs[\big]{\mkern6mu f(x_{n_a}) - f(x_{n_b})\,} \ge ε_0\),其中\(ε_0\)为\kenten{某个正数}.矛盾.
  \end{proof}
  % TODO: Add the proof from the lecture
\end{theorem}

% https://mathworld.wolfram.com/TopologistsSineCurve.html
\begin{example}[拓扑学家的正弦曲线]
  \label{eg:uniconttoposine}
  函数\(\sin(1/x)\)在区间\(\paren{0,1}\)上连续且有界,但它在此区间上不是一致连续的.

  \begin{proof}
    对于任意的\(δ > 0\),取\(1/x = \pi/2 + 2k\pi,\ 1/y = \pi/2 + (2k+1)\pi\),则有
    \begin{equation*}
      x - y = (1/y - 1/x)xy = \pi xy = \bigOp{1/k^2}.
    \end{equation*}
    当\(k\)取的足够大时,自然能使\(\abs*{\,x-y\,} < δ\),但是\(\abs[\big]{\,\sin(1/x) - \sin(1/y)\,} = 2 > 1\).
  \end{proof}
\end{example}

\begin{theorem*}
  有界区间上的一致连续函数,其乘积函数也一致连续.

  \begin{proof}
    仿照定理~\ref{thm:seq4ops}中对\enumparen{2}的证明,再使用定理~\ref{thm:unicontbndintvbnd}中的结果来完成放缩,即可得证.
  \end{proof}
\end{theorem*}

\begin{theorem}[Heine--Cantor定理]
  \label{thm:hc}
  函数\(\,f\mkern2mu\)在闭区间上一致连续的充分必要条件是它在此区间上连续.

  \begin{proof}
    必要性见定理~\ref{thm:unicont2cont}.下面使用反证法来证明一下充分性.假设函数\(\,f\mkern2mu\)在闭区间上不是一致连续的,那么就能找到这样的两个有界数列\(\Seq{x_n}\)和\(\Seq{y_n}\)使得\(\abs*{\,x_n - y_n\,} < 1/2^n\)但\(\abs*{\mkern6mu f(x_n) - f(y_n)\,}\)却恒大于等于某个正数\(ε_0\).由例~\ref{eg:seqbndcmnidx}可知,存在收敛子列\(\Seq{x_{n_k}}\)和\(\Seq{y_{n_k}}\).因为~\(\abs[\big]{\,x_{n_k} - y_{n_k}} < 1/2^{n_k}\),所以它们都收敛于同一个数,将此数记作\(L\).显然数\(L\)也在闭区间上.又因为函数\(\,f\mkern2mu\)在闭区间上连续,所以由定理~\ref{thm:contequiv}可知,数列\(\Seq{\,f(x_{n_k})}\)和\(\Seq{\,f(y_{n_k})}\)都收敛于\(f(L)\),从而数列\(\Seq[\big]{\,\abs[\big]{\mkern6mu f(x_{n_k}) - f(y_{n_k})\,}\,}\)收敛于零.矛盾.
    % 所以存在\(δ > 0\)使得当\(\abs[\big]{\,x-L\,} < δ\)时就有\(\abs*{\mkern6mu f(x)-f(L)\,} < ε_0/2\).那么根据三角不等式,就能找到正整数\(K\)使得当\(k > K\)时都有\(\abs[\big]{\mkern6mu f(x_{n_k}) - f(y_{n_k})\,} < ε_0\).矛盾.
  \end{proof}
\end{theorem}

\begin{theorem}
  \label{thm:unicontbothlim}
  函数\(\,f\mkern2mu\)在区间上一致连续的充分条件是:它在此区间上连续且左右端点的单侧极限都存在.特别地,当区间为有界区间时,此条件也是必要条件.

  \begin{proof}
    先证对有界区间的必要性.由定理~\ref{thm:unicont2cont}可知,只需证明它在端点处的单侧极限存在即可.实际上,因为区间端点是确定的数且函数在区间上一致连续,所以对于任意在区间上的非端点数列\(\Seq{x_n}\),当它收敛于端点时,数列\(\Seq{\,f(x_n)}\)就一定是柯西列.由于定理~\ref{thm:seqcvgcauchy}和三角不等式,这些柯西列都收敛于同一个数.再由定理~\ref{thm:limequiv}可知,函数在端点的单侧极限存在.

    再证充分性.当区间是独立点时,显然是成立的.下面仅考虑区间不是孤立点的情况.记函数在左右端点处的极限分别为\(A\)和\(B\).对于任意的正数\(ε\),能找到左右端点的单侧邻域使得当\(x\)落在邻域上时都有\(\abs*{\mkern6mu f(x)-A\,} < ε/2\)和\(\abs*{\mkern6mu f(x) - B\,} < ε/2\).在左端点的邻域上任取一点\(a\),在右端点的邻域上任取一点\(b\),就找到了一个闭区间\(\brkt{a,b}\).由定理~\ref{thm:hc}可知,能找到\(δ_{\text{闭}}\! > 0\)使得函数在\(\brkt{a,b}\)上满足一致连续的条件.令\(δ_{\text{左}}\!\)为\(a\)到左端点邻域右端的距离,\(δ_{\text{右}}\!\)为\(b\)到右端点邻域左端的距离.取\(δ = \minb{δ_{\text{左}}, δ_{\text{闭}}, δ_{\text{右}}}\)即可得证.
  \end{proof}

  \begin{remark}
    上述充分性证明中的邻域,有可能是在无穷处的邻域,因为区间的端点可能是正负无穷.
  \end{remark}
\end{theorem}

\subpdfbookmark{思考}{B1.3.3.P}
\subsection*{思考}

\begin{enumerate}
\item 函数在区间上的处处连续和一致连续有什么区别,有什么联系?

  \ifshowsolp
    一致连续是比处处连续更强的条件,它需要的更多,给出的也更多.一致连续一定有处处连续,但处处连续未必一致连续,需要在加上一些条件才能得到一致连续.
  \fi

\item 在有界开区间或者无穷区间上的连续函数未必是一致连续的.请举例说明.

  \ifshowsolp
    正切函数\(\tan x\)在有界开区间\(\paren{-\pi/2,\pi/2}\)上处处连续,但不是一致连续的.平方函数\(x^2\)在无穷区间\(\paren{-\infty,+\infty}\)上处处连续,但也不是一致连续的.
  \fi
\end{enumerate}

\ifshowex
\currentpdfbookmark{练习}{B1.3.3.E}
\subsection*{练习}

\begin{enumerate}
\item 设函数\(\,f\mkern2mu\)在区间\(I\)上有定义.下列说法中,与函数\(\,f\mkern2mu\)在区间\(I\)上一致连续不等价的是\uline{\hfill}.
  \begin{itemize}
    \renewcommand{\labelitemi}{\faCircleThin}
  \item
    \begin{math}
      \paren[\big]{\forall ε > 0}
      \paren[\big]{\exists δ > 0}
      \paren[\big]{\forall x,y \in I}
      \paren[\big]{\forall \abs[\big]{\,x-y\,} < δ}
      \paren[\big]{\abs[\big]{\mkern6mu f(x) - f(y)\,} < ε}
    \end{math}
  \item
    \begin{math}
      \paren[\big]{\forall ε \in \paren{0,1}}
      \paren[\big]{\exists δ > 0}
      \paren[\big]{\forall x,y \in I}
      \paren[\big]{\forall \abs[\big]{\,x-y\,} < δ}
      \paren[\big]{\abs[\big]{\mkern6mu f(x) - f(y)\,} < ε}
    \end{math}
  \item
    \begin{math}
      \paren[\big]{\forall ε > 0}
      \paren[\big]{\exists δ > 0}
      \paren[\big]{\forall x,y \in I}
      \paren[\big]{\forall \abs[\big]{\,x-y\,} < δ}
      \paren[\big]{\abs[\big]{\mkern6mu f(x) - f(y)\,} < kε}
    \end{math},
    其中\(k\)是某个正数
    \ifshowsol
    \item[\faCircle]
    \else
    \item
    \fi
    对于无穷多个正数\(ε\),存在\(δ > 0\)使得:对于区间\(I\)上的任意两点\(x\)和\(y\),当\(\abs*{\,x-y\,} < δ\)时都有\(\abs*{\mkern6mu f(x) - f(y)\,} < ε\)
  \end{itemize}

  \ifshowsol
    单位阶跃函数\(\Fn H\)是选项~D的一个反例.
  \fi

\item 关于函数的一致连续性,下列说法中,错误的是\uline{\makebox[10em]{}}.
  \begin{itemize}
    \renewcommand{\labelitemi}{\faCircleThin}
  \item 若函数\(\,f\mkern2mu\)在\(\brkt{a,b}\)上连续,则函数\(\,f\mkern2mu\)在\(\brkt{a,b}\)上一致连续
  \item 若函数\(\,f\mkern2mu\)在\(\brkt{a,b}\)上连续,则函数\(\,f\mkern2mu\)在\(\paren{a,b}\)上一致连续
    \ifshowsol
    \item[\faCircle]
    \else
    \item
    \fi
    若函数\(\,f\mkern2mu\)在\(\paren{a,b}\)上连续,则函数\(\,f\mkern2mu\)在\(\paren{a,b}\)上一致连续
  \item 若函数\(\,f\mkern2mu\)在\(\brkt{a,b}\)上一致连续,则函数\(\,f\mkern2mu\)在\(\brkt{a,b}\)上连续
  \end{itemize}

\item 设函数\(\,f\mkern2mu\)在区间\(I\)上有定义.\uline{\hfill}可以表明函数\(\,f\mkern2mu\)在区间\(I\)上不是一致连续的.
  \begin{itemize}
    \renewcommand{\labelitemi}{\faCircleThin}
  \item
    \begin{math}
      \paren[\big]{\exists ε_0 > 0}
      \paren[\big]{\exists δ > 0}
      \paren[\big]{\exists x,y \in I}
      \paren[\big]{\abs[\big]{\,x-y\,} < δ \tand \abs[\big]{\mkern6mu f(x) - f(y)\,} \ge ε_0}
    \end{math}
  \item
    \begin{math}
      \paren[\big]{\forall ε_0 > 0}
      \paren[\big]{\exists δ > 0}
      \paren[\big]{\exists x,y \in I}
      \paren[\big]{\abs[\big]{\,x-y\,} < δ \tand \abs[\big]{\mkern6mu f(x) - f(y)\,} \ge ε_0}
    \end{math}
    \ifshowsol
    \item[\faCircle]
    \else
    \item
    \fi
    存在\(ε_0 > 0\)使得区间\(I\)上除了有限个\(x\)和\(y\)之外都满足\(\abs*{\mkern6mu f(x) - f(y)\,} \ge ε_0\)
  \item 存在\(ε_0 > 0\)使得区间\(I\)上有无穷多个\(x\)和\(y\)满足\(\abs*{\mkern6mu f(x) - f(y)\,} \ge ε_0\)
  \end{itemize}

  \ifshowsol
    恒等函数\(x\)满足选项~A、B和~D,但却是一致连续的.
    % TODO: report.  Option C is actually unsatisfiable
  \fi

\item 下列函数中,在区间\(\paren{0,+\infty}\)上一致连续的是\uline{\makebox[6em]{}}.
  \begin{itemize}
    \renewcommand{\labelitemi}{\faCircleThin}
  \item \(y = x^2\)
    \ifshowsol
    \item[\faCircle]
    \else
    \item
    \fi
    \(y = \sqrt{x\mathstrut}\)
  \item \(y = 1/x\)
  \item \(y = \ln x\)
  \end{itemize}

  \ifshowsol
    存在\(ε > 0\)使得对于任意的\(2δ > 0\)都能找到\(x \in \paren{0,+\infty}\)满足\(\paren{x+δ}^2 - x^2 = 2δ x + δ^2 > ε\),只要取\(x = ε/δ\)即可.因此,选项~A中的函数并不一致连续.选项~B见例~\ref{eg:unicontsqrt}.由定理~\ref{thm:unicontbndintvbnd}可知,选项~C和~D中函数并不一致连续.
  \fi

\item 下列函数中,在区间\(\paren{0,1}\)上一致连续的是\uline{\makebox[6em]{}}.
  \begin{itemize}
    \renewcommand{\labelitemi}{\faCircleThin}
    \ifshowsol
    \item[\faCircle]
    \else
    \item
    \fi
    \(y = e^{-x}\)
  \item \(y = \ln x\)
  \item \(y = \sin\frac1x\)     % topology's sine curve
  \item \(y = 1/x^2\)
  \end{itemize}

  \ifshowsol
    由定理~\ref{thm:unicontbothlim}可知,选项~A中的函数在此区间上是一致连续的.由定理~\ref{thm:unicontbndintvbnd}可知,选项~B和~D中的函数并不一致收敛.选项~C见例~\ref{eg:uniconttoposine}.
  \fi

\item 函数\uline{\hspace{10em}}在区间\(\brkt{0,1}\)上不是一致连续的.
  \begin{itemize}
    \renewcommand{\labelitemi}{\faCircleThin}
  \item
    \begin{math}
      f(x) =
      \begin{cases}
        \dfrac{\ln(1+x)}{x}, & x \in \parenbrkt{0,1}, \\
        1, & x = 0
      \end{cases}
    \end{math}
  \item
    \begin{math}
      f(x) =
      \begin{cases}
        \dfrac{\sin x}{x}, & x \in \parenbrkt{0,1}, \\
        1, & x = 0
      \end{cases}
    \end{math}
  \item
    \begin{math}
      f(x) =
      \begin{cases}
        \dfrac{e^x-1}{x}, & x \in \parenbrkt{0,1}, \\
        1, & x = 0
      \end{cases}
    \end{math}
    \ifshowsol
    \item[\faCircle]
    \else
    \item
    \fi
    \begin{math}
      f(x) =
      \begin{cases}
        x \sin\dfrac1x, & x \in \parenbrkt{0,1}, \\
        1, & x = 0
      \end{cases}
    \end{math}
  \end{itemize}

  \ifshowsol
    由定理~\ref{thm:hc}可知,前三个选项中的函数是一致连续的.关于选项~D中的函数,令\(1/x = 2\,k\pi\),当\(k\)足够大时,自然能使\(\abs{x-0}\)小于任意的\(δ\),然而\(\abs*{\mkern6mu f(x) - f(0)\,} = 1\).因此它不是一致连续的.实际上,如果修改定义,让此函数在原点处的值为零,同样可由定理~\ref{thm:hc}得到一致连续性.
  \fi

\item 关于函数\(\lnp{1+x}\)的连续性,下列说法中,错误的是\uline{\hspace{10em}}.
  \begin{itemize}
    \renewcommand{\labelitemi}{\faCircleThin}
  \item 它在区间\(\paren{-1,+\infty}\)上处处连续
    \ifshowsol
    \item[\faCircle]
    \else
    \item
    \fi
    它在区间\(\paren{-1,+\infty}\)上一致连续
  \item 它在区间\(\paren{-1,+\infty}\)上没有最大值和最小值
  \item 它在区间\(\brktparen{1,+\infty}\)上一致连续
  \end{itemize}

  % https://math.stackexchange.com/a/5098943/147999
  \ifshowsol
    取出子区间\(\paren{-1,0}\),再由定理~\ref{thm:unicontbndintvbnd}可知,选项~B中的函数在此子区间上不是一致连续的,自然在原区间上也不是一致连续的.关于选项~D,对于任意的\(ε > 0\),有
    \begin{equation*}
      \lnp{1+x+\fdx} - \lnp{1+x} = \lnp[\bigg]{1 + \frac{\fdx}{1+x}} \le \frac{\fdx}{1+x} \le \frac{\fdx}{2} \le \fdx,
    \end{equation*}
    取\(δ = ε\),则当\(0 \le \fdx < δ\)时,自然有\(\,f(x+\fdx) - f(x) < δ = ε\).因此它在区间\(\brktparen{1,+\infty}\)上是一致连续的.
  \fi

\item 已知函数\(\,f\mkern2mu\)和\(\mkern2mu g\)在区间\(\brkt{a,b}\)上连续.下列说法中,错误的是\uline{\hspace{8em}}.
  \begin{itemize}
    \renewcommand{\labelitemi}{\faCircleThin}
  \item 函数\(\maxb{\,f, g}\)在区间\(\brkt{a,b}\)上一致连续
  \item 函数\(\minb{\,f, g}\)在区间\(\brkt{a,b}\)上一致连续
  \item 函数\(\,fg\)在区间\(\brkt{a,b}\)上一致连续
    \ifshowsol
    \item[\faCircle]
    \else
    \item
    \fi
    函数\(\sgnp{\,fg}\)在区间\(\brkt{a,b}\)上一致连续
  \end{itemize}

  \ifshowsol
    令\(\,f(x) = x,\ g(x) = x^2\),则选项~D中的函数\(\sgnp[\big]{\,f(x)\,g(x)} = \sgnp{x^3} = \sgnp{x}\)在原点处间断,自然也就不是一致连续的.因为\(\brkt{a,b}\)是个闭区间,所以根据定理~\ref{thm:hc},只要证明前三个选项中的函数是连续的,就能说明它们也是一致连续的.选项~C中的函数使用定理~\ref{thm:cont4ops}即可.前两个选项中的函数,利用练习~\ref{B1.1.1.E}题~\ref{B1.1.1.E9}中的公式,再加上定理~\ref{thm:cont4ops},只需证明:若函数\(\,f\mkern2mu\)连续,则函数\(\abs*{\mkern6mu f\ }\)也连续.对于任意的\(ε > 0\),因为函数\(\,f\mkern2mu\)在点\(x_0\)处连续,所以存在\(δ > 0\)使得当\({}\abs*{\,x-x_0\,} < δ\)时就有
    \begin{equation*}
      \abs[\Big]{\,\abs*{\mkern6mu f(x)\,} - \abs*{\mkern6mu f(x_0)\,}\,} \le \abs*{\mkern6mu f(x) - f(x_0)\,} < ε.
    \end{equation*}
  \fi
\end{enumerate}
\fi

% Local Variables:
% TeX-engine: luatex
% TeX-master: "微积分B"
% LaTeX-indent-begin-exceptions-list: ("ifshowex")
% End:
