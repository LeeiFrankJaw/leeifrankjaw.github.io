\documentclass[a4paper,punct=CCT]{ctexbook}

\title{微积分B}
\author{赵磊}

\usepackage[T1]{fontenc}
\usepackage{textcomp}
\usepackage{mathtools,amssymb,amsthm}
\usepackage[hmargin=1in,vmargin=1in]{geometry}
\usepackage{graphicx,xcolor}
\usepackage[pdfusetitle]{hyperref}
\hypersetup{%
  colorlinks=true,
  urlcolor=[rgb]{0,0.2,0.6},
  linkcolor={.},
  bookmarksdepth=2}
\usepackage{bookmark}
\usepackage{float}

\frenchspacing

\newcommand*{\parasp}{\setlength{\parskip}{10pt plus 2pt minus 3pt}}
\newcommand*{\noparasp}{\setlength{\parskip}{0pt plus 1pt}}
\newcommand*{\setparasp}[1]{\setlength{\parskip}{#1}}
\newcommand*{\pskip}{\vskip 10pt plus 2pt minus 3pt}
% \newcommand\LEFTRIGHT[3]{\left#1 #3 \right#2}
\newcommand\SetSymbol[1][]{%
  \nonscript\:#1\vert
  \allowbreak
  \nonscript\:
  \mathopen{}}
% \newcommand*{\paren}[1]{\LEFTRIGHT(){#1}}
\DeclarePairedDelimiterX{\paren}[1]{\lparen}{\rparen}{%
  \renewcommand{\mid}{\SetSymbol[\delimsize]}#1}
% \newcommand*{\brkt}[1]{\LEFTRIGHT[]{#1}}
\DeclarePairedDelimiterX{\brkt}[1]{\lbrack}{\rbrack}{%
  \renewcommand{\mid}{\SetSymbol[\delimsize]}#1}
\DeclarePairedDelimiterX{\brce}[1]{\lbrace}{\rbrace}{%
  \renewcommand{\mid}{\SetSymbol[\delimsize]}#1}
\DeclarePairedDelimiterX{\parenbrkt}[1]{\lparen}{\rbrack}{%
  \renewcommand{\mid}{\SetSymbol[\delimsize]}#1}
\DeclarePairedDelimiterX{\brktparen}[1]{\lbrack}{\rparen}{%
  \renewcommand{\mid}{\SetSymbol[\delimsize]}#1}
\newcommand*{\unit}[1]{\,\mathrm{#1}}
\newcommand*{\DeclareUnit}[1]{\expandafter\def\csname#1\endcsname{\unit{#1}}}
\DeclareUnit{cm}
% \renewcommand*{\m}{\unit{m}}
\DeclareUnit{m}
\DeclareUnit{kg}
\DeclareUnit{s}
\newcommand*{\R}{\mathbb{R}}
\newcommand*{\Z}{\mathbb{Z}}
\newcommand*{\N}{\mathbb{N}}
\newcommand*{\Q}{\mathbb{Q}}
% \newcommand*{\Rp}{(0,+\infty)}
% \newcommand*{\Rm}{(-\infty,0)}
\newcommand*{\deduce}{\mathrel{\Downarrow}}
\DeclarePairedDelimiter{\ceil}{\lceil}{\rceil}
\DeclarePairedDelimiter{\floor}{\lfloor}{\rfloor}
\newcommand*{\textop}[1]{\mathbin{\text{#1}}}
\newcommand*{\tand}{\textop{and}}
\newcommand*{\tor}{\textop{or}}
\newcommand*{\txt}[2][\quad]{#1 \text{#2} #1}
\newcommand*{\qand}{\txt{and}}
\newcommand*{\iand}{\intertext{and}}
\newcommand*{\siand}{\shortintertext{and}}
\newcommand*{\DeclareText}[1]{\expandafter\def\csname#1\endcsname{\text{#1}}}
\DeclareText{otherwise}
\newcommand*{\tfor}{\text{for }}
\newcommand*{\qfor}{\txt{for}}
\let\reason\text

\newcommand*{\enumparen}[1]{\textnormal{(}\makebox[0.6em][c]{#1}\textnormal{)}}
\renewcommand{\labelenumii}{\enumparen{\theenumii}}
\newcommand*{\upstar}{\textsuperscript{\normalfont\textasteriskcentered}}%
\makeatletter
\newcommand*{\bonus}{\@itemlabel\upstar}%
\def\contitem{%
  \def\H@item{%
    \@inmatherr\item
    \@noitemargtrue
    \@ifnextchar[\@item{\@item[\@itemlabel]}}}
\makeatother

\let\sin\relax
\DeclareMathOperator{\sin}{\smash{sin}}
\DeclareMathOperator{\arccosh}{arccosh}
\DeclareMathOperator{\arcsinh}{arcsinh}
\DeclareMathOperator{\arctanh}{arctanh}
\DeclareMathOperator{\arccoth}{arccoth}
\DeclareMathOperator{\arccot}{arccot}
\DeclareMathOperator{\atantwo}{atan2}
\DeclareMathOperator{\sech}{sech}
\DeclareMathOperator{\arcsech}{arcsech}
\DeclareMathOperator{\sgn}{sgn}
\DeclareMathOperator{\var}{var}
\DeclareMathOperator{\Ber}{Bernoulli}
\DeclareMathOperator{\Cov}{Cov}
\DeclareMathOperator{\E}{E}
\DeclareMathOperator{\ran}{ran}
\def\argmax{\qopname\relax m{arg\,max}}
\DeclarePairedDelimiterXPP{\Eb}[1]{\E}{\lbrack}{\rbrack}{}{%
  \renewcommand{\mid}{\SetSymbol[\delimsize]}#1}
\DeclarePairedDelimiterXPP{\varp}[1]{\var}{\lparen}{\rparen}{}{%
  \renewcommand{\mid}{\SetSymbol[\delimsize]}#1}
\DeclarePairedDelimiterXPP{\Covp}[1]{\Cov}{\lparen}{\rparen}{}{%
  \renewcommand{\mid}{\SetSymbol[\delimsize]}#1}
\DeclarePairedDelimiterXPP{\expb}[1]{\exp}{\lbrace}{\rbrace}{}{#1}
\DeclarePairedDelimiterXPP{\maxb}[1]{\max}{\lbrace}{\rbrace}{}{#1}
\DeclarePairedDelimiterXPP{\minb}[1]{\min}{\lbrace}{\rbrace}{}{#1}
\DeclarePairedDelimiterXPP{\lnp}[1]{\ln}{\lparen}{\rparen}{}{#1}
\DeclarePairedDelimiterXPP{\Set}[1]{\mathop{}}{\lbrace}{\rbrace}{}{%
  \renewcommand{\mid}{\SetSymbol[\delimsize]}#1}
\let\Seq\Set
\DeclarePairedDelimiterXPP{\abs}[1]{\mathop{}}{\lvert}{\rvert}{}{#1}
\let\card\abs
\DeclarePairedDelimiterXPP{\sinp}[1]{\sin}{\lparen}{\rparen}{}{#1}
\DeclarePairedDelimiterXPP{\cosp}[1]{\cos}{\lparen}{\rparen}{}{#1}
\DeclarePairedDelimiterXPP{\tanp}[1]{\tan}{\lparen}{\rparen}{}{#1}
\DeclarePairedDelimiterXPP{\sgnp}[1]{\sgn}{\lparen}{\rparen}{}{#1}
\newcommand*{\Fn}[1]{\mathop{\relax #1}\nolimits}
\newcommand*{\fn}[1]{\mathop{\relax\kern0pt #1}\nolimits}
\newcommand*{\gammaf}{\Fn{\Gamma}}
\renewcommand*{\Pr}{\Fn{P}}
\newcommand*{\littleo}{\Fn{o}}
\DeclarePairedDelimiterXPP{\littleop}[1]{\littleo}{\lparen}{\rparen}{}{#1}
\newcommand*{\bigO}{\Fn{O}}
\DeclarePairedDelimiterXPP{\bigOp}[1]{\bigO}{\lparen}{\rparen}{}{#1}
\DeclarePairedDelimiterXPP{\Prp}[1]{\Pr}{\lparen}{\rparen}{}{%
  \renewcommand{\mid}{\SetSymbol[\delimsize]}#1}
\newcommand*{\pnorm}{\Fn{\Phi}}
\DeclarePairedDelimiterXPP{\pnormp}[1]{\pnorm}{\lparen}{\rparen}{}{#1}
\newcommand*{\dnorm}{\fn{\varphi}}
\DeclarePairedDelimiterXPP{\dnormp}[1]{\dnorm}{\lparen}{\rparen}{}{#1}
\newcommand*{\qnorm}{\Fn{\Phi}^{-1}}
%\newcommand*{\diff}{\mathop{}\!d}
\newcommand*{\diff}{\mathop{}\!\mathit{d}}
%\newcommand*{\diff}{\mathop{}\!\mathrm{d}}
\newcommand*{\dx}{\diff x}
\newcommand*{\dy}{\diff y}
\newcommand*{\dz}{\diff z}
\newcommand*{\ds}{\diff s}
\newcommand*{\dt}{\diff t}
\newcommand*{\du}{\diff u}
\newcommand*{\dv}{\diff v}
\newcommand*{\dtheta}{\diff \theta}
\newcommand*{\dd}[2][]{\frac{\diff#1}{\diff#2}}
\newcommand*{\ddn}[3][]{\frac{\diff^{#3}#1}{\diff#2^{#3}}}
\newcommand*{\ddx}{\frac{\diff}{\dx}}
\newcommand*{\ddt}{\frac{\diff}{\dt}}
\newcommand*{\ddy}{\dd y}
\newcommand*{\ddtheta}{\frac{\diff}{\dtheta}}
\newcommand*{\ddz}{\dd z}
\newcommand*{\fwdf}{\mathop{}\!\Delta}
\newcommand*{\fdx}{\fwdf x}
\newcommand*{\fdy}{\fwdf y}
\newcommand*{\fdu}{\fwdf u}
\newcommand*{\dydx}{\frac\dy\dx}
\newcommand*{\pdpd}[2][]{\frac{\partial#1}{\partial#2}}
\newcommand*{\pdpdx}{\frac\partial{\partial x}}
\newcommand*{\pdpdy}{\frac\partial{\partial y}}
\newcommand*{\pdpdz}{\frac\partial{\partial z}}
\newcommand*{\pdpdu}{\frac\partial{\partial u}}
\newcommand*{\pdpdv}{\frac\partial{\partial v}}
\newcommand*{\pdpdt}{\frac\partial{\partial t}}
\newcommand*{\pdzpdx}{\frac{\partial z}{\partial x}}
\newcommand*{\pdzpdy}{\frac{\partial z}{\partial y}}
\newcommand*{\pdzpdt}{\frac{\partial z}{\partial t}}
\newcommand*{\pdxpdt}{\frac{\partial x}{\partial t}}
\newcommand*{\pdypdt}{\frac{\partial y}{\partial t}}


\newcommand*{\mreason}[1]{#1}
\newcommand*{\ex}[2]{\textbf{例#1:}#2}
\newcommand*{\disp}[1]{\( \displaystyle #1 \)}
\newcommand*{\exds}[2]{\ex{#1}\disp{#2}}

\usepackage{cancel}

% \usepackage[lite,subscriptcorrection,nofontinfo]{mtpro2}
\usepackage{fontspec}

\defaultfontfeatures{Ligatures=TeX}
\setmainfont{Palatino Linotype}
%\usepackage[integrals]{wasysym}
% \usepackage{fontawesome}

% \xeCJKsetup{CJKecglue=\,}
% \ltjsetparameter{xkanjiskip={0.13\zw plus 1pt minus 1pt}}
\setCJKmainfont{Songti SC}[
  BoldFont = * Black,
  ItalicFont = * Bold
]

\usepackage[math-style=TeX]{unicode-math}
\setmathfont{TeX Gyre Pagella Math}

\usepackage{microtype}

\newif\ifshowsol
\newif\ifshowex
% \showsoltrue
\showextrue

\let\reason\text

\AtBeginDocument{%
  % \renewcommand{\perp}{\mathrel{\bot}}
  \let\leq\leqslant
  \let\le\leq
  \let\geq\geqslant
  \let\ge\geq}

\begin{document}
\frontmatter

\maketitle
\thispagestyle{empty}
% \cleardoublepage

\chapter*{前言}

这是我在首期“乘风计划”中《微积分B》课程中做的学习笔记.

\clearpage
\thispagestyle{empty}

\mainmatter

\part{极限理论与一元函数}

\setcounter{chapter}{5}

\chapter{原函数与不定积分}

\section{概念与性质}

\section{换元积分法}

\subsection{第一换元法}

设$ \int f(u) \, dx = F(u) + C $且$\varphi(x) \in C^1$,
则$ \int f(\varphi(x)) \varphi'(x) \, dx = F(\varphi(x)) +C $.

\textbf{证明:}$ (\text{左边})' = f(\varphi(x)) \varphi'(x) = (\text{右边})' $. \qed

在实际使用中, 换元法体现了莱布尼茨微分记号的优越性. 将
\begin{align*}
  \varphi'(x) \, dx
  & = d\varphi(x) \\
  \intertext{代入原式, 得}
  \int f(\varphi(x)) \varphi'(x) \, dx
  & = \int f(\varphi(x)) \, d\varphi(x)
    \intertext{用$ u $代替$ \varphi(x) $}
  & = \int f(u) \, du \\
  & = F(u) + C \\
  & = F(\varphi(x)) + C.
\end{align*}

\exds{1}{ \int x \sin(x^2) \, dx. }

将$x \dx = \frac12 \diff(x^2)$代入得
\begin{align*}
  \int x \sin(x^2) \, dx
  &= \frac{1}{2} \int \sin(x^2) \, d(x^2) \\
  \intertext{用$u$代替$x^2$}
  &= \frac{1}{2} \int \sin u \, du \\
  &= -\frac{1}{2} \cos(u) + C \\
  &= -\frac{1}{2} \cos(x^2) + C.
\end{align*}

\exds{2}{ \int \cot x \, dx. }
\begin{align*}
  \text{原式}
  &= \int \frac{\cos x}{\sin x} \, dx
  && \mreason{\cot x = \frac{\cos x}{\sin x} }\\
  &= \int \frac{1}{\sin x} \, d(\sin x)
  && \mreason{ d\sin x = \cos x \, dx } \\
  &= \int \frac{du}{u}
  && \mreason{ u = \sin x } \\
  &= \ln \abs{u} + C
  && \mreason{ d\ln \abs{u} = \frac{1}{u} \, du } \\
  &= \ln \abs{\sin x} + C.
  && \mreason{ u = \sin x }
\end{align*}

\exds{3}{ \int \frac{dx}{a^2 + x^2} \qquad (a \ne 0). }
\begin{align*}
  \text{原式} &= \frac{1}{a^2} \int \frac{dx}{1+\paren*{\frac{x}{a}}^2}
  && \reason{因为$a \ne 0$, 提取因子$\frac{1}{a^2}$} \\
              &= \frac{1}{a^2} \int \frac{d(au)}{1+u^2}
  && \mreason{ u = \frac{x}{a} } \\
              &= \frac{1}{a} \int \frac{du}{1+u^2}
  && \mreason{ d(au) = a \, du } \\
              &= \frac{1}{a}\arctan u + C
  && \mreason{ d\arctan u = \frac{1}{1+u^2} \, du } \\
              &= \frac{1}{a}\arctan \frac{x}{a} + C.
  && \mreason{ u = \frac{x}{a} }
\end{align*}

\exds{4}{ \int \frac{dx}{\sqrt{a^2 - x^2}} \qquad (a > 0). }
\begin{align*}
  \text{原式} &= \frac{1}{a} \int \frac{dx}{\sqrt{1-\paren*{\frac{x}{a}}^2}}
  && \reason{因为$a > 0$, 提取因子$\frac{1}{a}$} \\
              &= \frac{1}{a} \int \frac{d(au)}{\sqrt{1-u^2}}
  && \mreason{ u = \frac{x}{a} } \\
              &= \int \frac{du}{\sqrt{1-u^2}}
  && \mreason{ d(au) = a \, du } \\
              &= \arcsin u + C
  && \mreason{ d\arcsin u = \frac{1}{\sqrt{1-u^2}} \, du } \\
              &= \arcsin \frac{x}{a} + C.
  && \mreason{ u = \frac{x}{a} }
\end{align*}

\subsection{第二换元法}

设$ f(x) $为连续函数, $ x = \varphi(t) $连续可导且有反函数, 则
\[ \int f(x) \, dx = \int f(\varphi(t)) \varphi'(t) \, dt. \]

若右边的原函数可求得, 记\disp{ G(t) = \int f(\varphi(t)) \varphi'(t) \, dt }, 则
\[ \int f(x) \, dx = G(\varphi^{-1}(x)) + C. \]

\exds{1}{ \int \sqrt{a^2 - x^2} \, dx. }
\begin{align*}
  \intertext{假设$ a > 0 $, 则$ -a \le x \le a$}
  \text{原式} &= \int a(\sqrt{1-\sin^2 t}) \, d(a\sin t)
  && \mreason{x = a \sin t, \  t \in [-\frac{\pi}{2}, \frac{\pi}{2}] } \\
              &= a^2 \int \cos^2 t \, dt
  && \mreason{ \cos t = \sqrt{1-\sin^2 t},\ d(a\sin t) = a\cos t \, dt } \\
              &= a^2 \int \frac{1+\cos 2t}{2} \, dt
  && \mreason{ \cos^2 t = \frac{1+\cos 2t}{2} } \\
              &= \frac{a^2}{2} t + \frac{\sin 2t}{4} + C
  && \mreason{ d\frac{t}{2} = \frac{1}{2} \, dt,\ d\frac{\sin 2t}{4} = \frac{\cos 2t}{2} \, dt } \\
              &= \frac{a^2}{2} \arcsin \frac{x}{a} + \frac{ \sin(2\arcsin\frac{x}{a})}{2} + C.
  && \mreason{ t = \arcsin\frac{x}{a} }
\end{align*}

\exds{2}{ \int \frac{dx}{\sqrt{x^2 - a^2}}. }

假设$a>0$, 则$x>a$或$x<-a$
\begin{align*}
  \text{原式} &= \int \frac{dx}{a \sqrt{\sec^2 t - 1}}
  && \mreason{ x = a\sec t, \  t \in (0, \frac{\pi}{2}) } \\
              &= \int \frac{ \cancel{a \tan t} \sec t }{\cancel{a \tan t}} \, dt
  && \mreason{ d(a\sec t) = a \sec t \tan t \, dt } \\
              &= \int \frac{dt}{\cos t}
  && \mreason{ \sec t = \frac{1}{\cos t} } \\
              &= \int \frac{\cos t \, dt}{\cos^2 t}
  && \reason{分子分母同乘以$\cos t$} \\
              &= \int \frac{d\sin t}{1 - \sin^2 t}
  && \mreason{ d\sin t = \cos t \, dt, \  \cos^2 t = 1 - \sin^2 t} \\
              &= \int \frac{du}{1-u^2}
  && \mreason{ u = \sin t } \\
              &= \frac{1}{2}\paren*{\int \frac{du}{1-u} + \int \frac{du}{1+u} }
  && \mreason{ \frac{1}{1-u^2} = \frac{1}{2}\paren*{\frac{1}{1-u} + \frac{1}{1+u}} } \\
              &= \frac{1}{2}(\ln \abs{1+u} - \ln \abs{1-u} + C)
  && \mreason{ d\ln \abs{1+u} = \frac{1}{1+u},\ d\ln \abs{1-u} = -\frac{1}{1-u} } \\
              &= \frac{1}{2} \ln \abs*{\frac{1+\sin t}{1 - \sin t}} + C
  && \reason{对数的性质, \(u = \sin t\)} \\
              &= \frac{1}{2} \ln \abs*{ \frac{1 + \sqrt{1-\paren*{\frac{a}{x}}^2}}
                {1 - \sqrt{1-\paren*{\frac{a}{x}}^2}} } + C
  && \mreason{ \sin^2 t = 1 - \paren*{\frac{a}{x}}^2} \\
              &= \frac{1}{2} \ln \abs*{ \frac{\paren*{1 + \sqrt{1-\paren*{\frac{a}{x}}^2}}^2}
                {\paren*{\frac{a}{x}}^2} } +C
  && \reason{分母有理化} \\
              &= \ln \paren[\Big]{x + \sqrt{x^2-a^2}} + C.
  && \reason{对数的性质}
\end{align*}

\exds{3}{ \int \frac{dx}{x^2 \sqrt{x^2+1}}. }
\begin{align*}
  \text{原式} &= \int \frac{d\tan t}{\tan^2 t \sqrt{\tan^2 t + 1}}
  && \reason{$x = \tan t$,
     因为$x \ne 0$,
     所以$t \in (-\frac{\pi}{2},\frac{\pi}{2})\setminus\{0\}$} \\
              &= \int \frac{\cancel{\sec t} \sec t \, dt}{\tan^2 t \,\cancel{\sec t}}
  && \mreason{ \sec^2 t - \tan^2 t = 1 } \\
              &= \int \frac{\cos t \, dt}{\sin^2 t}
  && \mreason{ \sec t = \frac{1}{\cos t},\  \tan t = \frac{\sin t}{\cos t} } \\
              &= \int \frac{d\sin t}{\sin^2 t}
  && \mreason{ d\sin t = \cos t \, dt } \\
              &= -\frac{1}{\sin t} + C
  && \mreason{ \int u^p \, du = \frac{u^{p+1}}{p+1} + C } \\
              &= -\frac{\sqrt{1+x^2}}{x} + C.
  && \frac{1}{\sin t} = \frac{\sqrt{1+x^2}}{x}
\end{align*}

\ifshowex
\currentpdfbookmark{练习}{B1.6.2.E}
\subsection*{练习}

\begin{enumerate}
\item 若\(f(x)\)的导函数是\(\sin x\), 求\(f(x)\)的原函数.

  \ifshowsol
  这题相当于是求函数\(\sin x\)的原函数的原函数, 所以有
  \[
    \int \paren*{\int \sin x \dx} \dx = \int \paren[\big]{-\cos x + C_1} \dx = -\sin x + C_1 x + C_2.
  \]
  \fi

\item 已知\(f(x)\)的一个原函数是\(\sin x\), \(g(x)\)的一个原函数是\(x^2\), 求复函数\(f(g(x))\)的原函数.

  \ifshowsol
  因为
  \[
    f(x) = \ddx \sin x = \cos x, \quad g(x) = \ddx x^2 = 2x,
  \]
  所以
  \begin{align*}
    \int f(g(x)) \dx
    &= \int \cos 2x \dx \\
    &= \frac12 \int \cos t \dt && \mreason{x = t/2} \\
    &= \frac12 \sin t + C = \frac12 \sin 2x + C.
  \end{align*}
  \fi

\item \disp{\int e^{-\abs{x}} \dx}.

  \ifshowsol
  \begin{align*}
    \int e^{-\abs{x}} \dx
    &=
      \begin{cases}
        \int e^{-x} \dx, & x \ge 0, \\
        \int e^x \dx, & x < 0,
      \end{cases} \\
    &=
      \begin{cases}
        - \int e^{t} \dt, & x \ge 0, \qquad \mreason{x = -t} \\
        e^x + C_2, & x < 0,
      \end{cases} \\
    &=
      \begin{cases}
        -e^t + C_1, & x \ge 0, \\
        e^x + C_2, & x < 0,
      \end{cases} \\
    &=
      \begin{cases}
        -e^{-x} + C_1, & x \ge 0, \qquad \mreason{t = -x} \\
        e^x + C_2, & x < 0,
      \end{cases} \\
    &= C +
      \begin{cases}
        -e^{-x} + 2, & x \ge 0, \\
        e^x, & x <0.
      \end{cases} \quad \reason{\(-1 + C_1 = 1 + C_2\), 取\(C_2 = 0\)}
  \end{align*}
  \fi

\item 已知曲线上任一点的二阶导数\(y'' = 6x\), 且在曲线上点\((0, -2)\)处的切线为\(2x - 3y = 6\), 求这条曲线的方程.

  \ifshowsol
  将切线的方程化为点斜式, 得到\(y = 2x/3 - 2\), 所以\(y'\big\vert_{x=0} = 2/3\).  有
  \[
    y' = \int y'' \dx = \int 6x \dx = 3x^2 + c_1,
  \]
  所以\(c_1 = 2/3\). 又有
  \[
    y = \int y' \dx = \int \paren[\Big]{3x^2 + \frac23} \dx = x^3 + \frac23 x + c_2,
  \]
  所以\(c_2 = -2\).  这条曲线的方程是
  \[
    y = x^3 + \frac23 x - 2.
  \]
  \fi

\item \disp{\int \frac{\ln x}{x^2} \dx}.

  \ifshowsol
  最好的办法是用\(x = e^t\)换元后用分部积分法, 可惜这里还没学到.  有
  \begin{align*}
    \int \frac{\ln x}{x^2} \dx
    &= \int t e^{-t} \dt
    && \mreason{x = e^t} \\
    &= - \int t \diff e^{-t} \\
    &= - \paren[\Big]{t e^{-t} - \int e^{-t} \dt} \\
    &= - t e^{-t} - e^{-t} + C \\
    &= - \frac{\ln x}{x} - \frac1x + C.
  \end{align*}
  \fi

\item 设\(f(x) = e^{-x}\), 求\disp{\int \frac{f'(\ln x)}{x} \dx}.

  \ifshowsol
  使用第一换元法, 有
  \begin{equation*}
    \int \frac{f'(\ln x)}{x} \dx
    = \int f'(\ln x) \diff (\ln x)
    = f(\ln x) = \frac1x + C.
  \end{equation*}

  也可以直接求导\(f'(x) = - e^{-x}\), 然后直接代入, 有
  \begin{equation*}
    \int \frac{f'(\ln x)}{x} \dx
    = - \int \frac{1}{x^2} \dx
    = \frac1x + C.
  \end{equation*}
  \fi

\item 若\disp{\int f(x) \dx = x^2 + C}, 求\disp{\int x \, f(1-x^2) \dx}.

  \ifshowsol
  使用第一换元法, 有
  \begin{align*}
    \int x \, f(1-x^2) \dx
    = - \frac12 \int f(1-x^2) \diff (1-x^2)
    = - \frac12 \paren[\big]{1-x^2}^2 + C.
  \end{align*}

  或者通过求导得到
  \[
    f(x) = \ddx \int f(x) \dx = \ddx (x^2 + C) = 2x,
  \]
  所以
  \[
    \int x \, f(1-x^2) \dx
    = 2 \int x (1-x^2) \dx
    = x^2 - \frac{x^4}{2} + C.
  \]

  上面两种方法得到的答案形式不一样, 但实际上是等价的, 只要把第一个答案中外部的平方展开, 把产生的常数项吸收到后面的任意常数\(C\)中, 就是第二个答案中的形式.
  \fi

\item \disp{\int \frac{x}{(1-x)^3} \dx}.

  \ifshowsol
  尝试使用\(u = 1/(1-x)^2\)和\(t = 1/(1-x)\)来做换元, 观察到
  \[
    \du = \frac{2}{(1-x)^3} \dx
    \quad \text{和} \quad
    \dt = \frac{1}{(1-x)^2} \dx,
  \]
  所以有
  \begin{align*}
    \int \frac{x}{(1-x)^3} \dx
    &= \int \paren[\bigg]{\frac{x-1}{(1-x)^3} + \frac{1}{(1-x)^3}} \dx \\
    &= - \int \frac{1}{(1-x)^2} \dx + \int \frac{1}{(1-x)^3} \dx \\
    &= - \int \dt + \frac12 \int \du \\
    &= - t + \frac u2 + C \\
    &= - \frac1{1-x} + \frac1{2 (1-x)^2} + C.
  \end{align*}
  \fi

\item \disp{\int \frac{\sin x \cos x}{1 + \sin^4 x} \dx}.

  \ifshowsol
  使用\(u = \sin x\)和\(t = u^2\)来做换元, 就有
  \begin{align*}
    \int \frac{\sin x \cos x}{1 + \sin^4 x} \dx
    &= \int \frac{u}{1+u^4} \du
    && \mreason{\du = \cos x \dx} \\
    &= \frac12 \int \frac1{1+t^2} \dt
    && \mreason{\dt = 2u \du} \\
    &= \frac12 \arctan t + C \\
    &= \frac12 \arctan \sin^2 x + C
    && \mreason{t = u^2,\ u = \sin x}
  \end{align*}
  \fi

\item \disp{\int \paren[\big]{2^x + x^2} \dx}.

  \ifshowsol
  \[
    \int \paren[\big]{2^x + x^2} \dx
    = \int 2^x \dx + \int x^2 \dx
    = \frac{2^x}{\ln 2} + \frac{x^3}3 + C.
  \]
  \fi
\end{enumerate}
\fi

\section{分部积分法}

若$u(x)$, $v(x)$连续可导, 则$[u(x) v(x)]' = u'(x)v(x) + u(x)v'(x)$, 就有
\[
  \int u(x) v'(x) \, dx = u(x) v(x) - \int v(x) u'(x) \, dx
\]
或者
\[
  \underset{\text{难}}{\underline{\int u(x) \, dv(x)}} =
  u(x) v(x) - \underset{\text{易}}{\underline{\int v(x) \, du(x)}}.
\]

这种方法适用于函数本身比较难, 但是其导函数比较简单. 这样的函数$ u(x) $一般有
\[
  \begin{matrix}
    \ln x, & \arctan x, & \arcsin x, & \text{函数复杂, 导数简单,} \\
    e^x, & \sin x, & \cos x, & \text{函数导数, 难度相同.}
  \end{matrix}
\]

\exds{1}{\int \ln x \, dx .}
\begin{align*}
  \int \underset{u(x)}{\underline{\ln x}} \, \underset{dv(x)}{\underline{dx}}
  &= x \ln x - \int \underset{v(x)}{\underline{x}} \, \underset{du(x)}{\underline{d\ln x}}
  && \reason{分部积分法} \\
  &= x \ln x - \int dx
  && \mreason{d\ln x = \frac{1}{x} \, dx} \\
  &= x \ln x - x + C .
  && \mreason{dx = dx}
\end{align*}

\exds{2}{\int x \arctan x \, dx .}
\begin{align*}
  \int x \, \underset{u(x)}{\underline{\arctan x}} \, dx
  &= \int \arctan x \, d\frac{x^2}{2}
  && \mreason{d\frac{x^2}{2} = x \, dx} \\
  &= \frac{x^2}{2} \arctan x - \int \frac{x^2}{2} \, d(\arctan x)
  && \reason{分部积分法} \\
  &= \frac{x^2}{2} \arctan x - \int \frac{x^2}{2(1+x^2)} \, dx
  && \mreason{d(\arctan x) = \frac{1}{1+x^2}} \\
  &= \frac{x^2}{2} \arctan x - \frac{1}{2} \int \paren[\bigg]{1 - \frac{1}{1+x^2}} \, dx
  && \mreason{\frac{x^2}{1+x^2} = 1 - \frac{1}{1+x^2}} \\
  &= \frac{1}{2} \paren[\bigg]{x^2 \arctan x - \int dx + \int \frac{dx}{1+x^2}}
  && \reason{积分的加法法则} \\
  &= \frac{1}{2} \brkt[\big]{(x^2+1) \arctan x - x} + C .
  && \mreason{dx = dx,\, d(\arctan x) = \frac{1}{1+x^2}}
\end{align*}

\exds{3}{\int x^2 \, e^x \dx.}
\begin{align*}
  \text{原式}
  &= \int x^2 \diff e^x \\
  &= x^2 e^x - \int e^x \, d(x^2)
  && \reason{分部积分法} \\
  &= x^2 e^x - 2\int x \, de^x \\
  &= x^2 e^x - 2\paren[\Big]{x e^x - \int e^x \, dx}
  && \reason{分部积分法} \\
  &= (x^2 - 2x + 2) e^x + C.
\end{align*}

\exds{4}{\int x \sin(2x) \, dx.}
\begin{align*}
  \text{原式}
  &= \frac{1}{2} \int x \, d[-\cos(2x)] \\
  &= \frac{1}{2} \paren*{\int \cos(2x) \, dx - x \cos(2x)}
  && \reason{分部积分法} \\
  &= \frac{1}{2} \paren[\bigg]{\frac{1}{2} \sin(2x) - x \cos(2x)} + C.
\end{align*}

\exds{5}{\int e^x \sin x \, dx.}
\begin{align*}
  -\int e^x \, d\cos x
  &= \text{原式}
    = \int \sin x \, de^x \\
  - e^x \cos x + \int \cos x \, de^x
  &= \text{原式}
    = e^x \sin x - \int e^x \, d\sin x
  && \reason{分部积分法} \\
  \intertext{因为$\int \cos x \, de^x = \int e^x \cos x \, dx = \int e^x \, d\sin x$, 所以}
  \int e^x \cos x \, dx
  &= \frac{\sin x + \cos x}{2} e^x + C, \\
  \text{原式}
  &= e^x \sin x - \frac{\sin x + \cos x}{2} e^x + C \\
  &= \frac{\sin x - \cos x}{2} e^x + C.
\end{align*}

\end{document}
