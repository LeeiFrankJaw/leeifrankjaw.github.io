\chapter{定积分}

\section{积分概念与积分存在条件}

\ifshowex
\subpdfbookmark{练习}{B1.7.1.E}
\subsection*{练习}

\begin{enumerate}
  \ifshowsol
  \setlength{\parskip}{9pt plus 3pt minus 2pt}
  \setlength{\itemsep}{7pt plus 2pt minus 2pt}
  \fi
\item 估算定积分\(\displaystyle \int_{\sqrt3/3}^{\sqrt3} x \arctan x \dx\)的上下界.

  \ifshowsol
  被积函数是单调递增函数, 这个积分上下限正好是常见的正切值, 求出上和与下和, 有
  \[
    \begin{split}
      (\sqrt3 - \sqrt3/3)\frac{\sqrt3}{3} \arctan \frac{\sqrt3}{3}
      \le
      &\int_{\sqrt3/3}^{\sqrt3} x \arctan x \dx
      \le
      (\sqrt3 - \sqrt3/3) \sqrt3 \arctan\sqrt3, \\
      \frac{\pi}{9}
      \le
      &\int_{\sqrt3/3}^{\sqrt3} x \arctan x \dx
      \le
      \frac23 \pi.
    \end{split}
  \]
  \fi

\item \(\displaystyle \lim_{n\to+\infty} \int_0^{1/2} \frac{x^n}{1+x^2} \dx\).

  \ifshowsol
  当\(n \ge 2\)时, 被积函数是单调递增函数, 所以通过求上和得到一个上界.这个积分显然是\(\ge 0\)的, 又当\(n\to\infty\)时, 这个上界是极限是\(0\), 所以这个积分的极限是\(0\).
  \fi

\item \(\displaystyle \lim_{n\to+\infty} \int_0^1 \frac{x^n e^x}{1+e^x} \dx\).

  \ifshowsol
  在区间\([0, 1]\)上,这个被积函数可以看成一个有界变量和\(x^n\)的乘积, 易知\(x^n\)是它的一个上界.那么只要证明\(\int_0^1 x^n \dx\)的极限是\(0\), 即可证明所求积分的极限也是\(0\).
  \fi

\item 设\(f(x)\)为\((0,+\infty)\)上的单调减函数, 试比较\(\sum_{k=1}^n f(k),\ \int_1^{n+1} f(x) \dx,\ \int_1^{n} f(x) \dx\)的大小关系.

  \ifshowsol
  因为\(f(x)\)是\((0,+\infty)\)上的单调减函数, 所以易知对于任意\(k \ge 1\)都有
  \[
    \int_k^{k+1} f(x) \dx \le f(k) \le \int_{k-1}^k f(x) \dx.
  \]
  那么
  \[
    \int_1^{n+1} f(x) \dx \le \sum_{k=1}^n f(k) \le \int_0^n f(x) \dx, \quad
    \int_2^{n+1} f(x) \dx \le \sum_{k=2}^n f(k) \le \int_1^n f(x) \dx.
  \]
  然后有
  \[
    \int_1^{n+1} f(x) \dx \le \sum_{k=1}^n f(k) = f(1) + \sum_{k=2}^n f(k) \le f(1) + \int_1^n f(x) \dx.
  \]
  \fi

\item \(\displaystyle \lim_{n\to+\infty} \sum_{k=1}^n \frac{n}{n^2+k^2} \).

  \ifshowsol
  将这个和变形得到
  \[
    \sum_{k=1}^n \frac{n}{n^2+k^2}
    = \sum_{k=1}^n \frac1n \frac{1}{1+\paren[\big]{\frac{k}{n}}^2}.
  \]
  这实际上就是函数\(\frac{1}{1+x^2}\)在\([0,1]\)上的一个下和.那么
  \[
    \lim_{n\to+\infty} \sum_{k=1}^n \frac{n}{n^2+k^2} = \int_0^1 \frac{\dx}{1+x^2} = \frac{\pi}{4}.
  \]
  \fi

\item \(\displaystyle \int_{-1}^1 \paren[\big]{\sin^5 x + \sqrt{1 - x^2}} \dx\).

  \ifshowsol
  将被积函数分解成一个奇函数和一个偶函数, 积分区间是\([-1,1]\), 奇函数抵消掉变成零, 偶函数变成
  \[
    \int_{-1}^1 \paren[\big]{\sin^5 x + \sqrt{1 - x^2}} \dx
    = \int_{-1}^1 \sin^5 x \dx + 2 \int_0^1 \sqrt{1-x^2} \dx
    = \frac{\pi}{2}.
  \]
  \fi

\item \(\displaystyle \int_{-1}^1 x^2 \ln\paren[\big]{x + \sqrt{1 + x^2}} \dx\).

  \ifshowsol
  因为\(\ln\paren[\big]{x + \sqrt{1 + x^2}} = \arcsinh x\)是一个奇函数, \(x^2\)是一个偶函数, 所以他们的乘积还是一个奇函数.因此
  \[
    \int_{-1}^1 x^2 \ln\paren[\big]{x + \sqrt{1 + x^2}} \dx = 0.
  \]
  \fi

\item 比较\(\displaystyle 0,\ \int_0^{\pi/2} \sin\sin x \dx,\ \int_0^{\pi/2} \cos\sin x \dx\)的大小.
\end{enumerate}
\fi

\section{定积分的性质}

\begin{theorem*}
  \begin{enumerate}
  \item[]
  \item 函数\(f, g \in R[a,b]\)且\(\forall \alpha, \beta \in \R,\ \alpha\,f + \beta\,g \in R[a,b]\), 则
    \[
      \int_a^b \paren{\alpha\,f + \beta\,g} \dx
      = \alpha \int_a^b f \dx + \beta \int_a^b g \dx.
    \]

  \item 函数\(f \in R[a,b]\), 那么\(\forall c \in (a,b)\)有
    \[
      \int_a^b f \dx = \int_a^c f \dx + \int_c^b f \dx.
    \]

  \item 函数\(f \in R[a,b]\), 那么
    \[
      \int_a^b f \dx = - \int_b^a \dx.
    \]
    推论:\(\int_a^a f \dx = 0\).

  \item 函数\(f \in R[a,b]\)且\(f(x) \ge 0\; \forall x \in [a,b]\), 则
    \[
      \int_a^b f \dx \ge 0.
    \]
    推论:函数\(f, g \in R[a,b]\)且\(f(x) \le g(x)\; \forall x \in [a,b]\), 则
    \[
      \int_a^b f \dx \le \int_a^b g \dx.
    \]

  \item 函数\(f \in R[a,b]\), 则\(\abs{\,f} \in R[a,b]\)且
    \[
      \abs[\Big]{\int_a^b f \dx} \le \int_a^b \abs{\,f} \dx.
    \]

  \item 函数\(f \in R[a,b]\), 若\(\forall x \in [a,b]\)都有\(m \le f < M\), 则
    \[
      m(b-a) \le \int_a^b f \dx \le M(b-a).
    \]

  \item 函数\(f \in R[a,b]\), 则\(\exists \xi \in (a,b)\)使得
    \[
      \int_a^b f \dx = f(\xi)(b-a).
    \]

  \item 函数\(f \in C[a,b],\ g \in R[a,b]\)且\(g\)在\([a,b]\)不变号, 则\(\exists \xi \in (a,b)\)使得
    \[
      \int_a^b fg \dx = f(\xi) \int_a^b g \dx.
    \]
  \end{enumerate}
\end{theorem*}

\section{变上限积分与Newton-Leibniz公式}

\ifshowex
\subpdfbookmark{练习}{B1.7.3.E}
\subsection*{练习}

\begin{enumerate}
\item 设
  \[
    f(x) =
    \begin{cases}
      x^2, & x \in [0,1), \\
      x+1, & x \in [1,2],
    \end{cases}
  \]
  试判断
  \[
    F(x) = \int_0^x f(t) \dt
  \]
  在\([0,2]\)上的连续性和可导性.

  \ifshowsol
  对于这个简单的函数, 我们可以写出它的表达式
  \[
    F(x) =
    \begin{cases}
      \frac13 x^3, & x \in [0,1), \\
      \frac12 x^2 + x - \frac76, & x \in [1,2].
    \end{cases}
  \]
  所以易知\(\lim_{x\to1} F(x) = F(1) = \frac13\)和\(F'_-(1) = 1 \ne 2 = F'_+(1)\).那么\(F\)在\([0,2]\)上连续, 导函数在\(x=1\)处存在第一类间断点.
  \fi

\item 设
  \[
    F(x) = \frac1{x-a} \int_a^x f(t) \dt,
  \]
  其中\(f\)在\([a,b]\)上连续, 在\((a,b)\)内可导且\(f'(x) \le 0\).试讨论\(F\)在\((a,b)\)上的单调性.

  \ifshowsol
  对该函数求导, 得到
  \[
    \ddx F(x) = \frac{f(x)(x-a) - \int_a^x f(t) \dt}{(x-a)^2}.
  \]
  因为\(f\)在\((a,b)\)上单调递减, 所以
  \[
    f(x)(x-a) \le \int_a^x f(t) \dt.
  \]
  这就意味着\(\ddx F(x) \le 0\), 也就是说\(F\)在\((a,b)\)上单调递减.
  \fi

\item \(\displaystyle \int_0^2 \frac{(x-1)^2 + 1}{(x-1)^2 + x^2 (x-2)^2} \dx\).

  \ifshowsol
  设被积函数为\(f\,\), 那么根据定积分的性质就有
  \[
    \begin{split}
      \int_0^2 f(x) \dx
      &= \int_{-1}^1 f(x+1) \dx
      = \int_{-1}^1 \frac{x^2 + 1}{x^2 + (x+1)^2 (x-1)^2} \dx \\
      &= 2 \int_0^1 \frac{x^2 + 1}{x^2 + (x+1)^2 (x-1)^2} \dx
      = 2 \int_0^1 \frac{1 + \frac{1}{x^2}}{1 + \paren[\big]{\frac{x^2-1}{x}}^2} \dx \\
      &= 2 \int_0^1 \frac{1 + \frac{1}{x^2}}{1 + \paren[\big]{x - \frac{1}{x}}^2} \dx
      = 2 \int_0^1 \frac{\diff\paren[\big]{x - \frac{1}{x}}}{1 + \paren[\big]{x - \frac{1}{x}}^2} \\
      &= 2 \arctan\paren[\Big]{x - \frac{1}{x}}\bigg\vert_0^1
      = 2 \brkt[\Big]{0 - \paren[\Big]{-\frac{\pi}{2}}} \\
      &= \pi.
    \end{split}
  \]
  易知平移后的函数\(f(x+1)\)是一个偶函数.
  \fi

\item 设\(f\)是\([0,+\infty)\)上的连续函数且恒有\(f(x) > 0\), 定义函数
  \[
    g(x) = \frac{\int_0^x t\,f(t) \dt}{\int_0^x f(t) \dt},
  \]
  则下列说法正确的是哪个?
  \begin{enumerate}
  \item \(\displaystyle \lim_{x\to0} g(x) = 1\).
  \item \(g(x)\)在\([0,+\infty)\)上单调增加.
  \item \(\displaystyle \lim_{x\to\infty} g(x) = +\infty\).
  \item \(\displaystyle \lim_{x\to0^+} g'(x)\)不存在.
  \end{enumerate}

  \ifshowsol
  易证\(\lim_{x\to0} g(x) = 0\)和\(\lim_{x\to0^+} g'(x) = 1/2\).设\(f(t) = e^{-t}\), 那么这个函数就是\enumparen{c}的反例, 因为\(\lim_{x\to\infty} g(x) = 1\).对于\(g\)求导, 有
  \[
    g'(x) = \frac{x\,f(x) \int_0^x f(t) \dt - f(x) \int_0^x t\,f(t) \dt}{\paren[\big]{\int_0^x f(t) \dt}^2}
    = f(x) \frac{\int_0^x (x-t)\,f(t) \dt}{\paren[\big]{\int_0^x f(t) \dt}^2}.
  \]
  对于所有的\(t \in (0, x)\), 都有\(x - t > 0\)和\(f(t) > 0\), 所以\(\int_0^x (x-t)\,f(t) \dt > 0\).这就是说在\((0,+\infty)\)上\(g'(x) > 0\), 所以\(g(x)\)在\([0,+\infty)\)上单调增加.
  \fi

\item 已知\(F(x) = \int_a^x f(t) \dt \; (a \le x \le b)\), 则下列结论正确的是哪个?
  \begin{enumerate}
  \item 若函数\(F\)连续, 则\(F' = f\).
  \item 若函数\(f\)连续, 则\(F\)一阶导函数连续.
  \item 函数\(F\)的连续点也是函数\(f\)的连续点.
  \item 若函数\(f\)连续, 则不一定有\(F' = f\).
  \end{enumerate}

\item 关于\(\int_a^x f(t) \dt\), 以下说法正确的是哪个?
  \begin{enumerate}
  \item 函数\(\int_a^x f(t) \dt\)是\(f(x)\)的某一个原函数.
  \item 函数\(\int_a^x f(t) \dt\)是\(f(x)\)的一类原函数族.
  \item 函数\(\int_a^x f(t) \dt\)不一定是\(f(x)\)的原函数.
  \item 若\(\int_a^x f(t) \dt\)是\(f(x)\)的原函数, 则\(f(x)\)连续.
  \end{enumerate}

\item \(\displaystyle \ddx \int_{1/x}^{\cos x} f(t) \dt\).

  \ifshowsol
  \[
    \ddx \int_{1/x}^{\cos x} f(t) \dt
    = \ddx \paren[\Big]{\int_{1/x}^0 f(t) \dt + \int_0^{\cos x} f(t) \dt}
    = \frac{f(1/x)}{x^2} - f(\cos x) \sin x .
  \]
  \fi

\item \(\displaystyle \ddx \int_0^x \sin x \cos t^2 \dt\).

  \ifshowsol
  \[
    \ddx \int_0^x \sin x \cos t^2 \dt
    = \ddx \sin x \int_0^x \cos t^2 \dt
    = \cos x \int_0^x \cos t^2 \dt + \sin x \cos x^2.
  \]
  \fi

\item 求\(x = \int_0^t \sin u \du,\ y = \int_0^t \cos u \du\)所确定的函数对\(x\)的导数\(\dd[y]x\).

  \ifshowsol
  \[
    \dd[y]x = \frac{\dy/\!\dt}{\dx/\!\dt}
    = \frac{\cos t}{\sin t}
    = \cot t.
  \]
  \fi

\item 求由\(\int_0^y e^t \dt + \int_0^x \cos t \dt = 0\)所确定的隐函数对\(x\)的导数\(\dd[y]x\).
  \[
    \begin{split}
      \diff\paren[\Big]{\int_0^y e^t \dt + \int_0^x \cos t \dt}
      &= 0 \\
      e^y \dy + \cos x \dx = 0 \\
      \dd[y]x = - e^{-y} \cos x.
    \end{split}
  \]
\end{enumerate}
\fi

\section{定积分的换元积分法与分部积分法}

\ifshowex
\subpdfbookmark{练习}{B1.7.4.E}
\subsection*{练习}

\begin{enumerate}
\item \(\displaystyle \int_0^a \sqrt{a^2 - x^2} \dx\).

  \ifshowsol
  根据被积函数的几何意义, 有
  \[
    \int_0^a \sqrt{a^2 - x^2} \dx = \frac{\pi a^2}{4}.
  \]
  \fi

\item \(\displaystyle \int_a^{2a} \frac{\sqrt{x^2 - a^2}}{x^4} \dx, \ (a > 0)\).

  \ifshowsol
  用\(x = a \sec t\)做换元, 有
  \[
    \begin{split}
      \int_a^{2a} \frac{\sqrt{x^2 - a^2}}{x^4} \dx
      &= \int_0^{\pi/3} \frac{\sqrt{a^2 \sec^2 t - a^2} \cdot a \tan t \sec t}{a^4 \sec^4 t} \dt
      = \frac{1}{a^2} \int_0^{\pi/3} \frac{\tan^2 t}{\sec^3 t} \dt \\
      &= \frac{1}{a^2} \int_0^{\pi/3} \sin^2 t \cos t \dt
      = \frac{1}{a^2} \int_0^{\sqrt3/2} u^2 \du \\
      &= \frac{1}{a^2} \frac{u^3}{3}\bigg\vert_0^{\sqrt3/2}
      = \frac{\sqrt3}{8a^2}.
    \end{split}
  \]
  也可以用\(x = a \cosh t\)做换元, 有
  \[
    \begin{split}
      \int_a^{2a} \frac{\sqrt{x^2 - a^2}}{x^4} \dx
      &= \int_0^{\arccosh2} \frac{\sqrt{a^2 \cosh^2 t - a^2} \cdot a \sinh t}{a^4 \cosh^4 t} \dt
      = \frac1{a^2} \int_0^{\arccosh2} \tanh^2 t \sech^2 t \dt \\
      &= \frac1{a^2} \int_0^{\tanh\arccosh2} u^2 \du
      = \frac1{a^2} \frac{u^3}{3} \bigg\vert_0^{\sqrt3/2}
      = \frac{\sqrt3}{8a^2}.
    \end{split}
  \]
  其中
  \[
    \tanh\arccosh2 = \tanh\arcsech\frac12 = \sqrt{1 - \sech^2\arcsech\frac12}
    = \sqrt{1 - \frac14} = \frac{\sqrt3}{2}.
  \]
  \fi

\item \(\displaystyle \int_0^3 \frac{x}{1 + \sqrt{1+x}} \dx\).

  \ifshowsol
  用\(t = \sqrt{1+x}\)做换元, 有
  \[
    \int_0^3 \frac{x}{1 + \sqrt{1+x}} \dx
    = \int_1^2 \frac{t^2-1}{1+t} \cdot 2t \dt
    = 2 \int_1^2 \paren{t^2 - t} \dt
    = \frac{2}{3} t^3 \bigg\vert_1^2 - t^2 \bigg\vert_1^2
    = \frac53.
  \]
  \fi

\item \(\displaystyle \int_1^e \frac{1 + \ln x}{x} \dx\).

  \ifshowsol
  \[
    \int_1^e \frac{1 + \ln x}{x} \dx
    = \int_1^e \frac{\dx}{x} + \int_1^e \frac{\ln x}{x} \dx
    = \ln x \bigg\vert_1^e + \frac{\ln^2 x}{2} \bigg\vert_1^e
    = 1 + \frac12 = \frac32.
  \]
  \fi

\item \(\displaystyle \int_0^2 \sqrt{\paren[\big]{4-x^2}^3} \dx\).

  \ifshowsol
  用\(x = 2 \sin t\)做换元, 有
  \[
    \begin{split}
      \int_0^{\pi/2} \sqrt{\paren[\big]{4-x^2}^3} \dx
      &= 16 \int_0^{\pi/2} \cos^4 t \dt
      = 4 \int_0^{\pi/2} \paren[\big]{1 + \cos 2t}^2 \dt \\
      &= 4 \int_0^{\pi/2} \paren{1 + 2 \cos 2t + \cos^2 2t} \dt \\
      &= 4 \paren[\bigg]{\frac{\pi}{2} + \sin 2t \Big\vert_0^{\pi/2} + \frac12 \int_0^\pi \cos^2 u \du} \\
      &= 2 \pi + \int_0^\pi \paren{1 + \cos 2u} \du \\
      &= 2 \pi + \pi + \frac{\sin 2u}{2} \bigg\vert_0^\pi
      = 3 \pi.
    \end{split}
  \]
  \fi
\item \(\displaystyle \int_0^4 \frac{\sqrt x}{1 + x \sqrt{x}} \dx\).

  \ifshowsol
  用\(t = \sqrt x\)做换元, 有
  \[
    \begin{split}
      \int_0^4 \frac{\sqrt x}{1 + x \sqrt{x}} \dx
      = \int_0^2 \frac{t}{1+t^3} \cdot 2t \dt
      = \frac23 \int_0^2 \frac{\diff(t^3)}{1+t^3}
      = \frac23 \ln(1+t^3) \Big\vert_0^2
      = \frac23 \ln9
      = \frac43 \ln3.
    \end{split}
  \]
  \fi

\item \(\displaystyle \int_1^2 x \sqrt{x^2 - 1} \dx\).

  \ifshowsol
  用\(x = \cosh t\)做换元, 有
  \[
    \int_1^2 x \sqrt{x^2 - 1} \dx
    = \int_0^{\arccosh2} \sinh^2 t \cosh t \dt
    = \int_0^{\sinh\arccosh2} u^2 \du
    = \frac{u^3}{3} \bigg\vert_0^{\sqrt3}
    = \sqrt3.
  \]
  其中
  \[
    \sinh\arccosh2 = \sqrt{\cosh^2 \arccosh 2 - 1} = \sqrt{2^2 - 1} = \sqrt3.
  \]
  或者用\(x = \sec t\)做换元, 有
  \[
    \int_1^2 x \sqrt{x^2 - 1} \dx
    = \int_0^{\pi/3} \tan^2 t \sec^2 t \dt
    = \int_0^{\sqrt3} u^2 \du
    = \frac{u^3}{3} \bigg\vert_0^{\sqrt3}
    = \sqrt3.
  \]
  \fi

\item \(\displaystyle \int_0^{2\pi} \frac{\dx}{1 + \cos^2 x}\).

  \ifshowsol
  先将被积函数降次, 然后把积分上下限变换到最小的范围, 最后用\(t = \tan \frac x2\)做换元, 有
  \[
    \begin{split}
      \int_0^{2\pi} \frac{\dx}{1 + \cos^2 x}
      &= \int_0^{2\pi} \frac{2}{3 + \cos 2x} \dx
      = \int_0^{4\pi} \frac{\dx}{3 + \cos x}
      = 2 \int_0^{2\pi} \frac{\dx}{3 + \cos x} \\
      &= 2 \int_{-\pi}^{\pi} \frac{\dx}{3 - \cos x}
      = 4 \int_0^{\pi} \frac{\dx}{3 - \cos x} \\
      &= 4 \int_0^{+\infty} \frac{1}{3 - \paren{1-t^2}/\paren{1+t^2}} \frac{2}{1+t^2} \dt
      = 4 \int_0^{+\infty} \frac{\dt}{1 + 2t^2} \\
      &= \frac{4}{\sqrt2} \arctan \sqrt2 t \bigg\vert_0^{+\infty}
      = \sqrt2 \pi.
    \end{split}
  \]
  \fi

\item 设函数
  \[
    f(x) =
    \begin{cases}
      x, & 0 \le x \le 1, \\
      x^2, & 1 < x \le 2,
    \end{cases}
  \]
  求\(\displaystyle \int_0^2 f(x) \dx\).

  \ifshowsol
  \[
    \int_0^2 f(x) \dx
    = \int_0^1 f(x) \dx + \int_1^2 f(x) \dx
    = \frac{x^2}{2} \bigg\vert_0^1 + \frac{x^3}{3} \bigg\vert_1^2
    = \frac{17}{6}.
  \]
  \fi

\item \(\displaystyle \int_0^3 \paren[\big]{\abs[\big]{x-1} + \abs[\big]{x-2}} \dx\).

  \ifshowsol
  \[
    \begin{split}
      \int_0^3 \paren[\big]{\abs[\big]{x-1} + \abs[\big]{x-2}} \dx
      &= \int_0^1 \paren{3-2x} \dx + \int_1^2 \dx + \int_2^3 \paren{2x-3} \dx \\
      &= \paren{3x-x^2} \Big\vert_0^1 + 1 + \paren{x^2 - 3x} \Big\vert_2^3 \\
      &= 2 + 1 + 2 = 5.
    \end{split}
  \]
  \fi

\item \(\displaystyle \int_0^{2\pi} \abs[\big]{\sin\paren{x-\pi}} \dx\).

  \ifshowsol
  \[
    \int_0^{2\pi} \abs[\big]{\sin\paren{x-\pi}} \dx
    = \int_0^{2\pi} \abs[\big]{- \sin x} \dx
    = \int_0^{2\pi} \abs[\big]{\sin x} \dx
    = 2 \int_0^{\pi} \abs[\big]{\sin x} \dx
    = 2 \int_0^{\pi} \sin x \dx
    = 4.
  \]
  \fi

\item \(\displaystyle \int_0^\pi \frac{x \sin^2 x}{1 + \sin x} \dx\).

  \ifshowsol
  \[
    \begin{split}
      \int_0^\pi \frac{x \sin^2 x}{1 + \sin x} \dx
      &= \int_{-\pi/2}^{\pi/2} \frac{(x+\pi/2) \cos^2 x}{1 + \cos x} \dx
      = \int_{-\pi/2}^{\pi/2} \frac{x \cos^2 x}{1 + \cos x} \dx + \frac\pi2 \int_{-\pi/2}^{\pi/2} \frac{\cos^2 x}{1 + \cos x} \dx \\
      &= \pi \int_0^{\pi/2} \frac{\cos^2 x}{1 + \cos x} \dx
      = \pi \int_0^1 \frac{\brkt[\big]{(1-t^2)/(1+t^2)}^2}{1 + (1-t^2)/(1+t^2)} \cdot \frac2{1+t^2} \dx \\
      &= \pi \int_0^1 \paren[\bigg]{\frac{1-t^2}{1+t^2}}^2 \dt
      = \pi \int_0^1 \brkt[\bigg]{1 - \frac{4t^2}{\paren{1+t^2}^2}} \dt \\
      &= \pi - 4 \pi \int_0^1 \brkt[\bigg]{\frac{1}{1+t^2} - \frac{1}{\paren{1+t^2}^2}} \dt \\
      &= \pi - 4 \pi \arctan t \Big\vert_0^1 + 4 \pi \int_0^1 \frac{\dt}{\paren{1+t^2}^2} \\
      &= \pi - \pi^2 + 4 \pi \int_0^1 \frac{\dt}{\paren{1+t^2}^2}.
    \end{split}
  \]
  现在只要把最后一行剩下的积分求出来就可以了, 注意到
  \[
    \begin{split}
      \int_0^1 \frac{\dt}{1+t^2}
      &= \frac{t}{1+t^2} \bigg\vert_0^1 + 2 \int_0^1 \frac{t^2}{\paren{1+t^2}^2} \dt
      = \frac12 + 2 \int_0^1 \brkt[\bigg]{\frac{1}{1+t^2} - \frac{1}{\paren{1+t^2}^2}} \dt \\
      &= \frac12 + 2 \int_0^1 \frac{\dt}{1+t^2} - 2 \int_0^1 \frac{\dt}{\paren{1+t^2}^2},
    \end{split}
  \]
  所以
  \[
    \int_0^1 \frac{\dt}{\paren{1+t^2}^2}
    = \frac14 + \frac12 \int_0^1 \frac{\dt}{1+t^2}
    = \frac14 + \frac12 \arctan t \Big\vert_0^1
    = \frac14 + \frac\pi8.
  \]
  因此,
  \[
    \int_0^\pi \frac{x \sin^2 x}{1 + \sin x} \dx
    = \pi - \pi^2 + 4 \pi \paren[\bigg]{\frac14 + \frac\pi8}
    = 2 \pi - \frac{\pi^2}{2}.
  \]
  \fi

\item \(\displaystyle \int_0^1 \ln\paren{1+x^2} \dx\).

  \ifshowsol
  \[
    \begin{split}
      \int_0^1 \ln\paren{1+x^2} \dx
      &= x \ln\paren{1+x^2} \Big\vert_0^1 - 2 \int_0^1 \frac{x^2}{1+x^2} \dx
      = \ln2 - 2 \int_0^1 \paren[\bigg]{1 - \frac{1}{1+x^2}} \dx \\
      &= \ln 2 - 2 + 2 \arctan x \Big\vert_0^1
      = \ln 2 - 2 + \frac\pi2.
    \end{split}
  \]
  \fi

\item \(\displaystyle \int_0^1 x \arctan x \dx\).

  \ifshowsol
  \[
    \begin{split}
      \int_0^1 x \arctan x \dx
      &= \frac12 \int_0^1 \arctan x \diff(x^2)
      = \frac12 \paren[\Big]{x^2 \arctan x \Big\vert_0^1 - \int_0^1 \frac{x^2}{1+x^2} \dx} \\
      &= \frac\pi8 - \frac12 + \frac12 \arctan x \Big\vert_0^1
      = \frac\pi4 - \frac12.
    \end{split}
  \]
  \fi

\item \(\displaystyle \int_0^1 x^3 e^{-x^2} \dx\).

  \ifshowsol
  \[
    \begin{split}
      \int_0^1 x^3 e^{-x^2} \dx
      &= - \frac12 \int_0^1 x^2 \diff e^{-x^2}
      = - \frac12 \paren[\Big]{x^2 e^{-x^2} \Big\vert_0^1 - \int_0^1 e^{-x^2} \cdot 2x \dx} \\
      &= - \frac12 \paren[\Big]{e^{-1} + e^{-x^2} \Big\vert_0^1}
      = \frac12 - \frac1e.
    \end{split}
  \]
  \fi

\item \(\displaystyle \int_0^\pi e^x \sin 2x \dx\).

  \ifshowsol
  因为
  \[
    \begin{split}
      \int_0^\pi e^x \sin 2x \dx
      &= e^x \sin 2x \Big\vert_0^\pi - 2 \int_0^\pi e^x \cos 2x \dx
      = -2 \paren[\Big]{e^x \cos 2x \Big\vert_0^\pi + 2 \int_0^\pi e^x \sin 2x \dx} \\
      &= -2(e^\pi - 1) - 4 \int_0^\pi e^x \sin 2x \dx.
    \end{split}
  \]
  所以
  \[
    \int_0^\pi e^x \sin 2x \dx = -\frac25 (e^\pi - 1).
  \]
  \fi
\end{enumerate}
\fi

\section{定积分的几何应用}

\ifshowex
\subpdfbookmark{练习}{B1.7.5.E}
\subsection*{练习}
\fi


\section{定积分的物理应用}

\section{反常积分}

\ifshowex
\subpdfbookmark{练习}{B1.7.7.E}
\subsection*{练习}
\fi

% Local Variables:
% TeX-engine: luatex
% TeX-master: "微积分B"
% End:
