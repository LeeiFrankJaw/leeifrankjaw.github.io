\documentclass[a4paper]{amsbook}

\usepackage[T1]{fontenc}
\usepackage{textcomp}
\usepackage{mathtools,amssymb,amsthm}
\usepackage[hmargin=1in,vmargin=1in]{geometry}
\usepackage{graphicx,xcolor}
\usepackage[pdfusetitle]{hyperref}
\hypersetup{%
  colorlinks=true,
  urlcolor=[rgb]{0,0.2,0.6},
  linkcolor={.},
  bookmarksdepth=2}
\usepackage{bookmark}
\usepackage{float}

\frenchspacing

\newcommand*{\parasp}{\setlength{\parskip}{10pt plus 2pt minus 3pt}}
\newcommand*{\noparasp}{\setlength{\parskip}{0pt plus 1pt}}
\newcommand*{\setparasp}[1]{\setlength{\parskip}{#1}}
\newcommand*{\pskip}{\vskip 10pt plus 2pt minus 3pt}
% \newcommand\LEFTRIGHT[3]{\left#1 #3 \right#2}
\newcommand\SetSymbol[1][]{%
  \nonscript\:#1\vert
  \allowbreak
  \nonscript\:
  \mathopen{}}
% \newcommand*{\paren}[1]{\LEFTRIGHT(){#1}}
\DeclarePairedDelimiterX{\paren}[1]{\lparen}{\rparen}{%
  \renewcommand{\mid}{\SetSymbol[\delimsize]}#1}
% \newcommand*{\brkt}[1]{\LEFTRIGHT[]{#1}}
\DeclarePairedDelimiterX{\brkt}[1]{\lbrack}{\rbrack}{%
  \renewcommand{\mid}{\SetSymbol[\delimsize]}#1}
\DeclarePairedDelimiterX{\brce}[1]{\lbrace}{\rbrace}{%
  \renewcommand{\mid}{\SetSymbol[\delimsize]}#1}
\newcommand*{\unit}[1]{\,\mathrm{#1}}
\newcommand*{\DeclareUnit}[1]{\expandafter\def\csname#1\endcsname{\unit{#1}}}
\DeclareUnit{cm}
% \renewcommand*{\m}{\unit{m}}
\DeclareUnit{m}
\DeclareUnit{kg}
\DeclareUnit{s}
\newcommand*{\R}{\mathbb{R}}
\newcommand*{\Z}{\mathbb{Z}}
\newcommand*{\N}{\mathbb{N}}
\newcommand*{\Q}{\mathbb{Q}}
% \newcommand*{\Rp}{(0,+\infty)}
% \newcommand*{\Rm}{(-\infty,0)}
\newcommand*{\deduce}{\mathrel{\Downarrow}}
% \newcommand*{\abs}[1]{\left\lvert #1 \right\rvert}
\DeclarePairedDelimiter{\abs}{\lvert}{\rvert}
% \newcommand*{\ceil}[1]{\left\lceil#1\right\rceil}
\DeclarePairedDelimiter{\ceil}{\lceil}{\rceil}
\DeclarePairedDelimiter{\floor}{\lfloor}{\rfloor}
\newcommand*{\textop}[1]{\mathbin{\text{#1}}}
\newcommand*{\tand}{\textop{and}}
\newcommand*{\tor}{\textop{or}}
\newcommand*{\txt}[2][\quad]{#1 \text{#2} #1}
\newcommand*{\qand}{\txt{and}}
\newcommand*{\iand}{\intertext{and}}
\newcommand*{\DeclareText}[1]{\expandafter\def\csname#1\endcsname{\text{#1}}}
\DeclareText{otherwise}
\newcommand*{\tfor}{\text{for }}
\newcommand*{\qfor}{\txt{for}}

\newcommand*{\enumparen}[1]{(\makebox[0.6em][c]{#1})}
\renewcommand{\labelenumii}{\enumparen{\theenumii}}
\newcommand*{\upstar}{\textsuperscript{\normalfont\textasteriskcentered}}%
\makeatletter
\newcommand*{\bonus}{\@itemlabel\upstar}%
\def\contitem{%
  \def\H@item{%
    \@inmatherr\item
    \@noitemargtrue
    \@ifnextchar[\@item{\@item[\@itemlabel]}}}
\makeatother

\DeclareMathOperator{\arccosh}{arccosh}
\DeclareMathOperator{\arcsinh}{arcsinh}
\DeclareMathOperator{\arctanh}{arctanh}
\DeclareMathOperator{\arccoth}{arccoth}
\DeclareMathOperator{\sech}{sech}
\DeclareMathOperator{\arcsech}{arcsech}
\DeclareMathOperator{\sgn}{sgn}
\DeclareMathOperator{\var}{var}
\DeclareMathOperator{\Ber}{Bernoulli}
\DeclareMathOperator{\Cov}{Cov}
\DeclareMathOperator{\E}{E}
\def\argmax{\qopname\relax m{arg\,max}}
\DeclarePairedDelimiterXPP{\Eb}[1]{\E}{\lbrack}{\rbrack}{}{%
  \renewcommand{\mid}{\SetSymbol[\delimsize]}#1}
\DeclarePairedDelimiterXPP{\varp}[1]{\var}{\lparen}{\rparen}{}{%
  \renewcommand{\mid}{\SetSymbol[\delimsize]}#1}
\DeclarePairedDelimiterXPP{\Covp}[1]{\Cov}{\lparen}{\rparen}{}{%
  \renewcommand{\mid}{\SetSymbol[\delimsize]}#1}
\DeclarePairedDelimiterXPP{\expp}[1]{\exp}{\lbrace}{\rbrace}{}{#1}
\newcommand*{\Fn}[1]{\mathop{\relax #1}\nolimits}
\newcommand*{\fn}[1]{\mathop{\relax\kern0pt #1}\nolimits}
\newcommand*{\gammaf}{\Fn{\Gamma}}
\renewcommand*{\Pr}{\Fn{P}}
\DeclarePairedDelimiterXPP{\Prp}[1]{\Pr}{\lparen}{\rparen}{}{%
  \renewcommand{\mid}{\SetSymbol[\delimsize]}#1}
\newcommand*{\pnorm}{\Fn{\Phi}}
\DeclarePairedDelimiterXPP{\pnormp}[1]{\pnorm}{\lparen}{\rparen}{}{#1}
\newcommand*{\dnorm}{\fn{\varphi}}
\DeclarePairedDelimiterXPP{\dnormp}[1]{\dnorm}{\lparen}{\rparen}{}{#1}
\newcommand*{\qnorm}{\Fn{\Phi}^{-1}}
%\newcommand*{\diff}{\mathop{}\!d}
\newcommand*{\diff}{\mathop{}\!\mathit{d}}
%\newcommand*{\diff}{\mathop{}\!\mathrm{d}}
\newcommand*{\dx}{\diff x}
\newcommand*{\dy}{\diff y}
\newcommand*{\dz}{\diff z}
\newcommand*{\ds}{\diff s}
\newcommand*{\dt}{\diff t}
\newcommand*{\du}{\diff u}
\newcommand*{\dv}{\diff v}
\newcommand*{\dtheta}{\diff \theta}
\newcommand*{\dd}[2][]{\frac{\diff#1}{\diff#2}}
\newcommand*{\ddx}{\frac{\diff}{\dx}}
\newcommand*{\ddt}{\frac{\diff}{\dt}}
\newcommand*{\ddy}{\dd y}
\newcommand*{\ddtheta}{\frac{\diff}{\dtheta}}
\newcommand*{\ddz}{\dd z}
\newcommand*{\fwdf}{\mathop{}\!\Delta}
\newcommand*{\dydx}{\frac\dy\dx}
\newcommand*{\pdpd}[2][]{\frac{\partial#1}{\partial#2}}
\newcommand*{\pdpdx}{\frac\partial{\partial x}}
\newcommand*{\pdpdy}{\frac\partial{\partial y}}
\newcommand*{\pdpdz}{\frac\partial{\partial z}}
\newcommand*{\pdpdu}{\frac\partial{\partial u}}
\newcommand*{\pdpdv}{\frac\partial{\partial v}}
\newcommand*{\pdpdt}{\frac\partial{\partial t}}
\newcommand*{\pdzpdx}{\frac{\partial z}{\partial x}}
\newcommand*{\pdzpdy}{\frac{\partial z}{\partial y}}
\newcommand*{\pdzpdt}{\frac{\partial z}{\partial t}}
\newcommand*{\pdxpdt}{\frac{\partial x}{\partial t}}
\newcommand*{\pdypdt}{\frac{\partial y}{\partial t}}


\title{Basics and Arithmetic of the Real Numbers}
\author{\textsc{Otto Stolz}}

\hypersetup{%
  pdfsubject={Real Analysis}}

\newtheorem*{theorem*}{Theorem}
\theoremstyle{remark}
\newtheorem*{remark}{Remark}

\usepackage{enumitem}

\usepackage{fontspec}

\setmainfont{Palatino Linotype}[Ligatures=TeX,Numbers=OldStyle]

\usepackage[euler-digits,euler-hat-accent]{eulervm}

\DeclareSymbolFont{greekletters}{OML}{cmr}{m}{it}
\DeclareMathSymbol{\varrho}{\mathalpha}{greekletters}{"25}

\usepackage{microtype}

\begin{document}
\frontmatter
\maketitle

\mainmatter

\begin{theorem*}
  \leavevmode
  \begin{enumerate}[label=\enumparen{\arabic*}]
  \item Let \(n\) be a whole positive number that grows without limit.  If the unique function \(\varphi(n)\) grows (decreases) continuously with \(n\) from a value \(n = n_1\), and thus has a limit at \(\lim = +\infty\) according to No. 6, which is supposed to be infinite; then it follows from the existence of the limit
    \begin{equation*}
      \lim_{n=+\infty} \frac{f(n+1)-f(n)}{\varphi(n+1)-\varphi(n)} = K,
    \end{equation*}
    where \(f(n)\) also means a unique function of \(n\), that there is also a limit for the fraction \(f(n):\varphi(n)\) at \(\lim n = +\infty\), namely that it is equal to \(K\).

  \item The same theorem applies if the unique functions \(f(n), \varphi(n)\) each have the limit \(0\) for \(\lim n = +\infty\) and \(\varphi(n)\) grows (decreases) continuously with \(n\) from \(n = n_1\).
  \end{enumerate}

  \begin{proof}
    For the proof, it is sufficient to assume that \(\varphi(n)\) increases continuously with \(n\) and that \(K\) does not have the sign \(-\). Then two cases can be distinguished for the first theorem, to which we restrict the following development.
    \begin{enumerate}[label=\alph*)]
    \item \label{item:1} \(K \geqq 0\).  According to the assumption, for every number \(\varepsilon > 0\) there belongs a number G such that
      \begin{equation*}
        \abs*{\frac{f(n+1)-f(n)}{\varphi(n+1)-\varphi(n)} - K} < \varepsilon
        \hspace{1.5em}
        n > G.
      \end{equation*}
      From this we conclude that, since \(\varphi(m + 1) > \varphi(n)\),
      \begin{equation*}
        (K-\varepsilon)\brkt{\varphi(n+1)-\varphi(n)}
        < f(n+1)-f(n)
        < (K+\varepsilon)\brkt{\varphi(n+1)-\varphi(n)}.
      \end{equation*}
      Now let \(m\) be a certain integer \(> G\), so that in this relation we can set \(n = m, m+1, \dots, m+r-1\) one after the other. If we add the \(r\) inequalities obtained in this way, we get
      \begin{gather*}
        (K-\varepsilon)\brkt{\varphi(m+r)-\varphi(m)}
        < f(m+r)-f(m)
        < (K+\varepsilon)\brkt{\varphi(m+r)-\varphi(m)} \\
        \iand
        -\varepsilon \brkt{\varphi(m+r)-\varphi(m)}
        < f(m+r)-f(m) - K \brkt{\varphi(m+r)-\varphi(m)}
        < \varepsilon \brkt{\varphi(m+r)-\varphi(m)}.
      \end{gather*}
      Since \(\varphi(m)\) and \(\varphi(m+r)\) can be assumed to be positive, we also find:
      \begin{gather*}
        -\varepsilon \varphi(m+r)
        < f(m+r)-f(m) - K \brkt{\varphi(m+r)-\varphi(m)}
        < \varepsilon \varphi(m+r), \\
        -\varepsilon
        < \frac{f(m+r)}{\varphi(m+r)} - K
        - \frac{f(m) - K \varphi(m)}{\varphi(m+r)}
        < \varepsilon.
      \end{gather*}
      However large the number \(m\) may be; then, because \(\lim\varphi(n) = +\infty\) we can find a number \(\varrho > 0\) such that
      \begin{equation}
        \label{eq:a}
        \frac{\abs{f(m) - K \varphi(m)}}{\varphi(m+r)} < \varepsilon
        \qfor
        r > \varrho.
      \end{equation}
      We therefore have
      \begin{equation}
        \label{eq:b}
        \abs*{\frac{f(m+r)}{\varphi(m+r)} - K} < 2\varepsilon
      \end{equation}
      or \(\abs{f(n):\varphi(n) — K} < 2\varepsilon\) for all \(n > m+\varrho\).  Now then, the number \(2\varepsilon\), which can be any positive number, indeed corresponds to the number \(m+\varrho\), since \(m\) results from \(\varepsilon\) initially and after \(m\) the number \(\varrho\) from \(\varepsilon\).  Thus we have
      \begin{equation*}
        \lim f(n):\varphi(n) = K
      \end{equation*}
      for \(\lim n = +\infty\).
    \item \(K = +\infty\).  Let \(0 < H < H'\).  According to the assumption, for \(H' > 0\) there should be a number~\(G'\) such that for \(n>G'\)
      \begin{equation*}
        f(n+1) - f(n) > H' \Seq{\varphi(n+1)-\varphi(n)},
      \end{equation*}
      from which we find as above---\(m > G'\)---
      \begin{gather}
        \label{eq:c}
        f(m+r) - f(m) > H' \Seq{\varphi(m+r)-\varphi(m)}, \\
        \frac{f(m+r)}{\varphi(m+r)}
        > H' + \frac{f(m)-H'\varphi(m)}{\varphi(m+r)}.
        \nonumber
      \end{gather}
      Now a number \(\varrho > 0\) can be specified such that
      \begin{equation*}
        r > \varrho
        \hspace{1.5em}
        \abs{f(m) - H' \varphi(m)}:\varphi(m+r) < H' - H;
      \end{equation*}
      therefore, for all \(n > m + \varrho\)
      \begin{equation*}
        \frac{f(n)}{\varphi(n)} > H' - (H' - H) = H
        \txt{i.e.}
        \lim\frac{f(n)}{\varphi(n)} = +\infty.
      \end{equation*}
      At the same time, we can see from~\eqref{eq:c} that \(\lim f(n) = +\infty\), so that we could also conclude as follows: According to theorem~\ref{item:1},
      \begin{equation*}
        \lim\Seq{\varphi(n):f(n)} = 0,
      \end{equation*}
      therefore, since \(\varphi(n):f(n) > 0\),
      \begin{equation*}
        \lim\Seq{f(n):\varphi(n)} = +\infty.
        \qedhere
      \end{equation*}
    \end{enumerate}
  \end{proof}
  \begin{remark}
    The above theorems can be extended to the case that the independent variable grows continuously over all limits.
  \end{remark}
\end{theorem*}
\end{document}

% Local Variables:
% TeX-engine: luatex
% End:
