\documentclass[a4paper]{ctexart}

\title{日语基础笔记}
\author{赵磊}

\usepackage[T1]{fontenc}
\usepackage{textcomp}
\usepackage{mathtools,amssymb,amsthm}
\usepackage[hmargin=1in,vmargin=1in]{geometry}
\usepackage{graphicx,xcolor}
\usepackage[pdfusetitle]{hyperref}
\hypersetup{%
  colorlinks=true,
  urlcolor=[rgb]{0,0.2,0.6},
  linkcolor={.},
  bookmarksdepth=2}
\usepackage{bookmark}
\usepackage{float}

\frenchspacing

\newcommand*{\parasp}{\setlength{\parskip}{10pt plus 2pt minus 3pt}}
\newcommand*{\noparasp}{\setlength{\parskip}{0pt plus 1pt}}
\newcommand*{\setparasp}[1]{\setlength{\parskip}{#1}}
\newcommand*{\pskip}{\vskip 10pt plus 2pt minus 3pt}
% \newcommand\LEFTRIGHT[3]{\left#1 #3 \right#2}
\newcommand\SetSymbol[1][]{%
  \nonscript\:#1\vert
  \allowbreak
  \nonscript\:
  \mathopen{}}
% \newcommand*{\paren}[1]{\LEFTRIGHT(){#1}}
\DeclarePairedDelimiterX{\paren}[1]{\lparen}{\rparen}{%
  \renewcommand{\mid}{\SetSymbol[\delimsize]}#1}
% \newcommand*{\brkt}[1]{\LEFTRIGHT[]{#1}}
\DeclarePairedDelimiterX{\brkt}[1]{\lbrack}{\rbrack}{%
  \renewcommand{\mid}{\SetSymbol[\delimsize]}#1}
\DeclarePairedDelimiterX{\brce}[1]{\lbrace}{\rbrace}{%
  \renewcommand{\mid}{\SetSymbol[\delimsize]}#1}
\newcommand*{\unit}[1]{\,\mathrm{#1}}
\newcommand*{\DeclareUnit}[1]{\expandafter\def\csname#1\endcsname{\unit{#1}}}
\DeclareUnit{cm}
% \renewcommand*{\m}{\unit{m}}
\DeclareUnit{m}
\DeclareUnit{kg}
\DeclareUnit{s}
\newcommand*{\R}{\mathbb{R}}
\newcommand*{\Z}{\mathbb{Z}}
\newcommand*{\N}{\mathbb{N}}
\newcommand*{\Q}{\mathbb{Q}}
% \newcommand*{\Rp}{(0,+\infty)}
% \newcommand*{\Rm}{(-\infty,0)}
\newcommand*{\deduce}{\mathrel{\Downarrow}}
% \newcommand*{\abs}[1]{\left\lvert #1 \right\rvert}
\DeclarePairedDelimiter{\abs}{\lvert}{\rvert}
% \newcommand*{\ceil}[1]{\left\lceil#1\right\rceil}
\DeclarePairedDelimiter{\ceil}{\lceil}{\rceil}
\DeclarePairedDelimiter{\floor}{\lfloor}{\rfloor}
\newcommand*{\textop}[1]{\mathbin{\text{#1}}}
\newcommand*{\tand}{\textop{and}}
\newcommand*{\tor}{\textop{or}}
\newcommand*{\txt}[2][\quad]{#1 \text{#2} #1}
\newcommand*{\qand}{\txt{and}}
\newcommand*{\iand}{\intertext{and}}
\newcommand*{\DeclareText}[1]{\expandafter\def\csname#1\endcsname{\text{#1}}}
\DeclareText{otherwise}
\newcommand*{\tfor}{\text{for }}
\newcommand*{\qfor}{\txt{for}}

\newcommand*{\enumparen}[1]{(\makebox[0.6em][c]{#1})}
\renewcommand{\labelenumii}{\enumparen{\theenumii}}
\newcommand*{\upstar}{\textsuperscript{\normalfont\textasteriskcentered}}%
\makeatletter
\newcommand*{\bonus}{\@itemlabel\upstar}%
\def\contitem{%
  \def\H@item{%
    \@inmatherr\item
    \@noitemargtrue
    \@ifnextchar[\@item{\@item[\@itemlabel]}}}
\makeatother

\DeclareMathOperator{\arccosh}{arccosh}
\DeclareMathOperator{\arcsinh}{arcsinh}
\DeclareMathOperator{\arctanh}{arctanh}
\DeclareMathOperator{\arccoth}{arccoth}
\DeclareMathOperator{\sech}{sech}
\DeclareMathOperator{\arcsech}{arcsech}
\DeclareMathOperator{\sgn}{sgn}
\DeclareMathOperator{\var}{var}
\DeclareMathOperator{\Ber}{Bernoulli}
\DeclareMathOperator{\Cov}{Cov}
\DeclareMathOperator{\E}{E}
\def\argmax{\qopname\relax m{arg\,max}}
\DeclarePairedDelimiterXPP{\Eb}[1]{\E}{\lbrack}{\rbrack}{}{%
  \renewcommand{\mid}{\SetSymbol[\delimsize]}#1}
\DeclarePairedDelimiterXPP{\varp}[1]{\var}{\lparen}{\rparen}{}{%
  \renewcommand{\mid}{\SetSymbol[\delimsize]}#1}
\DeclarePairedDelimiterXPP{\Covp}[1]{\Cov}{\lparen}{\rparen}{}{%
  \renewcommand{\mid}{\SetSymbol[\delimsize]}#1}
\DeclarePairedDelimiterXPP{\expp}[1]{\exp}{\lbrace}{\rbrace}{}{#1}
\newcommand*{\Fn}[1]{\mathop{\relax #1}\nolimits}
\newcommand*{\fn}[1]{\mathop{\relax\kern0pt #1}\nolimits}
\newcommand*{\gammaf}{\Fn{\Gamma}}
\renewcommand*{\Pr}{\Fn{P}}
\DeclarePairedDelimiterXPP{\Prp}[1]{\Pr}{\lparen}{\rparen}{}{%
  \renewcommand{\mid}{\SetSymbol[\delimsize]}#1}
\newcommand*{\pnorm}{\Fn{\Phi}}
\DeclarePairedDelimiterXPP{\pnormp}[1]{\pnorm}{\lparen}{\rparen}{}{#1}
\newcommand*{\dnorm}{\fn{\varphi}}
\DeclarePairedDelimiterXPP{\dnormp}[1]{\dnorm}{\lparen}{\rparen}{}{#1}
\newcommand*{\qnorm}{\Fn{\Phi}^{-1}}
%\newcommand*{\diff}{\mathop{}\!d}
\newcommand*{\diff}{\mathop{}\!\mathit{d}}
%\newcommand*{\diff}{\mathop{}\!\mathrm{d}}
\newcommand*{\dx}{\diff x}
\newcommand*{\dy}{\diff y}
\newcommand*{\dz}{\diff z}
\newcommand*{\ds}{\diff s}
\newcommand*{\dt}{\diff t}
\newcommand*{\du}{\diff u}
\newcommand*{\dv}{\diff v}
\newcommand*{\dtheta}{\diff \theta}
\newcommand*{\dd}[2][]{\frac{\diff#1}{\diff#2}}
\newcommand*{\ddx}{\frac{\diff}{\dx}}
\newcommand*{\ddt}{\frac{\diff}{\dt}}
\newcommand*{\ddy}{\dd y}
\newcommand*{\ddtheta}{\frac{\diff}{\dtheta}}
\newcommand*{\ddz}{\dd z}
\newcommand*{\fwdf}{\mathop{}\!\Delta}
\newcommand*{\dydx}{\frac\dy\dx}
\newcommand*{\pdpd}[2][]{\frac{\partial#1}{\partial#2}}
\newcommand*{\pdpdx}{\frac\partial{\partial x}}
\newcommand*{\pdpdy}{\frac\partial{\partial y}}
\newcommand*{\pdpdz}{\frac\partial{\partial z}}
\newcommand*{\pdpdu}{\frac\partial{\partial u}}
\newcommand*{\pdpdv}{\frac\partial{\partial v}}
\newcommand*{\pdpdt}{\frac\partial{\partial t}}
\newcommand*{\pdzpdx}{\frac{\partial z}{\partial x}}
\newcommand*{\pdzpdy}{\frac{\partial z}{\partial y}}
\newcommand*{\pdzpdt}{\frac{\partial z}{\partial t}}
\newcommand*{\pdxpdt}{\frac{\partial x}{\partial t}}
\newcommand*{\pdypdt}{\frac{\partial y}{\partial t}}

\input{preamble-ja}

\usepackage{tabularx}
\usepackage{diagbox}
\usepackage[column=C]{cellspace}
\addparagraphcolumntypes{X}
\setlength{\cellspacetoplimit}{1ex}
\setlength{\cellspacebottomlimit}{1ex}
\usepackage{luatexja-ruby}
\ltjsetruby{stretch=101}
\makeatletter
\def\arraycentercr{\let\newline\@centercr}
\makeatother
\newfontface{\ipafont}{Gentium Plus}
\newcommand{\ipa}[1]{{\footnotesize[{\ipafont#1}]}}

\let\reason\text
\let\vect\symbf

\AtBeginDocument{%
  % \renewcommand{\perp}{\mathrel{\bot}}
  \let\leq\leqslant
  \let\le\leq
  \let\geq\geqslant
  \let\ge\geq}


\begin{document}
\maketitle

\section[五十音図]{\ruby{五十音|図}{ごじゅうおん|ず}}

\begin{table}[H]
  \centering
  % \newcolumntype{Y}{>{\centering\arraybackslash\arraycentercr}X}
  \renewcommand{\tabularxcolumn}[1]{
    >{\centering\arraybackslash\arraycentercr}m{#1}}
  \begin{tabularx}{4in}{c*{5}{|C{X}}}
    \diagbox{\ruby{行}{ぎょう}}{\ruby{段}{だん}}
      & あ段
      & い段
      & う段
      & え段
      & お段 \\
    \hline
    あ行
      & あ \newline[-.8ex] \ipa{ä}
      & い \newline[-.8ex] \ipa{i}
      & う \newline[-.8ex] \ipa{ɯ̟}
      & え \newline[-.8ex] \ipa{e̞}
      & お \newline[-.8ex] \ipa{o̞} \\
    \hline
    か行
      & か \newline[-.8ex] \ipa{kä}
      & き \newline[-.8ex] \ipa{kʲi}
      & く \newline[-.8ex] \ipa{kɯ̟}
      & け \newline[-.8ex] \ipa{ke̞}
      & こ \newline[-.8ex] \ipa{ko̞} \\
    \hline
    さ行
      & さ \newline[-.8ex] \ipa{sä}
      & し \newline[-.8ex] \ipa{ɕi}
      & す \newline[-.8ex] \ipa{sɨ}
      & せ \newline[-.8ex] \ipa{se̞}
      & そ \newline[-.8ex] \ipa{so̞} \\
    \hline
    た行
      & た \newline[-.8ex] \ipa{tä}
      & ち \newline[-.8ex] \ipa{t͡ɕi}
      & つ \newline[-.8ex] \ipa{t͡sɨ}
      & て \newline[-.8ex] \ipa{te̞}
      & と \newline[-.8ex] \ipa{to̞} \\
    \hline
    な行
      & な \newline[-.8ex] \ipa{nä}
      & に \newline[-.8ex] \ipa{ɲ̟i}
      & ぬ \newline[-.8ex] \ipa{nɯ̟}
      & ね \newline[-.8ex] \ipa{ne̞}
      & の \newline[-.8ex] \ipa{no̞} \\
    \hline
    は行
      & は \newline[-.8ex] \ipa{hä}
      & ひ \newline[-.8ex] \ipa{çi}
      & ふ \newline[-.8ex] \ipa{ɸɯ̟}
      & へ \newline[-.8ex] \ipa{he̞}
      & ほ \newline[-.8ex] \ipa{ho̞} \\
    \hline
    ま行
      & ま \newline[-.8ex] \ipa{mä}
      & み \newline[-.8ex] \ipa{mʲi}
      & む \newline[-.8ex] \ipa{mɯ̟}
      & め \newline[-.8ex] \ipa{me̞}
      & も \newline[-.8ex] \ipa{mo̞} \\
    \hline
    や行
      & や \newline[-.8ex] \ipa{jä}
      & \
      & ゆ \newline[-.8ex] \ipa{jɯ̟}
      & \
      & よ \newline[-.8ex] \ipa{jo̞} \\
    \hline
    ら行
      & ら \newline[-.8ex] \ipa{ɾä}
      & り \newline[-.8ex] \ipa{ɾʲi}
      & る \newline[-.8ex] \ipa{ɾɯ̟}
      & れ \newline[-.8ex] \ipa{ɾe̞}
      & ろ \newline[-.8ex] \ipa{ɾo̞} \\
    \hline
    わ行
      & わ \newline[-.8ex] \ipa{ɰä}
      & \
      & \
      & \
      & を \newline[-.8ex] \ipa{o̞} \\
    \hline
    \ruby{撥|音}{はつ|おん}
      & ん \newline[-.8ex] \ipa{n m ŋ ɲ}
  \end{tabularx}
  \caption{\ruby{平|仮|名}{ひら|が|な}の五十音図}
\end{table}

此处虽然用{\ipafont [ɰ]}来表示日语的{\ipafont /w/},但要注意日语
的{\ipafont /w/}和国际音标的{\ipafont [ɰ]}并不完全一致。

\begin{table}[H]
  \centering
  \renewcommand{\tabularxcolumn}[1]{
    >{\centering\arraybackslash\arraycentercr}m{#1}}
  \begin{tabularx}{4in}{c*{5}{|C{X}}}
    \diagbox{行}{段}
      & ア段
      & イ段
      & ウ段
      & エ段
      & オ段 \\
    \hline
    ア行
      & ア
      & イ
      & ウ
      & エ
      & オ \\
    \hline
    カ行
      & カ
      & キ
      & ク
      & ケ
      & コ \\
    \hline
    サ行
      & サ
      & シ
      & ス
      & セ
      & ソ \\
    \hline
    タ行
      & タ
      & チ
      & ツ
      & テ
      & ト \\
    \hline
    ナ行
      & ナ
      & ニ
      & ヌ
      & ネ
      & ノ \\
    \hline
    ハ行
      & ハ
      & ヒ
      & フ
      & ヘ
      & ホ \\
    \hline
    マ行
      & マ
      & ミ
      & ム
      & メ
      & モ \\
    \hline
    ヤ行
      & ヤ
      & \
      & ユ
      & \
      & ヨ \\
    \hline
    ラ行
      & ラ
      & リ
      & ル
      & レ
      & ロ \\
    \hline
    ワ行
      & ワ
      & \
      & \
      & \
      & \  \\
    \hline
    撥音
      & ン
  \end{tabularx}
  \caption{\ruby{片|仮|名}{かた|か|な}の五十音図}
\end{table}

上面两个表里的都是\ruby{清|音}{せい|おん},日语除了清音,还
有\ruby{濁|音}{だく|おん}和\ruby{半}{はん}濁音,分别在仮名右上角
加\ruby{濁|点}{だく|てん}和半濁点来表示。

\begin{table}[H]
  \centering
  \renewcommand{\tabularxcolumn}[1]{
    >{\centering\arraybackslash\arraycentercr}m{#1}}
  \begin{tabularx}{4in}{c*{5}{|C{X}}}
    \diagbox{行}{段}
      & あ段
      & い段
      & う段
      & え段
      & お段 \\
    \hline
    が行
      & が \newline[-.8ex] \ipa{ɡä}
      & ぎ \newline[-.8ex] \ipa{ɡʲi}
      & ぐ \newline[-.8ex] \ipa{ɡɯ̟}
      & げ \newline[-.8ex] \ipa{ɡe̞}
      & ご \newline[-.8ex] \ipa{ɡo̞} \\
    \hline
    ざ行
      & ざ \newline[-.8ex] \ipa{d͡zä}
      & じ \newline[-.8ex] \ipa{d͡ʑi}
      & ず \newline[-.8ex] \ipa{d͡zɨ}
      & ぜ \newline[-.8ex] \ipa{d͡ze̞}
      & ぞ \newline[-.8ex] \ipa{d͡zo̞} \\
    \hline
    だ行
      & だ \newline[-.8ex] \ipa{dä}
      & ぢ \newline[-.8ex] \ipa{d͡ʑi}
      & づ \newline[-.8ex] \ipa{d͡zɨ}
      & で \newline[-.8ex] \ipa{de̞}
      & ど \newline[-.8ex] \ipa{do̞} \\
    \hline
    ば行
      & ば \newline[-.8ex] \ipa{bä}
      & び \newline[-.8ex] \ipa{bʲi}
      & ぶ \newline[-.8ex] \ipa{bɯ̟}
      & べ \newline[-.8ex] \ipa{be̞}
      & ぼ \newline[-.8ex] \ipa{bo̞} \\
    \hline
    ぱ行
      & ぱ \newline[-.8ex] \ipa{pä}
      & ぴ \newline[-.8ex] \ipa{pʲi}
      & ぷ \newline[-.8ex] \ipa{pɯ̟}
      & ぺ \newline[-.8ex] \ipa{pe̞}
      & ぽ \newline[-.8ex] \ipa{po̞} \\
  \end{tabularx}
  \caption{濁音と半濁音}
\end{table}

\begin{table}[H]
  \centering
  \renewcommand{\tabularxcolumn}[1]{
    >{\centering\arraybackslash\arraycentercr}m{#1}}
  \begin{tabularx}{\textwidth}{c*{11}{|C{X}}}
    \diagbox{段}{行}
      & か行
      & さ行
      & た行
      & な行
      & は行
      & ま行
      & ら行
      & が行
      & ざ行
      & ば行
      & ぱ行 \\
    \hline
    あ段
      & きゃ \newline[-.8ex] \ipa{kʲä}
      & じゃ \newline[-.8ex] \ipa{ɕä}
      & ちゃ \newline[-.8ex] \ipa{t͡ɕä}
      & にゃ \newline[-.8ex] \ipa{ɲ̟ä}
      & ひゃ \newline[-.8ex] \ipa{çä}
      & みゃ \newline[-.8ex] \ipa{mʲä}
      & りゃ \newline[-.8ex] \ipa{ɾʲä}
      & ぎゃ \newline[-.8ex] \ipa{gʲä}
      & じゃ \newline[-.8ex] \ipa{d͡ʑä}
      & びゃ \newline[-.8ex] \ipa{bʲä}
      & ぴゃ \newline[-.8ex] \ipa{pʲä} \\
    \hline
    う段
      & きゅ \newline[-.8ex] \ipa{kʲɨ}
      & しゅ \newline[-.8ex] \ipa{ɕɨ}
      & ちゅ \newline[-.8ex] \ipa{t͡ɕɨ}
      & にゅ \newline[-.8ex] \ipa{ɲ̟ɨ}
      & ひゅ \newline[-.8ex] \ipa{çɨ}
      & みゅ \newline[-.8ex] \ipa{mʲɨ}
      & りゅ \newline[-.8ex] \ipa{ɾʲɨ}
      & ぎゅ \newline[-.8ex] \ipa{gʲɨ}
      & じゅ \newline[-.8ex] \ipa{d͡ʑɨ}
      & びゅ \newline[-.8ex] \ipa{bʲɨ}
      & ぴゅ \newline[-.8ex] \ipa{pʲɨ} \\
    \hline
    お段
      & きょ \newline[-.8ex] \ipa{kʲo̞}
      & しょ \newline[-.8ex] \ipa{ɕo̞}
      & ちょ \newline[-.8ex] \ipa{t͡ɕo̞}
      & にょ \newline[-.8ex] \ipa{ɲ̟o̞}
      & ひょ \newline[-.8ex] \ipa{ço̞}
      & みょ \newline[-.8ex] \ipa{mʲo̞}
      & りょ \newline[-.8ex] \ipa{ɾʲo̞}
      & ぎょ \newline[-.8ex] \ipa{gʲo̞}
      & じょ \newline[-.8ex] \ipa{d͡ʑo̞}
      & びょ \newline[-.8ex] \ipa{bʲo̞}
      & ぴょ \newline[-.8ex] \ipa{pʲo̞}
  \end{tabularx}
  \caption{\ruby{拗}{よう}音}
\end{table}

\section[挨拶]{\ruby{挨|拶}{あい|さつ}の\ruby{言|葉}{こと|ば}}

\begingroup
\setlength{\parskip}{1em plus 1pt}

お\ruby{早}{はよ}う\ruby{御}{ご}\ruby{座}{ざ}います。

\ruby{今}{こん}\ruby{日}{にち}は。

さようなら。

\ruby{今}{こん}\ruby{晩}{ばん}は。

お\ruby{休}{やす}みなさい。

\ruby{初}{はじ}めまして、どうぞ、\ruby{宜}{よろ}しくお\ruby{願}{ねが}いします。

すみません。

\ruby{有}{あり}\ruby{難}{がと}う御座いました。

いいえ、どう\ruby{致}{いた}しまして。

\ruby{頂}{いただ}きます。

ご\ruby{馳|走}{ち|そう}\ruby{様}{さま}でした。
\endgroup

\section[語彙]{\ruby{語}{ご}\ruby{彙}{い}}

\begin{table}[H]
  \centering
  \renewcommand{\tabularxcolumn}[1]{
    >{\centering\arraybackslash\arraycentercr}m{#1}}
  \newcolumntype{Y}{>{\hsize=\dimexpr2\hsize+2\tabcolsep+\arrayrulewidth\relax}X}
  \begin{tabularx}{\textwidth}{*{9}{C{X}|}C{X}}
    \ruby{花}{はな}
    & \ruby{母}{はは}
    & \ruby{私}{わたし}
    & \ruby{川}{かわ}
    & \ruby{旗}{はた}
    & \ruby{学|生}{がく|せい}
    & \ruby{本}{ほん}
    & \ruby{映|画}{えい|が}
    & \ruby{駅}{えき}
    & \ruby{公|園}{こう|えん} \\
    \hline
    \ruby{部|屋}{へ|や}
    & \ruby{平|和}{へい|わ}
    & \ruby{学|校}{がっ|こう}
    & \ruby{工|場}{こう|じょう}
    & \ruby{お|菓|子}{|か|し}
    & \ruby{顔}{かお}
    & \ruby{応|接|間}{おう|せつ|ま}
    & \ruby{旗}{はた}
    & \ruby{砂|糖}{さ|とう}
    & \ruby{妹}{いもうと} \\
    \hline
    \ruby{日|曜|日}{にち|よう|び}
    & \ruby{帽|子}{ぼう|し}
    & \ruby{動|物}{どう|ぶつ}
    & \ruby{教|室}{きょう|しつ}
    & \ruby{大|きい}{おお|}
    & \ruby{多|い}{おお|}
    & \ruby{氷}{こおり}
    & \ruby{遠|い}{とお|}
    & \ruby{十}{とお}
    & \ruby{石|鹸}{せっ|けん} \\
    \hline
    \ruby{切|手}{きっ|て}
    & \ruby{雑|誌}{ざっ|し}
    & \ruby{辞|典}{じ|てん}
    & \ruby{漢|字}{かん|じ}
    & \ruby{縮|む}{ちぢ|}
    & \ruby{近|々}{ちか|ぢか}
    & \multicolumn{2}{C{Y}|}{\ruby{茶|飲|み|茶|碗}{ちゃ|の||ぢゃ|わん}}
    & \ruby{水}{みず}
    & \ruby{数}{かず} \\
    \hline
    \ruby{続|く}{つづ|}
    & \ruby{近|づく}{ちか|}
    & \ruby{缶|詰}{かん|づめ}
    & \ruby{飯|茶|碗}{めし|ぢゃ|わん}
  \end{tabularx}
\end{table}
\end{document}
