\documentclass[a4paper]{book}

\title{Solutions to Graph Theory with Applications}
\author{L. F. \textsc{Jaw}}

\usepackage[T1]{fontenc}
\usepackage{textcomp}
\usepackage{mathtools,amssymb,amsthm}
\usepackage[hmargin=1in,vmargin=1in]{geometry}
\usepackage{graphicx,xcolor}
\usepackage[pdfusetitle]{hyperref}
\hypersetup{%
  colorlinks=true,
  urlcolor=[rgb]{0,0.2,0.6},
  linkcolor={.},
  bookmarksdepth=2}
\usepackage{bookmark}
\usepackage{float}

\frenchspacing

\newcommand*{\parasp}{\setlength{\parskip}{10pt plus 2pt minus 3pt}}
\newcommand*{\noparasp}{\setlength{\parskip}{0pt plus 1pt}}
\newcommand*{\setparasp}[1]{\setlength{\parskip}{#1}}
\newcommand*{\pskip}{\vskip 10pt plus 2pt minus 3pt}
% \newcommand\LEFTRIGHT[3]{\left#1 #3 \right#2}
\newcommand\SetSymbol[1][]{%
  \nonscript\:#1\vert
  \allowbreak
  \nonscript\:
  \mathopen{}}
% \newcommand*{\paren}[1]{\LEFTRIGHT(){#1}}
\DeclarePairedDelimiterX{\paren}[1]{\lparen}{\rparen}{%
  \renewcommand{\mid}{\SetSymbol[\delimsize]}#1}
% \newcommand*{\brkt}[1]{\LEFTRIGHT[]{#1}}
\DeclarePairedDelimiterX{\brkt}[1]{\lbrack}{\rbrack}{%
  \renewcommand{\mid}{\SetSymbol[\delimsize]}#1}
\DeclarePairedDelimiterX{\brce}[1]{\lbrace}{\rbrace}{%
  \renewcommand{\mid}{\SetSymbol[\delimsize]}#1}
\newcommand*{\unit}[1]{\,\mathrm{#1}}
\newcommand*{\DeclareUnit}[1]{\expandafter\def\csname#1\endcsname{\unit{#1}}}
\DeclareUnit{cm}
% \renewcommand*{\m}{\unit{m}}
\DeclareUnit{m}
\DeclareUnit{kg}
\DeclareUnit{s}
\newcommand*{\R}{\mathbb{R}}
\newcommand*{\Z}{\mathbb{Z}}
\newcommand*{\N}{\mathbb{N}}
\newcommand*{\Q}{\mathbb{Q}}
% \newcommand*{\Rp}{(0,+\infty)}
% \newcommand*{\Rm}{(-\infty,0)}
\newcommand*{\deduce}{\mathrel{\Downarrow}}
% \newcommand*{\abs}[1]{\left\lvert #1 \right\rvert}
\DeclarePairedDelimiter{\abs}{\lvert}{\rvert}
% \newcommand*{\ceil}[1]{\left\lceil#1\right\rceil}
\DeclarePairedDelimiter{\ceil}{\lceil}{\rceil}
\DeclarePairedDelimiter{\floor}{\lfloor}{\rfloor}
\newcommand*{\textop}[1]{\mathbin{\text{#1}}}
\newcommand*{\tand}{\textop{and}}
\newcommand*{\tor}{\textop{or}}
\newcommand*{\txt}[2][\quad]{#1 \text{#2} #1}
\newcommand*{\qand}{\txt{and}}
\newcommand*{\iand}{\intertext{and}}
\newcommand*{\DeclareText}[1]{\expandafter\def\csname#1\endcsname{\text{#1}}}
\DeclareText{otherwise}
\newcommand*{\tfor}{\text{for }}
\newcommand*{\qfor}{\txt{for}}

\newcommand*{\enumparen}[1]{(\makebox[0.6em][c]{#1})}
\renewcommand{\labelenumii}{\enumparen{\theenumii}}
\newcommand*{\upstar}{\textsuperscript{\normalfont\textasteriskcentered}}%
\makeatletter
\newcommand*{\bonus}{\@itemlabel\upstar}%
\def\contitem{%
  \def\H@item{%
    \@inmatherr\item
    \@noitemargtrue
    \@ifnextchar[\@item{\@item[\@itemlabel]}}}
\makeatother

\DeclareMathOperator{\arccosh}{arccosh}
\DeclareMathOperator{\arcsinh}{arcsinh}
\DeclareMathOperator{\arctanh}{arctanh}
\DeclareMathOperator{\arccoth}{arccoth}
\DeclareMathOperator{\sech}{sech}
\DeclareMathOperator{\arcsech}{arcsech}
\DeclareMathOperator{\sgn}{sgn}
\DeclareMathOperator{\var}{var}
\DeclareMathOperator{\Ber}{Bernoulli}
\DeclareMathOperator{\Cov}{Cov}
\DeclareMathOperator{\E}{E}
\def\argmax{\qopname\relax m{arg\,max}}
\DeclarePairedDelimiterXPP{\Eb}[1]{\E}{\lbrack}{\rbrack}{}{%
  \renewcommand{\mid}{\SetSymbol[\delimsize]}#1}
\DeclarePairedDelimiterXPP{\varp}[1]{\var}{\lparen}{\rparen}{}{%
  \renewcommand{\mid}{\SetSymbol[\delimsize]}#1}
\DeclarePairedDelimiterXPP{\Covp}[1]{\Cov}{\lparen}{\rparen}{}{%
  \renewcommand{\mid}{\SetSymbol[\delimsize]}#1}
\DeclarePairedDelimiterXPP{\expp}[1]{\exp}{\lbrace}{\rbrace}{}{#1}
\newcommand*{\Fn}[1]{\mathop{\relax #1}\nolimits}
\newcommand*{\fn}[1]{\mathop{\relax\kern0pt #1}\nolimits}
\newcommand*{\gammaf}{\Fn{\Gamma}}
\renewcommand*{\Pr}{\Fn{P}}
\DeclarePairedDelimiterXPP{\Prp}[1]{\Pr}{\lparen}{\rparen}{}{%
  \renewcommand{\mid}{\SetSymbol[\delimsize]}#1}
\newcommand*{\pnorm}{\Fn{\Phi}}
\DeclarePairedDelimiterXPP{\pnormp}[1]{\pnorm}{\lparen}{\rparen}{}{#1}
\newcommand*{\dnorm}{\fn{\varphi}}
\DeclarePairedDelimiterXPP{\dnormp}[1]{\dnorm}{\lparen}{\rparen}{}{#1}
\newcommand*{\qnorm}{\Fn{\Phi}^{-1}}
%\newcommand*{\diff}{\mathop{}\!d}
\newcommand*{\diff}{\mathop{}\!\mathit{d}}
%\newcommand*{\diff}{\mathop{}\!\mathrm{d}}
\newcommand*{\dx}{\diff x}
\newcommand*{\dy}{\diff y}
\newcommand*{\dz}{\diff z}
\newcommand*{\ds}{\diff s}
\newcommand*{\dt}{\diff t}
\newcommand*{\du}{\diff u}
\newcommand*{\dv}{\diff v}
\newcommand*{\dtheta}{\diff \theta}
\newcommand*{\dd}[2][]{\frac{\diff#1}{\diff#2}}
\newcommand*{\ddx}{\frac{\diff}{\dx}}
\newcommand*{\ddt}{\frac{\diff}{\dt}}
\newcommand*{\ddy}{\dd y}
\newcommand*{\ddtheta}{\frac{\diff}{\dtheta}}
\newcommand*{\ddz}{\dd z}
\newcommand*{\fwdf}{\mathop{}\!\Delta}
\newcommand*{\dydx}{\frac\dy\dx}
\newcommand*{\pdpd}[2][]{\frac{\partial#1}{\partial#2}}
\newcommand*{\pdpdx}{\frac\partial{\partial x}}
\newcommand*{\pdpdy}{\frac\partial{\partial y}}
\newcommand*{\pdpdz}{\frac\partial{\partial z}}
\newcommand*{\pdpdu}{\frac\partial{\partial u}}
\newcommand*{\pdpdv}{\frac\partial{\partial v}}
\newcommand*{\pdpdt}{\frac\partial{\partial t}}
\newcommand*{\pdzpdx}{\frac{\partial z}{\partial x}}
\newcommand*{\pdzpdy}{\frac{\partial z}{\partial y}}
\newcommand*{\pdzpdt}{\frac{\partial z}{\partial t}}
\newcommand*{\pdxpdt}{\frac{\partial x}{\partial t}}
\newcommand*{\pdypdt}{\frac{\partial y}{\partial t}}

% \usepackage[lite,subscriptcorrection,nofontinfo]{mtpro2}
\usepackage{fontspec}

\setmainfont{Palatino Linotype}[Ligatures=TeX,Numbers=OldStyle]
\setmonofont{Source Code Pro}
% \usepackage[integrals]{wasysym}
\usepackage{fontawesome}

\usepackage[math-style=TeX]{unicode-math}
\setmathfont{TeX Gyre Pagella Math}

\usepackage{microtype}


\makeatletter
\renewenvironment{proof}[1][\proofname]{\par
  \pushQED{\qed}%
  \normalfont \topsep6\p@\@plus6\p@\relax
  \trivlist
  \item[]\ignorespaces
}{%
  \popQED\endtrivlist\@endpefalse
}
\makeatother

\let\reason\text
\let\vect\symbf

\AtBeginDocument{%
  % \renewcommand{\perp}{\mathrel{\bot}}
  \let\leq\leqslant
  \let\le\leq
  \let\geq\geqslant
  \let\ge\geq}


\begin{document}
\frontmatter
\maketitle

This is a collection of personal solutions to \textit{Graph theory
  with Applications} by Bondy and Murty.

% \tableofcontents
\mainmatter

\chapter{Graphs and Subgraphs}
\label{chap:1}

\section{Graphs and Simple Graphs}
\label{sec:1.1}

\section{Graph Isomorphism}
\label{sec:1.2}

\begin{enumerate}
  \setcounter{enumi}{7}
\item
  \begin{enumerate}
  \item
    \begin{proof}
      The set of vertices of \(K_{m,n}\) can be partitioned into two
      sets \(X\) and \(Y\), where \(\abs{X} = m\) and \(\abs{Y} = n\).
      Since \(K_{m,n}\) is also simple, we have
      \(E(K_{m,n}) = X \times Y\).  It follows that
      \[
        \varepsilon = \abs{E(K_{m,n})} = \abs{X \times Y} = \abs{X} \cdot \abs{Y} = mn. \qedhere
      \]
  \end{proof}
    
  \item
    \begin{proof}
      For any simple and bipartite graph \(G\), which can be
      partitioned into \(X\) and \(Y\), we always have
      \(\varepsilon \le \varepsilon(K_{m,n})\), where \(m = \abs{X}\)
      and \(n = \abs{Y}\).  Since \(\nu = m+n\), for any bipartition
      \((X, Y)\), we have
      \begin{align*}
        \varepsilon &\le \max \varepsilon(K_{m,n}) \\
                    &= \max m(\nu-m) \\
                    &= \frac{\nu}{2} \paren[\Big]{\nu - \frac{\nu}{2}}
                      = \frac{\nu^2}{4}. \qedhere
      \end{align*}
    \end{proof}
  \end{enumerate}
  
\item
  \begin{enumerate}
  \item
    \begin{proof}
      When \(n = m\), we have \(T_{m,n} = T_{m,m} = K_m\) and \(k = \brkt{n/m} = \brkt{m/m} = 1\).  Thus, 
      \begin{align*}
        \binom{n-k}{2} + (m-1) \binom{k+1}{2}
          &= \frac{(m-1)(m-2)}{2} + (m-1) \\
          &= \frac{(m-1)m}{2}
            = \binom{m}{2} \\
          &= \varepsilon(K_m)
            = \varepsilon(T_{m,n}).
      \end{align*}
      This is our induction base.  Suppose now the formula holds for
      some \(1 \le m \le n\).  Let \(k = \brkt{n/m}\).  Then
      \begin{align*}
        \varepsilon(T_{m,n+1})
          &= \varepsilon(T_{m,n}) + k(m-1) + (n+1 - km -1) \\
          &= \varepsilon(T_{m,n}) + n - k \\
          &= \binom{n-k}{2} + (m-1) \binom{k+1}{2} + n - k \\
          &= \frac{(n-k)(n-k-1)}{2} + n - k + (m-1) \binom{k+1}{2} \\
          &= \frac{(n-k)(n-k+1)}{2} + (m-1) \binom{k+1}{2} \\
          &= \binom{n+1-k}{2} + (m-1) \binom{k+1}{2}. \qedhere
      \end{align*}
      Notice that the choice of \(k\) is subtle here.
    \end{proof}
  \item
    \begin{proof}
      Let \(G\) be a complete \(m\)-partite graph on \(n\) vertices.
      Let \(n_1, n_2, \dotsc, n_m\) denote the size of the partitions
      for \(G\).  If \(\abs{n_i - n_j} \le 1\) for all \(i\) and \(j\),
      then \(G \cong T_{m,n}\).  The contrapositive of the previous
      proposition says if \(G \ncong T_{m,n}\), then there exist a pair
      \(i\) and \(j\) such that \(n_i - n_j > 1\).  Then we modify the
      graph \(G\) such that the \(i\)th partition one less vertex and
      the \(j\)th partition has one more vertex.  Outside the two
      partition concerned, the \emph{number} of edges is unchanged.
      Within the two partition concerned, the change of number of edges
      is
      \begin{align*}
        (n_i - 1)(n_j + 1) - n_i n_j
          &= n_i n_j + n_i - n_j - 1 - n_i n_j \\
          &= n_i - n_j - 1 \\
          &> 0.
      \end{align*}
      This completes the proof.
  \end{proof}
  \end{enumerate}
\end{enumerate}

\newpage
\null
\end{document}

