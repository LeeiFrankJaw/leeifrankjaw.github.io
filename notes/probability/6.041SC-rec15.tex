\documentclass{article}

\usepackage{amsmath, amssymb}
\usepackage[a4paper, hmargin=1.25in, vmargin=1in]{geometry}
\usepackage[colorlinks=true, urlcolor=blue]{hyperref}
\usepackage{graphicx, cancel}

\newcommand*{\parasp}{\setlength{\parskip}{10pt}}
\newcommand*{\paren}[1]{\left( #1 \right)}
\newcommand*{\Prb}[1]{\section*{Problem #1}}
\newcommand{\Q}[1]{\textbf{Question:} #1}
\newcommand*{\A}[1]{\textbf{Answer:} #1}
\newcommand{\Prblm}[3]{\Prb{#1} \Q{#2} \\[6pt] \A{#3}}
\newcommand*{\cm}{\,\mathrm{cm}}
\newcommand*{\R}{\mathbb{R}}
\newcommand*{\Rp}{(0,+\infty)}
\newcommand*{\Rm}{(-\infty,0)}
\newcommand*{\deduce}{\mathrel{\Downarrow}}
\newcommand*{\abs}[1]{\left\lvert #1 \right\rvert}
\newcommand*{\reason}[1]{\langle \, \text{#1} \, \rangle}

\DeclareMathOperator{\arccosh}{arccosh}

\everymath{\displaystyle}

\title{Solutions to Recitation 15}
\author{Lei Zhao}

\hyphenpenalty=687

\begin{document}
\maketitle

This is my own solutions to the problems from recitation 15 of
\href{https://ocw.mit.edu/courses/electrical-engineering-and-computer-science/6-041sc-probabilistic-systems-analysis-and-applied-probability-fall-2013/unit-iii/lecture-13/}{6.041\textsc{sc}}.

\begin{enumerate}
\item Beginning at time \(t = 0\), we begin using bulbs, one at a
  time, to illuminate a room.  Bulbs are replaced immediately upon
  failure.  Each new bulb is selected independently by an equally
  likely choice between a type-A bulb and a type-B bulb.  The
  lifetime, \(X\), of any particular bulb of a particular type is a
  random variable, independent of everything else, with the following
  \textsc{pdf}:
  \begin{center}
    \begin{tabular}{rl}
      for type-A bulbs:
      & \(f_X(x) =
        \begin{cases}
          \mathrlap{e^{-x},} \hphantom{3e^{-3x},} & x \geq 0, \\
          0,     & \text{otherwise;}
        \end{cases}\) \\[1.5em]
      for type-B bulbs:
      & \(f_X(x) =
        \begin{cases}
          3e^{-3x}, & x \geq 0, \\
          0,       & \text{otherwise.}
        \end{cases}\)
    \end{tabular}
  \end{center}

  \begin{enumerate} \parasp
  \item Find the expected time until the first failure.

    Let \(Y\) be the random variable such that \(Y=0\) whenever type-A
    bulb is chosen and \(Y=1\) whenever type-B bulb is chosen.  The
    expected time until the first failure actually is \(\E[X]\).  By
    the law of iterated expectation, we have
    \begin{align*}
      \E[X] &= \E[\E[X \mid Y]] \\
            &= p_Y(0) \E[X \mid Y=0] + p_Y(1) \E[X \mid Y=1] \\
            &= \frac12 + \frac12 \cdot \frac13 = \frac23.
    \end{align*}

  \item Find the probability that there are no bulb failures before
    time \(t\).

    This probabity is \(P(X > t)\).  By total probability theorem, we
    have
    \begin{align*}
      P(X > t) &= P(Y=0) P(X > t \mid Y=0) + P(Y=1) P(X > t \mid Y=1) \\
               &= \frac12  \int_t^{+\infty} e^{-x} \dx + \frac12 \int_t^{+\infty} 3e^{-3x} \dx \\
               &= \left.\frac12 e^{-x}\right\vert_{+\infty}^t + \left.\frac12 e^{-3x}\right\vert_{+\infty}^t \\
               &= \frac12 (e^{-t} + e^{-3t}).
    \end{align*}

  \item Given that there are no failures until time \(t\), determine
    the conditional probability that the first bulb used is a type-A
    bulb.

    By Bayes' theorem, we have
    \begin{align*}
      P(Y=0 \mid X > t) &= \frac{P(Y=0) P(X > t \mid Y=0)}{P(X > t)} \\
                        &= \frac{e^{-t}/2}{(e^{-t} + e^{-3t})/2} \\
                        &= \frac1{1+e^{-2t}}.
    \end{align*}

  \item Determine the probability that the total period of
    illumination provided by the first two type-B bulbs is longer than
    that provided by the first type-A bulb.

    Let \(X_1\), \(X_2\) be the lifetime of first two type-B bulbs and
    \(X_3\) be the lifetime of first type-A bulb.  The probability to be
    determined is actually \(P(X_1+X_2 > X_3) = P(X_1 + X_2 - X_3 > 0)\).
    Let \(S = X_1 + X_2\) and \(Z = S - X_3\).  By Erlang distribution,
    when \(s \geq 0\) we have
    \[ f_S(s) = \frac{3^2se^{-3s}}{(2-1)!} = 9se^{-3s}.\]
    By convolution formula, we have
    \begin{align*}
      f_Z(z) &= \int_{-\infty}^{+\infty} f_S(s) f_{-X_3}(z-s) \diff s \\
             &= \int_{-\infty}^{+\infty} f_S(s) f_{X_3}(s-z) \diff s \\
             &= \int_{\max(0, z)}^{+\infty} 9se^{-3s} e^{-(s-z)} \diff s \\
             &= 9e^z \int_{\max(0, z)}^{+\infty} se^{-4s} \diff s \\
             &= \frac94 e^z \left.\paren{s+\frac14} e^{-4s}\right\vert_{+\infty}^{\max(0, z)} \\
             &=
               \begin{cases}
                 \frac{9}{16} e^z, & z \leq 0, \\[1em]
                 \frac94 \paren{z+\frac14} e^{-3z}, & z > 0.
               \end{cases}
    \end{align*}
    Thus the probability to be determined is
    \begin{align*}
      P(Z > 0) &= 1 - P(Z \leq 0) \\
               &= 1 - \int_{-\infty}^0 \frac{9}{16} e^z \dz \\
               &= 1 - \frac{9}{16} = \frac{7}{16}.
    \end{align*}

  \item Suppose the process terminates as soon as a total of exactly
    \(12\) bulb failures have occurred. Determine the expected value and
    variance of the total period of illumination provided by type-B
    bulbs while the process is in operation.

    Let \(K\) be the number of type-B bulbs occurred in the process
    and \(X\) be the total period of illumination provided by type-B
    bulbs.  Fix \(K = k\), we have
    \begin{gather*}
      \E[X \mid K=k] = \frac{k}{3} \\
      \intertext{and}
      \var(X \mid K=k) = \frac{k}{9}.
    \end{gather*}
    Therefore,
    \begin{gather*}
      \E[X \mid K] = \frac{K}{3} \\
      \intertext{and}
      \var(X \mid K) = \frac{K}{9}.
    \end{gather*}
    By law of iterated expectation and law of total variance, we have
    \begin{align*}
      \E[X] &= \E[\E[X \mid K]] \\
            &= \E[K/3] \\
            &= \frac13\E[K] \\
            &= \frac13 \cdot 12 \cdot \frac12= 2 \\
      \intertext{and}
      \var(X) &= \E[\var(X \mid K)] + \var(\E[X \mid K]) \\
            &= \E[K/9] + \var(K/3) \\
            &= \frac19 \E[K] + \frac19 \var(K) \\
            &= \frac19 (6 + 3) = 1.
    \end{align*}

  \item Given that there are no failures until time \(t\), find the
    expected value of the time until the first failure.

    The entire process is not memoryless; however, condtioned upon chosen
    type of bulbs, the conditional probability of bulb lifetime is
    memeryless.  Using notation from (a)--(c), we have
    \begin{align*}
      \E[X \mid X > t] &= t + \E[X-t \mid X>t] \\
                       &= t + P(Y=0 \mid X>t) \E[X-t \mid X>t, Y=0] \\
                       &\phantom{{}=t} + P(Y=1 \mid X>t) \E[X-t \mid X>t, Y=1] \\
                       &= t + \frac1{1+e^{-2t}} + \frac1{3(1+e^{2t})}.
    \end{align*}
  \end{enumerate}

\item A service station handles jobs of two types, A and B.  (Multiple
  jobs can be processed simultane­ously.)  Arrivals of the two job
  types are independent Poisson processes with parameters \(λ_A=3\)
  and \(λ_B=4\) per minute, respectively.  Type A jobs stay in the
  service station for exactly one minute.  Each type B job stays in
  the service station for a random but integer amount of time which is
  geometrically distributed, with mean equal to \(2\), and independent
  of everything else.  The service station started operating at some
  time in the remote past.
  
  \begin{enumerate} \parasp
  \item What is the mean, variance, and \textsc{pmf} of the total
    number of jobs that arrive within a given three-minute interval?

    Let \(K\) be the total number of job arrivals within a given
    three-minute interval.  The merging of two Poisson process is
    another Poisson process with parameter \(\lambda_A + \lambda_B\),
    thus we have
    \begin{gather*}
      p_K(k) = \frac{(3(3+4))^k}{k!} e^{-3(3+4)} = \frac{21^k}{k!} e^{-21}, \\[1em]
      \E[K] = 21, \\
      \intertext{and}
      \var(K) = 21.
    \end{gather*}

  \item We are told that during a \(10\)-minute interval, exactly \(10\) new
    jobs arrived.  What is the probability that exactly \(3\) of them are
    of type A?

    Since the number of type A and type B job arrivals during a 10-minute
    interval is indepent from each other.  Let \(K_{\lambda \tau}\)
    denoate the Poisson random variable with parameter \(\lambda \tau\), the
    probability to be determined is
    \begin{align*}
      P(K_{3\times10} = 3 \mid K_{(3+4)\times10} = 10)
      &= \frac{P(K_{30} = 3) \, P(K_{40} = 7)}{P(K_{70} = 10)} \\
      &= \frac{\frac{30^{3}}{3!} e^{-30} \, \frac{40^{7}}{7!} e^{-40}}{\frac{70^{10}}{10!} e^{-70}} \\
      &= \binom{10}{3} \paren{\frac37}^3 \paren{\frac47}^7 \approx 18.79\%.
    \end{align*}

  \item At time \(0\), no job is present in the service station.  What is
    the \textsc{pmf} of the number of type B jobs that arrive in the
    future, but before the first type A arrival?

    Let \(K\) denote the number of type B job arrival before the first
    type A job arrival and \(X\) denote the time of first type A arrival.
    Then, we have
    \[P(K=k \mid X=t) = \frac{(4t)^k}{k!} e^{-4t}\]
    and
    \begin{align*}
      P(K) &= \int_0^{+\infty} f_X(t) P(K=k \mid X=t) \dt \\
           &= \frac{3 \times 4^k}{k!} \int_0^{+\infty} t^k \, e^{-7t} \dt.
    \end{align*}
  \end{enumerate}

\item Let \(X\), \(Y\), and \(Z\) be independent exponential random
  variables with parameters \(λ\), \(μ\), and \(ν\), respectively.  Find
  \(P(X < Y < Z)\).

  \begin{align*}
    P(X < Y < Z) &= \iiint_{\{(x,y,z) \mid x < y <z\}} f_X(x) f_Y(y) f_Z(z) \dx \dy \dz \\
                 &= \int_0^{+\infty} \int_x^{+\infty} \int_y^{+\infty} f_X(x) f_Y(y) f_Z(z) \dz \dy \dx \\
                 &= \int_0^{+\infty} f_X(x) \int_x^{+\infty} f_Y(y) \int_y^{+\infty} f_Z(z) \dz \dy \dx \\
                 &= \int_0^{+\infty} \lambda e^{-\lambda x} \int_x^{+\infty} \mu e^{-\mu y} \int_y^{+\infty} \nu e^{-\nu z} \dz \dy \dx \\
                 &= \int_0^{+\infty} \lambda e^{-\lambda x} \int_x^{+\infty} \mu e^{-(\mu+\nu)y} \dy \dx \\
                 &= \frac{\lambda \mu}{\mu + \nu} \int_0^{+\infty} e^{-(\lambda + \mu + \nu)x} \dx \\
                 &= \frac{\lambda \mu}{(\mu + \nu)(\lambda + \mu + \nu)}.
  \end{align*}
\end{enumerate}

\end{document}